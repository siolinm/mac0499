% Baseado no exemplo disponibilizado em: https://github.com/victorsenam/tcc/poster

\documentclass[final]{beamer}
\usepackage[size=custom,width=70.7,height=100,scale=1.0]{beamerposter}

\usepackage[brazil]{babel}
\usepackage[utf8]{inputenc}
\usepackage[T1]{fontenc}

\usepackage{tikz}
\usetikzlibrary{matrix,shapes,positioning,shadows,trees,patterns}

% \usepackage[shortlabels]{enumitem}
\usepackage{ragged2e}
\usepackage[numbers]{natbib}
\bibliographystyle{plainnat}

\sloppy

%----------------------------------------------------------------------------------------
%	SHORTCUTS
%----------------------------------------------------------------------------------------
\newcommand{\B}[1]{\mathbb{#1}}
\newcommand{\Cl}[1]{\ensuremath{\mathcal{#1}}}

\newcommand{\sse}{\Leftrightarrow}
\newcommand{\so}{\Rightarrow}
\newcommand{\se}{\Leftarrow}
\newcommand{\rec}{\leftarrow}

\newcommand{\Maxima}{\textit{Maxima}}
\newcommand{\Dom}{\textit{Dom}}
\newcommand{\Cands}{\textit{Cands}}
\newcommand{\Hits}{\textit{Hits}}
\newcommand{\lcand}{\textit{lcand}}
\newcommand{\up}{\textit{up}}
\newcommand{\low}{\textit{low}}
\newcommand{\Null}{\text{NULL}}

\newcommand{\tdots}{\,.\,.\,}

%----------------------------------------------------------------------------------------
%	BEAMER STYLE
%----------------------------------------------------------------------------------------

\usetheme{poster-exemplo}
\setbeamercolor{block title}{fg=dblue,bg=white}
\setbeamercolor{block body}{fg=black,bg=white}
\setbeamercolor{block alerted title}{fg=dblue,bg=gray!50}
\setbeamercolor{block alerted body}{fg=black,bg=gray!20}
\setbeamercolor{block prob}{fg=black,bg=white}
\setbeamertemplate{caption}[numbered]

%----------------------------------------------------------------------------------------
%	CUSTOM STYLING
%----------------------------------------------------------------------------------------

\newenvironment<>{prob}{
    \begin{beamercolorbox}[sep=1ex,center,dp={1ex}]{block prob}
    \textcolor{dblue}{\textbf{Problema:}}\itshape
}{\end{beamercolorbox}}

%----------------------------------------------------------------------------------------
%	POSTER
%----------------------------------------------------------------------------------------

\title{Estruturas de dados cinéticas}
\author{Marcos Siolin Martins \hspace{100pt} Orientadora: Profª. Drª. Cristina Gomes Fernandes}
\institute{\vspace{10pt}Departamento de Ciência da Computação,
Instituto de Matemática e Estatística, Universidade de São Paulo}


\begin{document}
\begin{frame}[t]
	%\vspace{-3ex}
	\begin{columns}[t]


		\begin{column}{0.35\paperwidth}% the right size for a 3-column layout

			\begin{block}{Introdução}

                Este trabalho discute algumas estruturas de dados cinéticas, propostas por Basch, Guibas e Hershberger~\cite{BGH99} para a
                utilização em algoritmos que resolvem os chamados problemas \textit{cinéticos}. Problemas \textit{cinéticos} são problemas
                em que deseja-se manter um determinado atributo sobre objetos que estão em movimento contínuo. Os problemas estudados neste
                trabalho envolvem objetos se movendo apenas em trajetórias lineares.
				% Os objetos nos problemas podem representar entidades do mundo físico: pontos podem representar pessoas, aviões,
                % estabelecimentos, entre outras coisas, retas podem representar trajetórias.

                % \vskip2ex
                % \begin{figure}
                %     \centering
                %     \begin{tikzpicture}[thick, scale=1.6]
                %         % \draw[step=1cm,lightgray,very thin] (-2,-2) grid (10,10);
                %         \draw[thick,->] (0,0) -- (7,0) node[anchor=north west]
                %             {$x$};
                %         \draw[thick,->] (0,0) -- (0,5) node[anchor=south east]
                %             {$y$};
                %         \foreach \Point in {(1, 0), (5, -1), (0, 2), (3, 2), (3, 1)}{
                %             \fill \Point circle[radius=2pt];
                %             % \node at \Point {\textbullet};
                %         }
                %         \draw[->] (1, 0) -- (1.5, 0.25);
                %         \draw[->] (5, -1) -- (4.75, -0.5);
                %         \draw[->] (0, 2) -- (0.25, 1.75);
                %         \draw[->] (3, 2) -- (3.25, 1.5);
                %         \draw[->] (3, 1) -- (2.5, 1);
                %         \foreach \Point in {(5, -2), (1, 1), (5, 2), (2,0), (3, 3)}{
                %             % \fill \Point circle[radius=2pt];
                %             \draw \Point circle[radius=2pt];
                %         }
                %         \draw[dashed] (3, 1) -- (3, 2);
                %         \draw[dotted] (1, 1) -- (2, 0);
                %     \end{tikzpicture}
                %     \caption{Exemplo de problema cinético: par mais próximo. Os pontos preenchidos em preto representam a coleção no
                %     instante $t = 0$. Os pontos não preenchidos representam a coleção no instante $t = 2$. As linhas representam o par mais
                %     próximo em cada instante.}
                %     \label{fig:parestatico:exemplo}
                % \end{figure}

				% Quando é dado um conjunto fixo de objetos geométricos, e deseja-se saber informações de um determinado atributo desses
				% objetos (como, por exemplo, em um conjunto dado de pontos, qual par de pontos possui distância mínima), dizemos que esse é
				% um problema \textit{estático}.
                % \vskip2ex

				% O mesmo problema pode ser formulado sobre um conjunto mutável. Por exemplo, pontos poderiam ser inseridos e removidos ao
				% longo do tempo. Queremos calcular o atributo sem ter que resolver do zero a nova instância do problema estático. Chamamos
				% esse tipo de problema de \textit{dinâmico} ou \textit{on-line}.
                % \vskip2ex

				% As \emph{estruturas de dados cinéticas} recebem esse nome para diferenciá-las das estruturas de dados \textit{estáticas} e
				% \textit{dinâmicas}. Elas têm como foco manter a descrição combinatória do problema que se altera frequentemente com a
				% passagem de tempo, já que os objetos estão em movimento contínuo.

			\end{block}

			\begin{block}{Ordenação cinética}

				Para entender a interface geral das estruturas de dados cinéticas precisamos entender o que são \textit{certificados},
				\textit{prazos de validade}, eventos e \textit{plano de vôo}. Vamos utilizar um problema e uma estrutura simples para
				exemplificar essas ideias: a \textit{ordenação cinética} e a \textit{lista ordenada cinética}.

                \vskip2ex

				Considere o seguinte problema cinético. Dada uma coleção de $n$ pontos em movimento retilíneo, o objetivo é responder
				consultas do tipo: para um certo $i$, com $1 \leq i \leq n$, quem é o $i$-ésimo maior valor da coleção no instante corrente.

				\begin{figure}[htb]
					\centering
					\begin{tikzpicture}[thick, scale=1.3]
						% \draw[step=1cm,lightgray,very thin] (-2,-2) grid (10,10);
						\draw[thick,->] (0,0) -- (8,0) node[anchor=north west]
						{$t$};
						\draw[thick,->] (0,0) -- (0,8) node[anchor=south east]
						{valor$(t)$};
						\foreach \x in {1, 2,..., 7}
							\draw (\x cm,1pt) -- (\x cm,-1pt)
								node[anchor=north] {$\x$};
						\foreach \y in {0, 1, ..., 7}
							\draw (1pt,\y cm) -- (-1pt,\y cm)
								node[anchor=east] {$\y$};
						\draw[solid, thick, color=blue] (0, 6) -- (6, 3)
							node[anchor=west] {elemento 1};
						\draw[densely dotted, thick, color=red] (0, 5) -- (6, 5)
							node[anchor=west] {elemento 2};
						\draw[densely dashed, thick, color=black!30!green]
							(0, 3) -- (6, 4.5) node[anchor=west] {elemento 3};
						\draw[loosely dashdotted, thick, color=black]
							(0, 0) -- (6, 8) node[anchor=west] {elemento 4};
						\draw[dotted, thick, color=black]
							(3, 0) -- (3, 8) node[anchor=south] {$t = 3$};
						\draw[dotted, thick, color=black]
							(5, 0) -- (5, 8) node[anchor=south] {$t = 5$};
					\end{tikzpicture}
					\caption{O segundo maior valor no instante $t = 0$ é do elemento~$2$, e no instante $t = 3$ é do elemento~$1$. O menor
							valor da coleção no instante $t = 3$ é do elemento~$3$, e no instante $t = 5$ é do elemento~$1$.}
				\end{figure}

				Queremos dar suporte às seguintes operações:
				\begin{itemize}
					\justifying
					\item \textsc{advance}$(t)$ $\rightarrow$ avança o tempo
					corrente para $t$;
					\item \textsc{change}$(j, v)$ $\rightarrow$ altera a
					velocidade do elemento $j$ para $v$;
					\item \textsc{query\_kth}$(i)$ $\rightarrow$ devolve o
					elemento cujo valor é o $i$-ésimo maior no instante atual.
				\end{itemize}

				Para implementar estas operações vamos manter um vetor com os elementos dados em ordem decrescente do valor no instante
				atual, ele será utilizado para responder as consultas de quem é o $i$-ésimo. O vetor começa ordenado com os valores iniciais
				dos elementos.

				\vskip2ex

				Uma vez de posse do vetor ordenado com os valores iniciais decrescentemente, construímos um certificado para cada par de
				elementos consecutivos no vetor.

				\vskip2ex

				Os certificados estabelecem que uma relação entre um objeto da estrutura e outro se mantém verdadeira até o seu vencimento.
				No caso da lista, o instante de tempo em que um elemento ultrapassa o valor do elemento seguinte na lista. Chamaremos o
				instante de tempo em que o certificado vence de valor ou \textit{prazo de validade} do certificado.

                \vskip2ex

				Esses prazos de validade determinam os \underline{eventos} que potencialmente causarão modificações na nossa estrutura e
				consequentemente em alguns certificados. Quando avançarmos no tempo trataremos dos eventos que surgem antes do momento em
				que queremos avançar.

				\begin{figure}[htb]
					\centering
					\begin{tikzpicture}[thick, scale=1.5]
						\edef\pos{0}
						\foreach \x in {1, 2,..., 5}{
							\pgfmathparse{\pos+2}
							\xdef\pos{\pgfmathresult}
							\node[circle,draw, minimum size=1cm] (\x) at
								(\pos, 0) {$\x$};
							% \node  at (\pos, -1) {$\x$};
						}
						% \foreach \x [evaluate=\x as \y using int(\x + 1)] in {1, 2,..., 4}{
						%     \ifthenelse{\x==2}{\draw[->, draw=red] (\x) -- (\y);}{\draw[->, draw=black] (\x) -- (\y);}
						% }

						% \foreach \x [evaluate=\x as \y using int(\x + 1)] in {1, 2,..., 4}{
							%     \ifthenelse{\x==2}{\draw[->, draw=red] (\x) -- (\y);}{\draw[->, draw=black] (\x) -- (\y);}
							% }
						\draw[->, color=blue, dashdotted] (1) -- (2);
						\draw[->, color=blue, dashdotted] (2) -- (3);
						\draw[->, color=blue, dashdotted] (3) -- (4);
						\draw[->] (4) -- (5);
					\end{tikzpicture}
					\caption{Ao avançar no tempo passamos por um evento gerado pelo vencimento do certificado do elemento 2. O tratamento
						desse evento resulta na atualização da estrutura e dos certificados afetados. }
				\end{figure}

                % \vskip2ex

				% Para implementar a operação \textsc{advance}, é necessário que nenhum certificado esteja vencido no instante $t$, ou seja, o
				% certificado de menor prazo de validade expira após o instante $t$. O certificado de menor prazo de validade será obtido
				% através de uma fila com prioridades que manteremos. Essa fila utiliza como prioridade o prazo de validade do certificado.

				% \vskip2ex

				Para calcular o prazo de validade dos certificados, utilizaremos o chamado \textit{plano de vôo} dos objetos. O plano de vôo
				de um objeto é uma função que, dado o instante de tempo atual, determina sua trajetória.

				\vskip2ex

				O plano de vôo pode sofrer mudanças. Essas mudanças serão feitas através da operação \textsc{change} e geram a necessidade
				de atualização de certificados e de ajustes nas estruturas.

				\vskip2ex

				Basch, Guibas e Hershberger propuseram alguns critérios de desempenho como forma de analisar e medir as estruturas. São
				eles: \textit{responsividade}, \textit{eficiência}, \textit{compacidade} e \textit{localidade}.

				% Agora que discutimos algumas definições podemos dar o panorama geral de como funcionam as estruturas de dados cinéticos. A
				% estrutura em que serão feitas as consultas é inicializada de acordo com o estado inicial dos objetos. Para cada objeto é
				% fornecida uma função que define a sua trajetória ao longo do tempo. Essa função é utilizada para calcular certificados que
				% serão colocados numa fila de prioridades. Para avançar no tempo é necessário tratar os eventos que vão surgindo.

				% % detalhe importante: a operação query só responde para um instante >= atual
				% Por fim, a operação \textsc{query} ficará responsável por responder o atributo geométrico que desejamos saber num dado
				% instante.
            \end{block}

			\begin{block}{Referências}
				\scriptsize{\begin{thebibliography}{50}
                        \bibitem{BGH99}
                        Basch, J., Guibas, L., Hershberger, J., ``\textbf{Data structures for mobile data},'' in \textit{Journal of Algorithms}, 1999, 31 (1), pp. 1--28
						% \bibitem{Aggarwal}
						% Aggarwal, Alok and Klawe, Maria M. and Moran, Shlomo and Shor, Peter and Wilber, Robert, ``\textbf{Geometric applications of a matrix-searching algorithm},`` in \textit{Algorithmica. Springer}, 1987, 2 (1-4), pp. 195--208.

						\bibitem{Eduardo}
						Freitas, E. G., ``\textbf{Problemas Cinéticos em Geometria Computacional},'' \textit{Dissertação de mestrado},
						Universidade de São Paulo, 2000. https://www.teses.usp.br

						\bibitem{Basch}
						Basch, J., ``\textbf{Kinetic data structures},'' \textit{PhD Thesis}, Stanford University, 1999. http://www.basch.org/phdthesis
						% Burkard, Rainer E. and Klinz, Bettina and Rudolf, R{\"u}diger, ``\textbf{Perspectives of Monge properties in optimization},`` in \textit{Discrete Applied Mathematics. Elsevier}, 1996, 70 (2), pp. 95--161.

						% Galil, Zvi and Park, Kunsoo, ``\textbf{Dynamic programming with convexity, concavity and sparsity},`` in \textit{Theoretical Computer Science. Elsevier}, 1992, 92 (1), pp. 49--76.

						% \bibitem{Knuth}
						% Knuth, Donald E., ``\textbf{Optimum binary search trees},`` in \textit{Acta Informatica. Springer,} 1971, 1 (1), pp. 14--25.

						% \bibitem{Yao}
						% Yao, F. Frances, ``\textbf{Efficient dynamic programming using quadrangle inequalities},`` in \textit{Proceedings of the Twelfth Annual ACM Symposium on Theory of Computing. ACM}, 1980, 3 (4), pp. 532--540.

					\end{thebibliography}}
			\end{block}


		\end{column}


		\begin{column}{0.60\paperwidth}

			\begin{columns}[c,totalwidth=0.60\paperwidth]

				\begin{column}{0.47\columnwidth}
					% \begin{block}{Critérios de desempenho}

					% 	Uma questão natural a ser feita a respeito destas estruturas é como medir o desempenho delas. Basch, Guibas e
					% 	Hershberger~\cite{BGH99} propuseram algumas formas de analisá-las e medi-las. São elas:
					% 	\begin{itemize}
					% 		\justifying
					% 		\item Responsividade: uma estrutura é dita \textit{responsiva} se o custo de atualizar os certificados e as
					% 		outras estruturas necessárias é pequeno;
					% 		\item Eficiência: uma estrutura é dita \textit{eficiente} se a razão entre a
					% 		quantidade total de eventos processados e a quantidade de eventos
					% 		\textit{externos} é pequena. Um evento diz respeito ao vencimento de um
					% 		certificado, os eventos chamados \textit{externos} são eventos que geram
					% 		mudanças na descrição combinatória, enquanto eventos chamados
					% 		\textit{internos} não geram mudanças na descrição combinatória, mas ainda
					% 		são necessários para manter a estrutura. O total de eventos processados é a
					% 		soma da quantidade de eventos externos e internos;
					% 		\item Compacidade: uma estrutura é dita \textit{compacta} se a quantidade máxima de
					% 		certificados que podem estar na fila com prioridades em um determinado é
					% 		instante é linear;
					% 		\item Localidade: uma estrutura é dita \textit{local} se a quantidade máxima
					% 		de certificados na fila que estão relacionados com um determinado objeto é
					% 		pequena.
					% 	\end{itemize}
					% \end{block}

					\begin{block}{Par mais próximo cinético}
						Vamos agora falar de um outro problema cinético interessante, o problema do par mais próximo cinético. Num conjunto
						dado de pontos em movimento, determinar qual par de pontos possui distância mínima. Explicaremos as principais
						ideias de um algoritmo proposto por Basch, Guibas e Hershberger que utiliza a técnica de linha de varredura para
						resolver o problema estático e depois mostraremos algumas ideias de como tratar os eventos na sua versão cinética.

						\vskip2ex

						O algoritmo é baseado na ideia de dividir o plano, para cada ponto, em seis cones iguais. Os cones são delimitados
						pela reta paralela ao eixo $y$ que passa pelo ponto e pelas retas $x \pm 30^\circ$, isto é, as retas que passam pelo
						ponto e formam $\pm 30^\circ$ com o eixo $x$.

						\begin{figure}
							\centering
							\begin{tikzpicture}[thick, scale=1.3]
								\draw (0, -3) -- (0, 3) node[anchor=north west] {$y$};
								\draw (-3, 1.7302) -- (3, -1.7302)
									node[anchor=north west] {$x - 30^\circ$};
								\draw (-3, -1.7302) -- (3, 1.7302)
									node[anchor=south west] {$x + 30^\circ$};
								\node[label=250:$p$] (p) at (0, 0) {\textbullet};
							\end{tikzpicture}
							\caption{A reta paralela ao eixo $y$ que passa por $p$ e as retas~$x \pm 30^\circ$.}
						\end{figure}

						Tendo dividido o plano em cones, a ideia é achar o ponto mais próximo de $p$ dentro de cada um desses cones. Se
						assim o fizermos para todos os pontos, um desses pares possui a menor distância entre si e será o par mais próximo
						que buscamos. Na prática, utilizaremos apenas os cones a direita do ponto.

						\vskip2ex

						Precisamos também definir alguns conjuntos que serão utilizados no algoritmo:
						\begin{itemize}
							\justifying
							\item $\Dom(p)$: \textit{dominância de p}. É o cone cujo eixo central é paralelo ao eixo $x$;
       						\item $\Maxima(p)$: o conjunto dos pontos à direita de $p$ que não pertencem a \textit{dominância} de nenhum
							ponto à direita de $p$;
							\item $\Cands(p)$: os \textit{candidatos} de $p$ são aqueles pontos à direita de $p$ que não pertencem a
							\textit{dominância} de nenhum ponto à direita de $p$ e pertencem a \textit{dominância} de $p$.
						\end{itemize}

						\begin{figure}
							\centering
							\begin{tikzpicture}[thick, scale=1.3]
								\node[label={[label distance = -4mm]160:$p$}]
									at (0.00, 0.00) {\textbullet};
								\node[label={[label distance = -4mm]90:$a$}]
									(a) at (3.00, 3.00) {\textbullet};
								\node[label={[label distance = -4mm]160:$b$}]
									(b) at (2.50, 2.00) {\textbullet};
								\node[label={[label distance = -4mm]160:$c$}]
									(c) at (3.50, 1.00) {\textbullet};
								\node[label={[label distance = -4mm]160:$d$}]
									at (4.00, -0.60) {\textbullet};
								\node[label={[label distance = -4mm]160:$e$}]
									at (5.00, -2.00) {\textbullet};
								\node[label={[label distance = -4mm]220:$f$}]
									(f) at (4.50, -3.00) {\textbullet};

								% e cone
								\draw (5.00, -2.00) -- (6.00, -2.58);
								\draw (5.00, -2.00) -- (6.00, -1.42);
								% f cone
								\draw (4.50, -3.00) -- (6.00, -3.87);
								\draw (4.50, -3.00) -- (5.62, -2.36);
								% d cone
								\draw (4.00, -0.60) -- (5.71, -1.59);
								\draw (4.00, -0.60) -- (6.00, 0.55);
								% c cone
								\draw (3.50, 1.00) -- (5.14, 0.06);
								\draw (3.50, 1.00) -- (6.00, 2.44);
								% a cone
								\draw (3.00, 3.00) -- (4.98, 1.86);
								\draw (3.00, 3.00) -- (6.00, 4.73);
								% b cone
								\draw (2.50, 2.00) -- (3.87, 1.21);
								\draw (2.50, 2.00) -- (3.62, 2.64);
								% p cone
								\draw[line width = 0.5mm] (0.00, 0.00) -- (4.85, -2.80);
								\draw[line width = 0.5mm] (0.00, 0.00) -- (2.98, 1.72);

								\draw[dashed] (6,-2.5) -- (8, -2.5)
									node[anchor=west, label=90:$\Cands(p)$] {};
								\draw[dashed] (8, 1.5) -- (8, -2.5);
								\draw[dashed] (5,1.5) -- (8, 1.5);

								\draw[dashed] (4, -4) -- (10, -4);
								\draw[dashed] (10, -4) -- (10, 4);
								\draw[dashed] (5, 4) -- (10, 4)
									node[anchor=south, label=$\Maxima(p)$] {};

								\node[label={[label distance = -3mm]270:$\low(p)$}]
									(low) at (3, -4) {};
								\draw[->] (low) edge[out=90,in=160] (f);

								\node[label={[label distance = -5mm]270:$\lcand(p)$}]
									(lc) at (2, 0) {};
								\draw[->] (lc) edge[out=90,in=200] (c);

								\node[label={[label distance = -3mm]90:$\up(p)$}]
									(up) at (1.5, 3) {};
								\draw[->] (up) edge[out=270,in=180] (b);
							\end{tikzpicture}
							\caption{Os pontos $c$, $d$ e $e$ pertencem a $\Cands(p)$, e todos os pontos exceto $p$ pertencem a
							$\Maxima(p)$. O ponto $b$ é $\up(p)$ e o ponto $f$ é $\low(p)$. O ponto $c$ é $\lcand(p)$.}
						\end{figure}

						Chamaremos o ponto de $\Maxima$ de menor ordenada que está acima de $\Dom(p)$ de $\up(p)$ e chamaremos o ponto de
						$\Maxima$ de maior ordenada que está abaixo de $\Dom(p)$ de $\low(p)$. Os pontos estritamente entre $\low(p)$ e $\up(p)$
						são justamente os de $\Cands(p)$.

						\vskip2ex

						Dentre os \textit{candidatos} de $p$, chamaremos o ponto com menor coordenada $x$ de $lcand(p)$. Consideraremos
						apenas os pares $(p, \lcand(p))$ como candidatos a par mais próximo.

						\vskip2ex

						O algoritmo varre o plano da direita para esquerda três vezes: com o sistema de coordenadas rotacionado $60^\circ$,
						$0^\circ$ e -$60^\circ$. Em cada varredura, computa as estruturas necessárias para comparar apenas os pares $(p,
						lcand(p))$ para determinar o par mais próximo em tempo $O(n\lg{n})$.

					\end{block}

				\end{column}

				\begin{column}{0.5\columnwidth}

					\begin{block}{``Cinetização'' do algoritmo estático}
						Para ``cinetizar'' o algoritmo estático utilizaremos duas estruturas de dados cinéticas. Primeiramente, teremos os
						certificados das três \textit{listas ordenadas cinéticas}, que guardarão a ordem dos pontos de acordo com os eixos
						$x$, $x + 60^\circ$ e $x - 60^\circ$.

						\vskip2ex

						Para garantir qual, dentre os pares $(p, \lcand(p))$, é o par mais próximo usaremos um \textit{torneio cinético},
						estrutura de dados cinética utilizada em algoritmos para resolver o problema do máximo (ou mínimo) cinético.

						\vskip2ex

						Também precisaremos manter informação guardada para atualizar com eficiência mudanças provocadas por trocas na ordem
						dos pontos em relação a um dos três eixos. Por exemplo, uma troca na ordem dos pontos pode acarretar na mudança nos
						conjuntos $\Cands(p)$ e $\Cands(q)$.

						\vskip2ex

						Para que consigamos manter $\lcand(p)$ de maneira eficiente, cada ponto terá três árvores binárias de busca
						associadas a ele com os conjuntos $\Cands(p)$, $\Hits_{up}(p)$ e $\Hits_{low}(p)$. A árvore $\Hits_{up}(p)$ guarda
						os pontos $q$ tais que $\up(q) = p$, enquanto a árvore $\Hits_{low}(p)$ guarda os pontos $q$ tais que $\low(q) = p$.

						\vskip2ex

						Cada uma das três árvores têm sua raiz apontando para o nó $p$, e cada nó das árvores aponta para o seu nó pai. Na
						árvore $\Cands(p)$ cada nó deve apontar para o descendente que contém o ponto mais à esquerda na ordenação
						horizontal. Ademais, cada ponto tem um ponteiro para os nós que o tem como chave.

						\vskip2ex

						Trocas na ordem dos pontos geram mudanças em três casos:

						\begin{figure}
							\centering
							\begin{tikzpicture}[thick, scale=0.6]
								\node[label={[label distance = -3mm]160:$p$}] at
									(2.00, 2.00) {\textbullet};
								\node[label={[label distance = -3mm]160:$q$}] at
									(3.00, 5.00) {\textbullet};
								\node[label={[label distance = -2mm]90:$a$}] at
									(6.00, 5.00) {\textbullet};
								\node[label={[label distance = -3mm]90:$b$}] at
									(8.00, 3.00) {\textbullet};
								\node[label={[label distance = -3mm]90:$c$}] at
									(7.00, 1.00) {\textbullet};
								\node[label={[label distance = -3mm]90:$d$}] at
									(9.00, -1.00) {\textbullet};
								\node[label={[label distance = -3mm]90:$e$}] at
									(5.00, 8.00) {\textbullet};
								\node[label={[label distance = -3mm]90:$f$}] at
									(6.00, 10.00) {\textbullet};
								\node[label={[label distance = -3mm]90:$g$}] at
									(6.00, -2.00) {\textbullet};
								% d cone
								\draw (9.00, -1.00) -- (15.00, -4.46);
								\draw (9.00, -1.00) -- (15.00, 2.46);
								% b cone
								\draw (8.00, 3.00) -- (11.96, 0.71);
								\draw (8.00, 3.00) -- (15.00, 7.04);
								% c cone
								\draw (7.00, 1.00) -- (9.73, -0.58);
								\draw (7.00, 1.00) -- (9.23, 2.29);
								% f cone
								\draw (6.00, 10.00) -- (13.06, 5.92);
								\draw (6.00, 10.00) -- (15.00, 15.20);
								% g cone
								\draw (6.00, -2.00) -- (15.00, -7.20);
								\draw (6.00, -2.00) -- (9.10, -0.21);
								% a cone
								\draw (6.00, 5.00) -- (8.73, 3.42);
								\draw (6.00, 5.00) -- (10.33, 7.50);
								% e cone
								\draw (5.00, 8.00) -- (8.10, 6.21);
								\draw (5.00, 8.00) -- (7.23, 9.29);
								% q cone
								\draw[line width = 0.5mm] (3.00, 5.00) -- (8.46, 1.85);
								\draw[line width = 0.5mm] (3.00, 5.00) -- (6.60, 7.08);
								% p cone
								\draw[line width = 0.5mm] (2.00, 2.00) -- (7.46, -1.15);
								\draw[line width = 0.5mm] (2.00, 2.00) -- (5.10, 3.79);

								\draw[dashed] (2, -3) -- (2, 8);
								\draw[dashed] (3, 3) -- (3, 8);

								\node at (15, 5) {$ \leftrightarrow$};

								\node[label={[label distance = -3mm]160:$p$}] at
									(18.00, 2.00) {\textbullet};
								\node[label={[label distance = -3mm]160:$q$}] at
									(17.00, 5.00) {\textbullet};
								\node[label={[label distance = -2mm]90:$a$}] at
									(21.00, 5.00) {\textbullet};
								\node[label={[label distance = -3mm]90:$b$}] at
									(23.00, 3.00) {\textbullet};
								\node[label={[label distance = -1mm]90:$c$}] at
									(22.00, 1.00) {\textbullet};
								\node[label={[label distance = -3mm]90:$d$}] at
									(24.00, -1.00) {\textbullet};
								\node[label={[label distance = -3mm]90:$e$}] at
									(20.00, 8.00) {\textbullet};
								\node[label={[label distance = -3mm]90:$f$}] at
									(21.00, 10.00) {\textbullet};
								\node[label={[label distance = -3mm]90:$g$}] at
									(21.00, -2.00) {\textbullet};
								% d cone
								\draw (24.00, -1.00) -- (30.00, -4.46);
								\draw (24.00, -1.00) -- (30.00, 2.46);
								% b cone
								\draw (23.00, 3.00) -- (26.96, 0.71);
								\draw (23.00, 3.00) -- (30.00, 7.04);
								% c cone
								\draw (22.00, 1.00) -- (24.73, -0.58);
								\draw (22.00, 1.00) -- (24.23, 2.29);
								% f cone
								\draw (21.00, 10.00) -- (28.06, 5.92);
								\draw (21.00, 10.00) -- (30.00, 15.20);
								% g cone
								\draw (21.00, -2.00) -- (30.00, -7.20);
								\draw (21.00, -2.00) -- (24.10, -0.21);
								% a cone
								\draw (21.00, 5.00) -- (23.73, 3.42);
								\draw (21.00, 5.00) -- (25.33, 7.50);
								% e cone
								\draw (20.00, 8.00) -- (23.10, 6.21);
								\draw (20.00, 8.00) -- (22.23, 9.29);
								% p cone
								\draw[line width = 0.5mm] (18.00, 2.00) -- (22.96, -0.87);
								\draw[line width = 0.5mm] (18.00, 2.00) -- (22.10, 4.37);
								% q cone
								\draw[line width = 0.5mm] (17.00, 5.00) -- (20.10, 3.21);
								\draw[line width = 0.5mm] (17.00, 5.00) -- (21.10, 7.37);

								\draw[dashed] (17, -3) -- (17, 8);
								\draw[dashed] (18, -1) -- (18, 4);
							\end{tikzpicture}
							\caption{Troca na ordem horizontal. Da esquerda para direita, o caso em que $p$ está em $\Hits_{up}(q)$. Da
							direita para esquerda, o caso em que $q$ está em $Hits_{low}(p)$.}
						\end{figure}

						\begin{figure}
							\centering
							\begin{tikzpicture}[thick, scale=0.6]
								\node[label={[label distance = -3mm]160:$p$}] at
									(2.00, 4.00) {\textbullet};
								\node[label={[label distance = -3mm]160:$q$}] at
									(6.00, 1.00) {\textbullet};
								\node[label={[label distance = -2mm]90:$a$}] at
									(5.25, -2.00) {\textbullet};
								% \node[label={[label distance = -3mm]160:$b$}] at (-5.00, 0.00) {\textbullet};
								\node[label={[label distance = -3mm]90:$c$}] at
									(-1.00, -4.00) {\textbullet};
								\node[label={[label distance = -1mm]90:$b$}] at
									(-2.00, 0.00) {\textbullet};
								\node[label={[label distance = -1mm]90:$e$}] at
									(0.50, -4.00) {\textbullet};

								% q cone
								\draw[line width = 0.5mm] (6.00, 1.00) -- (10.00, -1.31);
								\draw[line width = 0.5mm] (6.00, 1.00) -- (10.00, 3.31);
								% a cone
								\draw (5.25, -2.00) -- (10.00, -4.74);
								\draw (5.25, -2.00) -- (8.22, -0.28);
								% p cone
								\draw[line width = 0.5mm] (2.00, 4.00) -- (6.60, 1.35);
								\draw[line width = 0.5mm] (2.00, 4.00) -- (10.00, 8.62);
								% c cone
								\draw (-1.00, -4.00) -- (10.00, -10.35);
								\draw (-1.00, -4.00) -- (6.83, 0.52);
								% e cone
								\draw (0.50, -4.00) -- (10.00, -9.48);
								\draw (0.50, -4.00) -- (7.58, 0.09);
								% b cone
								\draw (-2.00, 0.00) -- (1.96, -2.29);
								\draw (-2.00, 0.00) -- (3.46, 3.15);
								% b cone
								% \draw (-5.00, 0.00) -- (0.46, -3.15);
								% \draw (-5.00, 0.00) -- (10.00, 8.66);

								\draw[dashed] (2.00, 4.00) -- (-1.00, 5.73);
								\draw[dashed] (6.00, 1.00) -- (2.00, 3.30);
								\node at (11, 0) {$ \leftrightarrow$};

								\node[label={[label distance = -3mm]160:$p$}] at
									(17.00, 4.00) {\textbullet};
								\node[label={[label distance = -3mm]160:$q$}] at
									(22.00, 2.00) {\textbullet};
								\node[label={[label distance = -2mm]90:$a$}] at
									(20.25, -2.00) {\textbullet};
								% \node[label={[label distance = -3mm]160:$b$}] at (15.00, 0.00) {\textbullet};
								\node[label={[label distance = -3mm]90:$c$}] at
									(14.00, -4.00) {\textbullet};
								\node[label={[label distance = -1mm]90:$b$}] at
									(13.00, 0.00) {\textbullet};
								\node[label={[label distance = -1mm]90:$e$}] at
									(15.50, -4.00) {\textbullet};

								% q cone
								\draw[line width = 0.5mm] (22.00, 2.00) -- (25.00, 0.27);
								\draw[line width = 0.5mm] (22.00, 2.00) -- (25.00, 3.73);
								% a cone
								\draw (20.25, -2.00) -- (25.00, -4.74);
								\draw (20.25, -2.00) -- (24.59, 0.51);
								% p cone
								\draw[line width = 0.5mm] (17.00, 4.00) -- (23.82, 0.06);
								\draw[line width = 0.5mm] (17.00, 4.00) -- (25.00, 8.62);
								% e cone
								\draw (15.50, -4.00) -- (25.00, -9.48);
								\draw (15.50, -4.00) -- (23.18, 0.43);
								% c cone
								\draw (14.00, -4.00) -- (25.00, -10.35);
								\draw (14.00, -4.00) -- (22.43, 0.87);
								% b cone
								\draw (13.00, 0.00) -- (16.96, -2.29);
								\draw (13.00, 0.00) -- (18.46, 3.15);
								% b cone
								% \draw (15.00, 0.00) -- (20.46, -3.15);
								% \draw (15.00, 0.00) -- (30.00, 8.66);

								\draw[dashed] (17.00, 4.00) -- (14.00, 5.73);
								\draw[dashed] (22.00, 2.00) -- (17.00, 4.88);
							\end{tikzpicture}
							\caption{Troca na ordem $60^\circ$. Da esquerda para direita, o caso em que $p$ está em $\Hits_{low}(q)$, ou
							seja, $q$ está entrando em $Dom(p)$. Da direita para esquerda, o caso em que $q$ está em $Cands(p)$, saindo de
							$Dom(p)$.}
						\end{figure}

						\begin{figure}[h]
							\centering
							\begin{tikzpicture}[thick, scale=0.6]
								\node[label={[label distance = -3mm]160:$p$}] at
									(6.00, 0.00) {\textbullet};
								\node[label={[label distance = -3mm]160:$q$}] at
									(8.00, 2.00) {\textbullet};
								\node[label={[label distance = -1mm]90:$a$}] at
									(0.00, 5.00) {\textbullet};
								\node[label={[label distance = -2mm]90:$b$}] at
									(2.00, 5.00) {\textbullet};
								\node[label={[label distance = 0mm]90:$c$}] at
									(4.00, 5.00) {\textbullet};
								\node[label={[label distance = 0mm]90:$d$}] at
									(10.00, 4.00) {\textbullet};
								\node[label={[label distance = 0mm]90:$e$}] at
									(8.00, 4.00) {\textbullet};

								% d cone
								\draw (10.00, 4.00) -- (14.00, 1.69);
								\draw (10.00, 4.00) -- (14.00, 6.31);
								% q cone
								\draw[dashed] (8.00, 2.00) -- (4.00, -0.30);
								\draw[line width = 0.5mm] (8.00, 2.00) -- (14.00, -1.46);
								\draw[line width = 0.5mm] (8.00, 2.00) -- (10.73, 3.58);
								% e cone
								\draw (8.00, 4.00) -- (9.73, 3.00);
								\draw (8.00, 4.00) -- (14.00, 7.46);
								% p cone
								\draw[dashed] (6.00, 0.00) -- (2.00, -2.30);
								\draw[line width = 0.5mm] (6.00, 0.00) -- (14.00, -4.62);
								\draw[line width = 0.5mm] (6.00, 0.00) -- (8.73, 1.58);
								% c cone
								\draw (4.00, 5.00) -- (8.60, 2.35);
								\draw (4.00, 5.00) -- (14.00, 10.77);
								% b cone
								\draw (2.00, 5.00) -- (8.33, 1.35);
								\draw (2.00, 5.00) -- (14.00, 11.93);
								% a cone
								\draw (0.00, 5.00) -- (7.33, 0.77);
								\draw (0.00, 5.00) -- (14.00, 13.08);

								\node at (14, 5) {$ \leftrightarrow$};

								\node[label={[label distance = -3mm]160:$p$}] at
									(22.00, 0.00) {\textbullet};
								\node[label={[label distance = -3mm]160:$q$}] at
									(25.00, 1.00) {\textbullet};
								\node[label={[label distance = -1mm]90:$a$}] at
									(16.00, 5.00) {\textbullet};
								\node[label={[label distance = -2mm]90:$b$}] at
									(18.00, 5.00) {\textbullet};
								\node[label={[label distance = 0mm]90:$c$}] at
									(20.00, 5.00) {\textbullet};
								\node[label={[label distance = 0mm]90:$d$}] at
									(26.00, 4.00) {\textbullet};
								\node[label={[label distance = 0mm]90:$e$}] at
									(24.00, 4.00) {\textbullet};

								% d cone
								\draw (26.00, 4.00) -- (31.00, 1.11);
								\draw (26.00, 4.00) -- (31.00, 6.89);
								% q cone
								\draw[dashed] (25.00, 1.00) -- (22.00, -0.73);
								\draw[line width = 0.5mm] (25.00, 1.00) -- (31.00, -2.46);
								\draw[line width = 0.5mm] (25.00, 1.00) -- (28.10, 2.79);
								% e cone
								\draw (24.00, 4.00) -- (27.10, 2.21);
								\draw (24.00, 4.00) -- (30.00, 7.46);
								% p cone
								\draw[dashed] (22.00, 0.00) -- (20.00, -1.15);
								\draw[line width = 0.5mm] (22.00, 0.00) -- (31.00, -5.20);
								\draw[line width = 0.5mm] (22.00, 0.00) -- (26.46, 2.58);
								% c cone
								\draw (20.00, 5.00) -- (25.33, 1.92);
								\draw (20.00, 5.00) -- (31.00, 11.35);
								% b cone
								\draw (18.00, 5.00) -- (24.33, 1.35);
								\draw (18.00, 5.00) -- (31.00, 12.51);
								% a cone
								\draw (16.00, 5.00) -- (23.33, 0.77);
								\draw (16.00, 5.00) -- (31.00, 13.66);
							\end{tikzpicture}
							\caption{Troca na ordem -$60^\circ$. Da esquerda para direita, o caso em que $p$ está em $\Hits_{up}(q)$, ou
							seja, $q$ está entrando em $\Dom(p)$. Da direita para esquerda, o caso em que $q$ está em $\Cands(p)$, saindo de
							$\Dom(p)$.}
						\end{figure}

					\end{block}
				\end{column}

			\end{columns}


			\begin{block}{Mais informações}
				Além dos problemas da ordenação cinética e do par mais próximo cinético, também foram estudadas estruturas para resolver o
				problema do máximo cinético e da triangulação de delaunay cinética. Algumas dessas estruturas foram implementadas e encontram-se
				em: \textcolor{jblue}{{\url{https://github.com/siolinm/mac0499}}}.

				Para mais informações acesse: \textcolor{jblue}{{\url{https://www.linux.ime.usp.br/~siolinm/mac0499}}}.
			\end{block}


		\end{column}

	\end{columns}
	% ---------------------------------------------------------------------------- %
\end{frame}
\end{document}
