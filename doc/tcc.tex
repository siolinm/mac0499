% Arquivo LaTeX de exemplo de dissertação/tese a ser apresentada à CPG do IME-USP
%
% Criação: Jesús P. Mena-Chalco
% Revisão: Fabio Kon e Paulo Feofiloff
% Adaptação para UTF8, biblatex e outras melhorias: Nelson Lago
%
% Except where otherwise indicated, these files are distributed under
% the MIT Licence. The example text, which includes the tutorial and
% examples as well as the explanatory comments in the source, are
% available under the Creative Commons Attribution International
% Licence, v4.0 (CC-BY 4.0) - https://creativecommons.org/licenses/by/4.0/


%%%%%%%%%%%%%%%%%%%%%%%%%%%%%%%%%%%%%%%%%%%%%%%%%%%%%%%%%%%%%%%%%%%%%%%%%%%%%%%%
%%%%%%%%%%%%%%%%%%%%%%%%%%%%%%% PREÂMBULO LaTeX %%%%%%%%%%%%%%%%%%%%%%%%%%%%%%%%
%%%%%%%%%%%%%%%%%%%%%%%%%%%%%%%%%%%%%%%%%%%%%%%%%%%%%%%%%%%%%%%%%%%%%%%%%%%%%%%%

% A opção twoside (frente-e-verso) significa que a aparência das páginas pares
% e ímpares pode ser diferente. Por exemplo, as margens podem ser diferentes ou
% os números de página podem aparecer à direita ou à esquerda alternadamente.
% Mas nada impede que você crie um documento "só frente" e, ao imprimir, faça
% a impressão frente-e-verso.
%
% Aqui também definimos a língua padrão do documento
% (a última da lista) e línguas adicionais.
%\documentclass[12pt,twoside,brazilian,english]{book}
\documentclass[12pt,twoside,english,brazilian]{book}

% Ao invés de definir o tamanho das margens, vamos definir os tamanhos do
% texto, do cabeçalho e do rodapé, e deixamos a package geometry calcular
% o tamanho das margens em função do tamanho do papel. Assim, obtemos o
% mesmo resultado impresso, mas com margens diferentes, se o tamanho do
% papel for diferente.
\usepackage[a4paper]{geometry}

\geometry{
  textwidth=152mm,
  hmarginratio=12:17, % 24:34 -> com papel A4, 24mm + 152mm + 34mm = 210mm
  textheight=237mm,
  vmarginratio=8:7, % 32:28 -> com papel A4, 32mm + 237mm + 28mm = 297mm
  headsep=11mm, % distância entre a base do cabeçalho e o texto
  headheight=21mm, % qualquer medida grande o suficiente, p.ex., top - headsep
  footskip=10mm,
  marginpar=20mm,
  marginparsep=5mm,
}

% Vários pacotes e opções de configuração genéricos; para personalizar o
% resultado, modifique estes arquivos.
%%%%%%%%%%%%%%%%%%%%%%%%%%%%%%%%%%%%%%%%%%%%%%%%%%%%%%%%%%%%%%%%%%%%%%%%%%%%%%%%
%%%%%%%%%%%%%%%%%%%%%%% CONFIGURAÇÕES E PACOTES BÁSICOS %%%%%%%%%%%%%%%%%%%%%%%%
%%%%%%%%%%%%%%%%%%%%%%%%%%%%%%%%%%%%%%%%%%%%%%%%%%%%%%%%%%%%%%%%%%%%%%%%%%%%%%%%

% Vários comandos auxiliares para o desenvolvimento de packages e classes;
% aqui, usamos em alguns comandos de formatação e condicionais.
\usepackage{etoolbox}
%\RequirePackage{pdftexcmds}
\usepackage{letltxmacro}
%\usepackage{ltxcmds}

% LaTeX 3
\usepackage{expl3}
\usepackage{xparse}

% Detecta se estamos usando pdftex, luatex, xetex etc.
\usepackage{iftex}

%\usepackage{xfp} % Floating-point calculations

\usepackage{regexpatch}

% O projeto LaTeX3 renomeou algumas macros em 2019-03-05 e removeu
% a compatibilidade com os nomes antigos em 2020-07-17 a partir de
% 2021-01-01 (veja o arquivo l3deprecation.dtx e o changelog em
% https://github.com/latex3/latex3/blob/main/l3kernel/CHANGELOG.md).
% Isso afetou a package regexpatch: versões antigas da package não
% funcionam com versões novas de LaTeX e vice-versa. Infelizmente,
% ubuntu 21.04 (hirsute) e debian 11 (bullseye) incluem essas versões
% incompatíveis e, portanto, a package regexpatch não funciona nesses
% ambientes. Talvez fosse possível contornar esse problema com a
% package latexrelease, mas isso afetaria muitos outros recursos.
% Ao invés disso, vamos restaurar manualmente a compatibilidade.
% TODO: remover isto após debian bullseye se tornar obsoleta,
%       provavelmente no final de 2024.
\makeatletter
\ExplSyntaxOn

\@ifpackagelater{regexpatch}{2021/03/21}
  {} % Se regexpatch é "nova", expl3 deve ser também; nada a fazer
  {
    % Talvez o correto seja 2021/01/01, mas na prática o resultado é o mesmo
    \@ifpackagelater{expl3}{2020/07/17}
      {
        % As versões são incompatíveis; vamos recuperar as macros preteridas
        \cs_gset:Npn \token_get_prefix_spec:N { \cs_prefix_spec:N }
        \cs_gset:Npn \token_get_arg_spec:N { \cs_argument_spec:N }
        \cs_gset:Npn \token_get_replacement_spec:N { \cs_replacement_spec:N }
      }
      {} % As duas packages são antigas e, portanto, compatíveis entre si
  }
\ExplSyntaxOff
\makeatother

% Algumas packages dependem de xpatch e tentam carregá-la, causando conflitos
% com regexpatch. Como regexpatch oferece todos os recursos de xpatch (ela
% é uma versão estendida de xpatch, mas ainda considerada experimental), vamos
% fazê-las acreditar que xpatch já foi carregada.
\expandafter\xdef\csname ver@xpatch.sty\endcsname{2012/10/02}

% Acrescenta a correção deste bug (contida na release 2018-11-28):
% https://github.com/latex3/latex2e/issues/94 . Se a correção não
% puder ser aplicada, temos uma versão de LaTeX que já a incorpora.
% Esse bug afeta apenas textos em duas colunas.
% TODO: remover após ubuntu 18.04 se tornar obsoleta (abril/2023)
\makeatletter
\patchcmd\@combinedblfloats{\box\@outputbox}{\unvbox\@outputbox}{}{}
\makeatother

% Arithmetic expressions in \set{length,counter} & \addto{length,counter};
% commands \widthof, \heightof, \depthof, \totalheightof, \settototalheight
\usepackage{calc}

% Algumas packages "padrão" da AMS, que são praticamente obrigatórias.
% Algumas delas devem ser carregadas antes de unicode-math ou das
% definições das fontes do documento. É preciso carregar amsthm após
% amsmath para o comando \qedhere funcionar dentro do ambiente align.
\usepackage{amssymb}
\usepackage{amsmath}
\usepackage{amsthm}

% "fontenc" é um parâmetro do NFSS (sistema de gestão de fontes do
% LaTeX; consulte "texdoc fntguide" e "texdoc fontenc"). O default
% é OT1, mas ele tem algumas limitações; a mais importante é que,
% com ele, palavras acentuadas não podem ser hifenizadas. Por
% conta disso, quase todos os documentos LaTeX utilizam o fontenc
% T1. A escolha do fontenc tem consequências para as fontes que
% podem ser usadas com NFSS; hoje em dia T1 tem mais opções de
% qualidade, então não se perde nada em usá-lo. A package fontspec
% (para gestão de fontes usando outro mecanismo, compatível apenas
% com lualatex e xelatex) carrega fontenc automaticamente, mas
% usando outra codificação ("TU" e não "T1"). Ainda assim, é útil
% carregar o fontenc T1 (antes de carregar fontspec!) para o caso
% de alguma fonte "antiga" ser utilizada no documento (embora isso
% não seja recomendado: lualatex e xelatex só são capazes de
% hifenizar palavras acentuadas com o fontenc TU).
\usepackage[T1]{fontenc}

\ifPDFTeX
  % O texto está escrito em utf8.
  \usepackage[utf8]{inputenc}

  % Permitem utilizar small caps + itálico (e outras pequenas
  % melhorias). Em geral, desnecessário com fontspec, a menos
  % que alguma package utilize especificamente. Algumas raras
  % packages de fontes podem causar conflitos com fontaxes, em
  % geral por utilizarem a package "concorrente" nfssext-cfr.
  \usepackage{fontaxes}
  \usepackage{mweights}

  % LaTeX substitui algumas sequências de caracteres, como
  % "fi", "fl" e outras, por caracteres especiais ("ligaduras").
  % Para que seja possível fazer copiar/colar ou buscas por
  % textos contendo essas ligaduras, o arquivo PDF precisa
  % conter uma tabela indicando quais são elas. Com fontes
  % OTF (LuaLaTeX ou XeLaTeX) isso não costuma ser um problema,
  % mas com pdfLaTeX pode ser. Estes dois comandos (que só
  % existem no pdfLaTeX) incluem uma tabela genérica que
  % funciona para a maioria das fontes. Veja a seção 5 de
  % http://www.tug.org/TUGboat/Articles/tb29-1/tb91thanh-fonts.pdf
  % Note que alguns visualizadores de PDF tentam "adivinhar"
  % o conteúdo da tabela quando ela está incompleta ou não
  % existe, então copiar/colar e buscas podem funcionar em
  % alguns visualizadores e em outros não.
  \input glyphtounicode.tex
  \pdfgentounicode=1
\else
  % Não é preciso carregar inputenc com LuaTeX e XeTeX, pois
  % com eles utf8 é obrigatório.
  \usepackage{fontspec}

  % Ao invés de usar o sistema tradicional de LaTeX para gerir
  % as fontes matemáticas, utiliza as extensões matemáticas do
  % formato otf definidas pela microsoft. Ao ativar esta package
  % o mecanismo tradicional não funciona mais! Há poucas fontes
  % com suporte a unicode-math.
  \usepackage{unicode-math}
\fi

% Acesso a símbolos adicionais, como \textrightarrow, \texteuro etc.,
% disponíveis na maioria das fontes através do fontenc TS1 ou mudando
% momentaneamente para computer modern/latin modern. Raramente útil
% com lualatex/xelatex, mas não causa problemas. Várias packages de
% fontes carregam textcomp, às vezes com opções específicas; assim,
% para evitar problemas, vamos carregá-la no final do preâmbulo para
% o caso de ela não ter sido carregada antes.
\AtBeginDocument{\usepackage{textcomp}}

% TeXLive 2018 inclui a versão 2.7a da package microtype e a versão
% 1.07 de luatex. Essa combinação faz aparecer um bug:
% https://tex.stackexchange.com/a/476742
% Aqui, aplicamos a solução sugerida, que não tem "contra-indicações".
\ifLuaTeX
  \usepackage{luatexbase}
\fi

% microajustes no tamanho das letras, espaçamento etc. para melhorar
% a qualidade visual do resultado.
\usepackage{microtype}

% Alguns "truques" (sujos?) para minimizar over/underfull boxes.
%
% Para fazer um texto justificado, é preciso modificar o tamanho dos espaços
% em cada linha para mais ou para menos em relação ao seu tamanho ideal. Para
% escolher as quebras de linha, TeX vai percorrendo o texto procurando lugares
% possíveis para quebrar as linhas considerando essa flexibilidade mas dentro
% de um certo limite mínimo/máximo. Nesse processo, ele associa a cada possível
% linha o valor *badness*, que é o nível de distorção do tamanho dos espaços
% daquela linha em relação ao ideal, e ignora opções que tenham badness muito
% grande (esse limite é dado por \tolerance). Depois de encontradas todas
% as possíveis quebras de linha e a badness de cada uma, TeX calcula as
% *penalties* das quebras encontradas, que são uma medida de quebras "ruins".
% Por exemplo, na configuração padrão, quebrar uma linha hifenizando uma
% palavra gera uma penalty de 50; já uma quebra que faça a última linha
% do parágrafo ficar sozinha na página seguinte gera uma penalty de 150.
% Finalmente, TeX calcula a "feiúra" de cada possível linha (demerits)
% com base na badness e nas penalties e escolhe a solução que minimiza os
% demerits totais do parágrafo. Os comandos \linebreak e \pagebreak funcionam
% simplesmente acrescentando uma penalty negativa ao lugar desejado para a
% quebra.
%
% Para cada fonte, o espaço entre palavras tem um tamanho ideal, um
% tamanho mínimo e um tamanho máximo (é possível obter os valores com
% \number\fontdimenX\font\relax, veja https://tex.stackexchange.com/q/88991 ).
% TeX nunca reduz um espaço para menos que o mínimo da fonte, mas pode
% aumentá-lo para mais que o máximo. Se os espaços de uma linha ficam
% com o tamanho ideal, a badness da linha é 0; se o tamanho é
% reduzido/aumentado 50% do mínimo/máximo, a badness da linha é 12; se
% o tamanho é reduzido/aumentado para o mínimo/máximo, a badness é 100,
% e assim por diante. O valor máximo possível para badness é 10.000, que
% significa "badness infinita". Como é feito o cálculo: se as medidas
% do espaço definidas pela fonte são "x plus y minus z" e o tamanho
% final do espaço é "x + k*y" ou "x - k*z", a badness é 100*(k^3). Com
% Libertinus corpo 12, os valores são "3pt plus 1.5pt minus .9996pt",
% Então se o espaço tiver sido aumentado para 3.75pt, o fator é 0.5 e
% a badness é 100*(.5^3) = 12.
%
% \tolerance indica a badness máxima que TeX aceita para uma linha; seu valor
% default é 200. Assim, aumentar para, digamos, 300 ou 400, permite que
% TeX escolha parágrafos com maior variação no espaçamento entre as palavras.
% No entanto, no cálculo de demerits, a badness e as penalties de cada linha
% são elevadas ao quadrado, então TeX geralmente prefere escolher outras
% opções no lugar de uma linha com espaçamento ruim. Por exemplo, órfãs/viúvas
% têm demerit de 22.500 e dois hífens seguidos têm demerit de 10.000; já uma
% linha com badness 400 tem demerit 160.000. Portanto, não é surpreendente que
% a maioria dos parágrafos tenha demerits abaixo de 40.000, quase todos abaixo
% de 100.000 e praticamente nenhum acima de 1.000.000. Isso significa que, para
% a grande maioria dos parágrafos, aumentar \tolerance não faz diferença: uma
% linha com badness 400 nunca será efetivamente escolhida se houver qualquer
% outra opção com badness menor. Também fica claro que não há muita diferença
% real entre definir \tolerance como 800 ou 9.999 (a não ser fazer TeX
% trabalhar mais desnecessariamente).
%
% O problema muda de figura se TeX não consegue encontrar uma solução. Isso
% pode acontecer em dois casos: (1) o parágrafo tem ao menos uma linha que não
% pode ser quebrada com badness < 10.000 ou (2) o parágrafo tem ao menos uma
% linha que não pode ser quebrada com badness < tolerance (mas essa badness é
% menor que 10.000).
%
% No primeiro caso, se houver várias possibilidades de linhas que não podem ser
% quebradas, TeX não vai ser capaz de compará-las e escolher a melhor: todas
% têm a badness máxima (10.000) e, portanto, a que gerar menos deméritos no
% restante do parágrafo será a escolhida. Na realidade, no entanto, essas
% linhas *não* são igualmente ruins entre si, o que pode levar TeX a fazer uma
% má escolha. Para evitar isso, TeX tenta novamente aplicando
% \emergencystretch, que "faz de conta" que o tamanho máximo ideal dos espaços
% da linha é maior que o definido na fonte. Isso reduz a badness de todas as
% linhas, o que soa parecido com aumentar \tolerance. Há três diferenças, no
% entanto: (1) essa mudança só afeta os parágrafos que falharam; (2) soluções
% que originalmente teriam badness = 10.000 (e, portanto, seriam vistas como
% equivalentes) podem ser avaliadas e comparadas entre si; e (3) como a badness
% de todas as linhas diminui, a possibilidade de outras linhas que
% originalmente tinham badness alta serem escolhidas aumenta. Esse último ponto
% significa que \emergencystretch pode fazer TeX escolher linhas mais
% espaçadas, fazendo o espaçamento do parágrafo inteiro aumentar e, portanto,
% tornando o resultado mais homogêneo mesmo com uma linha particularmente ruim.
%
% É esse último ponto que justifica o uso de \emergencystretch no segundo caso
% também: apenas aumentar a tolerância, nesse caso, poderia levar TeX a
% diagramar uma linha ruim em meio a um parágrafo bom, enquanto
% \emergencystretch pode fazer TeX aumentar o espaçamento de maneira geral no
% parágrafo, minimizando o contraste da linha problemática com as demais.
% Colocando a questão de outra maneira, aumentar \tolerance para lidar com
% esses parágrafos problemáticos pode fazê-los ter uma linha especialmente
% ruim, enquanto \emergencystretch pode dividir o erro entre várias linhas.
% Assim, definir \tolerance em torno de 800 parece razoável: no caso geral,
% não há diferença e, se um desses casos difíceis não pode ser resolvido com
% uma linha de badness até 800, \emergencystretch deve ser capaz de gerar um
% resultado igual ou melhor.
%
% Penalties & demerits: https://tex.stackexchange.com/a/51264
% Definições (fussy, sloppy etc.): https://tex.stackexchange.com/a/241355
% Mais definições (hfuzz, hbadness etc.): https://tex.stackexchange.com/a/50850
% Donald Arseneau defendendo o uso de \sloppy: https://groups.google.com/d/msg/comp.text.tex/Dhf0xxuQ66E/QTZ7aLYrdQUJ
% Artigo detalhado sobre \emergencystretch: https://www.tug.org/TUGboat/tb38-1/tb118wermuth.pdf
% Esse artigo me leva a crer que algo em torno de 1.5em é suficiente

\tolerance=800
\hyphenpenalty=100 % Default 50; se o texto é em 2 colunas, 50 é melhor
\setlength{\emergencystretch}{1.5em}

% Não gera warnings para Overfull menor que 1pt
\hfuzz=1pt
\vfuzz\hfuzz

% Não gera warnings para Underfull com badness < 1000
\hbadness=1000
\vbadness=1000

% Por padrão, o algoritmo LaTeX para textos não-justificados é (muito) ruim;
% este pacote implementa um algoritmo bem melhor
\usepackage[newcommands]{ragged2e}

% ragged2e funciona porque permite que LaTeX hifenize palavras em textos
% não-justificados quando necessário. No caso de textos centralizados,
% no entanto, isso em geral não é desejável. Assim, newcommands não é
% interessante para \centering e \begin{center}. newcommands também
% causa problemas com legendas se o float correspondente usa \centering
% (o que é muito comum). Assim, vamos voltar \centering e \begin{center}
% à definição padrão.
\let\centering\LaTeXcentering
\let\center\LaTeXcenter
\let\endcenter\endLaTeXcenter

% Com ragged2e e a opção "newcommands", textos curtos não-justificados
% podem gerar warnings sobre "underfull \hbox". Não há razão para pensar
% muito nesses warnings, então melhor desabilitá-los.
% https://tex.stackexchange.com/a/18019
\makeatletter
\gappto{\raggedright}{\hbadness=\@M}
\gappto{\RaggedRight}{\hbadness=\@M}
\gappto{\raggedleft}{\hbadness=\@M}
\gappto{\RaggedLeft}{\hbadness=\@M}
\gappto{\Centering}{\hbadness=\@M} % not \centering
\gappto{\flushleft}{\hbadness=\@M}
\gappto{\FlushLeft}{\hbadness=\@M}
\gappto{\flushright}{\hbadness=\@M}
\gappto{\FlushRight}{\hbadness=\@M}
\gappto{\Center}{\hbadness=\@M} % not \center
\makeatother

% Espaçamento entre linhas configurável (\singlespacing, \onehalfspacing etc.)
\usepackage{setspace}

% LaTeX às vezes coloca notas de rodapé logo após o final do texto da
% página ao invés de no final da página; este pacote evita isso e faz
% notas de rodapé funcionarem corretamente em títulos de seções.
% Esta package deve ser carregada depois de setspace.
\usepackage[stable,bottom]{footmisc}

% Se uma página está vazia, não imprime número de página ou cabeçalho
\usepackage{emptypage}

% hyperref deve preferencialmente ser carregada próximo ao final
% do preâmbulo mas, para o caso de alguma package forçar a sua
% carga antes de executarmos \usepackage explicitamente, vamos
% garantir que estas opções estejam ativas.
\PassOptionsToPackage{
  unicode=true,
  pdfencoding=unicode,
  plainpages=false,
  pdfpagelabels,
  bookmarksopen=true,
  breaklinks=true,
  %hyperfootnotes=false, % polui desnecessariamente com bordercolor
}{hyperref}

% Carrega nomes de cores disponíveis (podem ser usados com hyperref e listings)
\usepackage[hyperref,svgnames,x11names,table]{xcolor}

% LaTeX define os comandos "MakeUppercase" e "MakeLowercase", mas eles têm
% algumas limitações; esta package define os comandos MakeTextUppercase e
% MakeTextLowercase que resolvem isso.
\usepackage{textcase}

% Em documentos frente-e-verso, LaTeX faz o final da página terminar sempre
% no mesmo lugar (exceto no final dos capítulos). Esse comportamento pode ser
% ativado explicitamente com o comando "\flushbottom". Mas se, por alguma
% razão, o volume de texto na página é "pequeno", essa página vai ter espaços
% verticais artificialmente grandes. Uma solução para esse problema é utilizar
% "\raggedbottom" (padrão em documentos que não são frente-e-verso): com essa
% opção, as páginas podem terminar em alturas ligeiramente diferentes. Outra
% opção é corrigir manualmente cada página problemática, por exemplo com o
% comando "\enlargethispage".
%\raggedbottom
\flushbottom

% Por padrão, LaTeX coloca uma espaço aumentado após sinais de pontuação;
% Isso não é tão bom quanto alguns TeX-eiros defendem :) .
% Esta opção desabilita isso e, consequentemente, evita problemas com
% "id est" (i.e.) e "exempli gratia" (e.g.)
\frenchspacing

% Trechos de texto "puro" (tabs, quebras de linha etc. não são modificados)
\usepackage{verbatim}

% Durante o processamento, LaTeX procura por arquivos adicionais necessários
% (tanto componentes do próprio LaTeX, como packages e fontes, quanto partes
% do conteúdo em si, como imagens carregadas com \includegraphics ou arquivos
% solicitados com \input ou \include) no diretório de instalação e também
% no diretório atual (ou seja, o diretório do projeto). Assim, normalmente
% é preciso usar caminhos relativos para incluir arquivos de subdiretórios:
% "\input{diretorio/arquivo}". No entanto, há duas limitações:
%
% 1. É necessário dizer "\input{diretorio/arquivo}" mesmo quando o arquivo
%    que contém esse comando já está dentro do subdiretório.
%
% 2. Isso não deve ser usado para packages ("\usepackage{diretorio/package}"),
%    embora na prática funcione.
%
% Há três maneiras recomendadas de resolver esses problemas:
%
% 1. Acrescentando os diretórios desejados ao arquivo texmf.cnf
%
% 2. Acrescentando os diretórios desejados às variáveis de ambiente
%    TEXINPUTS e BSTINPUTS
%
% 3. Colocando os arquivos adicionais na árvore TEXMF (geralmente, no
%    diretório texmf dentro do diretório do usuário).
%
% Essas soluções, no entanto, não podem ser automatizadas por este modelo
% e são um tanto complicadas para usuários menos experientes. Veja mais a
% respeito na seção 5 de "texdoc kpathsea" e em
% https://www.overleaf.com/learn/latex/Articles/An_introduction_to_Kpathsea_and_how_TeX_engines_search_for_files .
%
% A package import pode solucionar o primeiro problema, mas exige o uso
% de outro comando no lugar de \input, então não a usamos aqui.
%\usepackage{import}
%
% Uma solução mais simples é acrescentar os diretórios desejados à macro
% \input@path, originalmente criada para resolver um problema relacionado
% à portabilidade. Seu uso não é normalmente recomendado por razões de
% desempenho, mas no nosso caso (em que adicionamos apenas um diretório
% com poucos arquivos e com máquinas modernas) isso não é um problema. Veja
% https://tex.stackexchange.com/questions/241828/define-path-for-packages-in-the-latex-file-analog-of-inputpath-or-graphicspa#comment705011_241832
\csappto{input@path}{{extras/}}

%%%%%%%%%%%%%%%%%%%%%%%%%%%%%%%%%%%%%%%%%%%%%%%%%%%%%%%%%%%%%%%%%%%%%%%%%%%%%%%%
%%%%%%%%%%%%%%%%%%%%%%%%%%%%%%%%%%% LÍNGUAS %%%%%%%%%%%%%%%%%%%%%%%%%%%%%%%%%%%%
%%%%%%%%%%%%%%%%%%%%%%%%%%%%%%%%%%%%%%%%%%%%%%%%%%%%%%%%%%%%%%%%%%%%%%%%%%%%%%%%

\makeatletter
\ExplSyntaxOn

% We need to have at least some variant of Portuguese and of English
% loaded to generate the abstract/resumo, palavras-chave/keywords etc.
% We will make sure that both languages are present in the class options
% list by adding them if needed. With this, these options become global
% and therefore are seen by all packages (among them, babel).
%
% babel traditionally uses "portuguese", "brazilian", "portuges", or
% "brazil" to support the Portuguese language, using .ldf files. babel
% is also in the process of implementing a new scheme, using .ini
% files, based on the concept of "locales" instead of "languages". This
% mechanism uses the names "portuguese-portugal", "portuguese-brazil",
% "portuguese-pt", "portuguese-br", "portuguese", "brazilian", "pt",
% "pt-PT", and "pt-BR" (i.e., neither "portuges" nor "brazil"). To avoid
% compatibility problems, let's stick with "brazilian" or "portuguese"
% by substituting portuges and brazil if necessary.

\NewDocumentCommand\@IMEportugueseAndEnglish{m}{

  % Make sure any instances of "portuges" and "brazil" are replaced
  % by "portuguese" e "brazilian"; other options are unchanged.
  \seq_gclear_new:N \l_tmpa_seq
  \seq_gclear_new:N \l_tmpb_seq
  \seq_gset_from_clist:Nc \l_tmpa_seq {#1}

  \seq_map_inline:Nn \l_tmpa_seq{
    \def\@tempa{##1}
    \ifstrequal{portuges}{##1}
      {
        \GenericInfo{sbc2019}{}{Substituting~language~portuges~->~portuguese}
        \def\@tempa{portuguese}
      }
      {}
    \ifstrequal{brazil}{##1}
      {
        \GenericInfo{}{Substituting~language~brazil~->~brazilian}
        \def\@tempa{brazilian}
      }
      {}
    \seq_gput_right:NV \l_tmpb_seq {\@tempa}
  }

  % Remove the leftmost duplicates (default is to remove the rightmost ones).
  % Necessary in case the user did "portuges,portuguese", "brazil,brazilian"
  % or some variation: When we substitute the language, we end up with the
  % exact same language twice, which may mess up the main language selection.
  \seq_greverse:N \l_tmpb_seq
  \seq_gremove_duplicates:N \l_tmpb_seq
  \seq_greverse:N \l_tmpb_seq

  % If the user failed to select some variation of English and Portuguese,
  % we add them here. We also remember which ones of portuguese/brazilian,
  % english/american/british etc. were selected.
  \exp_args:Nnx \regex_extract_all:nnNTF
    {\b(portuguese|brazilian)\b}
    {\seq_use:Nn \l_tmpb_seq {,}}
    \l_tmpa_tl
    {
      \tl_reverse:N \l_tmpa_tl
      \xdef\@IMEpt{\tl_head:N \l_tmpa_tl}
    }
    {
      \seq_gput_left:Nn \l_tmpb_seq {brazilian}
      \gdef\@IMEpt{brazilian}
    }

  \exp_args:Nnx \regex_extract_all:nnNTF
    {\b(english|american|USenglish|canadian|british|UKenglish|australian|newzealand)\b}
    {\seq_use:Nn \l_tmpb_seq {,}}
    \l_tmpa_tl
    {
      \tl_reverse:N \l_tmpa_tl
      \xdef\@IMEen{\tl_head:N \l_tmpa_tl}
    }
    {
      \seq_gput_left:Nn \l_tmpb_seq {english}
      \gdef\@IMEen{english}
    }

  \exp_args:Nc \xdef {#1} {\seq_use:Nn \l_tmpb_seq {,}}
}


% https://tex.stackexchange.com/a/43541
% This message is part of a larger thread that discusses some
% limitations of this method, but it is enough for us here.
\def\@getcl@ss#1.cls#2\relax{\def\@currentclass{#1}}
\def\@getclass{\expandafter\@getcl@ss\@filelist\relax}
\@getclass

% The three class option lists we need to update: \@unusedoptionlist,
% \@classoptionslist and one of \opt@book.cls, \opt@article.cls etc.
% according to the current class. Note that beamer.cls (and maybe
% others) does not use \@unusedoptionlist; with it, we incorrectly
% add "english,brazilian" to \@unusedoptionlist, but that does not
% cause problems.
\@IMEportugueseAndEnglish{@unusedoptionlist}
\@IMEportugueseAndEnglish{@classoptionslist}
\@IMEportugueseAndEnglish{opt@\@currentclass .cls}

\ExplSyntaxOff
\makeatother

% Internacionalização dos nomes das seções ("chapter" X "capítulo" etc.),
% hifenização e outras convenções tipográficas. babel deve ser um dos
% primeiros pacotes carregados. É possível passar a língua do documento
% como parâmetro aqui, mas já fizemos isso ao carregar a classe, no início
% do documento.
\usepackage{babel}
\usepackage{iflang}

% É possível personalizar as palavras-chave que babel utiliza, por exemplo:
%\addto\extrasbrazilian{\renewcommand{\chaptername}{Chap.}}
% Com BibTeX, isso vale também para a bibliografia; com BibLaTeX, é melhor
% usar o comando "DefineBibliographyStrings".

% Para línguas baseadas no alfabeto latino, como o inglês e o português,
% o pacote babel funciona muito bem, mas com outros alfabetos ele às vezes
% falha. Por conta disso, o pacote polyglossia foi criado para substituí-lo.
% polyglossia só funciona com LuaTeX e XeTeX; como babel também funciona com
% esses sistemas, provavelmente não há razão para usar polyglossia, mas é
% possível que no futuro esse pacote se torne o padrão.
%\usepackage{polyglossia}
%\setdefaultlanguage{brazilian}
%\setotherlanguage{english}

% Alguns pacotes (espeficicamente, tikz) usam, além de babel, este pacote
% como auxiliar para a tradução de palavras-chave, como os meses do ano.
\usepackage{translator}

%%%%%%%%%%%%%%%%%%%%%%%%%%%%%%%%%%%%%%%%%%%%%%%%%%%%%%%%%%%%%%%%%%%%%%%%%%%%%%%%
%%%%%%%%%%%%%%%%%%%%%%%%%%%%%%%%%%% FONTE %%%%%%%%%%%%%%%%%%%%%%%%%%%%%%%%%%%%%%
%%%%%%%%%%%%%%%%%%%%%%%%%%%%%%%%%%%%%%%%%%%%%%%%%%%%%%%%%%%%%%%%%%%%%%%%%%%%%%%%

% LaTeX normalmente usa quatro tipos de fonte:
%
% * uma fonte serifada, para o corpo do texto;
% * uma fonte com design similar à anterior, para modo matemático;
% * uma fonte sem serifa, para títulos ou "entidades". Por exemplo, "a classe
%   \textsf{TimeManager} é responsável..." ou "chamamos \textsf{primos} os
%   números que...". Observe que em quase todos os casos desse tipo é mais
%   adequado usar negrito ou itálico;
% * uma fonte "teletype", para trechos de programas.
%
% A escolha de uma família de fontes para o documento normalmente é feita
% carregando uma package específica que, em geral, seleciona as quatro fontes
% de uma vez.
%
% LaTeX usa por default a família de fontes "Computer Modern". Essas fontes
% precisaram ser re-criadas diversas vezes em formatos diferentes, então há
% diversas variantes dela. Com o fontenc OT1 (default "ruim" do LaTeX), a
% versão usada é a BlueSky Computer Modern, que é de boa qualidade, mas com
% os problemas do OT1. Com fontenc T1 (padrão deste modelo e recomendado), o
% LaTeX usa o conjunto "cm-super". Com fontspec (ou seja, com LuaLaTeX e
% XeLaTeX), LaTeX utiliza a versão "Latin Modern". Ao longo do tempo, versões
% diferentes dessas fontes foram recomendadas como "a melhor"; atualmente, a
% melhor opção para usar a família Computer Modern é a versão "Latin Modern".
%
% Você normalmente não precisa lidar com isso, mas pode ser útil saber: O
% mecanismo tradicionalmente usado por LaTeX para gerir fontes é o NFSS
% (veja "texdoc fntguide"). Ele funciona com todas as versões de LaTeX,
% mas só com fontes que foram adaptadas para funcionar com LaTeX. LuaLaTeX
% e XeLaTeX podem usar NFSS mas também são capazes de utilizar um outro
% mecanismo (através da package fontspec), que permite utilizar quaisquer
% fontes instaladas no computador.

\ifPDFTeX
    % Usando pdfLaTeX

    % Ativa Latin Modern como a fonte padrão.
    \usepackage{lmodern}

    % Alguns truques para melhorar a aparência das fontes Latin Modern;
    % eles não funcionam com LuaLaTeX e XeLaTeX.

    % Latin Modern não tem fontes bold + Small Caps, mas cm-super sim;
    % assim, vamos ativar o suporte às fontes cm-super (sem ativá-las
    % como a fonte padrão do documento) e configurar substituições
    % automáticas para que a fonte Latin Modern seja substituída por
    % cm-super quando o texto for bold + Small Caps.
    \usepackage{fix-cm}

    % Com Latin Modern, é preciso incluir substituições para o encoding TS1
    % também por conta dos números oldstyle, porque para inclui-los nas fontes
    % computer modern foi feita uma hack: os dígitos são declarados como sendo
    % os números itálicos da fonte matemática e, portanto, estão no encoding TS1.
    %
    % Primeiro forçamos o LaTeX a carregar a fonte Latin Modern (ou seja, ler
    % o arquivo que inclui "DeclareFontFamily") e, a seguir, definimos a
    % substituição
    \fontencoding{TS1}\fontfamily{lmr}\selectfont
    \DeclareFontShape{TS1}{lmr}{b}{sc}{<->ssub * cmr/bx/n}{}
    \DeclareFontShape{TS1}{lmr}{bx}{sc}{<->ssub * cmr/bx/n}{}

    \fontencoding{T1}\fontfamily{lmr}\selectfont
    \DeclareFontShape{T1}{lmr}{b}{sc}{<->ssub * cmr/bx/sc}{}
    \DeclareFontShape{T1}{lmr}{bx}{sc}{<->ssub * cmr/bx/sc}{}

    % Latin Modern não tem "small caps + itálico", mas tem "small caps + slanted";
    % vamos definir mais uma substituição aqui.
    \fontencoding{T1}\fontfamily{lmr}\selectfont % já feito acima, mas tudo bem
    \DeclareFontShape{T1}{lmr}{m}{scit}{<->ssub * lmr/m/scsl}{}
    \DeclareFontShape{T1}{lmr}{bx}{scit}{<->ssub * lmr/bx/scsl}{}

    % Se fizermos mudanças manuais na fonte Latin Modern, estes comandos podem
    % vir a ser úteis
    %\newcommand\lmodern{%
    %  \renewcommand{\oldstylenums}[1]{{\fontencoding{TS1}\selectfont ##1}}%
    %  \fontfamily{lmr}\selectfont%
    %}
    %
    %\DeclareRobustCommand\textlmodern[1]{%
    %  {\lmodern #1}%
    %}
\else
    % Com LuaLaTex e XeLaTeX, Latin Modern é a fonte padrão. Existem
    % diversas packages e "truques" para melhorar alguns aspectos de
    % Latin Modern, mas eles foram feitos para pdflatex (veja o "else"
    % logo abaixo). Assim, se você pretende usar Latin Modern como a
    % fonte padrão do documento, é melhor usar pdfLaTeX. Deve ser
    % possível implementar essas melhorias com fontspec também, mas
    % este modelo não faz isso, apenas ativamos Small Caps aqui.

    \ifLuaTeX
      % Com LuaTeX, basta indicar o nome de cada fonte; para descobrir
      % o nome "certo", use o comando "otfinfo -i" e veja os itens
      % "preferred family" e "full name"
      \setmainfont{Latin Modern Roman}[
        SmallCapsFont = {LMRomanCaps10-Regular},
        ItalicFeatures = {
          SmallCapsFont = {LMRomanCaps10-Oblique},
        },
        SlantedFont = {LMRomanSlant10-Regular},
        SlantedFeatures = {
          SmallCapsFont = {LMRomanCaps10-Oblique},
          BoldFont = {LMRomanSlant10-Bold}
        },
      ]
    \fi

    \ifXeTeX
      % Com XeTeX, é preciso informar o nome do arquivo de cada fonte.
      \setmainfont{lmroman10-regular.otf}[
        SmallCapsFont = {lmromancaps10-regular.otf},
        ItalicFeatures = {
          SmallCapsFont = {lmromancaps10-oblique.otf},
        },
        SlantedFont = {lmromanslant10-regular.otf},
        SlantedFeatures = {
          SmallCapsFont = {lmromancaps10-oblique.otf},
          BoldFont = {lmromanslant10-bold.otf}
        },
      ]
    \fi
\fi

% Algumas packages mais novas que tratam de fontes funcionam corretamente
% tanto com fontspec (LuaLaTeX/XeLaTeX) quanto com NFSS (qualquer versão
% de LaTeX, mas menos poderoso que fontspec). No entanto, muitas funcionam
% apenas com NFSS. Nesse caso, em LuaLaTeX/XeLaTeX é melhor usar os
% comandos de fontspec, como exemplificado mais abaixo.

% É possível mudar apenas uma das fontes. Em particular, a fonte
% teletype da família Computer Modern foi criada para simular
% as impressoras dos anos 1970/1980. Sendo assim, ela é uma fonte (1)
% com serifas e (2) de espaçamento fixo. Hoje em dia, é mais comum usar
% fontes sem serifa para representar código-fonte. Além disso, ao imprimir,
% é comum adotar fontes que não são de espaçamento fixo para fazer caber
% mais caracteres em uma linha de texto. Algumas opções de fontes para
% esse fim:
%\usepackage{newtxtt} % Não funciona com fontspec (lualatex / xelatex)
%\usepackage{DejaVuSansMono}
% inconsolata é uma boa fonte, mas não tem variante itálico
%\ifPDFTeX
%  \usepackage[narrow]{inconsolata}
%\else
%  \setmonofont{inconsolatan}
%\fi
\usepackage[scale=.85]{sourcecodepro}

% Ao invés da família Computer Modern, é possível usar outras como padrão.
% Uma ótima opção é a libertine, similar (mas não igual) à Times mas com
% suporte a Small Caps e outras qualidades. A fonte teletype da família
% é serifada, então é melhor definir outra; a opção "mono=false" faz
% o pacote não carregar sua própria fonte, mantendo a escolha anterior.
% Versões mais novas de LaTeX oferecem um fork desta fonte, libertinus.
% As packages libertine/libertinus funcionam corretamente com pdfLaTeX,
% LuaLaTeX e XeLaTeX.
% TODO: remover suporte a Libertine no final de 2022
\makeatletter
\IfFileExists{libertinus.sty}
    {
      \usepackage[mono=false]{libertinus}
      % Com LuaLaTeX/XeLaTeX, Libertinus configura também
      % a fonte matemática; aqui só precisamos corrigir \mathit
      \ifLuaTeX
        \setmathfontface\mathit{Libertinus Serif Italic}
      \fi
      \ifXeTeX
        % O nome de arquivo da fonte mudou na versão 2019-04-04
        \@ifpackagelater{libertinus-otf}{2019/04/03}
            {\setmathfontface\mathit{LibertinusSerif-Italic.otf}}
            {\setmathfontface\mathit{libertinusserif-italic.otf}}
      \fi
    }
    {
      % Libertinus não está disponível; vamos usar libertine
      \usepackage[mono=false]{libertine}

      % Com Libertine, é preciso modificar também a fonte
      % matemática, além de \mathit
      \ifLuaTeX
        \setmathfont{Libertinus Math}
        \setmathfontface\mathit{Linux Libertine O Italic}
      \fi

      \ifXeTeX
        \setmathfont{libertinusmath-regular.otf}
        \setmathfontface\mathit{LinLibertine_RI.otf}
      \fi
    }
\makeatother

\ifPDFTeX
  % A família libertine por padrão não define uma fonte matemática
  % específica para pdfLaTeX; uma opção que funciona bem com ela:
  %\usepackage[libertine]{newtxmath}
  % Outra, provavelmente melhor:
  \usepackage{libertinust1math}
\fi

% Ativa apenas a fonte biolinum, que é a fonte sem serifa da família.
%\IfFileExists{libertinus.sty}
%  \usepackage[sans]{libertinus}
%\else
%  \usepackage{biolinum}
%\fi

% Também é possível usar a Times como padrão; nesse caso, a fonte
% sem serifa usualmente é a Helvetica. Mas provavelmente libertine
% é uma opção melhor.
%\ifPDFTeX
%  \usepackage[helvratio=0.95,largesc]{newtxtext}
%  \usepackage{newtxtt} % Fonte teletype
%  \usepackage{newtxmath}
%\else
%  % Clone da fonte Times como fonte principal
%  \setmainfont{TeX Gyre Termes}
%  \setmathfont[Scale=MatchLowercase]{TeX Gyre Termes Math}
%  % TeX Gyre Termes Math tem um bug e não define o caracter
%  % \setminus; Vamos contornar esse problema usando apenas
%  % esse caracter da fonte STIX Two Math
%  \setmathfont[range=\setminus]{STIX Two Math}
%  % Clone da fonte Helvetica como fonte sem serifa
%  \setsansfont{TeX Gyre Heros}
%  % Clone da Courier como fonte teletype, mas provavelmente
%  % é melhor utilizar sourcecodepro
%  %\setmonofont{TeX Gyre Cursor}
%\fi

% Cochineal é outra opção de qualidade; ela define apenas a fonte
% com serifa.
%
% Com NFSS (recomendado no caso de cochineal):
%\usepackage{cochineal}
%\usepackage[cochineal,vvarbb]{newtxmath}
%\usepackage[cal=boondoxo]{mathalfa}
%
% Com fontspec (até a linha "setmathfontface..."):
%
%\setmainfont{Cochineal}[
%  Extension=.otf,
%  UprightFont=*-Roman,
%  ItalicFont=*-Italic,
%  BoldFont=*-Bold,
%  BoldItalicFont=*-BoldItalic,
%  %Numbers={Proportional,OldStyle},
%]
%
%\DeclareRobustCommand{\lfstyle}{\addfontfeatures{Numbers=Lining}}
%\DeclareTextFontCommand{\textlf}{\lfstyle}
%\DeclareRobustCommand{\tlfstyle}{\addfontfeatures{Numbers={Tabular,Lining}}}
%\DeclareTextFontCommand{\texttlf}{\tlfstyle}
%
%% Cochineal não tem uma fonte matemática; com fontspec, provavelmente
%% o melhor a fazer é usar libertinus.
%\setmathfont{Libertinus Math}
%\setmathfontface\mathit{Cochineal-Italic.otf}

% gentium inclui apenas uma fonte serifada, similar a Garamond, que busca
% cobrir todos os caracteres unicode
%\usepackage{gentium}

% LaTeX normalmente funciona com fontes que foram adaptadas para ele, ou
% seja, ele não usa as fontes padrão instaladas no sistema: para usar
% uma fonte é preciso ativar o pacote correspondente, como visto acima.
% É possível escapar dessa limitação e acessar as fontes padrão do sistema
% com XeTeX ou LuaTeX. Com eles, além dos pacotes de fontes "tradicionais",
% pode-se usar o pacote fontspec para usar fontes do sistema.
%\usepackage{fontspec}
%\setmainfont{DejaVu Serif}
%\setmainfont{Charis SIL}
%\setsansfont{DejaVu Sans}
%\setsansfont{Libertinus Sans}[Scale=1.1]
%\setmonofont{DejaVu Sans Mono}

% fontspec oferece vários recursos interessantes para manipular fontes.
% Por exemplo, Garamond é uma fonte clássica; a versão EBGaramond é muito
% boa, mas não possui versões bold e bold-italic; aqui, usamos
% CormorantGaramond ou Gentium para simular a versão bold.
%\setmainfont{EBGaramond12}[
%  Numbers        = {Lining,} ,
%  Scale          = MatchLowercase ,
%  UprightFont    = *-Regular ,
%  ItalicFont     = *-Italic ,
%  BoldFont       = gentiumbasic-bold ,
%  BoldItalicFont = gentiumbasic-bolditalic ,
%%  BoldFont       = CormorantGaramond Bold ,
%%  BoldItalicFont = CormorantGaramond Bold Italic ,
%]
%
%\newfontfamily\garamond{EBGaramond12}[
%  Numbers        = {Lining,} ,
%  Scale          = MatchLowercase ,
%  UprightFont    = *-Regular ,
%  ItalicFont     = *-Italic ,
%  BoldFont       = gentiumbasic-bold ,
%  BoldItalicFont = gentiumbasic-bolditalic ,
%%  BoldFont       = CormorantGaramond Bold ,
%%  BoldItalicFont = CormorantGaramond Bold Italic ,
%]

% Crimson tem Small Caps, mas o recurso é considerado "em construção".
% Vamos utilizar Gentium para Small Caps
%\setmainfont{Crimson}[
%  Numbers           = {Lining,} ,
%  Scale             = MatchLowercase ,
%  UprightFont       = *-Roman ,
%  ItalicFont        = *-Italic ,
%  BoldFont          = *-Bold ,
%  BoldItalicFont    = *-Bold Italic ,
%  SmallCapsFont     = Gentium Plus ,
%  SmallCapsFeatures = {Letters=SmallCaps} ,
%]
%
%\newfontfamily\crimson{Crimson}[
%  Numbers           = {Lining,} ,
%  Scale             = MatchLowercase ,
%  UprightFont       = *-Roman ,
%  ItalicFont        = *-Italic ,
%  BoldFont          = *-Bold ,
%  BoldItalicFont    = *-Bold Italic ,
%  SmallCapsFont     = Gentium Plus ,
%  SmallCapsFeatures = {Letters=SmallCaps} ,
%]

% Com o pacote fontspec, também é possível usar o comando "\fontspec" para
% selecionar uma fonte temporariamente, sem alterar as fontes-padrão do
% documento.

%%%%%%%%%%%%%%%%%%%%%%%%%%%%%%%%%%%%%%%%%%%%%%%%%%%%%%%%%%%%%%%%%%%%%%%%%%%%%%%%
%%%%%%%%%%%%%%%%%%%%%%%%%%%%% FIGURAS / FLOATS %%%%%%%%%%%%%%%%%%%%%%%%%%%%%%%%%
%%%%%%%%%%%%%%%%%%%%%%%%%%%%%%%%%%%%%%%%%%%%%%%%%%%%%%%%%%%%%%%%%%%%%%%%%%%%%%%%

% LaTeX escolhe automaticamente o "melhor" lugar para colocar cada float.
% Por padrão, ele tenta colocá-los no topo da página e depois no pé da
% página; se não tiver sucesso, vai para a página seguinte e recomeça.
% Se esse algoritmo não tiver sucesso "logo", LaTeX cria uma página só
% com floats. É possível modificar esse comportamento com as opções de
% posicionamento: "tp", por exemplo, instrui LaTeX a considerar apenas
% o topo da página ou uma página só de floats (ignorando o pé da página),
% e "htbp" o instrui para tentar "aqui" como a primeira opção. A ordem
% dessas opções não é relevante: dentre as opções disponíveis, LaTeX
% sempre tenta "aqui, topo, pé, página". Os pacotes "float" e "floatrow"
% acrescentam a opção "H", que significa "aqui, incondicionalmente".
%
% A escolha do "melhor" lugar leva em conta os parâmetros abaixo, mas é
% possível ignorá-los com a opção de posicionamento "!". Dado que os
% valores default não são muito bons para floats "grandes" ou documentos
% com muitos floats, é muito comum usar "!" ou "H". No entanto, modificando
% estes parâmetros o algoritmo automático tende a funcionar melhor. Ainda
% assim, vale ler a discussão a respeito na seção "Limitações do LaTeX"
% deste modelo.

% Fração da página que pode ser ocupada por floats no topo. Default: 0.7
\renewcommand{\topfraction}{.8}
% Idem para documentos em colunas e floats que tomam as 2 colunas. Default: 0.7
%\renewcommand{\dbltopfraction}{.7}
% Fração da página que pode ser ocupada por floats no pé. Default: 0.3
%\renewcommand{\bottomfraction}{.3}
% Fração mínima da página que deve conter texto. Default: 0.2
%\renewcommand{\textfraction}{.2}
% Numa página só de floats, fração mínima que deve ser ocupada. Default: 0.5
% floatpagefraction *deve* ser menor que topfraction.
\renewcommand{\floatpagefraction}{.66}
% Idem para documentos em colunas e floats que tomam as 2 colunas. Default: 0.5
\renewcommand{\dblfloatpagefraction}{.66}
% Máximo de floats no topo da página. Default: 2
\setcounter{topnumber}{3}
% Idem para documentos em colunas e floats que tomam as 2 colunas. Default: 2
%\setcounter{dbltopnumber}{2}
% Máximo de floats no pé da página. Default: 1
\setcounter{bottomnumber}{2}
% Máximo de floats por página. Default: 3
\setcounter{totalnumber}{5}

% A package float é amplamente utilizada; ela permite definir novos tipos
% de float e também acrescenta a possibilidade de definir "H" como opção de
% posicionamento do float, que significa "aqui, incondicionalmente". No
% entanto, ela tem algumas fragilidades e não é atualizada desde 2001.
% floatrow é uma versão aprimorada e com mais recursos da package "float",
% mas também não é atualizada desde 2009. Aqui utilizamos alguns recursos
% disponibilizados por ambas e é possível escolher qual delas utilizar.
% Um dos principais recursos dessas packages é permitir a criação de novos
% tipos de float; veja o arquivo source-code.tex para um exemplo.
%\usepackage{float}
\usepackage{floatrow}

% Por padrão, LaTeX prefere colocar floats no topo da página que
% onde eles foram definidos; vamos mudar isso. Este comando depende
% do pacote "floatrow", carregado logo acima.
\floatplacement{table}{htbp}
\floatplacement{figure}{htbp}

% Em alguns casos, um float pode aparecer antes do local do texto em que
% foi definido (ou seja, no topo da página ao invés do meio da página).
% Esta package garante que floats (tabelas e figuras) só apareçam após
% o local no texto em que foram definidos; veja os detalhes em
% https://tex.stackexchange.com/a/297580 . Note que, se o float tem a
% opção "h", normalmente LaTeX *não* coloca o float no topo da página
% atual: se o float não pode ser colocado "here", ele é delegado para
% a página seguinte.
\usepackage{flafter}

% Às vezes um float pode ser adiado por muitas páginas; é possível forçar
% LaTeX a imprimir todos os floats pendentes com o comando \clearpage mas,
% para isso, o usuário deve identificar os casos problemáticos e inserir
% \clearpage manualmente. Esta package acrescenta o comando \FloatBarrier,
% que executa \clearpage apenas se necessário no local em que é chamado.
% "above" e "below" desabilitam a barreira quando os floats estão na mesma
% página. A desvantagem de placeins é que, para funcionar, ela gera quebras
% de página que muitas vezes são inesperadas.
\usepackage[above,below]{placeins}

% Em documentos com duas colunas, floats normalmente são colocados como
% parte de uma das colunas. No entanto, é possível usar "\begin{figure*}"
% ou "\begin{table*}" para criar floats que ocupam as duas colunas. Floats
% "duplos" desse tipo têm algumas limitações:
%
% 1. Mesmo que haja espaço disponível na página atual, eles são sempre
%    inseridos na página seguinte ao lugar em que foram definidos (então
%    é comum defini-los antes do lugar "certo" para compensar isso)
%
% 2. Eles só podem aparecer no topo da página ou em uma página de floats,
%    ou seja, nunca "here" nem no pé da página.
%
% 3. Em alguns casos, eles podem aparecer fora da ordem em relação aos
%    demais floats do mesmo tipo (o que não acontece com floats "normais")
%
% Esta package:
%
% 1. Soluciona parcialmente o primeiro problema: floats "duplos" podem
%    aparecer na página em que são definidos se sua definição está contida
%    no texto da coluna da esquerda;
%
% 2. Soluciona o segundo problema: floats "duplos" podem aparecer tanto no
%    topo quanto no pé da página. Observe que eles *não* podem aparecer
%    "here" porque isso não faz sentido: a figura interromperia o fluxo
%    do texto da "outra" coluna (ainda assim, as packages midfloat e cuted
%    permitem fazer isso).
%
% 3. Soluciona o terceiro problema.
\usepackage{stfloats}

% Às vezes é interessante utilizar uma imagem mais larga que o texto.
% Por padrão, \centering *não* vai centralizar a imagem corretamente
% nesse caso. Com esta package, podemos acrescentar a opção "center"
% ao comando \includegraphics para resolver esse problema
% (ou seja, \includegraphics[width=1.2\textwidth,center]{imagem}.
% A package tem muitos outros recursos também
\usepackage[export]{adjustbox}

% Define o ambiente "\begin{landscape} -- \end{landscape}"; o texto entre
% esses comandos é impresso em modo paisagem, podendo se estender por várias
% páginas. A rotação não inclui os cabeçalhos e rodapés das páginas.
% O principal uso desta package é em conjunto com a package longtable: se
% você precisa mostrar uma tabela muito larga (que precisa ser impressa em
% modo paisagem) e longa (que se estende por várias páginas), use
% "\begin{landscape}" e "\begin{longtable}" em conjunto. Note que o modo
% landscape entra em ação imediatamente, ou seja, "\begin{landscape}" gera
% uma quebra de página no local em que é chamado. Na maioria dos casos, o
% que se quer não é isso, mas sim um "float paisagem"; isso é o que a
% package rotating oferece (veja abaixo).
\usepackage{pdflscape}

% Define dois novos tipos de float: sidewaystable e sidewaysfigure, que
% imprimem a figura ou tabela sozinha em uma página em modo paisagem. Além
% disso, permite girar elementos na página de diversas outras maneiras.
\usepackage[figuresright,clockwise]{rotating}

% Captions com fonte menor, indentação normal, corpo do texto
% negrito e nome do caption itálico
\usepackage[
  font=small,
  format=plain,
  labelfont=bf,up,
  textfont=it,up]{caption}

% Em geral, a package caption é capaz de "adivinhar" se o caption
% está acima ou abaixo da figura/tabela, mas isso não funciona
% corretamente com longtable. Aqui, forçamos a package a considerar
% que os captions ficam abaixo das tabelas.
\captionsetup*[longtable]{position=bottom}

% Sub-figuras (e seus captions) - observe que existe uma package chamada
% "subfigure", mas ela é obsoleta; use esta no seu lugar.
\usepackage{subcaption}

% Permite criar imagens com texto ao redor
%\usepackage{wrapfig}
% Esta é similar, mas me parece uma opção melhor:
%\usepackage{cutwin}

% Permite incorporar um arquivo PDF como uma página adicional. Útil se
% for necessário importar uma imagem ou tabela muito grande ou ainda
% para definir uma capa personalizada.
\usepackage{pdfpages}

% Permite importar figuras. LaTeX "tradicional" só é capaz de trabalhar com
% figuras EPS; hoje em dia não há nenhuma boa razão para usar essa versão.
% Já pdfTeX, XeTeX e LuaTeX podem usar figuras nos formatos PDF, JPG e PNG.
% Em algumas instalações, essas versões conseguem converter automaticamente
% arquivos EPS para PDF, mas não isso é garantido, então é melhor evitar o
% formato EPS.
\usepackage{graphicx}

% Caixas de texto coloridas
%\usepackage{tcolorbox}


%%%%%%%%%%%%%%%%%%%%%%%%%%%%%%%%%%%%%%%%%%%%%%%%%%%%%%%%%%%%%%%%%%%%%%%%%%%%%%%%
%%%%%%%%%%%%%%%%%%%%%%%%%%%%%%%%%% TABELAS %%%%%%%%%%%%%%%%%%%%%%%%%%%%%%%%%%%%%
%%%%%%%%%%%%%%%%%%%%%%%%%%%%%%%%%%%%%%%%%%%%%%%%%%%%%%%%%%%%%%%%%%%%%%%%%%%%%%%%

% Tabelas simples são fáceis de fazer em LaTeX; tabelas com alguma sofisticação
% são trabalhosas, pois é difícil controlar alinhamento, largura das colunas,
% distância entre células etc. Ou seja, é muito comum que a tabela final fique
% "torta". Por isso, em muitos casos, vale mais a pena gerar a tabela em uma
% planilha, como LibreOffice calc ou excel, transformar em PDF e importar como
% figura, especialmente se você quer controlar largura/altura das células
% manualmente etc. No entanto, se você quiser fazer as tabelas em LaTeX para
% garantir a consistência com o tipo e o tamanho das fontes, é possível e o
% resultado é muito bom. Aqui há alguns pacotes que incrementam os recursos de
% tabelas do LaTeX e alguns comandos pré-prontos que podem facilitar um pouco
% seu uso.

% LaTeX por padrão não permite notas de rodapé dentro de tabelas. De maneira
% geral, notas de rodapé em tabelas são consideradas "ruins" em termos de
% tipografia, mas às vezes são necessárias. Se esse é o caso, o recomendado
% é que as notas de rodapé apareçam no "rodapé" da tabela, com numeração
% própria, e não no rodapé da página. Você pode fazer isso com esta package:
\usepackage{threeparttable}
% Formatação personalizada das notas de threeparttable:
\appto{\TPTnoteSettings}{\footnotesize\itshape}
\def\TPTtagStyle{\textit}
% Outra opção é a package ctable, que ainda oferece vários outros
% recursos mas usa uma sintaxe diferente.
%\usepackage{ctable}

% Se você realmente quer notas de rodapé em tabelas que aparecem como as
% demais notas de rodapé (no final da página e mantendo a sequência numérica),
% você pode usar a package abaixo. No entanto, ela não funciona com floats
% duplos (floats que ocupam toda a largura da página em um documento de duas
% colunas) e, em alguns casos, a nota pode desaparecer ou aparecer em uma
% página diferente da tabela (mova o lugar do texto em que ela é definida
% para resolver esse problema).
\usepackage{tablefootnote}

% Por padrão, cada coluna de uma tabela tem a largura do maior texto contido
% nela, ou seja, se uma coluna contém uma célula muito larga, LaTeX não
% força nenhuma quebra de linha e a tabela "estoura" a largura do papel. A
% solução simples, nesses casos, é inserir uma ou mais quebras de linha
% manualmente, o que além de deselegante não é totalmente trivial (é preciso
% usar \makecell).
% Esta package estende o ambiente tabular para permitir definir um tamanho
% fixo para uma ou mais colunas; nesse caso, LaTeX quebra as linhas se uma
% célula é larga demais para a largura definida. Encontrar valores "bons"
% para as larguras das colunas, no entanto, também é um trabalho manual
% um tanto penoso. As packages tabularx e tabulary permitem configurar
% algumas colunas como "largura automática", evitando a necessidade da
% definição manual. Finalmente, ltxtable permite utilizar tabularx e
% longtable juntas. Neste modelo, não usamos tabularx/tabulary, mas você
% pode carregá-las se quiser.
\usepackage{array}

% Se você quer ter um pouco mais de controle sobre o tamanho de cada coluna da
% tabela, utilize estes tipos de coluna (criados com base nos recursos do pacote
% array). É só usar algo como M{número}, onde "número" (por exemplo, 0.4) é a
% fração de \textwidth que aquela coluna deve ocupar. "M" significa que o
% conteúdo da célula é centralizado; "L", alinhado à esquerda; "J", justificado;
% "R", alinhado à direita. Obviamente, a soma de todas as frações não pode ser
% maior que 1, senão a tabela vai ultrapassar a linha da margem.
\newcolumntype{M}[1]{>{\centering}m{#1\textwidth}}
\newcolumntype{L}[1]{>{\RaggedRight}m{#1\textwidth}}
\newcolumntype{R}[1]{>{\RaggedLeft}m{#1\textwidth}}
\newcolumntype{J}[1]{m{#1\textwidth}}

% Permite alinhar os elementos de uma coluna pelo ponto decimal; dê
% preferência à package siunitx (carregada em utils.tex), que também
% oferece esse recurso e muitos outros.
\usepackage{dcolumn}

% Define tabelas do tipo "longtable", similares a "tabular" mas que podem ser
% divididas em várias páginas. "longtable" também funciona corretamente com
% notas de rodapé. Note que, como uma longtable pode se estender por várias
% páginas, não faz sentido colocá-las em um float "table". Por conta disso,
% longtable define o comando "\caption" internamente.
\usepackage{longtable}

% Permite agregar linhas de tabelas, fazendo colunas "compridas"
\usepackage{multirow}

% Cria comando adicional para possibilitar a inserção de quebras de linha
% em uma célula de tabela, entre outros
\usepackage{makecell}

% Às vezes a tabela é muito larga e não cabe na página. Se os cabeçalhos da
% tabela é que são demasiadamente largos, uma solução é inclinar o texto das
% células do cabeçalho. Para fazer isso, use o comando "\rothead".
\renewcommand{\rothead}[2][60]{\makebox[11mm][l]{\rotatebox{#1}{\makecell[c]{#2}}}}

% Se quiser criar uma linha mais grossa no meio de uma tabela, use
% o comando "\thickhline".
\newlength\savedwidth
\newcommand\thickhline{
  \noalign{
    \global\savedwidth\arrayrulewidth
    \global\arrayrulewidth 1.5pt
  }
  \hline
  \noalign{\global\arrayrulewidth\savedwidth}
}

% Modifica (melhora) o layout default das tabelas e acrescenta os comandos
% \toprule, \bottomrule, \midrule e \cmidrule
\usepackage{booktabs}

% Permite colorir linhas, colunas ou células
\usepackage{colortbl}

% Ao invés de digitar os dados de uma tabela dentro do seu documento,
% você pode fazer LaTeX ler os dados de um arquivo CSV e criar uma
% tabela automaticamente com uma destas duas packages:
%\usepackage{csvsimple}     % mais simples
%\usepackage{pgfplotstable} % mais complexa

% Você também pode se interessar pelo ambiente "tabbing", que permite
% criar tabelas simples com algumas vantagens em relação a "tabular",
% ou por esta package, que permite criar tabulações.
%\usepackage{tabto}

%%%%%%%%%%%%%%%%%%%%%%%%%%%%%%%%%%%%%%%%%%%%%%%%%%%%%%%%%%%%%%%%%%%%%%%%%%%%%%%%
%%%%%%%%%%%%%%% CAPA E PÁGINAS PRELIMINARES (TESE/DISSERTAÇÃO)  %%%%%%%%%%%%%%%%
%%%%%%%%%%%%%%%%%%%%%%%%%%%%%%%%%%%%%%%%%%%%%%%%%%%%%%%%%%%%%%%%%%%%%%%%%%%%%%%%

% Formatação de datas de acordo com a língua
\usepackage[useregional]{datetime2}
\DTMusemodule{brazilian}{portuges}
\DTMnewdatestyle{month-year}{%
  \renewcommand*{\DTMdisplaydate}[4]{##2,\space##1}%
  \renewcommand*{\DTMDisplaydate}{\DTMdisplaydate}%
}

\makeatletter

%%%%%%%%%%%%%%%%%%%%%%%%%%%%%%%%%%%%%%%%%%%%%%%%%%%%%%%%%%%%%%%%%%%%%%%%%%%%%%%%
%%%%%%%%%%%%%%%%%%%%% TEXTOS PADRÃO EM PT E EN PARA A CAPA %%%%%%%%%%%%%%%%%%%%%
%%%%%%%%%%%%%%%%%%%%%%%%%%%%%%%%%%%%%%%%%%%%%%%%%%%%%%%%%%%%%%%%%%%%%%%%%%%%%%%%

% \extrasLANGUAGE vs \captionsLANGUAGE: https://tex.stackexchange.com/a/354197

% Palavras fixas a serem traduzidas
\providecommand\keywordsname{} % Keywords / Palavras-chave
\providecommand\programname{} % Program / Programa
\providecommand\committeename{} % Examining committee / Comissão julgadora
\providecommand\advisorname{} % Advisor / Orientador(a)
\providecommand\coadvisorname{} % Co-advisor / Coorientador(a)
\providecommand\workname{} % Report, Thesis / Tese, Dissertação, Monografia
\providecommand\degreename{} % Masters, Doctorate, Bachelor / Mestrado, Doutorado, Bacharelado
\providecommand\titlename{} % Master, Doctor, Bachelor / Mestre(a), Doutor(a), Bacharel
\providecommand\@assembleLicenseText[1]{O conteúdo deste trabalho
                                        é publicado sob a licença #1}

% Textos longos a serem traduzidos
\providecommand\@coverTCCText{}
\providecommand\@coverQualiText{}
\providecommand\@coverThesisText{}
\providecommand\@institutionBlockText{} % Só para TCC
\providecommand\@provisionalFrontmatterText{}
\providecommand\@finalFrontmatterText{}
\providecommand\@institution{}

% Este não precisa ser traduzido, o texto em inglês não utiliza
\providecommand\@bywhom{%
  \ifdefstring{\@authorGender}{masc}
    {pelo candidato \@author}
    {pela candidata \@author}%
}

%%%%%%%%%% PORTUGUÊS %%%%%%%%%%
\expandafter\addto\csname captions\@IMEpt\endcsname{%
  \DTMrenewdatestyle{month-year}{%
    \renewcommand*{\DTMdisplaydate}[4]
      {\DTMportugesmonthname{##2}\space de\space##1}%
  }%
  \renewcommand\keywordsname{Palavras-chave}%
  \renewcommand\programname{Programa}%
  \renewcommand\committeename{Comissão julgadora}%
  \renewcommand\advisorname{%
    \iftoggle{@tcc}{%
      \ifdefstring{\@advisorGender}{masc}
        {Supervisor}
        {Supervisora}%
    }{%
      \ifdefstring{\@advisorGender}{masc}
        {Orientador}
        {Orientadora}%
    }%
  }%
  \renewcommand\coadvisorname[1]{%
    \iftoggle{@tcc}{%
      \ifcsstring{@coadvisor#1Gender}{masc}
        {Cossupervisor}
        {Cossupervisora}%
    }{%
      \ifcsstring{@coadvisor#1Gender}{masc}
        {Coorientador}
        {Coorientadora}%
    }%
  }%
  \renewcommand\workname{%
    \iftoggle{@tcc}
      {Monografia}
      {\iftoggle{@qualificacao}
        {Exame de Qualificação}
        {\iftoggle{@doutorado}
          {Tese}
          {Dissertação}%
        }%
      }%
  }%
  \renewcommand\degreename{%
    \iftoggle{@doutorado}
      {Doutorado}
      {\iftoggle{@mestrado}
        {Mestrado}
        {\iftoggle{@tcc}
          {Bacharelado}
          {Nível não definido!}%
        }%
      }%
  }%
  \renewcommand\titlename{%
    \iftoggle{@doutorado}
      {\ifdefstring{\@authorGender}{masc}{Doutor}{Doutora}}
      {\iftoggle{@mestrado}
        {\ifdefstring{\@authorGender}{masc}{Mestre}{Mestra}}
        {\iftoggle{@tcc}
          {Bacharel}{Nível não definido!}%
        }%
      }%
  }%
  %
  %
  \renewcommand\@coverTCCText{%
    Monografia Final\vspace{.5\baselineskip}\\
    \@macCDXCIX{} --- Trabalho de\\
    Formatura Supervisionado%
  }%
  \renewcommand\@coverQualiText{%
    Relatório apresentado ao\\
    Instituto de Matemática e Estatística\\
    da Universidade de São Paulo\\
    para exame de qualificação de\\
    \degreename{} em Ciências%
  }%
  \renewcommand\@coverThesisText{%
    \workname{} apresentada ao\\
    Instituto de Matemática e Estatística\\
    da Universidade de São Paulo\\
    para obtenção do título de\\
    \titlename{} em Ciências%
  }%
  \renewcommand\@institutionBlockText{%
    Universidade de São Paulo\\
    Instituto de Matemática e Estatística\\
    Bacharelado em Ciência da Computação%
  }%
  \renewcommand\@provisionalFrontmatterText{%
    \iftoggle{@qualificacao}{%
      Esta é a versão original do texto de qualificação elaborado
      \@bywhom{}, tal como submetido à Comissão Julgadora.%
    }{%
      Esta é a versão original da \MakeLowercase{\workname} elaborada
      \@bywhom{}, tal como submetida à Comissão Julgadora.%
    }%
  }%
  \renewcommand\@finalFrontmatterText{%
    Esta versão da \MakeLowercase{\workname} contém as correções e alterações
    sugeridas pela Comissão Julgadora durante a defesa da versão
    original do trabalho, realizada em \DTMusedate{@defensedate}.\\[1\baselineskip]
    Uma cópia da versão original está disponível no Instituto de
    Matemática e Estatística da Universidade de São Paulo.%
  }%
  \renewcommand\@institution{%
    Instituto de Matemática e Estatística,
    Universidade de São Paulo%
  }%
  \renewcommand{\@assembleLicenseText}[1]{O conteúdo deste trabalho
                                          é publicado sob a licença #1}%
}


%%%%%%%%%% INGLÊS %%%%%%%%%%
\expandafter\addto\csname captions\@IMEen\endcsname{%
  \DTMrenewdatestyle{month-year}{%
    \renewcommand*{\DTMdisplaydate}[4]
      {\DTMenglishmonthname{##2},\space##1}%
  }%
  \renewcommand\keywordsname{Keywords}%
  \renewcommand\programname{Program}%
  \renewcommand\committeename{Examining Committee}
  \renewcommand\advisorname{%
    \iftoggle{@tcc}{Supervisor}{Advisor}%
  }%
  \renewcommand\coadvisorname[1]{%
    \iftoggle{@tcc}{Co-supervisor}{Coadvisor}%
  }%
  % "Tese" e "dissertação" têm sentido contrário em língua inglesa:
  % http://guides.lib.berkeley.edu/dissertations_theses
  % https://www.grad.ubc.ca/handbook-graduate-supervision/graduate-thesis
  % Como "Thesis" é o nome genérico, vamos usar para mestrado e doutorado
  %
  %%%%%
  %
  % Nomes possíveis para o TCC em inglês:
  %
  % * monograph/monography
  %     usado para trabalho de alto nível de um autor "senior",
  %     então não faz sentido para um trabalho de graduação.
  %
  % * undergraduate thesis / bachelor's thesis
  %     plausível, mas no nosso caso report parece melhor.
  %
  % * senior project / senior thesis / honor thesis
  %     usado para "TCCs" de caráter fortemente acadêmico;
  %     não é o caso aqui.
  %
  % * essay / report
  %     razoável, porque trata-se de um texto/relato
  %     sobre o projeto de TCC.
  \renewcommand\workname{%
    \iftoggle{@tcc}
      {Capstone Project Report}
      {\iftoggle{@qualificacao}
        {Qualifying Exam}
        {Thesis}%
      }%
  }%
  \renewcommand\degreename{%
    \iftoggle{@doutorado}
      {Doctorate}
      {\iftoggle{@mestrado}
        {Master's}
        {\iftoggle{@tcc}
          {Bachelor}
          {Nível não definido!}%
        }%
      }%
  }%
  \renewcommand\titlename{%
    \iftoggle{@doutorado}
      {Doctor}
      {\iftoggle{@mestrado}
        {Master}
        {\iftoggle{@tcc}
          {Bachelor}%
          {Nível não definido!}%
        }%
      }%
  }%
  %
  %
  \renewcommand\@coverTCCText{%
    Final Essay\vspace{.5\baselineskip}\\
    \@macCDXCIX{} --- Capstone Project%
  }%
  \renewcommand\@coverQualiText{%
    Report presented to the\\
    Institute of Mathematics and Statistics\\
    of the University of São Paulo\\
    for the \titlename{} of Science\\
    qualifying examination\\%
  }%
  \renewcommand\@coverThesisText{%
    \workname{} presented to the\\
    Institute of Mathematics and Statistics\\
    of the University of São Paulo\\
    in partial fulfillment\\
    of the requirements\\
    for the degree of\\
    \titlename{} of Science%
  }%
  \renewcommand\@institutionBlockText{%
    University of São Paulo\\
    Institute of Mathematics and Statistics\\
    Bachelor of Computer Science%
  }%
  \renewcommand\@provisionalFrontmatterText{%
    \iftoggle{@qualificacao}{%
      This is the original version of the qualifying text prepared
      by candidate \@author, as submitted to the Examining Committee.%
    }{%
      This is the original version of the \MakeLowercase{\workname} prepared
      by candidate \@author, as submitted to the Examining Committee.%
    }%
  }%
  \renewcommand\@finalFrontmatterText{%
    This version of the \MakeLowercase{\workname} includes the corrections
    and modifications suggested by the Examining Committee during
    the defense of the original version of the work, which took
    place on \DTMusedate{@defensedate}.\\[1\baselineskip]
    A copy of the original version is available at the Institute of
    Mathematics and Statistics of the University of São Paulo.%
  }%
  \renewcommand\@institution{%
    Institute of Mathematics and Statistics,
    University of São Paulo%
  }%
  \renewcommand{\@assembleLicenseText}[1]{The content of this work is
                                          published under the #1 license}%
}


%%%%%%%%%%%%%%%%%%%%%%%%%%%%%%%%%%%%%%%%%%%%%%%%%%%%%%%%%%%%%%%%%%%%%%%%%%%%%%%%
%%%%%%%%%%%%%%%%%%%%%%% COLETA E DEFINIÇÃO DE METADADOS %%%%%%%%%%%%%%%%%%%%%%%%
%%%%%%%%%%%%%%%%%%%%%%%%%%%%%%%%%%%%%%%%%%%%%%%%%%%%%%%%%%%%%%%%%%%%%%%%%%%%%%%%

\renewcommand\author[2][masc]{
  \gdef\@author{#2}
  \gdef\@authorGender{#1}
  \hypersetup{pdfauthor={\@author}}
}

\NewDocumentCommand{\orientador}{O{masc} m}{
  \gdef\@advisor{#2}
  \gdef\@advisorGender{#1}
}

% Mais de um coorientador é raro, mas acontece
\ExplSyntaxOn
\newcounter{numberOfCoadvisors}
\NewDocumentCommand\coorientador{O{masc} m}{
    \stepcounter{numberOfCoadvisors}
    \tl_gclear_new:c {@coadvisor\Roman{numberOfCoadvisors}}
    \tl_gclear_new:c {@coadvisor\Roman{numberOfCoadvisors}Gender}

    \tl_set:cn {@coadvisor\Roman{numberOfCoadvisors}} {#2}
    \tl_set:cn {@coadvisor\Roman{numberOfCoadvisors}Gender} {#1}
}

\seq_gclear_new:N \@committeeMembers

\newtoggle{@mestrado}
\newtoggle{@doutorado}
\newtoggle{@tcc}
\newtoggle{@qualificacao}
\newtoggle{@finalversion}

% Opções usando LaTeX3 (veja texdoc l3keys).
\keys_define:nn { IME / defense }
  {
    % Chaves à esquerda definem as variáveis à direita
    data .code:n= {\DTMsavedate{@defensedate}{#1}},
    data .value_required:n = true,
    nivel .choice:,
    nivel / mestrado .code:n = {\@mestrado},
    nivel / masters .code:n = {\@mestrado},
    nivel / dissertacao .code:n = {\@mestrado},
    nivel / doutorado .code:n = {\@doutorado},
    nivel / phd .code:n = {\@doutorado},
    nivel / tese .code:n = {\@doutorado},
    nivel / graduacao .code:n = {\@tcc},
    nivel / bachelor .code:n = {\@tcc},
    nivel / tcc .code:n = {\@tcc},
    nivel .value_required:n = true,
    quali .code:n = {\ifstrequal{#1}{true}{\toggletrue{@qualificacao}}{\togglefalse{@qualificacao}}},
    quali .default:n = {true},
    definitiva .code:n = {\ifstrequal{#1}{true}{\toggletrue{@finalversion}}{\togglefalse{@finalversion}}},
    definitiva .default:n = {true},
    provisoria .code:n = {\ifstrequal{#1}{true}{\togglefalse{@finalversion}}{\toggletrue{@finalversion}}},
    provisoria .default:n = {true},
    programa .tl_gset:N = \@program,
    program .value_required:n = true,
    apoio .tl_gset:N = \@financing,
    apoio .value_required:n = true,
    local .tl_gset:N = \@defenselocation,
    local .value_required:n = true,
    direitos .tl_gset:N = \@license,
    direitos .value_required:n = true,
    fichacatalografica .tl_gset:N = \@catalogingData,
    fichacatalografica .value_required:n = true,
    membrobanca .code:n = {\seq_gput_right:Nn \@committeeMembers {#1}},
    membrobanca .value_required:n = true,
  }

\NewDocumentCommand\defesa{+m}{%
  \keys_set:nn {IME/defense}{#1}

  \exp_args:NV \str_case:nnF \@license
    {
      {CC-BY}{\gdef\@license{\@assembleLicenseText{CC~BY~4.0}\\
        \href{https\c_colon_str//creativecommons.org/licenses/by/4.0/}{%
        (Creative~Commons~Attribution~4.0~International~License)}}
        \hypersetup{pdflicenseurl={https://creativecommons.org/licenses/by/4.0/}}
      }

      {CC-BY-NC}{\gdef\@license{\@assembleLicenseText{CC~BY-NC~4.0}\\
        \href{https\c_colon_str//creativecommons.org/licenses/by-nc/4.0/}{%
        (Creative~Commons~Attribution-NonCommercial~4.0~International~License)}}
        \hypersetup{pdflicenseurl={https://creativecommons.org/licenses/by-nc/4.0/}}
      }

      {CC-BY-ND}{\gdef\@license{\@assembleLicenseText{CC~BY-ND~4.0}\\
        \href{https\c_colon_str//creativecommons.org/licenses/by-nd/4.0/}{%
        (Creative~Commons~Attribution-NoDerivatives~4.0~International~License)}}
        \hypersetup{pdflicenseurl={https://creativecommons.org/licenses/by-nc-nd/4.0/}}
      }

      {CC-BY-SA}{\gdef\@license{\@assembleLicenseText{CC~BY-SA~4.0}\\
        \href{https\c_colon_str//creativecommons.org/licenses/by-sa/4.0/}{%
        (Creative~Commons~Attribution-ShareAlike~4.0~International~License)}}
        \hypersetup{pdflicenseurl={https://creativecommons.org/licenses/by-sa/4.0/}}
      }

      {CC-BY-NC-SA}{\gdef\@license{\@assembleLicenseText{CC~BY-NC-SA~4.0}\\
        \href{https\c_colon_str//creativecommons.org/licenses/by-nc-sa/4.0/}{%
        (Creative~Commons~Attribution-NonCommercial-ShareAlike~4.0~International~License)}}
        \hypersetup{pdflicenseurl={https://creativecommons.org/licenses/by-nc-sa/4.0/}}
      }

      {CC-BY-NC-ND}{\gdef\@license{\@assembleLicenseText{CC~BY-NC-ND~4.0}\\
        \href{https\c_colon_str//creativecommons.org/licenses/by-nc-nd/4.0/}{%
        (Creative~Commons~Attribution-NonCommercial-NoDerivatives~4.0~International~License)}}
        \hypersetup{pdflicenseurl={https://creativecommons.org/licenses/by-nc-nd/4.0/}}
      }
    }
    % If there is no match, use the user-supplied text
    {}
}

\seq_gclear_new:N \@seqkeywordspt
\seq_gclear_new:N \@seqkeywordsen
\newcommand*{\palavrachave}[1]{\seq_gput_right:Nn \@seqkeywordspt {#1}}
\newcommand*{\keyword}[1]{\seq_gput_right:Nn \@seqkeywordsen {#1}}

% Na impressão, as palavras-chave são separadas por pontos
\newcommand*{\@keywordspt}{\seq_use:Nn \@seqkeywordspt {.\space}.}
\newcommand*{\@keywordsen}{\seq_use:Nn \@seqkeywordsen {.\space}.}

% Para inclusão nos metadados com hyperxmp, são separadas por vírgulas
\newcommand*{\@commakeywordspt}{\seq_use:Nn \@seqkeywordspt {,}}
\newcommand*{\@commakeywordsen}{\seq_use:Nn \@seqkeywordsen {,}}

\ExplSyntaxOff

\NewDocumentCommand{\@doutorado}{}{
  \toggletrue{@doutorado}
  \togglefalse{@mestrado}
  \togglefalse{@tcc}
}

\NewDocumentCommand{\@mestrado}{}{
  \togglefalse{@doutorado}
  \toggletrue{@mestrado}
  \togglefalse{@tcc}
}

\NewDocumentCommand{\@tcc}{}{
  \togglefalse{@mestrado}
  \togglefalse{@doutorado}
  \toggletrue{@tcc}
}

% Defaults quando o usuário não define alguma dessas variáveis.
% Não podemos usar \title ou \author aqui porque esses comandos
% dependem de hyperref, que ainda não foi carregada.
\providecommand\@author{Autor não definido!}
\orientador{Orientador não definido!}
\DTMsavedate{@defensedate}{1970-01-01}
\providecommand\@program{Programa não definido!}
\providecommand\@financing{}
\providecommand\@defenselocation{Local não definido!}
\providecommand\@license{Direitos não definidos!}
\providecommand\@title{Título não definido!}
\providecommand\@titlept{Título em português não definido!}
\providecommand\@titleen{Título em inglês não definido!}
\providecommand\@resumo{Resumo não definido!}
\providecommand\@abstract{Abstract não definido!}


%%%%%%%%%%%%%%%%%%%%%%%%%%%%%%%%%%%%%%%%%%%%%%%%%%%%%%%%%%%%%%%%%%%%%%%%%%%%%%%%
%%%%%%%%%%%%%%%%%%%%%%%%%%%%%% TÍTULO E SUBTÍTULO %%%%%%%%%%%%%%%%%%%%%%%%%%%%%%
%%%%%%%%%%%%%%%%%%%%%%%%%%%%%%%%%%%%%%%%%%%%%%%%%%%%%%%%%%%%%%%%%%%%%%%%%%%%%%%%

\ExplSyntaxOn

% Opções usando LaTeX3 (veja texdoc l3keys).
\keys_define:nn { IME / title }
  {
    % Chaves à esquerda definem as variáveis à direita
    titlept .tl_gset:N = \@titlept,
    titlept .value_required:n = true,
    titleen .tl_gset:N = \@titleen,
    titleen .value_required:n = true,
    subtitlept .tl_gset:N = \@subtitlept,
    subtitlept .value_required:n = true,
    subtitleen .tl_gset:N = \@subtitleen,
    subtitleen .value_required:n = true,
  }

\RenewDocumentCommand\title{m}{
  \keys_set:nn {IME/title}{#1}

  \bgroup
  % O título deve existir nas duas línguas; o subtítulo é opcional,
  % mas se houver também deve existir nas duas línguas.
  \IfLanguagePatterns{brazilian}
    {
      \global\let\@title\@titlept
      \ifdefvoid{\@subtitlept}
        {}
        {\global\let\@subtitle\@subtitlept}

      \let\@mainlangtitle\@titlept
      \let\@mainlangsubtitle\@subtitlept
      \let\@otherlangtitle\@titleen
      \let\@otherlangsubtitle\@subtitleen
      \def\@otherlangname{en}
    }
    {
      \global\let\@title\@titleen
      \ifdefvoid{\@subtitleen}
        {}
        {\global\let\@subtitle\@subtitleen}

      \let\@mainlangtitle\@titleen
      \let\@mainlangsubtitle\@subtitleen
      \let\@otherlangtitle\@titlept
      \let\@otherlangsubtitle\@subtitlept
      \def\@otherlangname{pt}
    }

  % Insere os metadados XMP no arquivo PDF final.
  % \@IMEremoveLinebreaksEtc está definida em hyperlinks.tex.

  \@IMEremoveLinebreaksEtc{\@mainlangtitle}
  \@IMEremoveLinebreaksEtc{\@mainlangsubtitle}
  \@IMEremoveLinebreaksEtc{\@otherlangtitle}
  \@IMEremoveLinebreaksEtc{\@otherlangsubtitle}

  \hypersetup{
    pdftitle={\@mainlangtitle
              \ifdefvoid{\@mainlangsubtitle}{}{:~\@mainlangsubtitle}%
             },
  }

  \XMPLangAlt{\@otherlangname}{
      pdftitle={\@otherlangtitle
                \ifdefvoid{\@otherlangsubtitle}{}{:~\@otherlangsubtitle}%
               },
  }

  % XMPLangAlt undefines "\do"; this may cause
  % problems with biblatex, but we do not need
  % to worry here because we are in a group.
  % https://github.com/plk/biblatex/issues/1105
  \egroup
}

\ExplSyntaxOff


%%%%%%%%%%%%%%%%%%%%%%%%%%%%%%%%%%%%%%%%%%%%%%%%%%%%%%%%%%%%%%%%%%%%%%%%%%%%%%%%
%%%%%%%%%%%%%%%%%%%%%%%%%%%%%%%%%% DEDICATÓRIA %%%%%%%%%%%%%%%%%%%%%%%%%%%%%%%%%
%%%%%%%%%%%%%%%%%%%%%%%%%%%%%%%%%%%%%%%%%%%%%%%%%%%%%%%%%%%%%%%%%%%%%%%%%%%%%%%%

% A dedicatória vai em uma página separada, sem numeração,
% com o texto alinhado à direita e margens esquerda e
% superior muito grandes. Vamos fazer isso com uma minipage.
\newenvironment{dedicatoria} {
  \hypersetup{pageanchor=false} % Veja comentário em \maketitle

  \if@openright\cleardoublepage\else\clearpage\fi

  \thispagestyle{empty}
  \vspace*{140mm plus 0mm minus 100mm}
  \noindent
  \begin{FlushRight}
     \begin{minipage}[b][100mm][b]{100mm}
       \begin{FlushRight}
         \itshape
} {
       \end{FlushRight}
     \end{minipage}\hspace*{3em}
  \end{FlushRight}
  \vspace*{50mm plus 0mm minus 10mm}
  \if@openright\cleardoublepage\else\clearpage\fi

  \hypersetup{pageanchor=true}
}


%%%%%%%%%%%%%%%%%%%%%%%%%%%%%%%%%%%%%%%%%%%%%%%%%%%%%%%%%%%%%%%%%%%%%%%%%%%%%%%%
%%%%%%%%%%%%%%%%%%%%%%%%%%%%%%%%%%% RESUMO %%%%%%%%%%%%%%%%%%%%%%%%%%%%%%%%%%%%%
%%%%%%%%%%%%%%%%%%%%%%%%%%%%%%%%%%%%%%%%%%%%%%%%%%%%%%%%%%%%%%%%%%%%%%%%%%%%%%%%

% A página de resumo deve existir em português e inglês; ambas as versões
% utilizam o mesmo environment.

\NewDocumentCommand{\resumo}{+m}{%
  \long\gdef\@resumo{#1}%

  \bgroup
  \@IMEremoveLinebreaksEtc{\@resumo}
  \IfLanguagePatterns{brazilian}
    {
      \hypersetup{
        pdfsubject={\@resumo},
        pdfkeywords={\@commakeywordspt},
      }
    }
    {
      \XMPLangAlt{pt}{pdfsubject={\@resumo}}
      % o item "keywords" não pode ser traduzido

      % XMPLangAlt undefines "\do"; this may cause
      % problems with biblatex, but we do not need
      % to worry here because we are in a group.
      % https://github.com/plk/biblatex/issues/1105
    }
  \egroup

  \bgroup\bgroup % Dois grupos aninhados, veja a documentação da package babel
  \expandafter\selectlanguage\expandafter{\@IMEpt}
  \begin{IMEabstract}\@resumo\end{IMEabstract}
  \egroup\egroup
}

\DeclareDocumentCommand{\abstract}{+m}{%
  \long\gdef\@abstract{#1}%

  \bgroup
  \@IMEremoveLinebreaksEtc{\@abstract}
  \IfLanguagePatterns{brazilian}
    {
      \XMPLangAlt{en}{pdfsubject={\@abstract}}
      % o item "keywords" não pode ser traduzido

      % XMPLangAlt undefines "\do"; this may cause
      % problems with biblatex, but we do not need
      % to worry here because we are in a group.
      % https://github.com/plk/biblatex/issues/1105
    }
    {
      \hypersetup{
        pdfsubject={\@abstract},
        pdfkeywords={\@commakeywordsen},
      }
    }
  \egroup

  \bgroup\bgroup % Dois grupos aninhados, veja a documentação da package babel
  \expandafter\selectlanguage\expandafter{\@IMEen}
  \begin{IMEabstract}\@abstract\end{IMEabstract}
  \egroup\egroup
}

\NewDocumentEnvironment{IMEabstract}{} {
  \if@openright\cleardoublepage\else\clearpage\fi
  \thispagestyle{empty}

    \begin{center}\Large\bfseries\abstractname\end{center}

  \vspace*{2em plus 1em minus 1em}

  \footnotesize

  % Esse é o jeito mais simples de mudar as margens de um parágrafo:
  % faz de conta que é uma lista
  \begin{list}{}{\rightmargin 4em \leftmargin 4em}
    \item\@selfReference
  \end{list}

  \vspace*{1em plus 1em minus 0em}
} {
  % Impede uma quebra de página entre esta linha e a próxima, ou seja,
  % entre a última linha do resumo/abstract e as palavras-chave.
  \@afterheading

  \vspace*{1em plus 1em minus .5em}

  \begingroup

      \setlength{\leftmargini}{\widthof{\textbf{\keywordsname:}\quad}}
      \setlength{\labelwidth}{\widthof{\textbf{\keywordsname:}}}
      \setlength{\labelsep}{\widthof{\quad}}

      \begin{description}
        \item[\keywordsname:]\IfLanguagePatterns{brazilian}
                             {\@keywordspt}
                             {\@keywordsen}%
      \end{description}

  \endgroup
}


%%%%%%%%%%%%%%%%%%%%%%%%%%%%%%%%%%%%%%%%%%%%%%%%%%%%%%%%%%%%%%%%%%%%%%%%%%%%%%%%
%%%%%%%%%%%%%%%%%%%%%% IMPRIME A CAPA E A FOLHA DE ROSTO %%%%%%%%%%%%%%%%%%%%%%%
%%%%%%%%%%%%%%%%%%%%%%%%%%%%%%%%%%%%%%%%%%%%%%%%%%%%%%%%%%%%%%%%%%%%%%%%%%%%%%%%

\RenewDocumentCommand\maketitle{}{
  % Embora as páginas iniciais *pareçam* não ter numeração, a numeração
  % existe, só não é impressa. Os comandos \frontmatter, \mainmatter,
  % \pagenumbering etc. reiniciam a contagem de páginas quando os números
  % passam a ser impressos. Isso significa que há mais de uma página com
  % o número "1". O pacote hyperref não lida bem com essa situação, então
  % vamos desabilitar hyperlinks para números de páginas aqui.
  \hypersetup{pageanchor=false}
  \bgroup
  \onehalfspacing

  \@IMEcover
  \iftoggle{@tcc}{}{\@IMEtitlePage}
  \@IMEversoPage

  \egroup
  \if@openright\cleardoublepage\else\clearpage\fi
  \hypersetup{pageanchor=true}
}

% Layout da capa
\NewDocumentCommand{\@IMEcover}{} {
  \cleardoublepage
  \thispagestyle{empty}

  \begin{hyphenrules}{nohyphenation}
      \iftoggle{@tcc}{\@institutionBlock}{}
      \@titleBlock
      \vspace*{\fill}
      \@detailsBlock
  \end{hyphenrules}
}

% Layout para a página de rosto (duas versões, de acordo
% com a Resolução CoPGr 6018 de 13/10/2011)
\NewDocumentCommand{\@IMEtitlePage}{} {
  \if@openright\cleardoublepage\else\clearpage\fi
  \thispagestyle{empty}

  \begin{hyphenrules}{nohyphenation}
      \@titleBlock
      \vspace*{2cm plus 2cm minus 1cm}
      \@versionInfoBlock
      \vspace*{3.5cm plus 3cm minus 3.5cm}
      \iftoggle{@finalversion}{\@committeeBlock}{}
      \vspace*{2cm plus 2cm minus 2cm}
  \end{hyphenrules}
}

\NewDocumentCommand{\@IMEversoPage}{}{
  \clearpage
  \thispagestyle{empty}
  \ifcsvoid{@catalogingData}
    {\vspace*{4cm}}
    {\vspace*{2cm}}

  \begin{list}{}{\rightmargin 3em \leftmargin 3em}
    \onehalfspacing\centering\footnotesize\itshape
    \item\@license
  \end{list}

  \ifcsvoid{@catalogingData} {} {\vspace{\fill}\@catalogingData}

  \vspace{1cm minus 1cm}
}


%%%%%%%%%%%%%%%%%%%%%%%%%%%%%%%%%%%%%%%%%%%%%%%%%%%%%%%%%%%%%%%%%%%%%%%%%%%%%%%%
%%%%%%%%%%%%%%%%%%%%%%%% POSIÇÃO DOS ELEMENTOS NA CAPA %%%%%%%%%%%%%%%%%%%%%%%%%
%%%%%%%%%%%%%%%%%%%%%%%%%%%%%%%%%%%%%%%%%%%%%%%%%%%%%%%%%%%%%%%%%%%%%%%%%%%%%%%%

% O IME usa uma capa padrão de cartolina para todas as teses/dissertações.
% Essa capa tem uma janela recortada por onde se vê o título e o autor do
% trabalho. Ela fica centralizada na página, tem 100m de largura, 60mm de
% altura e começa 47mm abaixo do topo da página. Como o documento já tem
% margens definidas pelo usuário, precisamos calcular quanto precisamos
% acrescentar ou subtrair dessas margens para colocar o título e autor
% na posição exata (na verdade, com uma pequena folga: 49mm abaixo do topo
% da página, 96mm de largura e 56mm de altura).
%
% Para centralizar horizontalmente, poderíamos pensar em usar "\center",
% mas isso não funciona porque ele centraliza o texto em relação à coluna
% de texto, não à página. Assim, como as margens esquerda e direita do
% documento podem ser diferentes, a janela não ficaria na posição correta.
% O que faremos, então, é colocar essa janela em uma minipage e calcular
% a margem esquerda para que essa minipage fique centralizada.
%
% Além disso, outros elementos da capa também não podem ser centralizados
% com "\center", porque eles ficariam desalinhados em relação à janela
% com o título e autor. Vamos colocar esses outros elementos em uma
% minipage também, mas de tamanho diferente da anterior.
%
% Então, precisamos calcular três valores: a margem adicional em relação ao
% topo da página, a margem esquerda da janela com título e autor e a margem
% esquerda para os demais elementos centralizados da página.

\AtEndPreamble{
  % Calcula o valor das margens (após geometry ser carregada)

  % A distância entre o topo da página e o início do texto (fora o cabeçalho)
  % é dada por (1in + \voffset + \headsep + \topmargin + \headheight).
  % Queremos colocar a caixa com o título 49mm abaixo do topo, então:
  \dimgdef\@topTitleBlockMargin{49mm - (1in + \voffset + \headsep + \topmargin + \headheight)}

  % Quando \vspace é usado no início da página, ele não tem efeito; como
  % não é isso que queremos, vamos usar \vspace*. No entanto, \vspace*
  % é implementado inserindo uma \hrule de espessura zero e depois
  % acrescentando o espaço solicitado. O resultado não é exatamente
  % o esperado, pois \topskip, \parskip e \baselineskip interagem com
  % \vspace* de maneira um tanto complexa:
  % https://tex.stackexchange.com/a/247516
  %
  % Aqui, vamos compensar essa diferença. Note que, se a primeira linha
  % da página tivesse um tamanho de fonte especial, seria necessário
  % usar o valor de \baselineskip correspondente a essa fonte. Além
  % disso, definimos espaçamento simples porque o \vspace* mencionado
  % acima é executado com espaçamento simples.
  \bgroup
  \setstretch {\setspace@singlespace}% \singlespacing adds \baselineskip
  \dimgdef\@topTitleBlockMargin{\@topTitleBlockMargin - \baselineskip - \parskip}
  \egroup

  % Queremos colocar a caixa com o título centralizada na página. "\center"
  % centraliza em função da área de texto, não da página inteira, então
  % não podemos usá-lo, pois as margens esquerda e direita podem ser
  % diferentes. A distância entre a borda esquerda/interna do papel e o
  % início do texto é dada por (1in + \hoffset + \oddsidemargin), então:
  \dimgdef\@leftTitleBlockMargin{(\paperwidth - 96mm)/2 - (1in + \hoffset + \oddsidemargin)}
  \dimgdef\@coverLeftMargin{(\paperwidth - 160mm)/2 - (1in + \hoffset + \oddsidemargin)}
}


%%%%%%%%%%%%%%%%%%%%%%%%%%%%%%%%%%%%%%%%%%%%%%%%%%%%%%%%%%%%%%%%%%%%%%%%%%%%%%%%
%%%%%%%%%%%%% OS ELEMENTOS QUE COMPÕEM A CAPA E A FOLHA DE ROSTO %%%%%%%%%%%%%%%
%%%%%%%%%%%%%%%%%%%%%%%%%%%%%%%%%%%%%%%%%%%%%%%%%%%%%%%%%%%%%%%%%%%%%%%%%%%%%%%%

% Com fontspec (ou seja, lualatex/xelatex), o comando \oldstylenums funciona
% com qualquer fonte que tenha suporte a números old-style. Já com pdflatex,
% o comando para escolher números old style depende da fonte em uso. Nesse
% caso, se não soubermos qual a fonte atual (ou seja, não é nem libertine
% nem libertinus), vamos usar latin modern e torcer para o resultado não ser
% muito discrepante do restante do texto.

% 499 = CDXCIX
\@ifpackageloaded{fontspec}
  {\providecommand{\@macCDXCIX}{mac~\oldstylenums{499}}}
  {
    \providecommand{\@macCDXCIX}{{\fontfamily{lmr}\selectfont mac~\oldstylenums{499}}}

    \@ifpackageloaded{libertinus}
      {\renewcommand{\@macCDXCIX}{{\LibertinusSerifOsF mac~499}}}
      {}

    \@ifpackageloaded{libertine}
      {\renewcommand{\@macCDXCIX}{{\libertineOsF mac~499}}}
      {}
  }

\newcommand{\@coverText}{
  \bgroup
  \setstretch{.9}

  \iftoggle{@tcc}
    {\@coverTCCText}
    {\iftoggle{@qualificacao}{\@coverQualiText}{\@coverThesisText}}
  \par
  \egroup
}

\ExplSyntaxOn
\newcounter{@IMEtmpcnt}
\newcommand*{\@coverPeople} {%
  \begin{tabular}{rl}
    \iftoggle{@tcc}{}{\programname : & \@program \tabularnewline}
    \advisorname : & \@advisor \tabularnewline
    \setcounter{@IMEtmpcnt}{0}%
    \int_while_do:nNnn {\value{@IMEtmpcnt}} < {\value{numberOfCoadvisors}} {%
      \stepcounter{@IMEtmpcnt}%
      \coadvisorname{\Roman{@IMEtmpcnt}}: & \csuse{@coadvisor\Roman{@IMEtmpcnt}} \tabularnewline
    }%
  \end{tabular}
}

\ExplSyntaxOff

\newcommand{\@selfReference} {%
  \bgroup
  \IfLanguagePatterns{brazilian}
    {%
      \let\@currlangtitle\@titlept
      \let\@currlangsubtitle\@subtitlept
    }
    {%
      \let\@currlangtitle\@titleen
      \let\@currlangsubtitle\@subtitleen
    }%
  \@IMEremoveLinebreaksEtc{\@currlangtitle}%
  \@IMEremoveLinebreaksEtc{\@currlangsubtitle}%
  \@author.
  \textbf{\@currlangtitle\ifdefvoid{\@currlangsubtitle}{}{: \textit{\@currlangsubtitle}}}.
  \workname{} (\degreename).
  \@institution,
  São Paulo, \DTMfetchyear{@defensedate}.%
  \egroup
}

% Só para TCC
\newcommand{\@institutionBlock}{

    % A posição do quadro de título é fixa em relação à página;
    % a posição deste quadro é definida em função da posição do
    % quadro de título. Assim, primeiro vamos encontrar onde
    % deve começar o quadro do título. Veja os comentários em
    % \@titleBlock para entender o mecanismo.
    \bgroup
    \hfuzz=60pt % Não precisa avisar que estamos invadindo a margem direita aqui

    \setstretch {\setspace@singlespace}% \singlespacing adds \baselineskip

    \vspace*{\@topTitleBlockMargin}
    \ifdeflength{\@normalstrutheight}
      {}
      {\newlength{\@normalstrutheight}}
    \settoheight{\@normalstrutheight}{\strut}
    \vspace{-\@normalstrutheight}

    % Estamos alinhados com o quadro do título do trabalho,
    % mas não é isso que queremos: a parte inferior deste
    % quadro deve ficar 15mm acima do quadro de título e
    % este quadro tem 20mm de altura, então precisamos subir:
    \vspace{-20mm} % Espaço ocupado por este quadro
    \vspace{-15mm} % Espaço entre este quadro e o quadro de título

    \noindent\strut%
    \hspace*{\@coverLeftMargin}%
%    \fbox{%
      \begin{minipage}[t][20mm][s]{160mm}
        \vspace{0pt plus 20mm}

        \centering\large

        \textsc{\@institutionBlockText}

        \vspace{0pt plus 20mm}
      \end{minipage}
%    }% fbox
    \par

    % Agora precisamos voltar o "cursor" para o começo da página
    % para que o quadro de título seja inserido no lugar certo.
    % Para isso, vamos:
    %
    % 1. Chegar novamente ao início do quadro de título e
    %
    % 2. Retroceder o tamanho da margem superior

    % compensa o espaço inserido por \par logo acima
    \vspace{-\parskip}
    \egroup

    % A altura da minipage já compensou o \vspace{-20mm} acima;
    % ainda precisamos compensar o \vspace{-15mm}
    \vspace{15mm}

    % Agora estamos no início do quadro de título, então
    % podemos recuar exatamente o tamanho da margem superior.
    \vspace{-\@topTitleBlockMargin}
}

% O quadro com o título e o autor que deve ser visível
% através da janela na capa.
\NewDocumentCommand{\@titleBlock}{} {

    \bgroup
    \setstretch {\setspace@singlespace}% \singlespacing adds \baselineskip

    % Este espaço coloca o topo da próxima linha
    % na posição que queremos:
    \vspace*{\@topTitleBlockMargin}

    % No entanto, a próxima linha contém apenas
    % uma minipage, e definir o topo de uma linha
    % desse tipo é complicado. Assim, vamos:
    %
    % 1. Acrescentar um \strut a essa linha;
    %
    % 2. mover o baseline dessa linha para o topo do \strut;
    %
    % 3. Alinhar o topo da minipage ao baseline da linha.
    %
    % Sobre alinhamento de minipages:
    % https://en.wikibooks.org/wiki/LaTeX/Boxes

    \ifdeflength{\@normalstrutheight}
      {}
      {\newlength{\@normalstrutheight}}
    \settoheight{\@normalstrutheight}{\strut}
    \vspace{-\@normalstrutheight}

    \noindent\strut
    \hspace*{\@leftTitleBlockMargin}%
%    \fbox{%
      \begin{minipage}[t][56mm][s]{96mm}
          \vspace*{2cm plus 1.5cm minus 1.8cm}

          \centering\large

          \textbf{\@title}

          \vspace{0.3cm plus 0.2cm minus 0.1cm}

          \ifdefvoid{\@subtitle}{}{\textbf{\textit{\@subtitle}}}

          \vspace{1cm plus 1cm minus 0.6cm}

          \@author

          \vspace*{2cm plus 1.5cm minus 1.8cm}
      \end{minipage}%
%    }% fbox
    \par
    \egroup
}

% As demais informações da capa
\NewDocumentCommand{\@detailsBlock}{} {

  \bgroup
  \hfuzz=60pt % Não precisa avisar que estamos invadindo a margem direita aqui
  \onehalfspacing
  \noindent
  \hspace*{\@coverLeftMargin}%
%  \fbox{%
    \begin{minipage}[t][130mm][s]{160mm}
      \begin{center}
        \Large

        \vspace*{0.3cm plus 0.5cm minus 0.3cm}

        \textsc{\@coverText}

        \vspace*{1.5cm plus 0.5cm minus 0.5cm}

        \large\@coverPeople

        \vspace*{2.5cm plus 1cm minus 1cm}

        \normalsize

        \bgroup
        \singlespacing
        \@financing\par
        \egroup

        \vspace*{1cm plus 1cm minus 0.3cm}

        \@defenselocation

        \iftoggle{@tcc}
          {\DTMfetchyear{@defensedate}}
          {\DTMsetdatestyle{month-year}\DTMusedate{@defensedate}}

      \end{center}
    \end{minipage}%
%  }% fbox
  \par
  \egroup
}

% As informações da banca que vão apenas na versão definitiva
% da página de rosto
\ExplSyntaxOn
\NewDocumentCommand{\@committeeBlock}{} {
    \bgroup
    \onehalfspacing
    \begin{minipage}[t][][t]{\textwidth}
      \begin{quote}
        \normalsize\noindent\committeename :\par
        \begin{list}{}
        {
          \setlength{\leftmargin}{0pt}
          \setlength{\itemsep}{.1\baselineskip}
          \setlength{\topsep}{\baselineskip}
        }
          \item[] \seq_use:Nn \@committeeMembers {\item[]}
        \end{list}
      \end{quote}
    \end{minipage}
    \par
    \egroup
}
\ExplSyntaxOff

% A informação sobre a versão provisória ou definitiva
\NewDocumentCommand{\@versionInfoBlock}{} {%
  % As diretrizes dizem que "A natureza do trabalho, o grau pretendido, o
  % nome da instituição a que é submetido e a área de concentração devem
  % ser alinhados a partir do meio da parte impressa da página para a
  % margem direita, tanto na folha de rosto como na folha de avaliação."
  %
  % Assim, queremos alinhar o texto à direita com uma grande margem
  % à esquerda. Uma solução simples é alinhar o texto à direita
  % e inserir uma minipage. Dentro dela, definimos o texto
  % também alinhado à direita.

  \bgroup
  \onehalfspacing
  \begin{FlushRight}
    %\fbox{
      % Margem direita + 80mm de largura significa que a minipage
      % começa mais ou menos no meio da página.
      \begin{minipage}[t][50mm][s]{80mm}
        \begin{FlushRight}
          \normalsize
          \iftoggle{@finalversion}{%
            \@finalFrontmatterText%
          } {%
            \@provisionalFrontmatterText%
          }%
        \end{FlushRight}
        \vspace*{0pt plus 50mm}
      \end{minipage}
      \par
    %} % fbox
  \end{FlushRight}
  \egroup
}


%%%%%%%%%%%%%%%%%%%%%%%%%%%%%%%%%%%%%%%%%%%%%%%%%%%%%%%%%%%%%%%%%%%%%%%%%%%%%%%%
%%%%%%%%%%%%%%%%%%%%%%%%%%%%%% METADADOS XMP %%%%%%%%%%%%%%%%%%%%%%%%%%%%%%%%%%%
%%%%%%%%%%%%%%%%%%%%%%%%%%%%%%%%%%%%%%%%%%%%%%%%%%%%%%%%%%%%%%%%%%%%%%%%%%%%%%%%

% TODO: Com versões recentes de hyperxmp (final de 2020), não é
%       recomendado definir pdflang; no futuro, isto deve ser mudado.
\AtEndPreamble{
\IfLanguagePatterns{brazilian}
  {
    \hypersetup{
      pdflang={pt},
      pdfmetalang={pt},
    }
  }
  {
    \hypersetup{
      pdflang={en},
      pdfmetalang={en},
    }
  }
}


%%%%%%%%%%%%%%%%%%%%%%%%%%%%%%%%%%%%%%%%%%%%%%%%%%%%%%%%%%%%%%%%%%%%%%%%%%%%%%%%
%%%%%%%%%%%%%%%%%%%%%%%%%%%%% SUMÁRIO E SEÇÕES %%%%%%%%%%%%%%%%%%%%%%%%%%%%%%%%%
%%%%%%%%%%%%%%%%%%%%%%%%%%%%%%%%%%%%%%%%%%%%%%%%%%%%%%%%%%%%%%%%%%%%%%%%%%%%%%%%

% titlesec permite definir formatação personalizada de títulos, seções etc.
% Observe que titlesec é incompatível com os comandos refsection
% e refsegment do pacote biblatex!
% Esta package utiliza titlesec e implementa a possibilidade de incluir
% uma imagem no título dos capítulos com o comando \imgchapter (leia
% os comentários no arquivo da package).
\usepackage{imagechapter} % carregado do diretório extras (veja basics.tex)

\makeatother
 % capa, páginas preliminares e alguns detalhes
%%%%%%%%%%%%%%%%%%%%%%%%%%%%%%%%%%%%%%%%%%%%%%%%%%%%%%%%%%%%%%%%%%%%%%%%%%%%%%%%
%%%%%%%%%%%%%%%%%%%%%%% SUMÁRIO, CABEÇALHOS, SEÇÕES %%%%%%%%%%%%%%%%%%%%%%%%%%%%
%%%%%%%%%%%%%%%%%%%%%%%%%%%%%%%%%%%%%%%%%%%%%%%%%%%%%%%%%%%%%%%%%%%%%%%%%%%%%%%%

% Formatação personalizada do sumário, lista de tabelas/figuras etc.
%\usepackage{titletoc}

% titlesec permite definir formatação personalizada de títulos, seções etc.
% Observe que titlesec é incompatível com os comandos refsection
% e refsegment do pacote biblatex!
% Vamos usar titlesec apenas
% para fazer títulos, seções etc. não serem justificados.
\usepackage[raggedright]{titlesec}

% Permite saber o número total de páginas; útil para colocar no
% rodapé algo como "página 3 de 10" com "\thepage de \pageref{LastPage}"
%\usepackage{lastpage}

% Permite definir cabeçalhos e rodapés
%\usepackage{fancyhdr}

% Personalização de cabeçalhos e rodapés com o estilo deste modelo
\usepackage{imeusp-headers} % carregado do diretório extras (veja basics.tex)
\usepackage{commandstcc}
\usepackage{pseudocodetcc}

% biblatex pode ser configurado para inserir a bibliografia no sumário;
% bibtex não oferece essa possibilidade. Com esta package, resolvemos
% esse problema.
\usepackage[nottoc,notlot,notlof]{tocbibind}

% Só olha até o nível 2 (subseções) para gerar o sumário e os
% cabeçalhos, ou seja, não coloca nomes de subsubseções (nível 3)
% no sumário nem nos cabeçalhos.
\setcounter{tocdepth}{2}

% Só numera até o nível 2 (subseções, como 2.3.1), ou seja, não numera
% sub-subseções (como 2.3.1.1). Veja que isso afeta referências
% cruzadas: se você fizer \ref{uma-sub-subsecao} sem que ela seja
% numerada, a referência vai apontar para a seção um nível acima.
\setcounter{secnumdepth}{2}

% Normalmente, o capítulo de introdução não deve ser numerado, mas
% deve aparecer no sumário. Por padrão, LaTeX não oferece uma solução
% para isso, então criamos aqui os comandos \unnumberedchapter,
% \unnumberedsection e \unnumberedsubsection.
\newcommand{\unnumberedchapter}[2][]{
  \ifblank{#1}
    {
      \chapter*{#2}
      \phantomsection
      \addcontentsline{toc}{chapter}{#2}
      \chaptermark{#2}
    }
    {
      \chapter*{#2}
      \phantomsection
      \addcontentsline{toc}{chapter}{#1}
      \chaptermark{#1}
    }
}

\newcommand{\unnumberedsection}[2][]{
  \ifblank{#1}
    {
      \section*{#2}
      \phantomsection
      \addcontentsline{toc}{section}{#2}
      \sectionmark{#2}
    }
    {
      \section*{#2}
      \phantomsection
      \addcontentsline{toc}{section}{#1}
      \sectionmark{#1}
    }
}

\newcommand{\unnumberedsubsection}[2][]{
  \ifblank{#1}
    {
      \subsection*{#2}
      \phantomsection
      \addcontentsline{toc}{subsection}{#2}
    }
    {
      \subsection*{#2}
      \phantomsection
      \addcontentsline{toc}{subsection}{#1}
    }
}


%%%%%%%%%%%%%%%%%%%%%%%%%%%%%%%%%%%%%%%%%%%%%%%%%%%%%%%%%%%%%%%%%%%%%%%%%%%%%%%%
%%%%%%%%%%%%%%%%%%%%%%%%%% ESPAÇAMENTO E ALINHAMENTO %%%%%%%%%%%%%%%%%%%%%%%%%%%
%%%%%%%%%%%%%%%%%%%%%%%%%%%%%%%%%%%%%%%%%%%%%%%%%%%%%%%%%%%%%%%%%%%%%%%%%%%%%%%%

% LaTeX por default segue o estilo americano e não faz a indentação da
% primeira linha do primeiro parágrafo de uma seção; este pacote ativa essa
% indentação, como é o estilo brasileiro
\usepackage{indentfirst}

% A primeira linha de cada parágrafo costuma ter um pequeno recuo para
% tornar mais fácil visualizar onde cada parágrafo começa. Além disso, é
% possível colocar um espaço em branco entre um parágrafo e outro. Esta
% package coloca o espaço em branco e desabilita o recuo; como queremos
% o espaço *e* o recuo, é preciso guardar o valor padrão do recuo e
% redefini-lo depois de carregar a package.
% TODO: depois que ubuntu 18.04 se tornar obsoleta (abril/2023), remover
%       as linhas "oldparindent" e carregar a package com a opção "indent".
\newlength\oldparindent
\setlength\oldparindent\parindent
\usepackage[parfill]{parskip}
\setlength\parindent\oldparindent


%%%%%%%%%%%%%%%%%%%%%%%%%%%%%%%%%%%%%%%%%%%%%%%%%%%%%%%%%%%%%%%%%%%%%%%%%%%%%%%%
%%%%%%%%%%%%%%%%%%%%%%%%%% EPÍGRAFE E NOTAS DE RODAPÉ %%%%%%%%%%%%%%%%%%%%%%%%%%
%%%%%%%%%%%%%%%%%%%%%%%%%%%%%%%%%%%%%%%%%%%%%%%%%%%%%%%%%%%%%%%%%%%%%%%%%%%%%%%%

% O formato padrão do pacote epigraph é bem feinho...
% Outra opção para epígrafes é o pacote quotchap
\usepackage{epigraph}

\setlength\epigraphwidth{.85\textwidth}

% Sem linha entre o texto e o autor
\setlength{\epigraphrule}{0pt}

% Ambiente auxiliar para colocar margem à direita da epígrafe
% (como sempre, o modo mais simples de mudar as margens de um
% pagrágrafo é fazer de conta que é uma lista)
\newenvironment{epShiftLeft}
  {
    \par\begin{list}{}
      {
        \leftmargin 0pt
        \labelwidth 0pt
        \labelsep 0pt
        \itemsep 0pt
        \topsep 0pt
        \partopsep 0pt
        \rightmargin 2em
      }
    \item\FlushRight
  }
  {
    \end{list}
    % O espaço padrão que epigraph coloca entre a citação
    % e o autor é muito pequeno; vamos aumentar um pouco
    \vspace*{.3\baselineskip}
  }

\renewcommand\textflush{epShiftLeft}
\renewcommand\sourceflush{epShiftLeft}

\newcommand{\epigrafe}[2] {%
  \ifstrempty{#2}{
    \epigraph{\itshape #1}{}
  }{
    \epigraph{\itshape #1}{--- #2}
  }
}

% Formato personalizado para as notas de rodapé. Copiado quase
% literalmente do exemplo na documentação das classes-padrão de
% LaTeX2e (texdoc classes). Seria possível fazer algo similar
% usando list{} com um único item usando \@thefnmark como label.

% \footnotesep não é um espaço adicional, mas sim um strut que
% existe no começo de cada nota. É por isso que o valor é "grande"
% (\baselineskip) mas a separação de fato é pequena.
\makeatletter
\renewcommand\@makefntext[1]{%
    \setlength{\footnotesep}{1\baselineskip}%
    \@setpar{%
        \@@par
        \@tempdima = \hsize
        \advance\@tempdima-4pt\relax
        \parshape \@ne 4pt \@tempdima
    }%
    \par
    \parindent 1em\noindent
    \parskip .3\baselineskip
    \hbox to \z@{\hss\@makefnmark\,}#1%
}
\makeatother

% \maketitle redefine as notas de rodapé (\thanks) para usar símbolos
% ao invés de números, mas essa não é a única mudança. \maketitle
% também muda \@makefnmark para que a indicação de nota de rodapé
% não ocupe espaço horizontal (isso é feito com \rlap). Isso é feito
% porque a lista de autores em geral é similar a
% \author{Fulano\thanks{instituição 1}, Ciclano\thanks{instituição 2}}.
% Com essa mudança, a nota aparece acima da vírgula entre os autores.
% Mas isso significa que\maketitle precisa também modificar \@makefntext
% para que esse efeito aconteça apenas na lista de autores e não na
% nota em si. Assim, como criamos um novo formato para as notas de
% rodapé, precisamos mudar o formato em \maketitle também.
\makeatletter
\newcommand\@maketitlemakefntext[1]{%
    \setlength{\footnotesep}{1\baselineskip}%
    \@setpar{%
        \@@par
        \@tempdima = \hsize
        \advance\@tempdima-4pt\relax
        \parshape \@ne 4pt \@tempdima
    }%
    \par
    \parindent 1em\noindent
    \parskip .3\baselineskip
    \hbox to \z@{\hss\@textsuperscript{\normalfont\@thefnmark}\,}#1%
}

\patchcmd\maketitle
  {\long\def\@makefntext}
  {\let\@makefntext\@maketitlemakefntext\long\def\@disabledmakefntext}
  {}{}

\makeatother

%%%%%%%%%%%%%%%%%%%%%%%%%%%%%%%%%%%%%%%%%%%%%%%%%%%%%%%%%%%%%%%%%%%%%%%%%%%%%%%%
%%%%%%%%%%%%%%%%%%%%%%%%%%%%%% ÍNDICE REMISSIVO %%%%%%%%%%%%%%%%%%%%%%%%%%%%%%%%
%%%%%%%%%%%%%%%%%%%%%%%%%%%%%%%%%%%%%%%%%%%%%%%%%%%%%%%%%%%%%%%%%%%%%%%%%%%%%%%%

% Cria índice remissivo. Este pacote precisa ser carregado antes de hyperref.
% A criação do índice remissivo depende de um programa auxiliar, que pode ser
% o "makeindex" (default) ou o xindy. xindy é mais poderoso e lida melhor com
% línguas diferentes e caracteres acentuados, mas o programa não está mais
% sendo mantido e índices criados com xindy não funcionam em conjunto com
% hyperref. Se quiser utilizar xindy mesmo assim, é possível contornar esse
% segundo problema configurando hyperref para *não* gerar hyperlinks no
% índice (mais abaixo) e configurando xindy para que ele gere esses hyperlinks
% por conta própria. Para isso, modifique a chamada ao pacote imakeidx (aqui)
% e altere as opções do pacote hyperref.
\providecommand\theindex{} % evita erros de compilação se a classe não tem index
%\usepackage[xindy]{imakeidx} % usando xindy
\usepackage{imakeidx} % usando makeindex

% Cria o arquivo de configuração para xindy lidar corretamente com hyperlinks.
\begin{filecontents*}{hyperxindy.xdy}
(define-attributes ("emph"))
(markup-locref :open "\hyperpage{" :close "}" :attr "default")
(markup-locref :open "\textbf{\hyperpage{" :close "}}" :attr "textbf")
(markup-locref :open "\textit{\hyperpage{" :close "}}" :attr "textit")
(markup-locref :open "\emph{\hyperpage{" :close "}}" :attr "emph")
\end{filecontents*}

% Cria o arquivo de configuração para makeindex colocar um cabeçalho
% para cada letra do índice.
\begin{filecontents*}{mkidxhead.ist}
headings_flag 1
heading_prefix "{\\bfseries "
heading_suffix "}\\nopagebreak\n"
\end{filecontents*}

% Por padrão, o cabeçalho das páginas do índice é feito em maiúsculas;
% vamos mudar isso e deixar fancyhdr definir a formatação
\indexsetup{
  othercode={\chaptermark{\indexname}},
}

\makeindex[
  noautomatic,
  intoc,
  % Estas opções são usadas por xindy
  % "-C utf8" ou "-M lang/latin/utf8.xdy" são truques para contornar este
  % bug, que existe em outras distribuições tambem:
  % https://bugs.launchpad.net/ubuntu/+source/xindy/+bug/1735439
  % Se "-C utf8" não funcionar, tente "-M lang/latin/utf8.xdy"
  %options=-C utf8 -M hyperxindy.xdy,
  %options=-M lang/latin/utf8.xdy -M hyperxindy.xdy,
  % Estas opções são usadas por makeindex
  options=-s mkidxhead.ist -l -c,
]

\PassOptionsToPackage{
  % hyperref não gera hyperlinks corretos em índices remissivos criados
  % com xindy; assim, é possível desabilitar essa função aqui e gerar os
  % os hyperlinks através da configuração de xindy definida anteriormente.
  % Com makeindex (o default), quem precisa criar os hyperlinks é hyperref.
  %hyperindex=false,
}{hyperref}

%%%%%%%%%%%%%%%%%%%%%%%%%%%%%%%%%%%%%%%%%%%%%%%%%%%%%%%%%%%%%%%%%%%%%%%%%%%%%%%%
%%%%%%%%%%%%%%%%%%%%%%%%%%%%%%% BIBLIOGRAFIA %%%%%%%%%%%%%%%%%%%%%%%%%%%%%%%%%%%
%%%%%%%%%%%%%%%%%%%%%%%%%%%%%%%%%%%%%%%%%%%%%%%%%%%%%%%%%%%%%%%%%%%%%%%%%%%%%%%%

% Tradicionalmente, bibliografias no LaTeX são geradas com uma combinação entre
% LaTeX (muitas vezes usando o pacote natbib) e um programa auxiliar chamado
% bibtex. Nesse esquema, LaTeX e natbib são responsáveis por formatar as
% referências ao longo do texto e a formatação da bibliografia fica por conta
% do programa bibtex. A configuração dessa formatação é feita através de um
% arquivo auxiliar de "estilo", com extensão ".bst". Vários journals etc.
% fornecem o arquivo .bst que corresponde ao formato esperado da bibliografia.
%
% bibtex e natbib funcionam bem e, se você tiver alguma boa razão para usá-los,
% obterá bons resultados. No entanto, bibtex tem dois problemas: não lida
% corretamente com caracteres acentuados (embora, na prática, funcione com
% os caracteres usados em português) e o formato .bst, que define a formatação
% da bibliografia, é complexo e pouco flexível.
%
% Por conta disso, a comunidade está migrando para um novo sistema chamado
% biblatex. No biblatex, as formatações da bibliografia e das citações são
% feitas pelo próprio pacote biblatex, dentro do LaTeX. Assim, é bem mais fácil
% modificar e personalizar o estilo da bibliografia. biblatex usa o mesmo
% formato de arquivo de dados do bibtex (".bib") e, portanto, não é difícil
% migrar de um para o outro. biblatex também usa um programa auxiliar (biber),
% mas não para realizar a formatação da bibliografia. A maior desvantagem de
% biblatex é que ele é significativamente mais lento que bibtex.
%
% Observe que biblatex pode criar bibliografias independentes por capítulo
% ou outras divisões do texto. Normalmente é preciso indicar essas seções
% manualmente, mas as opções "refsection" e "refsegment" fazem biblatex
% identificar cada capítulo/seção/etc como uma nova divisão desse tipo.
% No entanto, refsection e refsegment são incompatíveis com o pacote
% titlesec, mencionado em imeusp-formatting.tex. Se você pretende criar
% bibliografias independentes por seções, há duas soluções: (1) desabilitar
% o pacote titlesec; (2) indicar as seções manualmente.
%
% Algumas dicas de configuração:
% https://tex.stackexchange.com/q/12806
% https://github.com/PaulStanley/biblatex-tutorial/releases

\PassOptionsToPackage{
  natbib=true, % Reconhece a sintaxe de natbib (\citet, \citep)
  hyperref=true, % Ativa o suporte ao pacote hyperref
  % Se um item da bibliografia tem língua definida (com langid), permite
  % hifenizar com base na língua selecionada.
  autolang=hyphen,
  % Inclui, em cada item da bibliografia, links para as páginas onde o
  % item foi citado
  backref=true,
}{biblatex}

% Este arquivo é executado antes de carregarmos biblatex, então precisamos
% adiar a execução deste comando. Não carregamos biblatex neste arquivo
% porque o usuário pode querer modificar o estilo bibliográfico, que é
% definido por um parâmetro na hora da carga da package.
\AtEndPreamble{
  % TODO: remover menção ao bug de atendvi/hyperxmp em 2024
  % Impede que um item da bibliografia seja dividido em duas páginas.
  % À parte a estética, isso contorna este bug, que afeta links na
  % úlima página do trabalho, ou seja, pode afetar a bibliografia
  % (atenddvi pode ser carregada por hyperxmp):
  % https://github.com/ho-tex/atenddvi/issues/1
  \AtBeginBibliography{\interlinepenalty=10000\raggedbottom}
}

\makeatletter
\AtEndPreamble{
  % backrefs só fazem sentido com documentos grandes
  \ifboolexpr{test {\@ifclassloaded{book}} or test {\@ifclassloaded{report}}}
    {}
    {\ExecuteBibliographyOptions{backref=false,}}

  % em apresentações e posters, a bibliografia deve ser o mais compacta possível
  \@ifclassloaded{beamer}
    {\ExecuteBibliographyOptions{maxbibnames=2,maxcitenames=2}}
    {}
}
\makeatother

%%%%%%%%%%%%%%%%%%%%%%%%%%%%%%%%%%%%%%%%%%%%%%%%%%%%%%%%%%%%%%%%%%%%%%%%%%%%%%%%
%%%%%%%%%%%%%%%%%%%%%%%%%% HIPERLINKS E REFERÊNCIAS %%%%%%%%%%%%%%%%%%%%%%%%%%%%
%%%%%%%%%%%%%%%%%%%%%%%%%%%%%%%%%%%%%%%%%%%%%%%%%%%%%%%%%%%%%%%%%%%%%%%%%%%%%%%%

% O comando \ref por padrão mostra apenas o número do elemento a que se
% refere; assim, é preciso escrever "veja a Figura~\ref{grafico}" ou
% "como visto na Seção~\ref{sec:introducao}". Usando o pacote hyperref
% (carregado mais abaixo), esse número é transformado em um hiperlink.
%
% Se você quiser mudar esse comportamento, ative as packages varioref
% e cleveref e também as linhas "labelformat" e "crefname" mais abaixo.
% Nesse caso, você deve escrever apenas "veja a \ref{grafico}" ou
% "como visto na \ref{sec:introducao}" etc. e o nome do elemento será
% gerado automaticamente como hiperlink.
%
% Se, além dessa mudança, você quiser usar os recursos de varioref ou
% cleveref, mantenha as linhas labelformat comentadas e use os comandos
% \vref ou \cref, conforme sua preferência, também sem indicar o nome do
% elemento, que é inserido automaticamente. Vale lembrar que o comando
% \vref de varioref pode causar problemas com hyperref, impedindo a
% geração do PDF final.
%
% ATENÇÃO: varioref, hyperref e cleveref devem ser carregadas nessa ordem!
%\usepackage{varioref}

%\labelformat{figure}{Figura~#1}
%\labelformat{table}{Tabela~#1}
%\labelformat{equation}{Equação~#1}
%% Isto não funciona corretamente com os apêndices; o comando seguinte
%% contorna esse problema
%%\labelformat{chapter}{Capítulo~#1}
%\makeatletter
%\labelformat{chapter}{\@chapapp~#1}
%\makeatother
%\labelformat{section}{Seção~#1}
%\labelformat{subsection}{Seção~#1}
%\labelformat{subsubsection}{Seção~#1}

% Cria hiperlinks para capítulos, seções, \ref's, URLs etc.
\usepackage{hyperref}

\providetoggle{IMEcolorlinks}
\providetoggle{IMEhidelinks}
\newcommand\hidelinks{\toggletrue{IMEhidelinks}}
\newcommand\colorlinks{\toggletrue{IMEcolorlinks}}
\newcommand\nocolorlinks{\togglefalse{IMEcolorlinks}}
\colorlinks
\AtEndPreamble{
  \iftoggle{IMEhidelinks}
    {\hypersetup{hidelinks}}
    {
      \iftoggle{IMEcolorlinks}{
        \hypersetup{
          colorlinks=true, % também desabilita pdfborderstyle
          %citecolor=black,
          %linkcolor=black,
          %urlcolor=black,
          %filecolor=black,
          citecolor=DarkGreen,
          linkcolor=NavyBlue,
          urlcolor=DarkRed,
          filecolor=green,
          anchorcolor=black,
        }
      }{
        \hypersetup{
          colorlinks=false,
          pdfborder={0 0 .6},
          pdfborderstyle={/S/U/W .6},
          urlbordercolor=DodgerBlue,
          citebordercolor=green!30!white,
          linkbordercolor=blue!30!white,
          filebordercolor=green!30!white,
        }
      }
    }
}

%\usepackage[nameinlink,noabbrev,capitalise]{cleveref}
%% cleveref não tem tradução para o português
%\crefname{figure}{Figura}{Figuras}
%\crefname{table}{Tabela}{Tabelas}
%\crefname{chapter}{Capítulo}{Capítulos}
%\crefname{section}{Seção}{Seções}
%\crefname{subsection}{Seção}{Seções}
%\crefname{subsubsection}{Seção}{Seções}
%\crefname{appendix}{Apêndice}{Apêndices}
%\crefname{subappendix}{Apêndice}{Apêndices}
%\crefname{subsubappendix}{Apêndice}{Apêndices}
%\crefname{line}{Linha}{Linhas}
%\crefname{subfigure}{Figura}{Figuras}
%\crefname{equation}{Equação}{Equações}
%\crefname{listing}{Código-fonte}{Códigos-fonte}
%\crefname{lstlisting}{Código-fonte}{Códigos-fonte}
%\crefname{lstnumber}{Linha}{Linhas}
%\crefrangelabelformat{chapter}{#3#1#4~a~#5#2#6}
%\crefrangelabelformat{section}{#3#1#4~a~#5#2#6}
%\newcommand{\crefrangeconjunction}{ e }
%\newcommand{\crefpairconjunction}{ e }
%\newcommand{\crefmiddleconjunction}{, }
%\newcommand{\creflastconjunction}{ e }
%\crefmultiformat{type}{first}{second}{middle}{last}
%\crefrangemultiformat{type}{first}{second}{middle}{last}

% ao criar uma referência hyperref para um float, a referência aponta
% para o final do caption do float, o que não é muito bom. Este pacote
% faz a referência apontar para o início do float (é possível personalizar
% também). Esta package é incompatível com a classe beamer (usada para
% criar posters e apresentações), então testamos a compatibilidade antes
% de carregá-la.
\ifboolexpr{
  test {\ifcsdef{figure}} and
  test {\ifcsdef{figure*}} and
  test {\ifcsdef{table}} and
  test {\ifcsdef{table*}}
}{\usepackage[all]{hypcap}}{}

% hyperref detecta url's definidas com \url que começam com "http" e
% "www" e cria links adequados. No entanto, quando a url não começa
% com essas strings (por exemplo, "usp.br"), hyperref considera que
% se trata de um link para um arquivo local. Isto força todas as
% \url's que não tem esquema definido a serem do tipo http.
\hyperbaseurl{http://}


%%%%%%%%%%%%%%%%%%%%%%%%%%%%%%%%%%%%%%%%%%%%%%%%%%%%%%%%%%%%%%%%%%%%%%%%%%%%%%%%
%%%%%%%%%%%%%%%%%%%%%%%%%%%%%%%% METADADOS XMP %%%%%%%%%%%%%%%%%%%%%%%%%%%%%%%%%
%%%%%%%%%%%%%%%%%%%%%%%%%%%%%%%%%%%%%%%%%%%%%%%%%%%%%%%%%%%%%%%%%%%%%%%%%%%%%%%%

% XMP (eXtensible Metadata Platform) é um mecanismo proposto pela Adobe
% para a inclusão dos metadados de um documento no próprio documento.
% Esta package se integra com hyperref e não depende de praticamente
% nenhuma modificação em documentos que já utilizam os mecanismos de
% hyperref para a definição de metadados PDF. Ela deve ser carregada
% depois de hyperref mas antes de as opções relacionadas a metadados
% de hyperref serem definidas.

% HACK ALERT! As versões 5.5 e 5.6 de hyperxmp podem "confundir" latexmk,
% fazendo a compilação do documento entrar em um laço infinito. Essas
% versões, assim como a 5.7 que corrige o problema, foram incluídas no
% TeXLive durante o ano de 2020, ou seja, nem a primeira versão nem a
% última atualização de TeXLive 2020 são afetadas. Ainda assim, aqui
% contornamos o problema desabilitando o recurso relacionado (o cálculo
% automático de "byteCount", ou seja, do tamanho aproximado do documento).
% TODO: remover esse código após 2024. Ao mesmo tempo, também podemos
%       desabilitar a package letltxmacro, que é usada apenas aqui.
\ifPDFTeX
  \LetLtxMacro\HACKoldpdffilesize\pdffilesize
  \renewcommand\pdffilesize[1]{}
\fi
\usepackage{hyperxmp}
\ifPDFTeX
  \LetLtxMacro\pdffilesize\HACKoldpdffilesize
\fi

% HACK ALERT! hyperxmp usa atenddvi, que tem este bug:
% https://github.com/ho-tex/atenddvi/issues/1 . Aparentemente, ele não
% afeta xelatex (cf https://github.com/latex3/latex2e/issues/94 ), então
% só precisamos nos preocupar com pdflatex e lualatex. Com pdflatex,
% hyperxmp não deveria utilizar atenddvi, mas às vezes usa. Com lualatex,
% atenddvi parece também não ser necessária, pois hyperxmp não usa
% \special ou \write com lualatex. Então, vamos deixar hyperxmp carregar
% atenddvi mas (1) vamos impedi-la de funcionar e (2) vamos garantir que
% hyperxmp sempre use AtEndDocument com pdflatex e lualatex. Há ainda um
% outro truque para contornar esse problema na configuração da bibliografia,
% mas não podemos ter certeza de que apenas a bibliografia pode ser afetada
% por este bug.

% TODO: esse bug foi resolvido com o lançamento do LaTeX 2020-10-01, que
%       deve ser incluído no TeXlive 2021. Assim, provavelmente podemos
%       remover isto em abril/2025 (veja o comentário em \NeedsTeXFormat)
\makeatletter
\ifXeTeX
\else
  \let\AtEndDvi@AtBeginShipout\relax
  \let\AtEndDvi@CheckImpl\relax
  \let\hyxmp@at@end\AtEndDocument
\fi
\makeatother

% Às vezes é necessário forçar quebras de linha no título para a capa, mas
% essas quebras não devem aparecer em outros lugares (especialmente nos
% metadados XMP). hyperref e hyperxmp removem essas quebras automaticamente,
% mas isso pode fazer duas palavras ficarem grudadas. Estas macros "apagam"
% esses comandos de maneira mais cuidadosa. Elas são utilizadas aqui e em
% imeusp-thesis.

\makeatletter
\ExplSyntaxOn

% Primeiro definimos a regex que vamos usar. O que estamos fazendo aqui é
% "(regex1)|(regex2)|(regex3)|(regex4)"
\regex_const:Nn \@IMEbreakRegex
  {
    ( \c{break} ) % Encontra "\break"
    |
    ( \c{newline} ) % Encontra "\newline"
    |
      % Isto encontra:
      % 1. A control sequence "\linebreak" -> "\c{linebreak}"
      % 2. Opcionalmente seguida de "[NÚMERO]" -> "( \s*\x{5B}\s*\d\s*\x{5D} )?"
    ( \c{linebreak} (\s*\x{5B}\s*\d\s*\x{5D})? )
    |
      % Isto encontra:
      % 1. A control sequence "\\" (quebra de linha) -> "\c{\\}"
      % 2. Optionalmente seguida de "[...]" -> "( \s*\x{5B}\s*...\s*\x{5D} )?"
      % 3. Onde "..." é qualquer dimensão TeX válida
      %    (veja o exemplo de regex em texdoc interface3)
    ( \c{\\}
      ( \s*\x{5B}\s*
          [\+\-]?(\d+|\d*\.\d+)\s*((?i)pt|in|[cem]m|ex|[bs]p|[dn]d|[pcn]c)
        \s*\x{5D}
      )?
    )
  }

\newcommand{\@IMEremoveLinebreaksEtc}[1]{
    % Primeiro, substitui todas as quebras de linha por espaços
    \regex_replace_all:NnN \@IMEbreakRegex {\ } #1

    % Depois elimina eventuais notas de rodapé
    \regex_replace_all:nnN {\s*\c{footnote}} {\c{@gobble}} #1
    \regex_replace_all:nnN {\s*\c{thanks}} {\c{@gobble}} #1

    % Depois transforma URLs em texto simples
    \regex_replace_all:nnN {\c{url}} {\c{@firstofone}} #1
    \regex_replace_all:nnN {\c{href}} {\c{@secondoftwo}} #1

    % Substitui \par (ou uma linha em branco) por um caractere de nova linha
    \regex_replace_all:nnN {\c{par}} {\n} #1

    % Depois elimina espaços repetidos
    \regex_replace_all:nnN {\h+} {\ } #1
    \regex_replace_all:nnN {\ \n} {\n} #1
    \regex_replace_all:nnN {\n+} {\n} #1
}

\ExplSyntaxOff
\makeatother

% Agora inserimos de fato os metadados essenciais XMP no arquivo PDF. Se
% o documento é uma tese/dissertação no formato do IME, outros metadados
% são definidos em imeusp-thesis e sobrescrevem o que está definido aqui.
\makeatletter

\let\@cleantitle\@title
\let\@cleanauthor\@author
\@IMEremoveLinebreaksEtc{\@cleantitle}
\@IMEremoveLinebreaksEtc{\@cleanauthor} % pode incluir \thanks

\hypersetup{
  pdfauthor={\@cleanauthor},
  pdftitle={\@cleantitle},
}

\makeatother

%\nocolorlinks % para impressão em P&B
% Para formatar código-fonte (ex. em Java). listings funciona bem mas
% tem algumas limitações (https://tex.stackexchange.com/a/153915 ).
% Se isso for um problema, a package minted pode oferecer resultados
% (muito) melhores; a desvantagem é que ela depende de um programa
% externo, o pygments (escrito em python).
%
% listings também não tem suporte específico a pseudo-código, mas
% incluímos uma configuração para isso que deve ser suficiente.
% Caso contrário, há diversas packages específicas para a criação
% de pseudocódigo:
%
%  * a mais comum é algorithmicx ("\usepackage{algpseudocode}");
%
%  * algorithm2e é bastante flexível, mas um tanto complexa;
%
%  * clrscode3e foi usada no livro "Introduction to Algorithms",
%    de Cormen, Leiserson, Rivert e Stein;
%
%  * pseudocode foi usada no livro "Combinatorial Algorithms",
%    de Kreher e Stinson;
%
%  * algpseudocodex é uma package relativamente nova similar
%    a algorithmicx/algpseudocode mas com diversas melhorias;
%
%  * pseudo também é relativamente nova; ela funciona de forma
%    um pouco diferente das demais e é bastante customizável.
%
% A diferença entre essas packages e listings/minted é que estas
% últimas "entendem" o código e aplicam a formatação automaticamente,
% enquanto com as packages acima o usuário precisa usar comandos LaTeX
% para definir a formatação.
%
% algorithmicx/algpseudocode, algorithm2e, clrscode3e, pseudocode
% e algpseudocodex usam uma "linguagem" própria baseada em comandos
% LaTeX que pode ser facilmente modificada pelo usuário (ou seja,
% é fácil fazer pseudocódigo em português). Segue um exemplo com
% algpseudocodex, provavelmente a opção mais interessante dentre
% este grupo (note o comando "\While", que imprime automaticamente
% a palavra-chave "while" e ajusta a indentação):
%
% \begin{algorithmic}[1]
%   \Function{Euclid}{$a, b$} \Comment{The g.c.d. of a and b}
%     \State $r\gets a\bmod b$
%     \While{$r\not=0$} \Comment{We have the answer if r is 0}
%       \State $a\gets b$
%       \State $b\gets r$
%       \State $r\gets a\bmod b$
%     \EndWhile
%     \State \textbf{return} $b$ \Comment{The gcd is b}
%   \EndFunction
% \end{algorithmic}
%
% pseudo não usa uma "linguagem" própria desse tipo; ao invés disso,
% ela oferece comandos para a formatação direta de palavras-chave,
% variáveis, indentação etc. Um exemplo ("\\", "\\+" e "\\-" controlam
% as quebras de linha e a indentação):
%
% \begin{pseudo}
%    \kw{Function} \fn{Euclid}(a, b) \ct{The g.c.d. of a and b} \\+
%      $r\gets a\bmod b$ \\
%      \kw{while} $r\not=0$ \ct{We have the answer if r is 0} \\+
%        $a\gets b$ \\
%        $b\gets r$ \\
%        $r\gets a\bmod b$ \\-
%      \kw{end} \\
%      \kw{return} $b$ \ct{The gcd is b} \\-
%    \kw{end}
% \end{pseudo}

\usepackage{listings}
\usepackage{lstautogobble}
% Carrega a "linguagem" pseudocode para listings
\appto{\lstaspectfiles}{,lstpseudocode.sty}
\appto{\lstlanguagefiles}{,lstpseudocode.sty}
% Estes dois são carregados do diretório extras (veja basics.tex)
\lstloadaspects{simulatex,invisibledelims,pseudocode}
\lstloadlanguages{[base]pseudocode,[english]pseudocode,[brazilian]pseudocode}

% O pacote listings não lida bem com acentos! No caso dos caracteres acentuados
% usados em português é possível contornar o problema com a tabela abaixo.
% From https://en.wikibooks.org/wiki/LaTeX/Source_Code_Listings#Encoding_issue
\lstset{literate=
  {á}{{\'a}}1 {é}{{\'e}}1 {í}{{\'i}}1 {ó}{{\'o}}1 {ú}{{\'u}}1
  {Á}{{\'A}}1 {É}{{\'E}}1 {Í}{{\'I}}1 {Ó}{{\'O}}1 {Ú}{{\'U}}1
  {à}{{\`a}}1 {è}{{\`e}}1 {ì}{{\`i}}1 {ò}{{\`o}}1 {ù}{{\`u}}1
  {À}{{\`A}}1 {È}{{\'E}}1 {Ì}{{\`I}}1 {Ò}{{\`O}}1 {Ù}{{\`U}}1
  {ä}{{\"a}}1 {ë}{{\"e}}1 {ï}{{\"i}}1 {ö}{{\"o}}1 {ü}{{\"u}}1
  {Ä}{{\"A}}1 {Ë}{{\"E}}1 {Ï}{{\"I}}1 {Ö}{{\"O}}1 {Ü}{{\"U}}1
  {â}{{\^a}}1 {ê}{{\^e}}1 {î}{{\^i}}1 {ô}{{\^o}}1 {û}{{\^u}}1
  {Â}{{\^A}}1 {Ê}{{\^E}}1 {Î}{{\^I}}1 {Ô}{{\^O}}1 {Û}{{\^U}}1
  {Ã}{{\~A}}1 {ã}{{\~a}}1 {Õ}{{\~O}}1 {õ}{{\~o}}1
  {œ}{{\oe}}1 {Œ}{{\OE}}1 {æ}{{\ae}}1 {Æ}{{\AE}}1 {ß}{{\ss}}1
  {ű}{{\H{u}}}1 {Ű}{{\H{U}}}1 {ő}{{\H{o}}}1 {Ő}{{\H{O}}}1
  {ç}{{\c c}}1 {Ç}{{\c C}}1 {ø}{{\o}}1 {å}{{\r a}}1 {Å}{{\r A}}1
  {€}{{\euro}}1 {£}{{\pounds}}1 {«}{{\guillemotleft}}1
  {»}{{\guillemotright}}1 {ñ}{{\~n}}1 {Ñ}{{\~N}}1 {¿}{{?`}}1
}

% Opções default para o pacote listings
% Ref: http://en.wikibooks.org/wiki/LaTeX/Packages/Listings
\lstset{
  columns=[l]fullflexible,            % do not try to align text with proportional fonts
  basicstyle=\footnotesize\ttfamily,  % the font that is used for the code
  numbers=left,                       % where to put the line-numbers
  numberstyle=\footnotesize\ttfamily, % the font that is used for the line-numbers
  stepnumber=1,                       % the step between two line-numbers. If it's 1 each line will be numbered
  numbersep=20pt,                     % how far the line-numbers are from the code
  autogobble,                         % ignore irrelevant indentation
  commentstyle=\color{Brown}\upshape,
  stringstyle=\color{black},
  identifierstyle=\color{DarkBlue},
  keywordstyle=\color{cyan},
  showspaces=false,                   % show spaces adding particular underscores
  showstringspaces=false,             % underline spaces within strings
  showtabs=false,                     % show tabs within strings adding particular underscores
  %frame=single,                       % adds a frame around the code
  framerule=0.6pt,
  tabsize=2,                          % sets default tabsize to 2 spaces
  captionpos=b,                       % sets the caption-position to bottom
  breaklines=true,                    % sets automatic line breaking
  breakatwhitespace=false,            % sets if automatic breaks should only happen at whitespace
  escapeinside={\%*}{*)},             % if you want to add a comment within your code
  backgroundcolor=\color[rgb]{1.0,1.0,1.0}, % choose the background color.
  rulecolor=\color{darkgray},
  extendedchars=true,
  inputencoding=utf8,
  xleftmargin=30pt,
  xrightmargin=10pt,
  framexleftmargin=25pt,
  framexrightmargin=5pt,
  framesep=5pt,
}

% Um exemplo de estilo personalizado para listings (tabulações maiores)
\lstdefinestyle{wider} {
  tabsize = 4,
  numbersep=15pt,
  xleftmargin=25pt,
  framexleftmargin=20pt,
}

% Outro exemplo de estilo personalizado para listings (sem cores)
\lstdefinestyle{nocolor} {
  commentstyle=\color{darkgray}\upshape,
  stringstyle=\color{black},
  identifierstyle=\color{black},
  keywordstyle=\color{black}\bfseries,
}

% Um exemplo de definição de linguagem para listings (XML)
\lstdefinelanguage{XML}{
  morecomment=[s]{<!--}{-->},
  morecomment=[s]{<!-- }{ -->},
  morecomment=[n]{<!--}{-->},
  morecomment=[n]{<!-- }{ -->},
  morestring=[b]",
  morestring=[s]{>}{<},
  morecomment=[s]{<?}{?>},
  morekeywords={xmlns,version,type}% list your attributes here
}

% Estilo padrão para a "linguagem" pseudocode
\lstdefinestyle{pseudocode}{
  basicstyle=\rmfamily\small,
  commentstyle=\itshape,
  keywordstyle=\bfseries,
  identifierstyle=\itshape,
  % as palavras "function" e "procedure"
  procnamekeystyle=\bfseries\scshape,
  % funções precedidas por function/procedure ou com \func{}
  procnamestyle=\ttfamily,
  specialidentifierstyle=\ttfamily\bfseries,
}
\lstset{defaultdialect=[english]{pseudocode}}

% A package listings tem seu próprio mecanismo para a criação de
% captions, lista de programas etc. Neste modelo não usamos esses
% recursos (veja mais abaixo), mas utilizamos estes nomes:
\addto\extrasbrazil{%
  \gdef\lstlistlistingname{Lista de Algoritmos}%
  \gdef\lstlistingname{Algoritmo}%
}
\addto\extrasbrazilian{%
  \gdef\lstlistlistingname{Lista de Algoritmos}%
  \gdef\lstlistingname{Algoritmo}%
}
\addto\extrasenglish{%
  \gdef\lstlistlistingname{List of Algorithms}%
  \gdef\lstlistingname{Algorithm}%
}

% Novo tipo de float para programas, possível graças à package float
% ou floatrow.
% Observe que a lista de floats de cada tipo é criada automaticamente
% pela package float/floatrow, mas precisamos:
%  1. Definir o nome do comando ("\begin{program}")
%  2. Definir o nome do float em cada língua ("Figura X", "Programa X")
%  3. Definir a extensão do arquivo temporário a ser usada. Pode ser
%     qualquer coisa, desde que não haja repetições. Aqui, usamos "lop";
%     lembre-se que LaTeX já usa várias outras, como "lof", "lot" etc.,
%     então seja cuidadoso na escolha!
%  4. Acrescentar os comandos correspondentes em paginas-preliminares.tex

\makeatletter
\@ifpackageloaded{floatrow}
  {
    \ifcsundef{chapter}
        % O novo ambiente se chama "program" ("\begin{program}") e a extensão
        % temporária é "lop"
        {\DeclareNewFloatType{program}{placement=htbp,fileext=lop}}
        {\DeclareNewFloatType{program}{placement=htbp,fileext=lop,within=chapter}}

    % Ajusta ligeiramente o espaçamento do estilo "ruled".
    \DeclareFloatVCode{customrule}{{\kern 0pt\hrule\kern 2.5pt\relax}}
    \floatsetup[program]{style=ruled,precode=customrule}
  }
  {
    % Não temos a package floatrow; vamos assumir que temos a package float.

    % O estilo padrão do novo float a ser criado (veja mais sobre isso na
    % documentação da package float). Para "program" usamos "ruled", mas
    % para outros floats provavelmente é melhor usar o mesmo formato de
    % Figuras e Tables (plain).
    \floatstyle{ruled}

    \ifcsundef{chapter}
        % O novo ambiente se chama "program" ("\begin{program}") e a extensão
        % temporária é "lop"
        {\newfloat{program}{htbp}{lop}}
        {\newfloat{program}{htbp}{lop}[chapter]}

    % Retorna o estilo dos floats para o padrão
    \floatstyle{plain}
  }
\makeatother

\captionsetup*[program]{style=ruled,position=top}

% "Program X / Programa X" e "Lista de Programas / List of Programs"
\floatname{algorithm}{\lstlistingname}
\gdef\programlistname{\lstlistlistingname}

% Se um programa é maior que uma página, ele não pode ser inserido em
% um float. Nesse caso, vamos criar o ambiente "programruledcaption",
% que cria a mesma estrutura visual e os mesmos captions que os floats
% do tipo "program", mas sem ser um float. Para isso, vamos usar recursos
% da package framed (a package tcolorbox poderia ter sido usada também).
%
% Observe que "programruledcaption" funciona *apenas* para os floats do
% tipo "program". Se quiser criar algo similar para outro tipo de float,
% você vai precisar criar um novo comando ("myfloatruledcaption")
% copiando os comandos abaixo e modificando-os conforme necessário.
\newsavebox{\programCaptionTextBox}
\usepackage{framed}
\newenvironment{programruledcaption}[2][]{
  % All spacing measurements were adjusted to visually reproduce
  % the float captions
  \setlength\fboxsep{0pt}

  % topsep means space before AND after
  \setlength\topsep{.28\baselineskip plus .3\baselineskip minus 0pt}

  \vspace{.3\baselineskip} % Some extra top space

  % For whatever reason, the framed package actually calls "\captionof"
  % multiple times, messing up the counter. We need to prevent this,
  % so we put the caption in a box once and reuse the box.

  \savebox{\programCaptionTextBox}{%
    \parbox[b]{\textwidth}{%
      \ifstrempty{#1}
        {\captionof{program}[#2]{#2}}%
        {\captionof{program}[#1]{#2}}%
    }
  }

  \def\fullcaption{
    \vspace*{-.325\baselineskip}
    \noindent\usebox{\programCaptionTextBox}%
    \vspace*{-.56\baselineskip}%
    \kern 2pt\hrule\kern 2pt\relax
  }

  \def\FrameCommand{
    \hspace{-.007\textwidth}%
    \CustomFBox
      {\fullcaption}
      {\vspace{.13\baselineskip}}
      {.8pt}{.4pt}{0pt}{0pt}
  }

  \def\FirstFrameCommand{
    \hspace{-.007\textwidth}%
    \CustomFBox
      {\fullcaption}
      {\hfill\textit{cont}\enspace$\longrightarrow$}
      {.8pt}{0pt}{0pt}{0pt}
  }

  \def\MidFrameCommand{
    \hspace{-.007\textwidth}%
    \CustomFBox
      {$\longrightarrow$\enspace\textit{cont}\par\vspace*{.3\baselineskip}}
      {\hfill\textit{cont}\enspace$\longrightarrow$}
      {0pt}{0pt}{0pt}{0pt}
  }

  \def\LastFrameCommand{
    \hspace{-.007\textwidth}%
    \CustomFBox
      {$\longrightarrow$\enspace\textit{cont}\par\vspace*{.3\baselineskip}}
      {\vspace{.13\baselineskip}}
      {0pt}{.4pt}{0pt}{0pt}
  }

  \MakeFramed{\FrameRestore}

}{
  \endMakeFramed
}

%%%%%%%%%%%%%%%%%%%%%%%%%%%%%%%%%%%%%%%%%%%%%%%%%%%%%%%%%%%%%%%%%%%%%%%%%%%%%%%%
%%%%%%%%%%%%%%%%%%%%%%%%%%%% OUTROS PACOTES ÚTEIS %%%%%%%%%%%%%%%%%%%%%%%%%%%%%%
%%%%%%%%%%%%%%%%%%%%%%%%%%%%%%%%%%%%%%%%%%%%%%%%%%%%%%%%%%%%%%%%%%%%%%%%%%%%%%%%

% Você provavelmente vai querer ler a documentação de alguns destes pacotes
% para personalizar algum aspecto do trabalho ou usar algum recurso específico.

% melhorias e recursos adicionais para o modo matemático; leia a documentação
\usepackage{mathtools}

% Permite mostrar itens "cancelados" em fórmulas matemáticas, como:
% 2a = 2(b+1)
% \cancel{2}a = \cancel{2}(b+1)
% a = b+1
\usepackage{cancel}

% A classe Book inclui o comando \appendix, que (obviamente) permite inserir
% apêndices no documento. No entanto, não há suporte similar para anexos. Esta
% package acrescenta alguns recursos adicionais para apêndices; vamos usá-la
% para permitir colocar a palavra "Apêndice" no sumário e também para definir
% o comando \annex.
\usepackage{appendix}
\noappendicestocpagenum

\makeatletter

% Altera a formatação da palavra "Apêndice" no sumário
%
% Não queremos a linha "Apêndice" como a última da página no sumário;
% para isso:
%
% 1. Acrescentamos um pouco de espaço elástico logo antes dela
%
% 2. Colocamos uma sugestão de quebra de página
%
% 3. Usamos \@afterheading
%
% 1 e 2 incentivam (mas não forçam) LaTeX a realizar a quebra antes
% dela e 3 força LaTeX a mantê-la na mesma página que a próxima linha.
%
% \addtocontents pode pregar peças se usamos \include; contornamos
% com \immediate: https://tex.stackexchange.com/a/13926
\renewcommand\addappheadtotoc{%
  \begingroup
    \let\origwrite\write
    \def\write{\immediate\origwrite}%
    \addtocontents{toc}{{\large\vspace{0pt plus 2\baselineskip minus 0pt}}}%
    \addtocontents{toc}{\protect\pagebreak[2]}%
    \addtocontents{toc}{\vspace{.8\baselineskip}}%
    \addtocontents{toc}{{\large\bfseries\hspace{-1.3em}\appendixtocname\par}}%
    \addtocontents{toc}{\protect\@afterheading}%
  \endgroup
}

\let\@IMErealAppendixname\appendixname
\let\@IMErealAppendixtocname\appendixtocname
\let\@IMErealAppendixpagename\appendixpagename
\def\appendixname{\@IMErealAppendixname}
\def\appendixtocname{\@IMErealAppendixtocname}
\def\appendixpagename{\@IMErealAppendixpagename}

\providecommand\annexname{Annex}
\providecommand\annextocname{Annexes}
\providecommand\annexpagename{Annexes}

\addto\captionsbrazil{%
  \renewcommand\annexname{Anexo}%
  \renewcommand\annextocname{Anexos}%
  \renewcommand\annexpagename{Anexos}%
  \renewcommand\@IMErealAppendixname{Apêndice}%
  \renewcommand\@IMErealAppendixtocname{Apêndices}%
  \renewcommand\@IMErealAppendixpagename{Apêndices}%
}

\addto\captionsbrazilian{%
  \renewcommand\annexname{Anexo}%
  \renewcommand\annextocname{Anexos}%
  \renewcommand\annexpagename{Anexos}%
  \renewcommand\@IMErealAppendixname{Apêndice}%
  \renewcommand\@IMErealAppendixtocname{Apêndices}%
  \renewcommand\@IMErealAppendixpagename{Apêndices}%
}

\addto\captionsenglish{%
  \renewcommand\annexname{Annex}%
  \renewcommand\annextocname{Annexes}%
  \renewcommand\annexpagename{Annexes}%
  \renewcommand\@IMErealAppendixname{Appendix}%
  \renewcommand\@IMErealAppendixtocname{Appendixes}%
  \renewcommand\@IMErealAppendixpagename{Appendixes}%
}

\let\@IMEorigAppendix\appendix
\renewcommand\appendix{%
    \def\appendixname{\@IMErealAppendixname}
    \def\appendixtocname{\@IMErealAppendixtocname}
    \def\appendixpagename{\@IMErealAppendixpagename}
    \def\Hy@appendixstring{appendix}%
    \@IMEorigAppendix
}

\newcommand\annex{%
    \def\Hy@appendixstring{annex}
    \def\appendixname{\annexname}
    \def\appendixtocname{\annextocname}
    \def\appendixpagename{\annexpagename}
    \@IMEorigAppendix
}

\makeatother

% Para inserir separações no texto que não correspondem a seções com um nome
% definido, é comum usar um ornamento ou florão (em inglês e francês: fleuron).
% Esta package define o comando \froufrou que insere um florão desse tipo.
\usepackage{froufrou} % carregado do diretório extras (veja basics.tex)

% Formatação personalizada das listas "itemize", "enumerate" e
% "description", além de permitir criar novos tipos de listas.
% Com a opção "inline", a package define os novos ambientes "itemize*",
% "description*" e "enumerate*", que fazem os itens da lista como
% parte de um único parágrafo. Como ela causa problemas com
% beamer, apenas a carregamos se não estivermos usando beamer.
\makeatletter
\@ifclassloaded{beamer}
  {}
  {\usepackage[inline]{enumitem}}
\makeatother

% Sublinhado e outras formas de realce de texto
\usepackage{soul}
\usepackage{soulutf8}

% Melhorias e personalização do sublinhado com soul (comando \ul)

% Distância e largura do sublinhado
\setul{1.4pt}{.5pt}

% btul -> "Better Underline" (https://alexwlchan.net/2017/10/latex-underlines/ )
% Sublinhado sem cruzar as linhas descendentes dos caracteres
\usepackage[outline]{contour}
\contourlength{1.1pt}
\newcommand{\btul}[2][white]{%
  \contourlength{1.1pt}%
  \setul{1.4pt}{.5pt}%
  \ul{{\phantom{#2}}}% Faz o sublinhado; precisa das chaves adicionais!
  \llap{\contour{#1}{#2}}% Escreve o texto com fundo branco/colorido
}

% Notação bra-ket
%\usepackage{braket}

% TODO: siunitx removed option "binary-units" in 2021;
%       we can remove this option here after, say, 2025.
% Vários recursos para apresentação de números e grandezas (unidades, notação
% científica, melhor apresentação de números longos etc.), além de permitir
% alinhar números em tabelas pelo ponto decimal (como a package dcolumn)
% através do tipo de coluna "S". Por exemplo, \SI{10}{\hertz} ou
% \num[round-mode=places,round-precision=2]{3.1415926}.
\usepackage[binary-units]{siunitx}
\sisetup{
  mode=text,
  round-mode=places,
}

\providetranslation[to=Portuguese]{to (numerical range)}{a}
\providetranslation[to=Portuguese]{and}{e}
\addto\extrasbrazil{\sisetup{output-decimal-marker = {,}}}
\addto\extrasbrazilian{\sisetup{output-decimal-marker = {,}}}

% Citações melhores; se você pretende fazer citações de textos
% relativamente extensos, vale a pena ler a documentação. biblatex
% utiliza recursos deste pacote.
\usepackage{csquotes}

\usepackage{url}
% URL com fonte sem serifa ao invés de teletype
\urlstyle{sf}

% Permite inserir comentários, muito bom durante a escrita do texto;
% você também pode se interessar pela package pdfcomment.
\usepackage[textsize=scriptsize,colorinlistoftodos,textwidth=2.5cm]{todonotes}
\presetkeys{todonotes}{color=orange!40!white}{}

% Comando para fazer notas com highlight no texto correspondente:
% \hltodo[texto][opções]{comentário}
\makeatletter
\if@todonotes@disabled
  \NewDocumentCommand{\hltodo}{O{} O{} +m}{#1}
\else
  \NewDocumentCommand{\hltodo}{O{} O{} +m}{
    \ifstrempty{#1}{}{\texthl{#1}}%
    \todo[#2]{#3}{}%
  }
\fi
\makeatother

% Vamos reduzir o espaçamento entre linhas nas notas/comentários
\makeatletter
\xpatchcmd{\@todo}
  {\renewcommand{\@todonotes@text}{#2}}
  {\renewcommand{\@todonotes@text}{\begin{spacing}{0.5}#2\end{spacing}}}
  {}
  {}
\makeatother

% Outras ferramentas que podem ser úteis durante a preparação do texto:

% Faz LaTeX mostrar um traço ao lado de linhas "overfull".
%\overfullrule=1mm

% Faz LaTeX mostrar labels e referências bibliográficas:
%\usepackage{showkeys}

% Faz LaTeX mostrar linhas e traços indicando espaçamento, kerning etc.
% Funciona apenas com lualatex.
%\usepackage{lua-visual-debug}

% Além disso, o programa checkcites, instalado juntamente com LaTeX,
% indica problemas com citações bibliográficas.

% Símbolos adicionais, como \celsius, \ohm, \perthousand etc.
%\usepackage{gensymb}

% Permite criar listas como glossários, listas de abreviaturas etc.
% https://en.wikibooks.org/wiki/LaTeX/Glossary
%\usepackage{glossaries}

% Permite formatar texto em colunas
\usepackage{multicol}

% Gantt charts; útil para fazer o cronograma para o exame de
% qualificação, por exemplo.
\usepackage{pgfgantt}

% Estes parâmetros definem a aparência das gantt charts e variam
% em função da fonte do documento.
\ganttset{
    vgrid,
    x unit=1.7em,
    y unit title=3ex,
    y unit chart=4ex,
    % O "strut" é necessário para alinhar o baseline dos nomes dos meses
    title label font=\strut\footnotesize,
    group label font=\footnotesize\bfseries,
    bar label font=\footnotesize,
    milestone label font=\footnotesize\itshape,
    % "align=right" é necessário para \ganttalignnewline funcionar
    group label node/.append style={align=right},
    bar label node/.append style={align=right},
    milestone label node/.append style={align=right},
    group incomplete/.append style={fill=black!50},
    bar/.append style={fill=black!25,draw=black},
    bar incomplete/.append style={fill=white,draw=black},
    % Não é preciso imprimir "0%"
    progress label text=\ifnumequal{#1}{0}{}{(#1\%)},
    % Formato e tamanho dos elementos
    title height=.9,
    group top shift=.4,
    group left shift=0,
    group right shift=0,
    group peaks tip position=0,
    group peaks width=.2,
    group peaks height=.3,
    milestone height=.4,
    milestone top shift=.4,
    milestone left shift=.8,
    milestone right shift=.2,
}

% Em inglês, tanto o nome completo quanto a abreviação do mês de maio
% são "May"; por conta disso, na tradução em português LaTeX erra a
% abreviação. Como talvez usemos o nome inteiro do mês em outro lugar,
% ao invés de forçar a tradução para "Mai" globalmente, fazemos isso
% apenas em ganttchart.
\AtBeginEnvironment{ganttchart}{\deftranslation[to=Portuguese]{May}{Mai}}

% Ilustrações, diagramas, gráficos etc. criados diretamente em LaTeX.
% Também é útil se você quiser importar gráficos gerados com GnuPlot.
\usepackage{tikz}

% Gráficos gerados diretamente em LaTeX; é possível usar tikz para
% isso também.
\usepackage{pgfplots}
% sobre níveis de compatibilidade do pgfplots, veja
% https://tex.stackexchange.com/a/81912
%\pgfplotsset{compat=1.14} % TeXLive 2016
%\pgfplotsset{compat=1.15} % TeXLive 2017
%\pgfplotsset{compat=1.16} % TeXLive 2019
\pgfplotsset{compat=newest}

% Importação direta de arquivos gerados por gnuplot com o
% driver/terminal "lua tikz"; esta package não faz parte da
% instalação padrão do LaTeX, mas sim do gnuplot.
%\usepackage{gnuplot-lua-tikz}

% O formato pdf permite anexar arquivos ao documento, que aparecem
% na página como ícones "clicáveis"; esta package implementa esse
% recurso em LaTeX.
%\usepackage{attachfile}

% Notas de rodapé "órfãs", ou seja, textos que aparecem junto
% das notas de rodapé mas que não têm referência em nenhum lugar.
% "0" desabilita o marcador porque não existe o 0-ésimo símbolo.
\newcommand\detachedfootnote[1]{%
    \bgroup
    \renewcommand\thefootnote{\fnsymbol{footnote}}%
    \renewcommand\thempfootnote{\fnsymbol{mpfootnote}}%
    \footnotetext[0]{#1}%
    \egroup
}

% Os comandos \TeX e \LaTeX são nativos do LaTeX; esta package acrescenta os
% comandos \XeLaTeX e \LuaLaTeX. Você provavelmente não precisa desse recurso
% e, portanto, pode removê-la.
\usepackage{metalogo}
\providecommand{\ConTeXt}{\textsc{Con\TeX{}t}}
% Outros logos da família TeX; você também pode remover estas linhas:
\usepackage{hologo}
\renewcommand{\ConTeXt}{\hologo{ConTeXt}}


% Diretórios onde estão as figuras; com isso, não é preciso colocar o caminho
% completo em \includegraphics (e nem a extensão).
\graphicspath{{figuras/},{logos/}}

% Comandos rápidos para mudar de língua:
% \en -> muda para o inglês
% \br -> muda para o português
% \texten{blah} -> o texto "blah" é em inglês
% \textbr{blah} -> o texto "blah" é em português
\babeltags{br = brazilian, en = english}

% Bibliografia
\usepackage[
  style=extras/plainnat-ime, % variante de autor-data, similar a plainnat
  %style=alphabetic, % similar a alpha
  %style=numeric, % comum em artigos
  %style=authoryear-comp, % autor-data "padrão" do biblatex
  %style=apa, % variante de autor-data, muito usado
  %style=abnt,
]{biblatex}


%%%%%%%%%%%%%%%%%%%%%%%%%%%%%%%%%%%%%%%%%%%%%%%%%%%%%%%%%%%%%%%%%%%%%%%%%%%%%%%%
%%%%%%%%%%%%%%%%%%%%%%%%%%%%%%%%%% METADADOS %%%%%%%%%%%%%%%%%%%%%%%%%%%%%%%%%%%
%%%%%%%%%%%%%%%%%%%%%%%%%%%%%%%%%%%%%%%%%%%%%%%%%%%%%%%%%%%%%%%%%%%%%%%%%%%%%%%%

% O arquivo com os dados bibliográficos para biblatex; você pode usar
% este comando mais de uma vez para acrescentar múltiplos arquivos
\addbibresource{bibliografia.bib}

% Este comando permite acrescentar itens à lista de referências sem incluir
% uma referência de fato no texto (pode ser usado em qualquer lugar do texto)
%\nocite{bronevetsky02,schmidt03:MSc, FSF:GNU-GPL, CORBA:spec, MenaChalco08}
% Com este comando, todos os itens do arquivo .bib são incluídos na lista
% de referências
%\nocite{*}

% É possível definir como determinadas palavras podem (ou não) ser
% hifenizadas; no entanto, a hifenização automática geralmente funciona bem
\babelhyphenation{documentclass latexmk soft-ware clsguide} % todas as línguas
\babelhyphenation[brazilian]{Fu-la-no}
\babelhyphenation[english]{what-ever}

% Estes comandos definem o título e autoria do trabalho e devem sempre ser
% definidos, pois além de serem utilizados para criar a capa, também são
% armazenados nos metadados do PDF.
\title{
    % Obrigatório nas duas línguas
    titlept={Estruturas de Dados Cinéticas},
    titleen={Kinetic Data Structures},
    % Opcional, mas se houver deve existir nas duas línguas
    subtitlept={},
    subtitleen={},
}

\author{Marcos Siolin Martins}

% Para TCCs, este comando define o supervisor
\orientador[fem]{Profª. Drª. Cristina Gomes Fernandes}

% A página de rosto da versão para depósito (ou seja, a versão final
% antes da defesa) deve ser diferente da página de rosto da versão
% definitiva (ou seja, a versão final após a incorporação das sugestões
% da banca).
\defesa{
  nivel=tcc, % mestrado, doutorado ou tcc
  % É a versão para defesa ou a versão definitiva?
  %definitiva,
  % É qualificação?
  %quali,
  programa={Ciência da Computação},
  membrobanca={Profª. Drª. Cristina Gomes Fernandes (orientadora) -- IME-USP [sem ponto final]},
  % Em inglês, não há o "ª"
  % membrobanca{Prof. Dr. Fulana de Tal (advisor) -- IME-USP [sem ponto final]},
  % membrobanca={Prof. Dr. Ciclano de Tal -- IME-USP [sem ponto final]},
  % membrobanca={Profª. Drª. Convidada de Tal -- IMPA [sem ponto final]},
  % Se não houve bolsa, remova
  %
  % Norma sobre agradecimento por auxílios da FAPESP:
  % https://fapesp.br/11789/referencia-ao-apoio-da-fapesp-em-todas-as-formas-de-divulgacao
  %
  % Norma sobre agradecimento por auxílios da CAPES (Portaria 206,
  % de 4 de Setembro de 2018):
  % https://www.in.gov.br/materia/-/asset_publisher/Kujrw0TZC2Mb/content/id/39729251/do1-2018-09-05-portaria-n-206-de-4-de-setembro-de-2018-39729135
  %
  %apoio={O presente trabalho foi realizado com apoio da Coordenação
  %       de Aperfeiçoamento\\ de Pessoal de Nível Superior -- Brasil
  %       (CAPES) -- Código de Financiamento 001}, % o código é sempre 001
  %
  %apoio={This study was financed in part by the Coordenação de
  %       Aperfeiçoamento\\ de Pessoal de Nível Superior -- Brasil
  %       (CAPES) -- Finance Code 001}, % o código é sempre 001
  %
  %apoio={Durante o desenvolvimento deste trabalho, o autor recebeu\\
  %       auxílio financeiro da FAPESP -- processo nº aaaa/nnnnn-d},
  %
  %apoio={During the development if this work, the author received\\
  %       financial support from FAPESP -- grant \#aaaa/nnnnn-d},
  % %
  % apoio={Durante o desenvolvimento deste trabalho o autor
  %        recebeu auxílio financeiro da XXXX},
  local={São Paulo},
  data=2022-12-12, % YYYY-MM-DD
  % A licença do seu trabalho. Use CC-BY, CC-BY-NC, CC-BY-ND, CC-BY-SA,
  % CC-BY-NC-SA ou CC-BY-NC-ND para escolher a licença Creative Commons
  % correspondente (o sistema insere automaticamente o texto da licença).
  % Se quiser estabelecer regras diferentes para o uso de seu trabalho,
  % converse com seu orientador e coloque o texto da licença aqui, mas
  % observe que apenas TCCs sob alguma licença Creative Commons serão
  % acrescentados ao BDTA. Se você tem alguma intenção de publicar o
  % trabalho comercialmente no futuro, sugerimos a licença CC-BY-NC-ND.
  direitos={CC-BY}, % Creative Commons Attribution 4.0 International License
  %direitos={Autorizo a reprodução e divulgação total ou parcial
  %          deste trabalho, por qualquer meio convencional ou
  %          eletrônico, para fins de estudo e pesquisa, desde que
  %          citada a fonte.},
  % Para gerar a ficha catalográfica, acesse https://fc.ime.usp.br/,
  % preencha o formulário e escolha a opção "Gerar Código LaTeX".
  % Basta copiar e colar o resultado aqui.
  fichacatalografica={},
}

%%%%%%%%%%%%%%%%%%%%%%%%%%%%%%%%%%%%%%%%%%%%%%%%%%%%%%%%%%%%%%%%%%%%%%%%%%%%%%%%
%%%%%%%%%%%%%%%%%%%%%%% AQUI COMEÇA O CONTEÚDO DE FATO %%%%%%%%%%%%%%%%%%%%%%%%%
%%%%%%%%%%%%%%%%%%%%%%%%%%%%%%%%%%%%%%%%%%%%%%%%%%%%%%%%%%%%%%%%%%%%%%%%%%%%%%%%

\begin{document}

%%%%%%%%%%%%%%%%%%%%%%%%%%% CAPA E PÁGINAS INICIAIS %%%%%%%%%%%%%%%%%%%%%%%%%%%%

% Aqui começa o conteúdo inicial que aparece antes do capítulo 1, ou seja,
% página de rosto, resumo, sumário etc. O comando frontmatter faz números
% de página aparecem em algarismos romanos ao invés de arábicos e
% desabilita a contagem de capítulos.
\frontmatter

\pagestyle{plain}

\onehalfspacing % Espaçamento 1,5 na capa e páginas iniciais

\maketitle % capa e folha de rosto

%%%%%%%%%%%%%%%% DEDICATÓRIA, AGRADECIMENTOS, RESUMO/ABSTRACT %%%%%%%%%%%%%%%%%%

% \begin{dedicatoria}
% Esta seção é opcional e fica numa página separada; ela pode ser usada para
% uma dedicatória ou epígrafe.
% \end{dedicatoria}

% Reinicia o contador de páginas (a próxima página recebe o número "i") para
% que a página da dedicatória não seja contada.
\pagenumbering{roman}

% Agradecimentos:
% Se o candidato não quer fazer agradecimentos, deve simplesmente eliminar
% esta página. A epígrafe, obviamente, é opcional; é possível colocar
% epígrafes em todos os capítulos. O comando "\chapter*" faz esta seção
% não ser incluída no sumário.
\chapter*{Agradecimentos}
\epigrafe{Whatever you do, do it well.}{Walt Disney}

Gostaria de colocar aqui os meus agradecimentos a todos aqueles que
me apoiaram e contribuíram, de forma direta ou indireta, com a
produção deste trabalho.

Gostaria de agradecer à minha família: as minhas irmãs, Melissa e
Milena, e os meus pais especialmente, Elke e Renato. Eles têm me
guiado e apoiado desde o início e foi com esse apoio e colaboração
que as minhas maiores realizações foram possíveis.

Gostaria de agradecer também a todos os amigos e colegas que
participaram da realização deste trabalho. Em particular, gostaria
de agradecer ao Davi, ao Luciano, ao Willian (Hiroshi) e ao Daniel
(Lawand).

Por fim, gostaria de agradecer a todos os meus professores que me
ensinaram e me fizeram evoluir nos últimos quatro anos. Em especial,
gostaria de agradecer a minha orientadora: a Cris, que me orientou
com muita paciência e dedicação, mesmo antes, quando este estudo era
ainda uma iniciação científica.


%!TeX root=../tcc.tex
%("dica" para o editor de texto: este arquivo é parte de um documento maior)
% para saber mais: https://tex.stackexchange.com/q/78101

% As palavras-chave são obrigatórias, em português e em inglês, e devem ser
% definidas antes do resumo/abstract. Acrescente quantas forem necessárias.
\palavrachave{Estruturas de dados cinéticas}
\palavrachave{Geometria computacional}
\palavrachave{Algoritmos}

\keyword{Kinetic data structures}
\keyword{Computational geometry}
\keyword{Algorithms}

% O resumo é obrigatório, em português e inglês. Estes comandos também
% geram automaticamente a referência para o próprio documento, conforme
% as normas sugeridas da USP.
\resumo{

    Estruturas de dados cinéticas são um modelo proposto por Basch, Guibas e
    Hershberger para a resolução de problemas cinéticos de maneira eficiente.
    Problemas cinéticos são problemas em que os objetos envolvidos estão em
    movimento contínuo e desejamos saber um determinado atributo destes objetos no
    instante atual. Por exemplo, consultar qual o par de pontos mais próximo no
    momento, num conjunto de pontos em movimento.

    A interface proposta para as estruturas oferece suporte a três operações:
    consulta, que retorna o atributo mantido, avançar no tempo, que altera o
    instante atual para um dado instante no futuro, e alterar a trajetória de objetos,
    caracterizada pela alteração de uma função cujo nome é plano de vôo.
    O plano de vôo é necessário para manter o atributo desejado sobre os elementos: ele
    define a trajetória dos pontos ao longo do tempo e será utilizado para calcular os chamados
    certificados.
    Os certificados são objetos que garantem que a organização interna da estrutura está correta até
    um determinado instante, denominado prazo de validade do certificado.
    As estruturas serão orientadas a eventos, que são o vencimento de certificados.
    Os certificados serão mantidos numa fila de prioridades com o seu prazo de
    validade como prioridade.
    O vencimento de um certificado significa que a estrutura ficou inválida e é necessário realizar
    mudanças para que ela volte a se tornar válida, removendo os certificados vencidos e gerando
    novos certificados no processo.

    Com a inclusão da dimensão tempo, a forma tradicional de analisar algoritmos não
    é adequada para a análise dessas estruturas. Por isso, Basch, Guibas e
    Hershberger também propuseram outros quatro critérios com o intuito de
    determinar a eficiência de cada estrutura: responsividade, eficiência,
    localidade e compacidade.

    Neste trabalho, estudaremos quatro problemas: ordenação, máximo, par mais próximo
    e triangulação de Delaunay, todos num contexto cinético. Também estudaremos as
    respectivas estruturas utilizadas na solução de cada problema e discutiremos
    sobre os critérios de eficiência em cada uma delas. Para algumas das estruturas
    também discutiremos um cenário dinâmico-cinético, ou seja, consideraremos operações
    de inserção e remoção dentro de um contexto cinético.

}

\abstract{

    Kinetic data structures are a model proposed by Basch, Guibas, and Hershberger to
    efficiently solve kinetic problems. These are problems in which the
    objects involved are in continuous motion and we want to know a certain
    attribute about these objects at the current moment. For example, we would like to
    query the closest pair of points at the moment, in a set of moving points.

    The proposed interface for the data structures supports three operations: query,
    that returns the maintained attribute, advance, that adjusts the current moment to a given
    instant in the future, and changing the object motion, given by updates on a function called
    flight plan.
    The flight plan is necessary to maintain the attribute of interest over
    the elements; it defines the object motion through time and it will be used for
    computing the so called certificates.
    The certificates are objects that ensure that the internal state of our data structure is
    correct until a certain instant of time: the certificate's expiration time.
    Our data structures will be event driven where the events are the certificates expiration times.
    The certificates will be kept in a priority queue with their expiration time as priority.
    The expiration of a certificate means the data structure has become invalid and we need to
    update it in order to be in a correct state again, deleting the expired certificates and
    generating new ones in the process.

    The traditional way of analyzing algorithms does not work well with these
    structures.
    Because of that, Basch, Guibas, and Hershberger also proposed four
    criteria to determine the quality of the structures: responsiveness, efficiency,
    locality, and compactness.

    In this work, we address four problems: sorting, maximum, closest pair, and Delaunay
    triangulation, all in a kinetic context.
    We study the respective kinetic data structures used to solve these problems and discuss
    the quality criteria for each one of them.
    For some of the structures we also consider a kinetic-dynamic scenario,
    where insert and delete operations will also be supported in a kinetic context.

}



%%%%%%%%%%%%%%%%%%%%%%%%%%% LISTAS DE FIGURAS ETC. %%%%%%%%%%%%%%%%%%%%%%%%%%%%%

% Como as listas que se seguem podem não incluir uma quebra de página
% obrigatória, inserimos uma quebra manualmente aqui.
\makeatletter
\if@openright\cleardoublepage\else\clearpage\fi
\makeatother

% Todas as listas são opcionais; Usando "\chapter*" elas não são incluídas
% no sumário. As listas geradas automaticamente também não são incluídas por
% conta das opções "notlot" e "notlof" que usamos para a package tocbibind.

% Normalmente, "\chapter*" faz o novo capítulo iniciar em uma nova página, e as
% listas geradas automaticamente também por padrão ficam em páginas separadas.
% Como cada uma destas listas é muito curta, não faz muito sentido fazer isso
% aqui, então usamos este comando para desabilitar essas quebras de página.
% Se você preferir, comente as linhas com esse comando e des-comente as linhas
% sem ele para criar as listas em páginas separadas. Observe que você também
% pode inserir quebras de página manualmente (com \clearpage, veja o exemplo
% mais abaixo).
\newcommand\disablenewpage[1]{{\let\clearpage\par\let\cleardoublepage\par #1}}

% Nestas listas, é melhor usar "raggedbottom" (veja basics.tex). Colocamos
% a opção correspondente e as listas dentro de um grupo para ativar
% raggedbottom apenas temporariamente.
\bgroup
\raggedbottom

%%%%% Listas criadas manualmente

%\chapter*{Lista de Abreviaturas}
% \disablenewpage{\chapter*{Lista de Abreviaturas}}

% \begin{tabular}{rl}
%    CFT & Transformada contínua de Fourier (\emph{Continuous Fourier Transform})\\
%    DFT & Transformada discreta de Fourier (\emph{Discrete Fourier Transform})\\
%   EIIP & Potencial de interação elétron-íon (\emph{Electron-Ion Interaction Potentials})\\
%   STFT & Transformada de Fourier de tempo reduzido (\emph{Short-Time Fourier Transform})\\
%   ABNT & Associação Brasileira de Normas Técnicas\\
%    URL & Localizador Uniforme de Recursos (\emph{Uniform Resource Locator})\\
%    IME & Instituto de Matemática e Estatística\\
%    USP & Universidade de São Paulo
% \end{tabular}

%\chapter*{Lista de Símbolos}
% \disablenewpage{\chapter*{Lista de Símbolos}}

% \begin{tabular}{rl}
%   $\omega$ & Frequência angular\\
%     $\psi$ & Função de análise \emph{wavelet}\\
%     $\Psi$ & Transformada de Fourier de $\psi$\\
% \end{tabular}

% Quebra de página manual
\clearpage

%%%%% Listas criadas automaticamente

% Você pode escolher se quer ou não permitir a quebra de página
\listoffigures
% \disablenewpage{\listoffigures}

% Você pode escolher se quer ou não permitir a quebra de página
\listoftables
% \disablenewpage{\listoftables}

% Esta lista é criada "automaticamente" pela package float quando
% definimos o novo tipo de float "program" (em utils.tex)
% Você pode escolher se quer ou não permitir a quebra de página
\listof{algorithm}{\programlistname}
% \disablenewpage{\listof{algorithm}{\programlistname}}

% Sumário (obrigatório)
\tableofcontents

\egroup % Final de "raggedbottom"

% Referências indiretas ("x", veja "y") para o índice remissivo (opcionais,
% pois o índice é opcional). É comum colocar esses itens no final do documento,
% junto com o comando \printindex, mas em alguns casos isso torna necessário
% executar texindy (ou makeindex) mais de uma vez, então colocar aqui é melhor.
% \index{Inglês|see{Língua estrangeira}}
% \index{Figuras|see{Floats}}
% \index{Tabelas|see{Floats}}
% \index{Código-fonte|see{Floats}}
% \index{Subcaptions|see{Subfiguras}}
% \index{Sublegendas|see{Subfiguras}}
% \index{Equações|see{Modo matemático}}
% \index{Fórmulas|see{Modo matemático}}
% \index{Rodapé, notas|see{Notas de rodapé}}
% \index{Captions|see{Legendas}}
% \index{Versão original|see{Tese/Dissertação, versões}}
% \index{Versão corrigida|see{Tese/Dissertação, versões}}
% \index{Palavras estrangeiras|see{Língua estrangeira}}
% \index{Floats!Algoritmo|see{Floats, ordem}}


%%%%%%%%%%%%%%%%%%%%%%%%%%%%%%%% CAPÍTULOS %%%%%%%%%%%%%%%%%%%%%%%%%%%%%%%%%%%%%

% Aqui vai o conteúdo principal do trabalho, ou seja, os capítulos que compõem
% a dissertação/tese. O comando mainmatter reinicia a contagem de páginas,
% modifica a numeração para números arábicos e ativa a contagem de capítulos.
\mainmatter

\pagestyle{mainmatter}

% Espaçamento simples
\singlespacing

%!TeX root=../tcc.tex
%("dica" para o editor de texto: este arquivo é parte de um documento maior)
% para saber mais: https://tex.stackexchange.com/q/78101

%% ------------------------------------------------------------------------- %%

% "\chapter" cria um capítulo com número e o coloca no sumário; "\chapter*"
% cria um capítulo sem número e não o coloca no sumário. A introdução não
% deve ser numerada, mas deve aparecer no sumário. Por conta disso, este
% modelo define o comando "\unnumberedchapter".
\unnumberedchapter{Introdução}
\label{cap:introducao}

\enlargethispage{.5\baselineskip}

Quando desejamos criar algoritmos para resolver problemas com o computador,
utilizamos maneiras de organizar os dados para que operações de acesso e
alteração desses dados possam ser realizadas rapidamente, as chamadas estruturas
de dados. A forma como serão organizados os dados depende altamente das
características do problema em questão.

Neste trabalho estudaremos \textit{estruturas de dados cinéticas} (em inglês,
\emph{KDS - Kinetic Data Structures}), propostas por Basch, Guibas e Hershberger
% referencia bibliográfica aqui
para a resolução dos chamados problemas \textit{cinéticos}.

Problemas \textit{cinéticos} são problemas em que deseja-se manter um
determinado atributo sobre objetos que estão em movimento contínuo. Por exemplo,
num conjunto dado de pontos em movimento, qual par de pontos possui distância
mínima. Os objetos nos problemas podem representar entidades do mundo físico:
pontos podem representar pessoas, aviões, estabelecimentos, entre outras coisas,
retas podem representar trajetórias. Devido à natureza desses problemas é comum
que olhemos para problemas de geometria computacional, mas dentro de um contexto
cinético.

Quando é dado um conjunto fixo de objetos geométricos, e deseja-se saber
informações de um determinado atributo desses objetos (como, por exemplo, em um
conjunto dado de pontos, qual par de pontos possui distância mínima), dizemos
que esse é um problema \textit{estático}.

O mesmo problema pode ser formulado sobre um conjunto mutável. Por exemplo,
pontos poderiam ser inseridos e removidos ao longo do tempo. Queremos calcular o
atributo sem ter que resolver do zero a nova instância do problema estático.
Chamamos esse tipo de problema de \textit{dinâmico} ou \textit{on-line}.

As \emph{estruturas de dados cinéticas} recebem esse nome para diferenciá-las
das estruturas de dados \textit{estáticas} e \textit{dinâmicas}, pois tem como
foco em manter a descrição combinatória do problema, que agora também se altera
% definir o que é descrição combinatória
com a passagem de tempo, já que os objetos estão em movimento contínuo.

% definir o que é o atributo geométrico dos objetos
Essas estruturas nos permitem realizar consultas de um determinado atributo dos
objetos, no instante atual. A garantia de que a estrutura permanece correta se
dá através do uso de instrumentos chamados \textit{certificados}. Os
certificados estabelecem que uma relação entre um objeto da estrutura e outro se
mantém verdadeira até o seu vencimento e devem ajudar na manutenção da estrutura
para permitir as consultas desejadas.

Nos problemas estudados neste trabalho os objetos serão pontos. Precisaremos
saber como os pontos se movimentam para determinar as relações entre os objetos
na estrutura e calcular o vencimento de certificados, o chamado \textit{plano de
vôo}. Uma operação de atualização pode ser realizada neste \emph{plano de vôo}, o que
implicará em mudanças a serem feitas nos certificados da estrutura. Neste
trabalho, a trajetória dos pontos será linear, descrita por uma função
$\gamma(t) = (x(t), y(t))$, sendo $x(t)$ e $y(t)$ funções afim. O modelo
proposto por Basch, Guibas e Hershberger também pode ser aplicado para
trajetórias não-lineares.

Falar sobre forma trivial de resolver, porque não funciona e usar de gatilho
para explicar as estruturas de dados e as medidas de eficiência.

% Poderíamos usar certificados por exemplo entre alguns pares de pontos que,
% considerando suas trajetórias atuais, garantissem que, até o instante $t$, a
% ordenada de um dos pontos é maior que a ordenada do outro ponto. Ao atingirmos o
% instante $t$, o certificado vence, implicando em uma mudança estrutural no
% conjunto de pontos que pode afetar o resultado de futuras consultas, requerendo
% assim possíveis ajustes na estrutura de dados, e eventual cálculo de novos
% certificados.

%!TeX root=../../tcc.tex

%% ------------------------------------------------------------------------- %%
\chapter{Ordenação cinética}
Considere o seguinte problema cinético. São dados $n$ pares de
valores. Cada par $(x_0, v)$ representa um valor que está mudando
linearmente com o tempo. Num instante arbitrário $t \geq 0$, o valor
correspondente ao par $(x_0, v)$ é $x_0 + tv$. O objetivo é
responder consultas do tipo: para um certo $i$, com $1 \leq i \leq
n$, quem é o $i$-ésimo maior valor da coleção no instante corrente.

Por exemplo, se tivermos quatro elementos na coleção, digamos
$\left(6, -\frac{1}{2}\right)$, $(5, 0)$, $\left(3,
\frac{1}{4}\right)$ e $\left(0, \frac{4}{3}\right)$, podemos
representar essa coleção como na Figura \ref{fig:ordenacao:exemplo}.

\begin{figure}[H]
    \centering
    \begin{tikzpicture}[thick, scale=0.7]
        \node[label={1},circle,draw,minimum size=1cm]
        (1) at (0,0) {$3$};
        \node[label={2},circle,draw,minimum size=1cm]
        (2) at (-4,-2) {$3$};
        \node[label={3},circle,draw,minimum size=1cm]
        (3) at (4,-2) {$4$};
        \node[label={4},circle,draw,minimum size=1cm]
        (4) at (-6,-4) {$1$};
        \node[label={5},circle,draw,minimum size=1cm]
        (5) at (-2,-4) {$3$};
        \node[label={6},circle,draw,minimum size=1cm]
        (6) at (2,-4) {$4$};
        \node[label={7},circle,draw,minimum size=1cm]
        (7) at (6,-4) {$6$};
        \node[label={8},circle,draw,minimum size=1cm]
        (8) at (-7,-6) {$8$};
        \node[label={9},circle,draw,minimum size=1cm]
        (9) at (-5,-6) {$1$};
        \node[label={10},circle,draw,minimum size=1cm]
        (10) at (-3,-6) {$2$};
        \node[label={11},circle,draw,minimum size=1cm]
        (11) at (-1,-6) {$3$};
        \node[label={12},circle,draw,minimum size=1cm]
        (12) at (1,-6) {$4$};
        \node[label={13},circle,draw,minimum size=1cm]
        (13) at (3,-6) {$5$};
        \node[label={14},circle,draw,minimum size=1cm]
        (14) at (5,-6) {$6$};
        \node[label={15},circle,draw,minimum size=1cm]
        (15) at (7,-6) {$7$};
        \node[label={16},circle,draw,minimum size=1cm]
        (16) at (-8,-8) {$8$};
        \node[label={17},circle,draw,minimum size=1cm]
        (17) at (-6,-8) {$9$};

        \draw[thick] (1) -- (2);
        \draw[thick] (2) -- (4);
        \draw[thick] (4) -- (8);
        \draw[thick] (4) -- (9);
        \draw[thick] (8) -- (16);
        \draw[thick] (8) -- (17);
        \draw[thick] (2) -- (5);
        \draw[thick] (5) -- (10);
        \draw[thick] (5) -- (11);
        \draw[thick] (1) -- (3);
        \draw[thick] (3) -- (6);
        \draw[thick] (3) -- (7);
        \draw[thick] (6) -- (12);
        \draw[thick] (6) -- (13);
        \draw[thick] (7) -- (14);
        \draw[thick] (7) -- (15);
    \end{tikzpicture}
    \caption[Representação da estrutura torneio]{Torneio com $9$
        elementos em que $3$ é o elemento com valor máximo.}
    \label{fig:torneio:exemplo}
\end{figure}

\newpage

Queremos dar suporte às seguintes operações:
\begin{itemize}
    \item \textsc{advance}$(t)$ $\rightarrow$ avança o tempo
    corrente para $t$;
    \item \textsc{change}$(j, v)$ $\rightarrow$ altera a
    velocidade do elemento $j$ para $v$;
    \item \textsc{query\_kth}$(i)$ $\rightarrow$ devolve o
    elemento cujo valor é o $i$-ésimo maior no instante atual.
\end{itemize}

%!TeX root=./ordenacao.tex

%% ------------------------------------------------------------------------- %%


\section{Lista ordenada cinética}
\label{sec:lista}
Um jeito natural de resolver o problema da ordenação cinética é por
meio de uma lista ordenada cinética, que é manter um vetor com os
elementos dados em ordem decrescente do valor no instante atual.

Inicialmente o vetor começa com os valores dos elementos no instante
$t = 0$, ou seja, com o valor $x_0$ de cada elemento, e este vetor é
ordenado em ordem decrescente.
Na verdade, o vetor pode armazenar não os valores, mas os índices dos elementos, e fazemos
ordenação indireta.
No caso de empates nos valores dos elementos, o desempate
será feito pela velocidade: se dois elementos, digamos $i$
e $j$, possuem o mesmo valor $x_0$, mas a velocidade de $i$ é maior
que a de $j$, então $i$ será tratado como se possuísse maior valor
que $j$ no instante inicial.
Esse mesmo critério de desempate será aplicado em todos os instantes e também em todos os
problemas daqui em diante.

\begin{figure}[H]
    \centering
    \begin{tikzpicture}[thick]
        \draw[thick,->] (0,0) -- (6.5,0) node[anchor=north west]
            {$t$};
%        \draw[thick,->] (0,0) -- (0,8) node[anchor=south east]
%            {valor$(t)$};
        \foreach \x in {0, 1,..., 6}
        \draw (\x cm,1pt) -- (\x cm,-1pt)
        node[anchor=north] {$\x$};
        \node[circle, draw] at (-1,1 cm) {$4$};
        \node[circle, draw] at (-1,2 cm) {$3$};
        \node[circle, draw] at (-1,3 cm) {$2$};
        \node[circle, draw] at (-1,4 cm) {$1$};

        \node at (0,1 cm) {$4$};
        \node at (0,2 cm) {$3$};
        \node at (0,3 cm) {$2$};
        \node at (0,4 cm) {$1$};

        \node at (2,1 cm) {$4$};
        \node at (2,2 cm) {$3$};
        \node at (2,3 cm) {$1$};
        \node at (2,4 cm) {$2$};

        \node at (2.76,1 cm) {$3$};
        \node at (2.76,2 cm) {$4$};
        \node at (2.76,3 cm) {$1$};
        \node at (2.76,4 cm) {$2$};

        \node at (3.27,1 cm) {$3$};
        \node at (3.27,2 cm) {$1$};
        \node at (3.27,3 cm) {$4$};
        \node at (3.27,4 cm) {$2$};

        \node at (3.75,1 cm) {$3$};
        \node at (3.75,2 cm) {$1$};
        \node at (3.75,3 cm) {$2$};
        \node at (3.75,4 cm) {$4$};

        \node at (4,1 cm) {$1$};
        \node at (4,2 cm) {$3$};
        \node at (4,3 cm) {$2$};
        \node at (4,4 cm) {$4$};


        \draw[dashed] (0, 0) -- (0, 0.8);
        \draw[dashed] (2, 0) -- (2, 0.8);
        \draw[dashed] (2.76, 0) -- (2.76, 0.8);
        \draw[dashed] (3.27, 0) -- (3.27, 0.8);
        \draw[dashed] (3.75, 0) -- (3.75, 0.8);
        \draw[dashed] (4, 0) -- (4, 0.8);


    \end{tikzpicture}
    \caption[Exemplo de trocas na lista ordenada]{Vetor com os índices dos elementos, ordenado
    pelos valores dos elementos no tempo $t = 0$ e suas
    alterações a medida que o tempo avança, para os quatro
    elementos da Figura~\ref{fig:ordenacao:exemplo}.}
    \label{fig:lista:vetores}
\end{figure}

Uma vez de posse do vetor ordenado com os valores iniciais
decrescentemente, construímos um certificado para cada par de
elementos consecutivos no vetor.
O $i$-ésimo certificado, denotado pelo par $(i, t)$, se refere ao par das posições $i$ e $i + 1$.
O valor $t$ consiste no instante de tempo em que o $i$-ésimo elemento
deixará de ter um valor maior que o valor do $(i + 1)$-ésimo
elemento do vetor, se esse instante for maior ou igual a 0, ou em
geral ao instante atual.
Do contrário, o valor $t$ consiste em $+\infty$.
O valor $t$ do certificado é o seu \textit{prazo de
validade}.

Esses prazos de validade determinam os \textit{eventos} que
potencialmente causarão modificações no vetor que mantém os
elementos ordenados pelo seu valor e, consequentemente, em alguns
certificados.

Esses $n - 1$ certificados são colocados em uma fila com
prioridades, com seu prazo de validade determinando a prioridade.
Estamos interessados nos certificados com menor prazo de validade.
Ou seja, a fila com prioridades pode ser implementada com um heap de
mínimo que usa os prazos de validade como chave.

Para descrever a implementação das três operações, precisamos
estabelecer o nome das variáveis usadas.
São elas:
\begin{enumerate}
    \item $n$: o número de elementos dados;
    \item $x_0$ e \textit{speed}: vetores com o valor e a velocidade
    inicial de cada um dos $n$ elementos;
    \item \now: instante atual.
    A variável \now\ será tratada como
    global, ou seja, será utilizada nas rotinas sem ser passada como
    argumento;
    \item \textit{sorted}: vetor com os índices dos $n$ elementos em
    ordem decrescente do seu valor no instante \textit{now};
%    \item \textit{indS}: vetor de $n$ posições; \textit{indS}[$j$]
%    guarda a posição em \textit{sorted} do elemento $j$;
    \item \textit{cert}: vetor com os $n-1$ certificados;
    \textit{cert}$[i]$ guarda o $i$-ésimo certificado, ou seja, o certificado
    entre $\sorted[i]$ e ${\sorted[i+1]}$, para~$1\leq i < n$;
    \item \textit{Q}: fila que guarda os inteiros $1, \ldots, n-1$,
    sendo \textit{cert}[$i$] a prioridade do inteiro $i$ na fila.
\end{enumerate}

\begin{figure}[H]
    \centering
    \begin{tikzpicture}[thick]
        \node at (0.5,0 cm) {$4$};
        \node at (0.5,0.5 cm) {$3$};
        \node at (0.5,1 cm) {$2$};
        \node at (0.5,1.5 cm) {$1$};

        \draw (0.75,-0.5) -- (0.75, 2);
        \draw (1.25,-0.5) -- (1.25, 2);
        \draw (0.75,-0.5) -- (1.25,-0.5);
        \draw (0.75, 2) -- (1.25, 2);
        \node at (1,0 cm) {$4$};
        \node at (1,0.5 cm) {$3$};
        \node at (1,1 cm) {$2$};
        \node at (1,1.5 cm) {$1$};

        \draw[->, shorten >= 0.1pt, shorten <= 0.1pt, >=stealth, line width=0.25mm]
        (1.25, 0) to[out=0,in=-30] node[right=1pt] {$\frac{36}{13}$} (1.25, 0.5);

        \draw[->, shorten >= 0.1pt, shorten <= 0.1pt, >=stealth, line width=0.25mm]
        (1.25, 0.5) to[out=0,in=-30] node[right=1pt] {$8$} (1.25, 1);

        \draw[->, shorten >= 0.1pt, shorten <= 0.1pt, >=stealth, line width=0.25mm]
        (1.25, 1) to[out=0,in=-30] node[right=1pt] {$2$} (1.25, 1.5);

        \node at (1,2.25 cm) {$\sorted$};

        \node at (2.5,0.5 cm) {$3$};
        \node at (2.5,1 cm) {$2$};
        \node at (2.5,1.5 cm) {$1$};

        \draw (2.75,0) -- (2.75, 2);
        \draw (3.25,0) -- (3.25, 2);
        \draw (2.75,0) -- (3.25,0);
        \draw (2.75, 2) -- (3.25, 2);
        \node at (3,0.5 cm) {$\frac{36}{13}$};
        \node at (3,1 cm) {$8$};
        \node at (3,1.5 cm) {$2$};

        \node at (3,2.25 cm) {$\cert$};

        \begin{scope}
            [shift={(5,0)}]
            \node at (0.5,0 cm) {$4$};
            \node at (0.5,0.5 cm) {$3$};
            \node at (0.5,1 cm) {$2$};
            \node at (0.5,1.5 cm) {$1$};

            \draw (0.75,-0.5) -- (0.75, 2);
            \draw (1.25,-0.5) -- (1.25, 2);
            \draw (0.75,-0.5) -- (1.25,-0.5);
            \draw (0.75, 2) -- (1.25, 2);
            \node at (1,0 cm) {$3$};
            \node at (1,0.5 cm) {$4$};
            \node at (1,1 cm) {$1$};
            \node at (1,1.5 cm) {$2$};

            \draw[->, shorten >= 0.1pt, shorten <= 0.1pt, >=stealth, line width=0.25mm]
            (1.25, 0) to[out=0,in=-30] node[right=1pt] {$\infty$} (1.25, 0.5);

            \draw[->, shorten >= 0.1pt, shorten <= 0.1pt, >=stealth, line width=0.25mm]
            (1.25, 0.5) to[out=0,in=-30] node[right=1pt] {$\frac{36}{11}$} (1.25, 1);

            \draw[->, shorten >= 0.1pt, shorten <= 0.1pt, >=stealth, line width=0.25mm]
            (1.25, 1) to[out=0,in=-30] node[right=1pt] {$\infty$} (1.25, 1.5);

            \node at (1,2.25 cm) {$\sorted$};

            \node at (2.5,0.5 cm) {$3$};
            \node at (2.5,1 cm) {$2$};
            \node at (2.5,1.5 cm) {$1$};

            \draw (2.75,0) -- (2.75, 2);
            \draw (3.25,0) -- (3.25, 2);
            \draw (2.75,0) -- (3.25,0);
            \draw (2.75, 2) -- (3.25, 2);
            \node at (3,0.5 cm) {$\infty$};
            \node at (3,1 cm) {$\frac{36}{11}$};
            \node at (3,1.5 cm) {$\infty$};

            \node at (3,2.25 cm) {$\cert$};

        \end{scope}
    \end{tikzpicture}
    \caption[Exemplo das estruturas utilizadas na lista ordenada]{Vetores $\sorted$ e $\cert$ para
        $\now = 0$ e $\now = 3$.
        O instante $t = \frac{36}{13}$ é quando as trajetórias dos elementos $3$ e $4$ se cruzam, e
        $t = \frac{36}{11}$ é quando as trajetórias dos elementos $1$ e $4$ se cruzam.}
    \label{fig:lista:variaveis}
\end{figure}



A interface da fila com prioridades que utilizaremos inclui as duas
seguintes operações:
\begin{enumerate}
    \item \textsc{minPQ}$(Q)$: devolve $i$ tal que
    \textit{cert}[$i$] é mínimo;
    \item \textsc{updatePQ}$(Q, i, t)$: altera a chave do
    $i$-ésimo certificado para $t$ e ajusta $Q$ de acordo.
\end{enumerate}

Note que não usaremos inserção ou remoção da fila com prioridades.

Para implementar a operação \textsc{change} de maneira eficiente, utilizaremos um vetor adicional
$\inds$ que guarda em $\inds[j]$ a posição em $\sorted$ do elemento $j$.
Utilizaremos $\textit{indS}$ pois, dado um elemento $j$, precisamos saber a posição $i$ do
elemento $j$ em $\sorted$ para recalcular os certificados relacionados com a posição $i$.
Para implementar a operação \textsc{updatePQ}$(Q, i, t)$ em tempo logarítmico no número de
elementos na fila $Q$, é necessário utilizar um vetor adicional \textit{indQ} que guarda em
\textit{indQ}$[i]$ a posição em $Q$ do $i$-ésimo certificado.

Com isso, a operação \textsc{advance}$(t)$, implementada no
Algoritmo~\ref{alg:lista-ordenada:advance}, segue uma ideia bem simples: enquanto
$t$ for maior que o prazo de validade do próximo evento, avançamos
\textit{now} para esse prazo de validade e tratamos esse evento.
Nos problemas seguintes, a operação \textsc{advance}$(t)$ será essencialmente a
mesma;
as únicas mudanças ocorrerão no tratamento de um evento.

\begin{algorithm}[H]
    \caption[Algoritmo \textsc{advance}]{Função \textsc{advance}.} \label{alg:lista-ordenada:advance}
\begin{algorithmic}[1]
    \Function{advance}{$t$}
        \If{$t < $ \now}
            \State \Return
        \EndIf
        \State $i \leftarrow \Call{minPQ}{$Q$}$
        \While{$t \geq$ \cert[$i$]}
            \State \now $~\leftarrow$ \cert[i]
            \State $\Call{event}$
            \State $i \leftarrow \Call{minPQ}{$Q$}$
        \EndWhile
        \State \now $~\leftarrow$ $t$
    \EndFunction
\end{algorithmic}
\end{algorithm}

Um evento está associado a um certificado $(i, t)$ que expira quando
$\now = t$.
O tratamento do evento correspondente ao certificado $(i, t)$ consiste em trocar de lugar os
índices das posições $i$ e $i + 1$ do vetor \textit{sorted}, recalcular o prazo de validade do
$(i-1)$-ésimo certificado se $i > 1$, e do $(i + 1)$-ésimo
certificado se $i < n - 1$.
O $i$-ésimo certificado também deve ser ajustado para $+\infty$.
Finalmente, é necessário fazer ajustes em $Q$, nas chaves dos certificados que sofreram
alteração.

Na implementação da operação \textsc{event}, utilizaremos a rotina
\textsc{update}$(i)$ para calcular o novo prazo de validade $t$ do
$i$-ésimo certificado, se $1 \leq i < n$, e fazer os devidos ajustes
em~$Q$.
Para calcular $t$, utilizaremos uma rotina chamada \textsc{expire}$(i,
j)$, que calcula o prazo de validade dos certificados entre os
elementos $i$ e $j$ no instante \now.
A rotina auxiliar \textsc{expire}$(i, j)$ não mudará para outros problemas, mantendo a
mesma definição.
As implementações estão nos Algoritmos~\ref{alg:lista-ordenada:update}
e~\ref{alg:lista-ordenada:evento} e a Figura~\ref{fig:lista:expire}
ilustra as atualizações feitas por essas rotinas.

\begin{algorithm}
    \caption{Função \textsc{update}.} \label{lista:update}
\begin{algorithmic}[1]
    \Function{update}{$i$}
        \If{$1 \leq i < n$}
            \State $t \leftarrow $ \Call{expire}{$i,i+1$}
            \State \Call{updatePQ}{$Q,i,t$}
        \EndIf
    \EndFunction
\end{algorithmic}
\end{algorithm}

\begin{algorithm}
    \caption{Função \textsc{event}.} \label{torneioi:evento}
    \begin{algorithmic}[1]
        \Function{event}{\nnull}
            \State $e \leftarrow  $ \Call{minPQ}{$Q$}
            \While{$e.\cert$ = \now}
                \State $j \leftarrow e.\lastmatch$
                \State $k \leftarrow 2\cdot \floor{\frac{j}{2}}
                + ((j + 1)\mod2)$ \Comment{adversário}
                \While{$j > 1$ \AND \Call{compare}{$j, k$}}
                    \State \torneio[$\floor{\frac{j}{2}}$]
                    $\leftarrow~$\torneio[$j$]
                    \State $\torneio[k].\lastmatch$ $\leftarrow k$
                    \State \Call{update}{$\torneio[k]$}
                    \State $j \leftarrow \floor{\frac{j}{2}}$
                    \State $k \leftarrow 2\cdot \floor{\frac{j}{2}}
                    + ((j + 1)\mod2)$ \Comment{adversário}
                \EndWhile
                \State $\torneio[j].\lastmatch \leftarrow j$
                \State \Call{update}{$\torneio[j]$}
                \State $e \leftarrow  $ \Call{minPQ}{$Q$}
            \EndWhile
        % \LineComment{swapHeap$(i, \floor{\frac{i}{2}})$ troca \heap[$i$] por \heap$\left[\floor{\frac{i}{2}}\right]$}
        \EndFunction
        \LineComment{\Call{compare}{$i, j$} retorna se o valor
        de $i$ é maior que o valor de $j$.}
    \end{algorithmic}
\end{algorithm}

\begin{figure}[H]
    \centering
    \begin{tikzpicture}[thick]
        \node at (0.5,0 cm) {$4$};
        \node at (0.5,0.5 cm) {$3$};
        \node at (0.5,1 cm) {$2$};
        \node at (0.5,1.5 cm) {\textbf{1}};

        \draw (0.75,-0.5) -- (0.75, 2);
        \draw (1.25,-0.5) -- (1.25, 2);
        \draw (0.75,-0.5) -- (1.25,-0.5);
        \draw (0.75, 2) -- (1.25, 2);
        \node at (1,0 cm) {$4$};
        \node at (1,0.5 cm) {$3$};
        \node at (1,1 cm) {$\textbf{2}$};
        \node at (1,1.5 cm) {$\textbf{1}$};

        \draw[->, shorten >= 0.1pt, shorten <= 0.1pt, >=stealth, line width=0.25mm]
        (1.25, 0) to[out=0,in=-30] node[right=1pt] {$\frac{36}{13}$} (1.25, 0.5);

        \draw[->, shorten >= 0.1pt, shorten <= 0.1pt, >=stealth, line width=0.25mm]
        (1.25, 0.5) to[out=0,in=-30] node[right=1pt] {$8$} (1.25, 1);

        \draw[->, shorten >= 0.1pt, shorten <= 0.1pt, >=stealth, line width=0.25mm]
        (1.25, 1) to[out=0,in=-30] node[right=1pt] {$2$} (1.25, 1.5);

        \node at (1,2.25 cm) {$\sorted$};

        \node at (2.5,0.5 cm) {$3$};
        \node at (2.5,1 cm) {$2$};
        \node at (2.5,1.5 cm) {$1$};

        \draw (2.75,0) -- (2.75, 2);
        \draw (3.25,0) -- (3.25, 2);
        \draw (2.75,0) -- (3.25,0);
        \draw (2.75, 2) -- (3.25, 2);
        \node at (3,0.5 cm) {$\frac{36}{13}$};
        \node at (3,1 cm) {$8$};
        \node at (3,1.5 cm) {$\textbf{2}$};

        \node at (3,2.25 cm) {$\cert$};

        \begin{scope}
            [shift={(5,0)}]
            \node at (0.5,0 cm) {$4$};
            \node at (0.5,0.5 cm) {$3$};
            \node at (0.5,1 cm) {$2$};
            \node at (0.5,1.5 cm) {$1$};

            \draw (0.75,-0.5) -- (0.75, 2);
            \draw (1.25,-0.5) -- (1.25, 2);
            \draw (0.75,-0.5) -- (1.25,-0.5);
            \draw (0.75, 2) -- (1.25, 2);
            \node at (1,0 cm) {$4$};
            \node at (1,0.5 cm) {$3$};
            \node at (1,1 cm) {$\textbf{1}$};
            \node at (1,1.5 cm) {$\textbf{2}$};

            \draw[->, shorten >= 0.1pt, shorten <= 0.1pt, >=stealth, line width=0.25mm]
            (1.25, 0) to[out=0,in=-30] node[right=1pt] {$\infty$} (1.25, 0.5);

            \draw[->, shorten >= 0.1pt, shorten <= 0.1pt, >=stealth, line width=0.25mm]
            (1.25, 0.5) to[out=0,in=-30] node[right=1pt] {$\frac{36}{11}$} (1.25, 1);

            \draw[->, shorten >= 0.1pt, shorten <= 0.1pt, >=stealth, line width=0.25mm]
            (1.25, 1) to[out=0,in=-30] node[right=1pt] {$\infty$} (1.25, 1.5);

            \node at (1,2.25 cm) {$\sorted$};

            \node at (2.5,0.5 cm) {$3$};
            \node at (2.5,1 cm) {$2$};
            \node at (2.5,1.5 cm) {$1$};

            \draw (2.75,0) -- (2.75, 2);
            \draw (3.25,0) -- (3.25, 2);
            \draw (2.75,0) -- (3.25,0);
            \draw (2.75, 2) -- (3.25, 2);
            \node at (3,0.5 cm) {$\frac{36}{13}$};
            \node at (3,1 cm) {$\bm{\frac{36}{11}}$};
            \node at (3,1.5 cm) {$\bm{\infty}$};

            \node at (3,2.25 cm) {$\cert$};

        \end{scope}
    \end{tikzpicture}
    \caption[Exemplo de expiração de certificado da lista ordenada]{No exemplo da
    Figura~\ref{fig:ordenacao:exemplo}, \cert[1] expirou no instante $\now = 2$, por isso
        $\sorted[1]$ e $\sorted[2]$ foram trocados e \cert[1] e \cert[2] foram atualizados.}
    \label{fig:lista:expire}
\end{figure}

A operação \textsc{query\_kth}$(i)$, implementada no Algoritmo~\ref{alg:lista:query}, consiste em
devolver \textit{sorted}$[i]$, enquanto que a operação \textsc{change}$(j, v)$ consiste em alterar
a posição $x_0[j]$ para $x_0[j] + (\textit{speed}[j] - v)\cdot now$,
a posição \textit{speed}[j] para \textit{v} e recalcular os
eventuais certificados de que $j$ participa.
O novo valor da posição $x_0[j]$ corresponde à posição inicial do elemento caso ele tivesse
começado com essa velocidade e estivesse na posição atual agora.
Além disso, a partir da posição $i$ em que $j$ se encontra no vetor
\textit{sorted}, podemos recalcular \textit{cert}$[i - 1]$ se $i >
1$ e \textit{cert}$[i]$ se $i < n$, como ilustrado na Figura~\ref{fig:lista:after}, acionando a
rotina \textsc{update} para fazer
os devidos acertos em~$Q$ correspondentes a estas modificações.
As instruções executadas pela operação \textsc{change} estão descritas
no Algoritmo~\ref{alg:lista-ordenada:change}.

\begin{algorithm}
    \caption{Função \textsc{query\_kth}.} \label{lista:query}
\begin{algorithmic}[1]
    \Function{query\_kth}{$i$}
        \If{$1 \leq i \leq n$}
            \State \Return \sorted[$i$]
        \EndIf
        \State \Return $-1$
    \EndFunction
\end{algorithmic}
\end{algorithm}

\begin{algorithm}
    \caption{Função \textsc{change}.} \label{torneioi:change}
    \begin{algorithmic}[1]
        \Function{change}{$j, v$}
            \State $e \leftarrow$ \Call{getObject}{$j$}
            \State $e.x_0 \leftarrow e.x_0+~(e.\speed -~v)~\cdot~\now$;
            \State $e.\speed \leftarrow v$
            \State $i \leftarrow e.\lastmatch$
            \State \Call{update}{$e$}
            \While{$i < n$}
                \If{$\torneio[i] = \torneio[2i]$}
                    \State $i \leftarrow 2i$
                \Else
                    \State $i \leftarrow 2i + 1$
                \EndIf
                \State $k \leftarrow 2\cdot \floor{\frac{i}{2}}
                + ((i + 1)\mod2)$ \Comment{adversário}
                \State \Call{update}{$\torneio[k]$}
            \EndWhile
        \EndFunction
    \end{algorithmic}
\end{algorithm}

\begin{figure}[H]
    \centering
    \begin{tikzpicture}[thick, scale=0.7]
        \node[label={1},circle,draw,minimum size=1cm]
        (1) at (0,0) {$9$};
        \node[label={2},circle,draw,minimum size=1cm]
        (2) at (-2,-2) {$5$};
        \node[label={3},circle,draw,minimum size=1cm]
        (3) at (2,-2) {$7$};
        \node[label={4},circle,draw,minimum size=1cm]
        (4) at (-3,-4) {$2$};
        \node[label={5},circle,draw,minimum size=1cm]
        (5) at (-1,-4) {$6$};
        \node[label={6},circle,draw,minimum size=1cm]
        (6) at (1,-4) {$3$};
        \node[label={7},circle,draw,minimum size=1cm]
        (7) at (3,-4) {$8$};
        \node[label={8},circle,draw,minimum size=1cm]
        (8) at (-4,-6) {$4$};
        \node[label={9},circle,draw,minimum size=1cm]
        (9) at (-2,-6) {$1$};

        \tikzstyle{cert}=[<-, dashdotted, blue, thick]
        \draw[thick] (1) -- (2);
        \draw[thick] (1) -- (3);
        \draw[cert] (2) -- (4);
        \draw[thick] (2) -- (5);
        \draw[thick] (3) -- (6);
        \draw[thick] (3) -- (7);
        \draw[cert] (4) -- (8);
        \draw[cert] (4) -- (9);
    \end{tikzpicture}
    \caption[Exemplo certificados do heap cinético após operação \textsc{change}]{Após a mudança de
    velocidade do elemento 2, que se encontra em \heap[$4$], os certificados
    \cert[$4$], \cert[$8$] e \cert[$9$] foram atualizados.}
    \label{fig:predeventheap}
\end{figure}

\subsection{Análise de desempenho}\label{subsec:analise-de-desempenho}

A lista ordenada cinética é uma estrutura \textit{responsiva}, pois o custo de
processar um certificado é exatamente o custo da rotina \textsc{event}, que é $O(\lg{n})$ pois a
rotina \textsc{update} consome $O(\lg{n})$ para atualizar a fila de prioridade dos certificados.

A lista ordenada cinética é uma estrutura \textit{eficiente}, pois todos os eventos
processados são eventos \textit{externos}, isto é, todo vencimento de
certificado representa a troca de ordem entre dois elementos na lista, que é uma
mudança na descrição combinatória do problema.

A lista ordenada cinética é uma estrutura \textit{compacta}, pois como cada
certificado está associado à relação de ordem entre um elemento e seu
predecessor, teremos no máximo $n-1$ certificados na fila de prioridades num
determinado instante.

A lista ordenada cinética é uma estrutura \textit{local}, pois cada elemento
está relacionado a no máximo dois certificados, o certificado entre ele e o
seu predecessor e o certificado entre o seu sucessor e ele.

%!TeX root=./ordenacao.tex

\section{Árvore binária balanceada de busca} \label{abb}

Manter um vetor ordenado é uma boa maneira de resolver o problema da
lista ordenada cinética dando suporte às operações
\textsc{advance}$(t)$, \textsc{change}$(j,v)$ e
\textsc{query\_kth}$(i)$. Poderíamos também querer dar suporte, além
das operações citadas, às seguintes operações:

\begin{itemize}
    \item \textsc{insert}$(v, x_t) \rightarrow$ insere um
    elemento com velocidade $v$ e valor $x_t$ no instante \now;
    \item \textsc{delete}$(i) \rightarrow$ remove o elemento
    $i$ no instante \now.
\end{itemize}

Para inserir um elemento no vetor ordenado, antes teríamos de
encontrar a posição que este deveria ocupar no vetor. Digamos que
seja a posição $j$. Após encontrar a posição, movemos todos os
elementos, a partir da posição $j$, uma posição à frente e colocamos
o elemento na posição~$j$. Após isso, os certificados de $j$~até~$n
- 1$ devem ser atualizados, pois esses elementos mudaram de posição
no vetor, e um novo certificado será criado, o $n$-ésimo
certificado, o total de elementos $n$ deve ser mudado para $n + 1$.
O novo certificado também deve ser inserido na fila com prioridades.

Só a operação de inserir um novo elemento no vetor já pode se tornar
pouco eficiente com uma grande quantidade de elementos sendo
inseridos no começo do vetor, consumindo tempo linear por inserção.
Como a remoção de um elemento no vetor ordenado envolve uma
sequência parecida de operações, da mesma maneira se torna pouco
eficiente, também consumindo tempo linear no pior caso.

Dessa forma, apesar da lista ordenada cinética implementada
manipulando um vetor ser uma estrutura eficiente para a operação
\textsc{query\_kth}$(i)$, com um consumo de tempo constante, o
consumo de tempo para as operações \textsc{insert}$(v, x_t)$ e
\textsc{delete}$(i)$ é, no pior caso, proporcional ao número de
elementos, o que pode ser ruim para uma grande quantidade de
elementos, inserções e remoções.

Podemos equilibrar o consumo de tempo das operações
\textsc{query\_kth}$(i)$, \textsc{insert}$(v, x_t)$ e
\textsc{delete}$(i)$ em tempo logarítmico no número de elementos,
usando uma ABBB (árvore binária balanceada de busca). Os pontos
serão armazenados na ABBB tendo o seu valor no instante \now~como
chave.

Além da ABBB, para garantirmos a eficiência das operações
\textsc{event}, \textsc{change}, \textsc{insert} e \textsc{delete},
cada elemento terá um apontador para o seu predecessor e um
apontador para o seu sucessor, formando uma lista duplamente ligada
ordenada pelo valor do elemento no instante \now; veja a Figura
\ref{fig:abb:exemplo}.

\begin{figure}[H]
    \centering
    \begin{tikzpicture}[thick, scale=0.7]
        \node[label={1},circle,draw,minimum size=1cm]
        (1) at (0,0) {$3$};
        \node[label={2},circle,draw,minimum size=1cm]
        (2) at (-4,-2) {$3$};
        \node[label={3},circle,draw,minimum size=1cm]
        (3) at (4,-2) {$4$};
        \node[label={4},circle,draw,minimum size=1cm]
        (4) at (-6,-4) {$1$};
        \node[label={5},circle,draw,minimum size=1cm]
        (5) at (-2,-4) {$3$};
        \node[label={6},circle,draw,minimum size=1cm]
        (6) at (2,-4) {$4$};
        \node[label={7},circle,draw,minimum size=1cm]
        (7) at (6,-4) {$6$};
        \node[label={8},circle,draw,minimum size=1cm]
        (8) at (-7,-6) {$8$};
        \node[label={9},circle,draw,minimum size=1cm]
        (9) at (-5,-6) {$1$};
        \node[label={10},circle,draw,minimum size=1cm]
        (10) at (-3,-6) {$2$};
        \node[label={11},circle,draw,minimum size=1cm]
        (11) at (-1,-6) {$3$};
        \node[label={12},circle,draw,minimum size=1cm]
        (12) at (1,-6) {$4$};
        \node[label={13},circle,draw,minimum size=1cm]
        (13) at (3,-6) {$5$};
        \node[label={14},circle,draw,minimum size=1cm]
        (14) at (5,-6) {$6$};
        \node[label={15},circle,draw,minimum size=1cm]
        (15) at (7,-6) {$7$};
        \node[label={16},circle,draw,minimum size=1cm]
        (16) at (-8,-8) {$8$};
        \node[label={17},circle,draw,minimum size=1cm]
        (17) at (-6,-8) {$9$};

        \draw[thick] (1) -- (2);
        \draw[thick] (2) -- (4);
        \draw[thick] (4) -- (8);
        \draw[thick] (4) -- (9);
        \draw[thick] (8) -- (16);
        \draw[thick] (8) -- (17);
        \draw[thick] (2) -- (5);
        \draw[thick] (5) -- (10);
        \draw[thick] (5) -- (11);
        \draw[thick] (1) -- (3);
        \draw[thick] (3) -- (6);
        \draw[thick] (3) -- (7);
        \draw[thick] (6) -- (12);
        \draw[thick] (6) -- (13);
        \draw[thick] (7) -- (14);
        \draw[thick] (7) -- (15);
    \end{tikzpicture}
    \caption[Representação da estrutura torneio]{Torneio com $9$
        elementos em que $3$ é o elemento com valor máximo.}
    \label{fig:torneio:exemplo}
\end{figure}

No que diz respeito aos certificados, antes um certificado estava
associado a uma posição e, no vetor, ao inserirmos um elemento em
uma determinada posição, teríamos que deslocar % que atualizar
todos os certificados conseguintes àquela posição. Agora, para que
consigamos alterar apenas uma quantidade constante de certificados
após uma inserção, os certificados não estarão mais associados a uma
posição e sim aos elementos.

O certificado $i$ se refere à relação estabelecida entre o elemento
$i$ e seu predecessor e consiste no instante de tempo em que o
elemento $i$ deixará de ter um valor maior que o valor do seu
predecessor, se esse instante for maior que o instante atual. Do
contrário, o certificado consiste em $+\infty$. Se o elemento $i$
não possui predecessor, então o certificado também consiste em
$+\infty$. Veja a Figura \ref{fig:abb:cert}.

Esses $n$ certificados também serão colocados em uma fila com
prioridades, com o prazo de validade determinando a prioridade.
A fila com prioridades agora também deverá suportar operações
como a inserção e remoção de certificados.

\begin{figure}[htb]
    \centering
    \begin{tikzpicture}[baseline=-2.25cm]
        \node[circle,draw,minimum size=1cm] (1) at (0,0)  {$2$};
        \node[circle,draw,minimum size=1cm] (2) at (-2,-2){$3$};
        \node[circle,draw,minimum size=1cm] (3) at (2,-2) {$4$};
        \node[circle,draw,minimum size=1cm] (4) at (-3,-4){$5$};
        \node[circle,draw,minimum size=1cm] (5) at (1,-4) {$1$};
        \node[circle,draw,minimum size=1cm] (6) at (3,-4) {$6$};
        \tikzstyle{filho}=[thick]
        \tikzstyle{pred}=[->, shorten >= 2pt, shorten <= 2pt,
        dashed, >=stealth, red]
        \tikzstyle{sucessor}=[->, shorten >= 2pt, shorten <= 2pt,
        dotted, >=stealth, line width=0.35mm]
        \draw[filho] (1) -- (2);
        \draw[filho] (1) -- (3);
        \draw[filho] (2) -- (4);
        \draw[filho] (3) -- (5);
        \draw[filho] (3) -- (6);
        \draw[pred] (6) edge[out=30,in=0] node[above=10pt, red] {$2$} (3);
        % \node[red] at (4,-3) {$2$};
        \draw[pred] (3) edge[out=190,in=120]
        node[above=10pt, red] {$+\infty$} (5);
        \draw[pred] (5) edge[out=170,in=285]
        node[left=5pt, red] {$1$} (1);
        \draw[pred] (1) edge[out=180,in=120]
        node[above=5pt, red] {$+\infty$} (2);
        \draw[pred] (2) edge[out=180,in=120]
        node[above=10pt, red] {$2$} (4);
        \draw[pred] (4) edge[out=180,in=40]
        node[above=5pt, red] {$+\infty$} (-4.5, -4);
    \end{tikzpicture}
    \qquad
    \qquad
    \qquad
    \begin{tabular}{|c|c|c|c|}
        \hline
        $i$ & $x_0$ & $v$   & $\cert[i]$ \\
        \hline
        $1$ & $6$   & $2$   & $1$        \\

        $2$ & $3$   & $5$   & $+\infty$  \\

        $3$ & $2$   & $1$   & $2$        \\

        $4$ & $7$   & $4$   & $+\infty$  \\

        $5$ & $-2$  & $3$   & $+\infty$  \\

        $6$ & $14$  & $0.5$ & $2$        \\
        \hline
    \end{tabular}
    \caption[Representação dos certificados da ABB]{Certificados
    representados pelas setas vermelhas tracejadas. O
    elemento $5$ é o último da lista e o seu certificado vale $+\infty$.}
    \label{fig:abb:cert}
\end{figure}

Para descrever as implementações das operações, vamos
estabelecer os nomes dos objetos, variáveis e rotinas
auxiliares utilizados:
\begin{enumerate}
    \item $n$: número de elementos no instante \now;
    \item \no: objeto que compõe a árvore binária balanceada
            de busca, com atributos:
    \begin{enumerate}
        \item \esq$:$ aponta para a raiz da subárvore
        esquerda do nó;
        \item \dir$:$ aponta para a raiz da subárvore
        direita do nó;
        \item \textit{key}$:$ aponta para um elemento;
        \item \children$:$ quantidade de nós que a subárvore
        enraizada neste nó possui. Este atributo será importante
        para a operação \textsc{query\_kth}$(i)$;
    \end{enumerate}
    \item \raiz: nó que é a raiz da árvore binária balanceada de
                busca;
    \item \elemento: objeto com os seguintes atributos:
    \begin{enumerate}
        \item \id: vem de \textit{identifier} e é o atributo
        para identificar o objeto. Assim, daqui
        em diante, usaremos elemento $i$ para nos
        referirmos ao elemento cujo \id~é $i$;
        \item \speed: velocidade do elemento;
        \item \initv: valor que o elemento possuía no
        instante~$t = 0$;
        \item \nex: apontador para o elemento imediatamente
        posterior a este na coleção, no instante \now. O
        elemento imediatamente posterior a $i$ é aquele
        que possui o menor valor dentre a coleção de
        elementos que possuem valor maior que o elemento
        $i$;
        \item \prev: apontador para o elemento imediatamente
        anterior a este na coleção, no instante \now;
        \item \pqpos: vem de \textit{priority queue position} e
        aponta para a posição do certificado associado
        ao elemento na fila com prioridades;
        \item \cert: vem de \textit{certificate} e é o prazo de
        validade do certificado entre este elemento e o elemento
        apontado por \prev; se \prev~não aponta para ninguém,
        \cert~vale $+\infty$;
        \item \no: apontador para o nó da árvore binária de busca em
        que o elemento se encontra;
    \end{enumerate}
    \item \Q: fila com prioridades que contém os elementos; o
    elemento com certificado de menor valor estará à frente da fila;

    \item \textsc{insertKey}$(\text{\raiz},e)\rightarrow$ insere
    $e$, um elemento, na árvore binária balanceada de busca com raiz
    \raiz~e retorna a, possivelmente nova, raiz da árvore. No
    processo também atualiza a lista ligada de elementos;

    \item \textsc{deleteKey}$(\text{\raiz},e)\rightarrow$ remove
    $e$, um elemento, da árvore binária balanceada de busca com raiz
    \raiz~e retorna a, possivelmente nova, raiz da árvore. No
    processo também atualiza a lista ligada de elementos.

\end{enumerate}
Para a implementação das operações \textsc{change}$(j, v)$ e
\textsc{delete}$(i)$, precisamos de alguma maneira recuperar um
elemento baseado no seu \id. Para tal, podemos utilizar uma tabela
de símbolos, implementada por uma árvore binária balanceada de busca
ou uma tabela de dispersão. A seguir~estão três operações que nos
ajudarão a recuperar os elementos:

\begin{enumerate}
    \item \textsc{getObject}$(i)\rightarrow$ retorna o elemento $i$;
    \item \textsc{insertObject}$(e) \rightarrow$ insere $e$,
    que é um elemento, na estrutura;
    \item \textsc{deleteObject}$(e) \rightarrow$ remove $e$,
    que é um elemento, da estrutura.
\end{enumerate}

Para permitir a inserção e remoção de certificados, a interface da
fila com prioridades será reformulada, contando com duas operações
extras:

\begin{enumerate}
    \item \textsc{insertPQ}$(Q, e) \rightarrow$ insere $(e, t)$
    na fila com prioridades $Q$;
    \item \textsc{deletePQ}$(Q, e) \rightarrow$ remove $e$
    da fila com prioridades $Q$;
    \item \textsc{updatePQ}$(Q,e,t) \rightarrow$ muda o prazo de
    validade do certificado de $e$ para $t$ e atualiza a fila com
    prioridades $Q$;
    \item \textsc{minPQ}$(Q) \rightarrow$ devolve o elemento com o
    certificado de menor prazo de validade da fila com prioridades
    $Q$.
\end{enumerate}

A operação \textsc{updatePQ}$(Q,e,t)$ pode ser implementada de modo
a consumir tempo logarítmico no número de elementos em $Q$ graças ao
atributo \pqpos~dos elementos.

Um evento está associado a um par $(e, t)$ que corresponde ao
certificado do elemento $e$ que expira no instante $t$. O tratamento
do evento correspondente a esse par $(e, t)$ consiste em trocar de
lugar o elemento $e$ e seu predecessor, digamos $e'$, na árvore
binária de busca e na lista ligada, e recalcular o prazo de validade
de até três certificados:

\begin{itemize}
    \item do certificado de $e$;
    \item do certificado de $e'$;
    \item do certificado do novo sucessor de $e'$, caso não seja \textsc{null}.
\end{itemize}

Na implementação da operação \textsc{event}, no Algoritmo
\ref{abb:evento}, utilizaremos a rotina $\textsc{update}(e)$, no
Algoritmo \ref{abb:update}, que calcula o novo prazo de validade $t$
do certificado do elemento $e$, e chama a
rotina~$\textsc{updatePQ}(Q, e, t)$. A função $\textsc{swap}(e_1,
e_2)$ troca a posição de $e_1$ e $e_2$ na árvore binária balanceada
de busca e na lista ligada e a função \Call{expire}{$e,e'$} calcula
a validade do certificado entre os elementos $e$ e $e'$; se $e'$ é
\textsc{null} retorna $+\infty$.

\begin{algorithm}
    \caption{Função \textsc{update}.} \label{lista:update}
\begin{algorithmic}[1]
    \Function{update}{$i$}
        \If{$1 \leq i < n$}
            \State $t \leftarrow $ \Call{expire}{$i,i+1$}
            \State \Call{updatePQ}{$Q,i,t$}
        \EndIf
    \EndFunction
\end{algorithmic}
\end{algorithm}

\begin{algorithm}
    \caption{Função \textsc{event}.} \label{torneioi:evento}
    \begin{algorithmic}[1]
        \Function{event}{\nnull}
            \State $e \leftarrow  $ \Call{minPQ}{$Q$}
            \While{$e.\cert$ = \now}
                \State $j \leftarrow e.\lastmatch$
                \State $k \leftarrow 2\cdot \floor{\frac{j}{2}}
                + ((j + 1)\mod2)$ \Comment{adversário}
                \While{$j > 1$ \AND \Call{compare}{$j, k$}}
                    \State \torneio[$\floor{\frac{j}{2}}$]
                    $\leftarrow~$\torneio[$j$]
                    \State $\torneio[k].\lastmatch$ $\leftarrow k$
                    \State \Call{update}{$\torneio[k]$}
                    \State $j \leftarrow \floor{\frac{j}{2}}$
                    \State $k \leftarrow 2\cdot \floor{\frac{j}{2}}
                    + ((j + 1)\mod2)$ \Comment{adversário}
                \EndWhile
                \State $\torneio[j].\lastmatch \leftarrow j$
                \State \Call{update}{$\torneio[j]$}
                \State $e \leftarrow  $ \Call{minPQ}{$Q$}
            \EndWhile
        % \LineComment{swapHeap$(i, \floor{\frac{i}{2}})$ troca \heap[$i$] por \heap$\left[\floor{\frac{i}{2}}\right]$}
        \EndFunction
        \LineComment{\Call{compare}{$i, j$} retorna se o valor
        de $i$ é maior que o valor de $j$.}
    \end{algorithmic}
\end{algorithm}

\begin{figure}
    \centering
    \begin{tikzpicture}[thick, scale=0.8]
        \node[label={1},circle,draw,minimum size=1cm]
            (1) at (0,0) {$3$};
        \node[label={2},circle,draw,minimum size=1cm]
            (2) at (-4,-2) {$3$};
        \node[label={3},circle,draw,minimum size=1cm]
            (3) at (4,-2) {$4$};
        \node[label={4},circle,draw,minimum size=1cm]
            (4) at (-6,-4) {$1$};
        \node[label={5},circle,draw,minimum size=1cm]
            (5) at (-2,-4) {$3$};
        \node[label={6},circle,draw,minimum size=1cm]
            (6) at (2,-4) {$4$};
        \node[label={7},circle,draw,minimum size=1cm]
            (7) at (6,-4) {$6$};
        \node[label={8},circle,draw,minimum size=1cm]
            (8) at (-7,-6) {$8$};
        \node[label={9},circle,draw,minimum size=1cm]
            (9) at (-5,-6) {$1$};
        \node[label={10},circle,draw,minimum size=1cm]
            (10) at (-3,-6) {$2$};
        \node[label={11},circle,draw,minimum size=1cm]
            (11) at (-1,-6) {$3$};
        \node[label={12},circle,draw,minimum size=1cm]
            (12) at (1,-6) {$4$};
        \node[label={13},circle,draw,minimum size=1cm]
            (13) at (3,-6) {$5$};
        \node[label={14},circle,draw,minimum size=1cm]
            (14) at (5,-6) {$6$};
        \node[label={15},circle,draw,minimum size=1cm]
            (15) at (7,-6) {$7$};
        \node[label={16},circle,draw,minimum size=1cm]
            (16) at (-8,-8) {$8$};
        \node[label={17},circle,draw,minimum size=1cm]
            (17) at (-6,-8) {$9$};

        \draw[thick] (1) -- (2);
        \draw[<-,line width=\thickness, red] (2) -- (4);
        \draw[<-,thick, dashed, red] (4) -- (8);
        \draw[thick] (4) -- (9);
        \draw[thick] (8) -- (16);
        \draw[<-,line width=\thickness, red] (8) -- (17);
        \draw[thick] (2) -- (5);
        \draw[<-,line width=\thickness, red] (5) -- (10);
        \draw[thick] (5) -- (11);
        \draw[<-,line width=\thickness, red] (1) -- (3);
        \draw[thick] (3) -- (6);
        \draw[<-,line width=\thickness, red] (3) -- (7);
        \draw[thick] (6) -- (12);
        \draw[<-,line width=\thickness, red] (6) -- (13);
        \draw[thick] (7) -- (14);
        \draw[<-,line width=\thickness, red] (7) -- (15);
    \end{tikzpicture}
    \caption[Representação de certificado expirado]{cert[$8$] expirou.}
    \label{fig:torneio:evento}
\end{figure}

\begin{figure}[H]
    \centering
    \begin{tikzpicture}[thick, scale=0.7]
        \node[label={1},circle,draw,minimum size=1cm]
        (1) at (0,0) {$9$};
        \node[label={2},circle,draw,minimum size=1cm]
        (2) at (-2,-2) {$5$};
        \node[label={3},circle,draw,minimum size=1cm]
        (3) at (2,-2) {$7$};
        \node[label={4},circle,draw,minimum size=1cm]
        (4) at (-3,-4) {$2$};
        \node[label={5},circle,draw,minimum size=1cm]
        (5) at (-1,-4) {$6$};
        \node[label={6},circle,draw,minimum size=1cm]
        (6) at (1,-4) {$3$};
        \node[label={7},circle,draw,minimum size=1cm]
        (7) at (3,-4) {$8$};
        \node[label={8},circle,draw,minimum size=1cm]
        (8) at (-4,-6) {$4$};
        \node[label={9},circle,draw,minimum size=1cm]
        (9) at (-2,-6) {$1$};

        \tikzstyle{cert}=[<-, dashdotted, blue, thick]
        \draw[thick] (1) -- (2);
        \draw[thick] (1) -- (3);
        \draw[cert] (2) -- (4);
        \draw[thick] (2) -- (5);
        \draw[thick] (3) -- (6);
        \draw[thick] (3) -- (7);
        \draw[cert] (4) -- (8);
        \draw[cert] (4) -- (9);
    \end{tikzpicture}
    \caption[Exemplo certificados do heap cinético após operação \textsc{change}]{Após a mudança de
    velocidade do elemento 2, que se encontra em \heap[$4$], os certificados
    \cert[$4$], \cert[$8$] e \cert[$9$] foram atualizados.}
    \label{fig:predeventheap}
\end{figure}

A operação \textsc{query\_kth}$(i)$ consiste em devolver o $i$-ésimo
maior elemento da lista ligada, ou seja, o $i$-ésimo da direita para
a esquerda, pois a árvore está em ordem crescente da esquerda para a
direita. Para tal, percorreremos a árvore binária balanceada de
busca utilizando o atributo $\children$ para, a cada iteração,
decidir em qual subárvore o $i$-ésimo está, ajustando $i$ quando
necessário. O Algoritmo \ref{abb:query} implementa esta operação e a
Figura \ref{fig:abb:queryexecution} simula a execução em um exemplo.
A rotina auxiliar \textsc{rsize}$(r)$ devolve o valor de
$r.right.\textit{size}$ caso este seja não nulo, caso contrário
devolve $0$.


% Estando em um determinado nó \no~da árvore, sabemos que todos os nós
% na subárvore direita tem valor maior que o nó atual e que os nós da
% subárvore esquerda. Portanto, se $i \leq node.right.\children$, a
% nossa resposta com certeza está na subárvore direita do nó. Caso
% contrário temos duas opções: \no~é a resposta ou a resposta está na
% subárvore esquerda. Para checar se \no~é a resposta, devemos
% perceber que \no~tem valor maior que os nós de sua subárvore
% esquerda, então se $i = node.right.\children + 1$, \no~é a resposta.
% Se $i > node.right.\children + 1$, então a nossa resposta está na
% subárvore esquerda e queremos o $[i - (node.right.\children +
% 1)]$-ésimo elemento da subárvore esquerda. Podemos repetir esse
% processo até encontrar a nossa resposta. O Algoritmo \ref{abb:query}
% utiliza a rotina auxiliar \textsc{rsize}$(r)$, que devolve o valor
% de \textit{size} de $r.right$ caso este seja não nulo, caso
% contrário devolve $0$.
% Se, dada uma raiz, soubermos a quantidade de filhos
% direitos,~\raiz$.rsize$, comparamos com o valor de $i$ e podemos
% ter três conclusões: se~$i > root.rsize$, significa que o
% $i$-ésimo elemento com certeza não está na subárvore direita, pois
% todos elementos da subárvore direita estão a frente de \raiz~e da
% subárvore esquerda. Nesse caso, calculamos $i - root.rsize$, se
% esse valor é igual a $1$, então \raiz~é o $i-ésimo$ elemento da
% coleção, pois \raiz~é o próximo elemento após todos elementos na
% subárvore direita. Se $root.rsize - i \neq 1$, então devemos

\begin{algorithm}
    \caption{Função \textsc{query\_kth}.} \label{lista:query}
\begin{algorithmic}[1]
    \Function{query\_kth}{$i$}
        \If{$1 \leq i \leq n$}
            \State \Return \sorted[$i$]
        \EndIf
        \State \Return $-1$
    \EndFunction
\end{algorithmic}
\end{algorithm}

\begin{figure}[H]
    \centering
    \begin{tikzpicture}[thick,scale=0.7]
        \node[label=280:{14},circle,draw,minimum size=1cm]
        (1) at (0,0) {$2$};
        \node[label=280:{6},circle,draw,minimum size=1cm]
        (2) at (-4,-2) {$1$};
        \node[label=280:{7},circle,draw,minimum size=1cm]
        (3) at (4,-2) {$4$};
        \node[label=320:{3},circle,draw,minimum size=1cm]
        (4) at (-6,-4) {$3$};
        \node[label=280:{2},circle,draw,minimum size=1cm]
        (5) at (-2,-4) {$11$};
        \node[label=320:{3},circle,draw,minimum size=1cm]
        (6) at (2,-4) {$14$};
        \node[label=320:{3},circle,draw,minimum size=1cm]
        (7) at (6,-4) {$13$};
        \node[label=280:{1},circle,draw,minimum size=1cm]
        (8) at (-7,-6) {$8$};
        \node[label=280:{1},circle,draw,minimum size=1cm]
        (9) at (-5,-6) {$7$};
        \node[label=280:{1},circle,draw,minimum size=1cm]
        (10) at (-3,-6) {$5$};
        % \node[label={11},circle,draw,minimum size=1cm] (11) at (-1,-6) {$3$};
        \node[label=280:{1},circle,draw,minimum size=1cm]
        (12) at (1,-6) {$6$};
        \node[label=280:{1},circle,draw,minimum size=1cm]
        (13) at (3,-6) {$12$};
        \node[label=280:{1},circle,draw,minimum size=1cm]
        (14) at (5,-6) {$9$};
        \node[label=280:{1},circle,draw,minimum size=1cm]
        (15) at (7,-6) {$10$};

        \draw[thick] (1) -- (2);
        \draw[thick] (2) -- (4);
        \draw[thick] (4) -- (8);
        \draw[thick] (4) -- (9);
        \draw[thick] (2) -- (5);
        \draw[thick] (5) -- (10);
        % \draw[thick] (5) -- (11);
        \draw[thick] (1) -- (3);
        \draw[thick] (3) -- (6);
        \draw[thick] (3) -- (7);
        \draw[thick] (6) -- (12);
        \draw[thick] (6) -- (13);
        \draw[thick] (7) -- (14);
        \draw[thick] (7) -- (15);
    \end{tikzpicture}
    \caption[Exemplo de árvore binária de busca com campo \children]{Exemplo de ABB com campo \children.}
    \label{fig:abb:query}
\end{figure}

\begin{figure}[H]
    \centering
    \begin{tikzpicture}[thick,scale=0.7]
        \tikzstyle{triangle} = [regular polygon, regular polygon sides=3]
        \node[
        label=280:{14},
        label=80:{$i = 7 \leq 7$},
        very thick,
        circle,
        draw,
        minimum size=1cm]
        (1) at (0,0) {$2$};
        \node[label=280:{6},triangle,draw,minimum size=1cm]
        (2) at (-4,-2) {};

        \node[label=280:{7},
        label=80:{$i = 7 > 3$},
        very thick,
        circle,draw,minimum size=1cm]
        (3) at (4,-2) {$4$};
        % \node[label=320:{3},circle,draw,minimum size=1cm]
        %     (4) at (-6,-4) {$3$};
        % \node[label=280:{2},circle,draw,minimum size=1cm]
        %     (5) at (-2,-4) {$11$};
        \node[label=320:{3},
        label=120:{$i = 3 > 1$},
        very thick,
        circle,draw,minimum size=1cm]
        (6) at (2,-4) {$14$};

        \node[label=320:{3},triangle,draw,minimum size=1cm]
        (7) at (6,-4) {};
        % \node[label=280:{1},circle,draw,minimum size=1cm]
        %     (8) at (-7,-6) {$8$};
        % \node[label=280:{1},circle,draw,minimum size=1cm]
        %     (9) at (-5,-6) {$7$};
        % \node[label=280:{1},circle,draw,minimum size=1cm]
        %     (10) at (-3,-6) {$5$};
        % \node[label={11},circle,draw,minimum size=1cm] (11) at (-1,-6) {$3$};
        \node[label=280:{1},
        label=120:{$i = 1 = 0 + 1$},
        very thick,
        circle,draw,minimum size=1cm]
        (12) at (1,-6) {$6$};

        \node[label=280:{1},triangle,draw,minimum size=1cm]
        (13) at (3,-6) {};
        % \node[label=280:{1},circle,draw,minimum size=1cm]
        %     (14) at (5,-6) {$9$};
        % \node[label=280:{1},circle,draw,minimum size=1cm]
        %     (15) at (7,-6) {$10$};

        \draw[thick] (1) -- (2);
        % \draw[thick] (2) -- (4);
        % \draw[thick] (4) -- (8);
        % \draw[thick] (4) -- (9);
        % \draw[thick] (2) -- (5);
        % \draw[thick] (5) -- (10);
        % \draw[thick] (5) -- (11);
        \draw[very thick] (1) -- node[above, sloped] {$i = 7$} (3);
        \draw[very thick] (3) -- node[above, sloped] {$i = 3$} (6);
        \draw[thick] (3) -- (7);
        \draw[very thick] (6) -- (12);% node[above, sloped] {$i = 1$} (12);
        \draw[thick] (6) -- (13);
        % \draw[thick] (7) -- (14);
        % \draw[thick] (7) -- (15);
    \end{tikzpicture}
    \caption[Exemplo de busca pelo $i$-ésimo]{Exemplo de busca pelo
    $7^\circ$ elemento da direita para a esquerda na árvore da
    Figura \ref{fig:abb:query}.}
    \label{fig:abb:queryexecution}
\end{figure}

A operação \textsc{change}$(j, v)$, como implementada no Algoritmo
\ref{abb:change}, consiste em recuperar o elemento $e$ com
identificador $j$, alterar seu atributo \initv~para $x_0 +
(\mathit{speed} - v)\cdot now$, \textit{speed} para \textit{v} e
recalcular os eventuais certificados de que $j$ participa, que
seriam $e.cert$ e $e.next.cert$, se $e.next$ existe. A Figura
\ref{fig:abb:change} ilustra um exemplo com os elementos afetados.

\begin{algorithm}
    \caption{Função \textsc{change}.} \label{torneioi:change}
    \begin{algorithmic}[1]
        \Function{change}{$j, v$}
            \State $e \leftarrow$ \Call{getObject}{$j$}
            \State $e.x_0 \leftarrow e.x_0+~(e.\speed -~v)~\cdot~\now$;
            \State $e.\speed \leftarrow v$
            \State $i \leftarrow e.\lastmatch$
            \State \Call{update}{$e$}
            \While{$i < n$}
                \If{$\torneio[i] = \torneio[2i]$}
                    \State $i \leftarrow 2i$
                \Else
                    \State $i \leftarrow 2i + 1$
                \EndIf
                \State $k \leftarrow 2\cdot \floor{\frac{i}{2}}
                + ((i + 1)\mod2)$ \Comment{adversário}
                \State \Call{update}{$\torneio[k]$}
            \EndWhile
        \EndFunction
    \end{algorithmic}
\end{algorithm}

\begin{algorithm}
    \caption{Função \textsc{change}.} \label{torneioi:change}
    \begin{algorithmic}[1]
        \Function{change}{$j, v$}
            \State $e \leftarrow$ \Call{getObject}{$j$}
            \State $e.x_0 \leftarrow e.x_0+~(e.\speed -~v)~\cdot~\now$;
            \State $e.\speed \leftarrow v$
            \State $i \leftarrow e.\lastmatch$
            \State \Call{update}{$e$}
            \While{$i < n$}
                \If{$\torneio[i] = \torneio[2i]$}
                    \State $i \leftarrow 2i$
                \Else
                    \State $i \leftarrow 2i + 1$
                \EndIf
                \State $k \leftarrow 2\cdot \floor{\frac{i}{2}}
                + ((i + 1)\mod2)$ \Comment{adversário}
                \State \Call{update}{$\torneio[k]$}
            \EndWhile
        \EndFunction
    \end{algorithmic}
\end{algorithm}

\begin{figure}[H]
    \centering
    \begin{tikzpicture}[baseline=-2cm,scale=0.8]
        \node[circle,draw,minimum size=1cm] (1) at (0,0)  {$1$};
        \node[circle,draw,minimum size=1cm] (2) at (-2,-2){$3$};
        \node[circle,draw,minimum size=1cm] (3) at (2,-2) {$4$};
        \node[circle,draw,minimum size=1cm] (4) at (-3,-4){$5$};
        \node[circle,draw,minimum size=1cm] (5) at (1,-4) {$2$};
        \node[circle,draw,minimum size=1cm] (6) at (3,-4) {$6$};
        \node[circle,draw,minimum size=1cm] (7) at (-1,-4){$7$};
        % \node[label={7},circle,draw,minimum size=1cm] (7) at (3,-4) {$8$};
        % \node[label={9},circle,draw,minimum size=1cm] (9) at (-2,-6) {$1$};
        \tikzstyle{filho}=[thick]
        \tikzstyle{pred}=[->, shorten >= 2pt, shorten <= 2pt,
        dashed, >=stealth]
        \tikzstyle{sucessor}=[->, shorten >= 2pt, shorten <= 2pt,
        dotted, >=stealth, line width=0.35mm]
        % \tikzstyle{p4}=[->, shorten >= 2pt, shorten <= 2pt, dotted, >=stealth]
        \draw[filho] (1) -- (2);
        \draw[filho] (1) -- (3);
        \draw[filho] (2) -- (4);
        \draw[filho] (2) -- (7);
        \draw[filho] (3) -- (5);
        \draw[filho] (3) -- (6);
        \draw[pred] (6) edge[out=30,in=0]
        node[above=10pt] {$2$} (3);
        \draw[sucessor] (3) edge[out=280,in=180] (6);
        \draw[pred] (3) edge[out=190,in=100]
        node[above=10pt] {$4$} (5);
        \draw[sucessor] (5) edge[out=0,in=260] (3);
        \draw[pred] (5) edge[out=120,in=300]
        node[above=10pt] {$+\infty$} (1);
        \draw[sucessor] (1) edge[out=290,in=150] (5);
        \draw[pred, loosely dashed, red] (7) edge[out=180,in=280]
        node[red, above=10pt] {$+\infty$} (2);
        \draw[sucessor] (2) edge[out=260,in=200] (7);
        \draw[pred, blue, dashdotted] (1) edge[out=235,in=60]
        node[blue, above=10pt] {$+\infty$} (7);
        \draw[sucessor] (7) edge[out=45,in=250] (1);
        \draw[pred] (2) edge[out=180,in=120]
        node[above=10pt] {$2$} (4);
        \draw[sucessor] (4) edge[out=100,in=200] (2);
        \draw[pred] (4) edge[out=180,in=40]
        node[above=10pt] {$+\infty$} (-4.5, -4);
    \end{tikzpicture}
    \begin{tabular}{|c|c|c|c|c|}
        \hline
        &       &       & $\now = 1$                  \\
        $i$ & $x_0$ & $v$   & $\cert[i]$                  \\
        \hline
        $1$ & $6$   & $2$   & \textcolor{blue}{$+\infty$} \\

        $2$ & $3$   & $5$   & $+\infty$                   \\

        $3$ & $2$   & $1$   & $2$                         \\

        $4$ & $7$   & $4$   & $4$                         \\

        $5$ & $-2$  & $3$   & $+\infty$                   \\

        $6$ & $14$  & $0.5$ & $2$                         \\

        $7$ & $3$   & $1$   & \textcolor{red}{$+\infty$}  \\
        \hline
    \end{tabular}
    \caption[ABB após chamar \textsc{insert}]{Após chamar
        {\normalfont \textsc{insert}$(1, 4)$}, no instante $1$, o elemento $7$ foi
    inserido na árvore.
    O certificado do elemento $7$ foi criado e o
    certificado do seu sucessor, o elemento $1$, atualizado.}
    \label{fig:abb:insert}
\end{figure}

A operação \textsc{insert}$(v, x_t)$, como ilustrado na Figura
\ref{fig:abb:insert}, consiste em criar um novo elemento,
inicializando seus atributos com os devidos valores, inseri-lo na
árvore binária balanceada de busca e na estrutura que usamos para
recuperá-lo depois, calcular o seu certificado e inseri-lo na fila
com prioridades e, por fim, atualizar o certificado de seu sucessor,
caso exista. Uma importante observação é que se \now~$\neq 0$, então
$x_t \neq$~\initv. Para calcular \initv, podemos utilizar a relação
${x_t = now\cdot speed + x_0}$, que implica que ${x_0 = x_t -
speed\cdot now}$. O Algoritmo \ref{abb:insert} implementa esta
operação.

\begin{figure}[H]
    \centering
    \begin{tikzpicture}[baseline=-2cm,scale=0.8]
        \node[circle,draw,minimum size=1cm] (1) at (0,0)  {$1$};
        \node[circle,draw,minimum size=1cm] (2) at (-2,-2){$3$};
        \node[circle,draw,minimum size=1cm] (3) at (2,-2) {$4$};
        \node[circle,draw,minimum size=1cm] (4) at (-3,-4){$5$};
        \node[circle,draw,minimum size=1cm] (5) at (1,-4) {$2$};
        \node[circle,draw,minimum size=1cm] (6) at (3,-4) {$6$};
        \node[circle,draw,minimum size=1cm] (7) at (-1,-4){$7$};
        % \node[label={7},circle,draw,minimum size=1cm] (7) at (3,-4) {$8$};
        % \node[label={9},circle,draw,minimum size=1cm] (9) at (-2,-6) {$1$};
        \tikzstyle{filho}=[thick]
        \tikzstyle{pred}=[->, shorten >= 2pt, shorten <= 2pt,
        dashed, >=stealth]
        \tikzstyle{sucessor}=[->, shorten >= 2pt, shorten <= 2pt,
        dotted, >=stealth, line width=0.35mm]
        % \tikzstyle{p4}=[->, shorten >= 2pt, shorten <= 2pt, dotted, >=stealth]
        \draw[filho] (1) -- (2);
        \draw[filho] (1) -- (3);
        \draw[filho] (2) -- (4);
        \draw[filho] (2) -- (7);
        \draw[filho] (3) -- (5);
        \draw[filho] (3) -- (6);
        \draw[pred] (6) edge[out=30,in=0]
        node[above=10pt] {$2$} (3);
        \draw[sucessor] (3) edge[out=280,in=180] (6);
        \draw[pred] (3) edge[out=190,in=100]
        node[above=10pt] {$4$} (5);
        \draw[sucessor] (5) edge[out=0,in=260] (3);
        \draw[pred] (5) edge[out=120,in=300]
        node[above=10pt] {$+\infty$} (1);
        \draw[sucessor] (1) edge[out=290,in=150] (5);
        \draw[pred, loosely dashed, red] (7) edge[out=180,in=280]
        node[red, above=10pt] {$+\infty$} (2);
        \draw[sucessor] (2) edge[out=260,in=200] (7);
        \draw[pred, blue, dashdotted] (1) edge[out=235,in=60]
        node[blue, above=10pt] {$+\infty$} (7);
        \draw[sucessor] (7) edge[out=45,in=250] (1);
        \draw[pred] (2) edge[out=180,in=120]
        node[above=10pt] {$2$} (4);
        \draw[sucessor] (4) edge[out=100,in=200] (2);
        \draw[pred] (4) edge[out=180,in=40]
        node[above=10pt] {$+\infty$} (-4.5, -4);
    \end{tikzpicture}
    \begin{tabular}{|c|c|c|c|c|}
        \hline
        &       &       & $\now = 1$                  \\
        $i$ & $x_0$ & $v$   & $\cert[i]$                  \\
        \hline
        $1$ & $6$   & $2$   & \textcolor{blue}{$+\infty$} \\

        $2$ & $3$   & $5$   & $+\infty$                   \\

        $3$ & $2$   & $1$   & $2$                         \\

        $4$ & $7$   & $4$   & $4$                         \\

        $5$ & $-2$  & $3$   & $+\infty$                   \\

        $6$ & $14$  & $0.5$ & $2$                         \\

        $7$ & $3$   & $1$   & \textcolor{red}{$+\infty$}  \\
        \hline
    \end{tabular}
    \caption[ABB após chamar \textsc{insert}]{Após chamar
        {\normalfont \textsc{insert}$(1, 4)$}, no instante $1$, o elemento $7$ foi
    inserido na árvore.
    O certificado do elemento $7$ foi criado e o
    certificado do seu sucessor, o elemento $1$, atualizado.}
    \label{fig:abb:insert}
\end{figure}

\begin{figure}[htb]
    \centering
    \begin{tikzpicture}[baseline=-2.25cm]
        \node[circle,draw,minimum size=1cm] (1) at (0,0)  {$1$};
        \node[circle,draw,minimum size=1cm] (2) at (-2,-2){$3$};
        \node[circle,draw,minimum size=1cm] (3) at (2,-2) {$4$};
        \node[circle,draw,minimum size=1cm] (4) at (-3,-4){$5$};
        \node[circle,draw,minimum size=1cm] (5) at (1,-4) {$2$};
        \node[circle,draw,minimum size=1cm] (6) at (3,-4) {$6$};
        % \node[label={7},circle,draw,minimum size=1cm] (7) at (3,-4) {$8$};
        % \node[label={8},circle,draw,minimum size=1cm] (8) at (-4,-6) {$4$};
        % \node[label={9},circle,draw,minimum size=1cm] (9) at (-2,-6) {$1$};
        \tikzstyle{filho}=[thick]
        \tikzstyle{pred}=[->, shorten >= 2pt, shorten <= 2pt,
        dashed, >=stealth]
        \tikzstyle{sucessor}=[->, shorten >= 2pt, shorten <= 2pt,
        dotted, >=stealth, line width=0.35mm]
        % \tikzstyle{p4}=[->, shorten >= 2pt, shorten <= 2pt, dotted, >=stealth]
        \draw[filho] (1) -- (2);
        \draw[filho] (1) -- (3);
        \draw[filho] (2) -- (4);
        \draw[filho] (3) -- (5);
        \draw[filho] (3) -- (6);
        \draw[pred] (6) edge[out=30,in=0]
        node[above=10pt] {$2$} (3);
        \draw[sucessor] (3) edge[out=280,in=180] (6);
        \draw[pred] (3) edge[out=190,in=120]
        node[above=10pt] {$4$} (5);
        \draw[sucessor] (5) edge[out=0,in=260] (3);
        \draw[pred] (5) edge[out=170,in=285]
        node[above=10pt] {$+\infty$} (1);
        \draw[sucessor] (1) edge[out=270,in=200] (5);
        \draw[pred, blue, dashdotted] (1) edge[out=180,in=120]
        node[blue, above=10pt] {$+\infty$} (2);
        \draw[sucessor] (2) edge[out=100,in=200] (1);
        \draw[pred] (2) edge[out=180,in=120]
        node[above=10pt] {$2$} (4);
        \draw[sucessor] (4) edge[out=100,in=200] (2);
        \draw[pred] (4) edge[out=180,in=40]
        node[above=10pt] {$+\infty$} (-4.5, -4);
    \end{tikzpicture}
    \begin{tabular}{|c|c|c|c|c|}
        \hline
        &       &       & $\now = 1.5$                \\
        $i$ & $x_0$ & $v$   & $\cert[i]$                  \\
        \hline
        $1$ & $6$   & $2$   & \textcolor{blue}{$+\infty$} \\

        $2$ & $3$   & $5$   & $+\infty$                   \\

        $3$ & $2$   & $1$   & $2$                         \\

        $4$ & $7$   & $4$   & $4$                         \\

        $5$ & $-2$  & $3$   & $+\infty$                   \\

        $6$ & $14$  & $0.5$ & $2$                         \\
        \hline
    \end{tabular}
    \caption[ABB após chamar \textsc{delete}]{Após chamar
        {\normalfont \textsc{delete}$(7)$} na árvore da Figura~\ref{fig:abb:insert}, o
    elemento $7$ foi retirado da árvore, a lista ligada foi
    ajustada, o certificado do seu sucessor, o elemento $1$, foi
    atualizado e o certificado do elemento $7$ foi destruído.}
    \label{fig:abb:delete}
\end{figure}

A operação \textsc{delete}$(i)$ consiste em recuperar o elemento
$i$, removê-lo da árvore binária balanceada de busca e da estrutura
que usamos para recuperá-lo, e depois removê-lo da fila com
prioridades. Após isso, basta atualizar o certificado de seu
sucessor, caso exista. Essa operação é ilustrada na Figura
\ref{fig:abb:delete} e implementada no Algoritmo \ref{abb:delete}.

\begin{figure}[htb]
    \centering
    \begin{tikzpicture}[baseline=-2.25cm]
        \node[circle,draw,minimum size=1cm] (1) at (0,0)  {$1$};
        \node[circle,draw,minimum size=1cm] (2) at (-2,-2){$3$};
        \node[circle,draw,minimum size=1cm] (3) at (2,-2) {$4$};
        \node[circle,draw,minimum size=1cm] (4) at (-3,-4){$5$};
        \node[circle,draw,minimum size=1cm] (5) at (1,-4) {$2$};
        \node[circle,draw,minimum size=1cm] (6) at (3,-4) {$6$};
        % \node[label={7},circle,draw,minimum size=1cm] (7) at (3,-4) {$8$};
        % \node[label={8},circle,draw,minimum size=1cm] (8) at (-4,-6) {$4$};
        % \node[label={9},circle,draw,minimum size=1cm] (9) at (-2,-6) {$1$};
        \tikzstyle{filho}=[thick]
        \tikzstyle{pred}=[->, shorten >= 2pt, shorten <= 2pt,
        dashed, >=stealth]
        \tikzstyle{sucessor}=[->, shorten >= 2pt, shorten <= 2pt,
        dotted, >=stealth, line width=0.35mm]
        % \tikzstyle{p4}=[->, shorten >= 2pt, shorten <= 2pt, dotted, >=stealth]
        \draw[filho] (1) -- (2);
        \draw[filho] (1) -- (3);
        \draw[filho] (2) -- (4);
        \draw[filho] (3) -- (5);
        \draw[filho] (3) -- (6);
        \draw[pred] (6) edge[out=30,in=0]
        node[above=10pt] {$2$} (3);
        \draw[sucessor] (3) edge[out=280,in=180] (6);
        \draw[pred] (3) edge[out=190,in=120]
        node[above=10pt] {$4$} (5);
        \draw[sucessor] (5) edge[out=0,in=260] (3);
        \draw[pred] (5) edge[out=170,in=285]
        node[above=10pt] {$+\infty$} (1);
        \draw[sucessor] (1) edge[out=270,in=200] (5);
        \draw[pred, blue, dashdotted] (1) edge[out=180,in=120]
        node[blue, above=10pt] {$+\infty$} (2);
        \draw[sucessor] (2) edge[out=100,in=200] (1);
        \draw[pred] (2) edge[out=180,in=120]
        node[above=10pt] {$2$} (4);
        \draw[sucessor] (4) edge[out=100,in=200] (2);
        \draw[pred] (4) edge[out=180,in=40]
        node[above=10pt] {$+\infty$} (-4.5, -4);
    \end{tikzpicture}
    \begin{tabular}{|c|c|c|c|c|}
        \hline
        &       &       & $\now = 1.5$                \\
        $i$ & $x_0$ & $v$   & $\cert[i]$                  \\
        \hline
        $1$ & $6$   & $2$   & \textcolor{blue}{$+\infty$} \\

        $2$ & $3$   & $5$   & $+\infty$                   \\

        $3$ & $2$   & $1$   & $2$                         \\

        $4$ & $7$   & $4$   & $4$                         \\

        $5$ & $-2$  & $3$   & $+\infty$                   \\

        $6$ & $14$  & $0.5$ & $2$                         \\
        \hline
    \end{tabular}
    \caption[ABB após chamar \textsc{delete}]{Após chamar
        {\normalfont \textsc{delete}$(7)$} na árvore da Figura~\ref{fig:abb:insert}, o
    elemento $7$ foi retirado da árvore, a lista ligada foi
    ajustada, o certificado do seu sucessor, o elemento $1$, foi
    atualizado e o certificado do elemento $7$ foi destruído.}
    \label{fig:abb:delete}
\end{figure}

%!TeX root=../../tcc.tex

\chapter{Máximo cinético}
Agora, considere o seguinte problema cinético. São dados $n$ pares
de valores em que cada par $(x_0, v)$ representa um valor que está
mudando linearmente com o tempo, assim como na lista cinética. Num
instante arbitrário $t \geq 0$, o valor correspondente ao par $(x_0,
v)$ é $x_0 + tv$. Desta vez, o objetivo é responder consultas mais
simples, do tipo: quem é o elemento com maior valor da coleção no
instante corrente. Veremos que é possível obter implementações mais
eficientes que para a ordenação cinética neste caso.

Utilizando o mesmo exemplo da lista, se tivermos quatro elementos na
coleção, digamos $\left(6, -\dfrac{1}{2}\right)$, $(5, 0)$,
$\left(3, \dfrac{1}{4}\right)$ e $\left(0, \dfrac{4}{3}\right)$,
podemos representar essa coleção da seguinte maneira:

\begin{figure}[H]
    \centering
    \begin{tikzpicture}[thick, scale=0.7]
        \node[label={1},circle,draw,minimum size=1cm]
        (1) at (0,0) {$3$};
        \node[label={2},circle,draw,minimum size=1cm]
        (2) at (-4,-2) {$3$};
        \node[label={3},circle,draw,minimum size=1cm]
        (3) at (4,-2) {$4$};
        \node[label={4},circle,draw,minimum size=1cm]
        (4) at (-6,-4) {$1$};
        \node[label={5},circle,draw,minimum size=1cm]
        (5) at (-2,-4) {$3$};
        \node[label={6},circle,draw,minimum size=1cm]
        (6) at (2,-4) {$4$};
        \node[label={7},circle,draw,minimum size=1cm]
        (7) at (6,-4) {$6$};
        \node[label={8},circle,draw,minimum size=1cm]
        (8) at (-7,-6) {$8$};
        \node[label={9},circle,draw,minimum size=1cm]
        (9) at (-5,-6) {$1$};
        \node[label={10},circle,draw,minimum size=1cm]
        (10) at (-3,-6) {$2$};
        \node[label={11},circle,draw,minimum size=1cm]
        (11) at (-1,-6) {$3$};
        \node[label={12},circle,draw,minimum size=1cm]
        (12) at (1,-6) {$4$};
        \node[label={13},circle,draw,minimum size=1cm]
        (13) at (3,-6) {$5$};
        \node[label={14},circle,draw,minimum size=1cm]
        (14) at (5,-6) {$6$};
        \node[label={15},circle,draw,minimum size=1cm]
        (15) at (7,-6) {$7$};
        \node[label={16},circle,draw,minimum size=1cm]
        (16) at (-8,-8) {$8$};
        \node[label={17},circle,draw,minimum size=1cm]
        (17) at (-6,-8) {$9$};

        \draw[thick] (1) -- (2);
        \draw[thick] (2) -- (4);
        \draw[thick] (4) -- (8);
        \draw[thick] (4) -- (9);
        \draw[thick] (8) -- (16);
        \draw[thick] (8) -- (17);
        \draw[thick] (2) -- (5);
        \draw[thick] (5) -- (10);
        \draw[thick] (5) -- (11);
        \draw[thick] (1) -- (3);
        \draw[thick] (3) -- (6);
        \draw[thick] (3) -- (7);
        \draw[thick] (6) -- (12);
        \draw[thick] (6) -- (13);
        \draw[thick] (7) -- (14);
        \draw[thick] (7) -- (15);
    \end{tikzpicture}
    \caption[Representação da estrutura torneio]{Torneio com $9$
        elementos em que $3$ é o elemento com valor máximo.}
    \label{fig:torneio:exemplo}
\end{figure}

Agora, além das operações \textsc{advance}$(t)$ e
\textsc{change}$(j, v)$, em vez da operação~\Call{query\_kth}{$i$},
queremos dar suporte à nova operação:
\begin{itemize}
    \item \textsc{query\_max}$()$ $\rightarrow$ devolve o elemento
    cujo valor é o maior no instante atual.
\end{itemize}
%!TeX root=./maximo.tex


\section{Heap cinético}\label{sec:heap-cinetico}
Um bom jeito de resolver o problema do máximo cinético é manter uma fila de
prioridades com os elementos da coleção tendo como prioridade o valor corrente
do elemento.
Dessa maneira, o elemento que se encontra na raiz da fila será o que possui o maior valor da
coleção.
Para implementar a fila utilizaremos um vetor organizado como um heap.

Inicialmente o vetor começa com os índices dos elementos e o reorganizamos como um
heap usando como chave o valor de cada elemento no instante $t = 0$, ou seja, o
valor $x_0$ de cada elemento.

Uma vez montado o heap, construímos um certificado para cada par $($filho,
pai$)$ no heap.
O $i$-ésimo certificado se refere ao par das posições $i$ e $\floor{\frac{i}{2}}$ e consiste no
instante de tempo em que o $i$-ésimo elemento passará a ter um valor maior que o valor do
$\floor{\frac{i}{2}}$-ésimo elemento do vetor, se esse instante for maior que o instante atual.
Do contrário, o certificado consiste em $+\infty$.

Esses $n - 1$ certificados são colocados em uma fila com prioridades $Q$, com o
prazo de validade como chave.
Estamos interessados nos certificados com menor prazo de validade.

Para descrever a implementação das três operações, precisamos estabelecer o nome
das novas variáveis usadas.
São elas:
\begin{enumerate}
    \item \textit{heap}: vetor com os índices dos $n$ elementos
    formando um heap de acordo com o seu valor no instante
    \textit{now};
    \item \textit{cert}: vetor com os certificados, onde
    \textit{cert}$[i]$ guarda o certificado entre $i$ e
    $\floor{\frac{i}{2}}$, para $1 < i \leq n$.
\end{enumerate}

A interface da fila com prioridades que utilizaremos não se altera.

Um evento está associado a um certificado $(i, t)$ que expira no instante $t$,
como pode ser visto na Figura~\ref{fig:maxdevent}.
O tratamento do evento correspondente ao certificado $(i, t)$ consiste em trocar de lugar os
índices armazenados nas posições $i$ e $\floor{\frac{i}{2}}$ do vetor \heap, e recalcular o prazo
de validade de até cinco certificados, ilustrados na Figura~\ref{fig:max:update}:
\begin{itemize}
    \item do $\floor{\frac{i}{2}}$-ésimo certificado, se $i > 1$;
    \item do $j$-ésimo certificado, se $i > 1$ e $j \leq n$,
    onde $j = 2\cdot \floor{\frac{i}{2}} + ((i + 1)\mod2)$
    é o irmão de $i$;
    \item do $(2i)$-ésimo certificado, se $2i \leq n$;
    \item do $(2i + 1)$-ésimo certificado, se $2i + 1 \leq n$.
\end{itemize}

\begin{table}[htb]
    \begin{tabular}{|c|c|c|c|c|}
        \hline
        &       &       & $\now = 0$ & $\now = 1$ \\
        $i$   & $x_0$ & $v$   & $\cert[i]$ & $\cert[i]$ \\
        \hline
        $1$   & $2$   & $2$   & $?$        & $?$        \\

        $2$   & $6$   & $1$   & $?$        & $?$        \\

        $3$   & $-6$  & $3$   & $?$        & $?$        \\

        $4$   & $-16$ & $4$   & $?$        & $?$        \\

        $5$   & $-1$  & $5$   & $?$        & $?$        \\

        $6$   & $1$   & $2.5$ & $?$        & $?$        \\

        $7$   & $-18$ & $6$   & $?$        & $?$        \\

        $8$   & $5$   & $1$   & $?$        & $?$        \\

        % $9$ & $-12$ & $3$ & $?$ & $?$ \\
        \hline
    \end{tabular}
    \caption{Tabela para as Figuras~\ref{fig:maxdevent}
    e~\ref{fig:max:update}. (incompleta)}\label{tab:table}
\end{table}

O $i$-ésimo certificado também deve ser ajustado para $+\infty$.
Finalmente, é necessário fazer ajustes em $Q$, alterando a chave dos certificados que sofreram
alteração.

\begin{table}[htb]
    \begin{tabular}{|c|c|c|c|c|}
        \hline
        &       &       & $\now = 0$ & $\now = 1$ \\
        $i$   & $x_0$ & $v$   & $\cert[i]$ & $\cert[i]$ \\
        \hline
        $1$   & $2$   & $2$   & $?$        & $?$        \\

        $2$   & $6$   & $1$   & $?$        & $?$        \\

        $3$   & $-6$  & $3$   & $?$        & $?$        \\

        $4$   & $-16$ & $4$   & $?$        & $?$        \\

        $5$   & $-1$  & $5$   & $?$        & $?$        \\

        $6$   & $1$   & $2.5$ & $?$        & $?$        \\

        $7$   & $-18$ & $6$   & $?$        & $?$        \\

        $8$   & $5$   & $1$   & $?$        & $?$        \\

        % $9$ & $-12$ & $3$ & $?$ & $?$ \\
        \hline
    \end{tabular}
    \caption{Tabela para as Figuras~\ref{fig:maxdevent}
    e~\ref{fig:max:update}. (incompleta)}\label{tab:table}
\end{table}

Novamente, na implementação da operação \textsc{event}, no Algoritmo~\ref{max:evento},
utilizaremos a rotina \textsc{update}$(i)$, do Algoritmo~\ref{max:update}, que calcula a nova
validade $t$ do $i$-ésimo certificado, se $1 < i \leq n$, e chama a rotina \textsc{updatePQ}$(Q, i,
t)$.

\begin{algorithm}
    \caption{Função \textsc{update}.} \label{lista:update}
\begin{algorithmic}[1]
    \Function{update}{$i$}
        \If{$1 \leq i < n$}
            \State $t \leftarrow $ \Call{expire}{$i,i+1$}
            \State \Call{updatePQ}{$Q,i,t$}
        \EndIf
    \EndFunction
\end{algorithmic}
\end{algorithm}

\begin{algorithm}
    \caption{Função \textsc{event}.} \label{torneioi:evento}
    \begin{algorithmic}[1]
        \Function{event}{\nnull}
            \State $e \leftarrow  $ \Call{minPQ}{$Q$}
            \While{$e.\cert$ = \now}
                \State $j \leftarrow e.\lastmatch$
                \State $k \leftarrow 2\cdot \floor{\frac{j}{2}}
                + ((j + 1)\mod2)$ \Comment{adversário}
                \While{$j > 1$ \AND \Call{compare}{$j, k$}}
                    \State \torneio[$\floor{\frac{j}{2}}$]
                    $\leftarrow~$\torneio[$j$]
                    \State $\torneio[k].\lastmatch$ $\leftarrow k$
                    \State \Call{update}{$\torneio[k]$}
                    \State $j \leftarrow \floor{\frac{j}{2}}$
                    \State $k \leftarrow 2\cdot \floor{\frac{j}{2}}
                    + ((j + 1)\mod2)$ \Comment{adversário}
                \EndWhile
                \State $\torneio[j].\lastmatch \leftarrow j$
                \State \Call{update}{$\torneio[j]$}
                \State $e \leftarrow  $ \Call{minPQ}{$Q$}
            \EndWhile
        % \LineComment{swapHeap$(i, \floor{\frac{i}{2}})$ troca \heap[$i$] por \heap$\left[\floor{\frac{i}{2}}\right]$}
        \EndFunction
        \LineComment{\Call{compare}{$i, j$} retorna se o valor
        de $i$ é maior que o valor de $j$.}
    \end{algorithmic}
\end{algorithm}

A operação \textsc{query\_max}$()$, no Algoritmo~\ref{max:heap:querymax},
consiste em devolver \textit{heap}$[1]$, enquanto que a operação
\textsc{change}$(j, v)$, no Algoritmo~\ref{alg:heap:change}, consiste em alterar
a posição $x_0[j]$ para ${x_0[j] + (\mathit{speed}[j] - v)\cdot now}$, a posição
\textit{speed}[j] para \textit{v} e recalcular os eventuais certificados de que
$j$ participa.
Para tanto, a partir da posição $i$ em que $j$ se encontra no vetor \textit{heap}, podemos
recalcular \textit{cert}$[i]$ se $i > 1$, \textit{cert}$[2i]$ se $2i \leq n$ e \textit{cert}$[2i +
1]$ se $2i + 1 \leq n$, acionando a rotina \textsc{update} para fazer os devidos acertos em $Q$
correspondentes a estas modificações.
Veja a Figura~\ref{fig:predeventheap}.

\begin{figure}[H]
    \centering
    \begin{tikzpicture}[thick, scale=0.7]
        \node[label={1},circle,draw,minimum size=1cm]
        (1) at (0,0) {$9$};
        \node[label={2},circle,draw,minimum size=1cm]
        (2) at (-2,-2) {$5$};
        \node[label={3},circle,draw,minimum size=1cm]
        (3) at (2,-2) {$7$};
        \node[label={4},circle,draw,minimum size=1cm]
        (4) at (-3,-4) {$2$};
        \node[label={5},circle,draw,minimum size=1cm]
        (5) at (-1,-4) {$6$};
        \node[label={6},circle,draw,minimum size=1cm]
        (6) at (1,-4) {$3$};
        \node[label={7},circle,draw,minimum size=1cm]
        (7) at (3,-4) {$8$};
        \node[label={8},circle,draw,minimum size=1cm]
        (8) at (-4,-6) {$4$};
        \node[label={9},circle,draw,minimum size=1cm]
        (9) at (-2,-6) {$1$};

        \tikzstyle{cert}=[<-, dashdotted, blue, thick]
        \draw[thick] (1) -- (2);
        \draw[thick] (1) -- (3);
        \draw[cert] (2) -- (4);
        \draw[thick] (2) -- (5);
        \draw[thick] (3) -- (6);
        \draw[thick] (3) -- (7);
        \draw[cert] (4) -- (8);
        \draw[cert] (4) -- (9);
    \end{tikzpicture}
    \caption[Exemplo certificados do heap cinético após operação \textsc{change}]{Após a mudança de
    velocidade do elemento 2, que se encontra em \heap[$4$], os certificados
    \cert[$4$], \cert[$8$] e \cert[$9$] foram atualizados.}
    \label{fig:predeventheap}
\end{figure}

\begin{algorithm}
    \caption{Função \textsc{query\_max}.} \label{torn:querymax}
    \begin{algorithmic}[1]
        \Function{query\_max}{\null}
            \State \Return \torneio$[1]$
        \EndFunction
    \end{algorithmic}
\end{algorithm}

\begin{algorithm}
    \caption{Função \textsc{change}.} \label{torneioi:change}
    \begin{algorithmic}[1]
        \Function{change}{$j, v$}
            \State $e \leftarrow$ \Call{getObject}{$j$}
            \State $e.x_0 \leftarrow e.x_0+~(e.\speed -~v)~\cdot~\now$;
            \State $e.\speed \leftarrow v$
            \State $i \leftarrow e.\lastmatch$
            \State \Call{update}{$e$}
            \While{$i < n$}
                \If{$\torneio[i] = \torneio[2i]$}
                    \State $i \leftarrow 2i$
                \Else
                    \State $i \leftarrow 2i + 1$
                \EndIf
                \State $k \leftarrow 2\cdot \floor{\frac{i}{2}}
                + ((i + 1)\mod2)$ \Comment{adversário}
                \State \Call{update}{$\torneio[k]$}
            \EndWhile
        \EndFunction
    \end{algorithmic}
\end{algorithm}

\FloatBarrier

\subsection{Análise de desempenho}\label{subsec:heap:analise-de-desempenho}

O heap cinético é uma estrutura \textit{responsiva}, pois o custo de processar
um certificado é $O(\lg{n})$, onde $n$ é a quantidade de pontos.
O custo de processar um certificado corresponde a uma iteração da linha $3$ da operação
\textsc{event}, que troca de posição os dois elementos envolvidos no certificado em $O(1)$ e
atualiza os cinco certificados afetados em $O(\lg{n})$.

O heap cinético é uma estrutura \textit{eficiente}, pois a quantidade de eventos
internos é $O(n\lg^2{n})$ para trajetórias lineares, mostrado em~\cite{basch-thesis}.
Como a quantidade de eventos externos é $\Theta(n)$ temos uma razão $O(\lg^2{n})$, que é
polilogarítmica em $n$, tornando o heap cinético uma estrutura eficiente em trajetórias lineares.

O heap cinético é uma estrutura \textit{compacta}, pois há no máximo $n$
certificados na fila com prioridades, um para cada objeto.

O heap cinético é uma estrutura \textit{local}, pois um elemento está envolvido
em no máximo três certificados da fila ao mesmo tempo.




%!TeX root=./maximo.tex

\newcommand{\thickness}{0.75mm}
\FloatBarrier
\section{Torneio cinético} \label{torneio:secao}

Considere o seguinte algoritmo para achar o valor máximo em um
conjunto de $n$ elementos: aloque um vetor \torneio~com $2n - 1$
posições. Inicializamos as últimas $n$ posições com os valores dos
$n$ elementos e uma variável $i$ com o valor da última posição, $i =
2n - 1$. Repita o seguinte processo até que $i$ seja igual a $1$: se
\torneio$[i] > $~\torneio$[i - 1]$, então
\torneio$\left[\floor{\frac{i}{2}}\right] =$~\torneio$[i]$, caso
contrário \torneio$\left[\floor{\frac{i}{2}}\right] =$~ \torneio$[i
- 1]$, e, por fim, subtraia~$2$ de $i$. Dessa maneira, ao fim da
execução do algoritmo, em \torneio$[1]$ estará o maior valor da
coleção. Na verdade, podemos fazer a comparação de maneira indireta,
e guardar os índices dos elementos no vetor \torneio~e não seus
valores. Veja a Figura \ref{fig:torneio:exemplo}.

Podemos pensar nessas comparações entre \torneio$[i]$ e
\torneio$[i-1]$ como sendo partidas de um torneio classificatório,
por isso o nome torneio. Chamaremos de ``partida'' as comparações
entre \torneio$[i]$ e \torneio$[i-1]$ e diremos que o elemento~$j$
``vence'' o elemento $k$ quando os elementos $j$ e $k$ disputaram
uma partida entre si e o elemento $j$ possuía maior valor nesse
instante. Utilizaremos esse ``torneio'' para resolver o problema do
máximo cinético e, para implementá-lo, será utilizado um vetor como
o citado no algoritmo acima.

\begin{figure}[H]
    \centering
    \begin{tikzpicture}[thick, scale=0.7]
        \node[label={1},circle,draw,minimum size=1cm]
        (1) at (0,0) {$3$};
        \node[label={2},circle,draw,minimum size=1cm]
        (2) at (-4,-2) {$3$};
        \node[label={3},circle,draw,minimum size=1cm]
        (3) at (4,-2) {$4$};
        \node[label={4},circle,draw,minimum size=1cm]
        (4) at (-6,-4) {$1$};
        \node[label={5},circle,draw,minimum size=1cm]
        (5) at (-2,-4) {$3$};
        \node[label={6},circle,draw,minimum size=1cm]
        (6) at (2,-4) {$4$};
        \node[label={7},circle,draw,minimum size=1cm]
        (7) at (6,-4) {$6$};
        \node[label={8},circle,draw,minimum size=1cm]
        (8) at (-7,-6) {$8$};
        \node[label={9},circle,draw,minimum size=1cm]
        (9) at (-5,-6) {$1$};
        \node[label={10},circle,draw,minimum size=1cm]
        (10) at (-3,-6) {$2$};
        \node[label={11},circle,draw,minimum size=1cm]
        (11) at (-1,-6) {$3$};
        \node[label={12},circle,draw,minimum size=1cm]
        (12) at (1,-6) {$4$};
        \node[label={13},circle,draw,minimum size=1cm]
        (13) at (3,-6) {$5$};
        \node[label={14},circle,draw,minimum size=1cm]
        (14) at (5,-6) {$6$};
        \node[label={15},circle,draw,minimum size=1cm]
        (15) at (7,-6) {$7$};
        \node[label={16},circle,draw,minimum size=1cm]
        (16) at (-8,-8) {$8$};
        \node[label={17},circle,draw,minimum size=1cm]
        (17) at (-6,-8) {$9$};

        \draw[thick] (1) -- (2);
        \draw[thick] (2) -- (4);
        \draw[thick] (4) -- (8);
        \draw[thick] (4) -- (9);
        \draw[thick] (8) -- (16);
        \draw[thick] (8) -- (17);
        \draw[thick] (2) -- (5);
        \draw[thick] (5) -- (10);
        \draw[thick] (5) -- (11);
        \draw[thick] (1) -- (3);
        \draw[thick] (3) -- (6);
        \draw[thick] (3) -- (7);
        \draw[thick] (6) -- (12);
        \draw[thick] (6) -- (13);
        \draw[thick] (7) -- (14);
        \draw[thick] (7) -- (15);
    \end{tikzpicture}
    \caption[Representação da estrutura torneio]{Torneio com $9$
        elementos em que $3$ é o elemento com valor máximo.}
    \label{fig:torneio:exemplo}
\end{figure}

Inicialmente o vetor começa com os índices dos elementos ocupando as
últimas posições e construímos o torneio de acordo com o valor de
cada elemento no instante $t = 0$, ou seja, com o valor $x_0$ de
cada elemento.

Uma vez montado o torneio, construímos um certificado para cada
elemento no torneio. O $i$-ésimo certificado, que se refere ao par
formado pelo $i$-ésimo elemento da entrada e quem o venceu na última
partida que disputou, consiste no instante de tempo em que o
$i$-ésimo elemento passará a ter um valor maior que o valor do
elemento que o venceu anteriormente, se esse instante for maior que
o instante atual. Do contrário, o certificado consiste em $+\infty$.
Veja a Figura \ref{fig:torneio:certificados}.

É importante observar que o elemento que se encontra na primeira
posição do torneio não é vencido por ninguém no instante \now. Dessa
forma, sendo $i$ o elemento que ocupa a primeira posição do torneio,
associamos ao $i$-ésimo certificado a chave $+\infty$.

\begin{figure}
    \centering
    \begin{tikzpicture}[thick]
        \node[label={1},circle,draw,minimum size=1cm]
            (1) at (0,0) {$3$};
        \node[label={2},circle,draw,minimum size=1cm]
            (2) at (-4,-2) {$3$};
        \node[label={3},circle,draw,minimum size=1cm]
            (3) at (4,-2) {$4$};
        \node[label={4},circle,draw,minimum size=1cm]
            (4) at (-6,-4) {$1$};
        \node[label={5},circle,draw,minimum size=1cm]
            (5) at (-2,-4) {$3$};
        \node[label={6},circle,draw,minimum size=1cm]
            (6) at (2,-4) {$4$};
        \node[label={7},circle,draw,minimum size=1cm]
            (7) at (6,-4) {$6$};
        \node[label={8},circle,draw,minimum size=1cm]
            (8) at (-7,-6) {$8$};
        \node[label={9},circle,draw,minimum size=1cm]
            (9) at (-5,-6) {$1$};
        \node[label={10},circle,draw,minimum size=1cm]
            (10) at (-3,-6) {$2$};
        \node[label={11},circle,draw,minimum size=1cm]
            (11) at (-1,-6) {$3$};
        \node[label={12},circle,draw,minimum size=1cm]
            (12) at (1,-6) {$4$};
        \node[label={13},circle,draw,minimum size=1cm]
            (13) at (3,-6) {$5$};
        \node[label={14},circle,draw,minimum size=1cm]
            (14) at (5,-6) {$6$};
        \node[label={15},circle,draw,minimum size=1cm]
            (15) at (7,-6) {$7$};
        \node[label={16},circle,draw,minimum size=1cm]
            (16) at (-8,-8) {$8$};
        \node[label={17},circle,draw,minimum size=1cm]
            (17) at (-6,-8) {$9$};

        \draw[thick] (1) -- (2);
        \draw[<-, line width=\thickness, red] (2) -- (4);
        \draw[<-, line width=\thickness, red] (4) -- (8);
        \draw[thick] (4) -- (9);
        \draw[thick] (8) -- (16);
        \draw[<-, line width=\thickness, red] (8) -- (17);
        \draw[thick] (2) -- (5);
        \draw[<-, line width=\thickness, red] (5) -- (10);
        \draw[thick] (5) -- (11);
        \draw[<-, line width=\thickness, red] (1) -- (3);
        \draw[thick] (3) -- (6);
        \draw[<-, line width=\thickness, red] (3) -- (7);
        \draw[thick] (6) -- (12);
        \draw[<-, line width=\thickness, red] (6) -- (13);
        \draw[thick] (7) -- (14);
        \draw[<-, line width=\thickness, red] (7) -- (15);
    \end{tikzpicture}
    \caption[Representação de certificados em torneio]{Torneio com 9
    elementos e os certificados visualmente representados pelas
    setas vermelhas mais grossas. O certificado correspondente ao
    elemento 3 terá a chave~$+\infty$.}
    \label{fig:torneio:certificados}
\end{figure}

Esses $n$ certificados são colocados em uma fila com prioridades,
com o prazo de validade como chave. Estamos interessados nos
certificados com menor prazo de validade.

Para descrever a implementação das operações \textsc{advance},
\textsc{change} e \textsc{query\_max}, precisamos estabelecer o nome
das novas variáveis usadas. São elas:
\begin{enumerate}
    \item \torneio: vetor, de $2n - 1$ posições, com os índices dos
    $n$ elementos formando um torneio de acordo com o seu valor no
    instante \textit{now};

    \item \textit{cert}: vetor com os certificados;
    \textit{cert}$[i]$ guarda o certificado entre o elemento $i$ e
    quem o venceu na última partida que disputou, para $1 \leq i
    \leq n$;

    \item \textit{indT}: vetor de $n$ posições; \indt[$i$] guarda a
    posição em \torneio~em que $i$ perde uma partida, com $1 \leq i
    \leq n$. Se $i$ não perde nenhuma partida, \indt[$i$] é igual a
    $-1$.
\end{enumerate}

A interface da fila com prioridades que utilizaremos não se altera.

Na implementação da operação \textsc{event}, no Algoritmo
\ref{torneio:evento}, utilizaremos a rotina \textsc{update}$(i)$, no
Algoritmo \ref{torneio:update}, que calcula a nova validade $t$ do
elemento $j$ que se encontra na $i$-ésima posição de \torneio, isto
é, $j =~$\torneio[$i$] certificado, se $1 \leq i \leq 2n - 1$, e
chama a rotina \textsc{updatePQ}$(Q, i, t)$.

\begin{algorithm}
    \caption{Função \textsc{update}.} \label{lista:update}
\begin{algorithmic}[1]
    \Function{update}{$i$}
        \If{$1 \leq i < n$}
            \State $t \leftarrow $ \Call{expire}{$i,i+1$}
            \State \Call{updatePQ}{$Q,i,t$}
        \EndIf
    \EndFunction
\end{algorithmic}
\end{algorithm}

\begin{algorithm}
    \caption{Função \textsc{event}.} \label{torneioi:evento}
    \begin{algorithmic}[1]
        \Function{event}{\nnull}
            \State $e \leftarrow  $ \Call{minPQ}{$Q$}
            \While{$e.\cert$ = \now}
                \State $j \leftarrow e.\lastmatch$
                \State $k \leftarrow 2\cdot \floor{\frac{j}{2}}
                + ((j + 1)\mod2)$ \Comment{adversário}
                \While{$j > 1$ \AND \Call{compare}{$j, k$}}
                    \State \torneio[$\floor{\frac{j}{2}}$]
                    $\leftarrow~$\torneio[$j$]
                    \State $\torneio[k].\lastmatch$ $\leftarrow k$
                    \State \Call{update}{$\torneio[k]$}
                    \State $j \leftarrow \floor{\frac{j}{2}}$
                    \State $k \leftarrow 2\cdot \floor{\frac{j}{2}}
                    + ((j + 1)\mod2)$ \Comment{adversário}
                \EndWhile
                \State $\torneio[j].\lastmatch \leftarrow j$
                \State \Call{update}{$\torneio[j]$}
                \State $e \leftarrow  $ \Call{minPQ}{$Q$}
            \EndWhile
        % \LineComment{swapHeap$(i, \floor{\frac{i}{2}})$ troca \heap[$i$] por \heap$\left[\floor{\frac{i}{2}}\right]$}
        \EndFunction
        \LineComment{\Call{compare}{$i, j$} retorna se o valor
        de $i$ é maior que o valor de $j$.}
    \end{algorithmic}
\end{algorithm}

No trecho das linhas $5$ - $11$ do Algoritmo \ref{torneio:evento}, o
resultado da partida entre o elemento~$j$ e seu adversário que se
encontra na posição $k$ de \torneio~é recalculado, e o certificado
correspondente é atualizado. Caso o resultado da partida tenha sido
alterado, a verificação se propaga para o nível de cima. A função
$\Call{value}{j}$ retorna ${\speed[j]~\cdot\now + x_0[j]}$.

\begin{figure}
    \centering
    \begin{tikzpicture}[thick, scale=0.8]
        \node[label={1},circle,draw,minimum size=1cm]
            (1) at (0,0) {$3$};
        \node[label={2},circle,draw,minimum size=1cm]
            (2) at (-4,-2) {$3$};
        \node[label={3},circle,draw,minimum size=1cm]
            (3) at (4,-2) {$4$};
        \node[label={4},circle,draw,minimum size=1cm]
            (4) at (-6,-4) {$1$};
        \node[label={5},circle,draw,minimum size=1cm]
            (5) at (-2,-4) {$3$};
        \node[label={6},circle,draw,minimum size=1cm]
            (6) at (2,-4) {$4$};
        \node[label={7},circle,draw,minimum size=1cm]
            (7) at (6,-4) {$6$};
        \node[label={8},circle,draw,minimum size=1cm]
            (8) at (-7,-6) {$8$};
        \node[label={9},circle,draw,minimum size=1cm]
            (9) at (-5,-6) {$1$};
        \node[label={10},circle,draw,minimum size=1cm]
            (10) at (-3,-6) {$2$};
        \node[label={11},circle,draw,minimum size=1cm]
            (11) at (-1,-6) {$3$};
        \node[label={12},circle,draw,minimum size=1cm]
            (12) at (1,-6) {$4$};
        \node[label={13},circle,draw,minimum size=1cm]
            (13) at (3,-6) {$5$};
        \node[label={14},circle,draw,minimum size=1cm]
            (14) at (5,-6) {$6$};
        \node[label={15},circle,draw,minimum size=1cm]
            (15) at (7,-6) {$7$};
        \node[label={16},circle,draw,minimum size=1cm]
            (16) at (-8,-8) {$8$};
        \node[label={17},circle,draw,minimum size=1cm]
            (17) at (-6,-8) {$9$};

        \draw[thick] (1) -- (2);
        \draw[<-,line width=\thickness, red] (2) -- (4);
        \draw[<-,thick, dashed, red] (4) -- (8);
        \draw[thick] (4) -- (9);
        \draw[thick] (8) -- (16);
        \draw[<-,line width=\thickness, red] (8) -- (17);
        \draw[thick] (2) -- (5);
        \draw[<-,line width=\thickness, red] (5) -- (10);
        \draw[thick] (5) -- (11);
        \draw[<-,line width=\thickness, red] (1) -- (3);
        \draw[thick] (3) -- (6);
        \draw[<-,line width=\thickness, red] (3) -- (7);
        \draw[thick] (6) -- (12);
        \draw[<-,line width=\thickness, red] (6) -- (13);
        \draw[thick] (7) -- (14);
        \draw[<-,line width=\thickness, red] (7) -- (15);
    \end{tikzpicture}
    \caption[Representação de certificado expirado]{cert[$8$] expirou.}
    \label{fig:torneio:evento}
\end{figure}

\begin{figure}[H]
    \centering
    \begin{tikzpicture}[thick, scale=0.7]
        \node[label={1},circle,draw,minimum size=1cm]
        (1) at (0,0) {$9$};
        \node[label={2},circle,draw,minimum size=1cm]
        (2) at (-2,-2) {$5$};
        \node[label={3},circle,draw,minimum size=1cm]
        (3) at (2,-2) {$7$};
        \node[label={4},circle,draw,minimum size=1cm]
        (4) at (-3,-4) {$2$};
        \node[label={5},circle,draw,minimum size=1cm]
        (5) at (-1,-4) {$6$};
        \node[label={6},circle,draw,minimum size=1cm]
        (6) at (1,-4) {$3$};
        \node[label={7},circle,draw,minimum size=1cm]
        (7) at (3,-4) {$8$};
        \node[label={8},circle,draw,minimum size=1cm]
        (8) at (-4,-6) {$4$};
        \node[label={9},circle,draw,minimum size=1cm]
        (9) at (-2,-6) {$1$};

        \tikzstyle{cert}=[<-, dashdotted, blue, thick]
        \draw[thick] (1) -- (2);
        \draw[thick] (1) -- (3);
        \draw[cert] (2) -- (4);
        \draw[thick] (2) -- (5);
        \draw[thick] (3) -- (6);
        \draw[thick] (3) -- (7);
        \draw[cert] (4) -- (8);
        \draw[cert] (4) -- (9);
    \end{tikzpicture}
    \caption[Exemplo certificados do heap cinético após operação \textsc{change}]{Após a mudança de
    velocidade do elemento 2, que se encontra em \heap[$4$], os certificados
    \cert[$4$], \cert[$8$] e \cert[$9$] foram atualizados.}
    \label{fig:predeventheap}
\end{figure}

A operação \textsc{query\_max}, no Algoritmo \ref{torn:querymax},
consiste em devolver \torneio$[1]$, enquanto que a operação
\textsc{change}$(j, v)$, no Algoritmo \ref{torn:change}, consiste em
alterar a posição $x_0[j]$ para ${x_0[j] + (\mathit{speed}[j] -
v)\cdot now}$, a posição \textit{speed}$[j]$ para \textit{v} e
recalcular os eventuais certificados de que $j$ participa. Para
tanto, a partir da posição $i$ em que $j$ se encontra no vetor
\torneio, podemos recalcular \textit{cert}$[j]$ e então continuamos
visitando as partidas em que $j$ participou para atualizar os
certificados daqueles que perderam de $j$, acionando a rotina
\textsc{update} para fazer os devidos acertos em $Q$ correspondentes
a estas modificações.

\begin{algorithm}
    \caption{Função \textsc{query\_max}.} \label{torn:querymax}
    \begin{algorithmic}[1]
        \Function{query\_max}{\null}
            \State \Return \torneio$[1]$
        \EndFunction
    \end{algorithmic}
\end{algorithm}

\begin{algorithm}
    \caption{Função \textsc{change}.} \label{torneioi:change}
    \begin{algorithmic}[1]
        \Function{change}{$j, v$}
            \State $e \leftarrow$ \Call{getObject}{$j$}
            \State $e.x_0 \leftarrow e.x_0+~(e.\speed -~v)~\cdot~\now$;
            \State $e.\speed \leftarrow v$
            \State $i \leftarrow e.\lastmatch$
            \State \Call{update}{$e$}
            \While{$i < n$}
                \If{$\torneio[i] = \torneio[2i]$}
                    \State $i \leftarrow 2i$
                \Else
                    \State $i \leftarrow 2i + 1$
                \EndIf
                \State $k \leftarrow 2\cdot \floor{\frac{i}{2}}
                + ((i + 1)\mod2)$ \Comment{adversário}
                \State \Call{update}{$\torneio[k]$}
            \EndWhile
        \EndFunction
    \end{algorithmic}
\end{algorithm}

\begin{algorithm}
    \caption{Função \textsc{change}.} \label{torneioi:change}
    \begin{algorithmic}[1]
        \Function{change}{$j, v$}
            \State $e \leftarrow$ \Call{getObject}{$j$}
            \State $e.x_0 \leftarrow e.x_0+~(e.\speed -~v)~\cdot~\now$;
            \State $e.\speed \leftarrow v$
            \State $i \leftarrow e.\lastmatch$
            \State \Call{update}{$e$}
            \While{$i < n$}
                \If{$\torneio[i] = \torneio[2i]$}
                    \State $i \leftarrow 2i$
                \Else
                    \State $i \leftarrow 2i + 1$
                \EndIf
                \State $k \leftarrow 2\cdot \floor{\frac{i}{2}}
                + ((i + 1)\mod2)$ \Comment{adversário}
                \State \Call{update}{$\torneio[k]$}
            \EndWhile
        \EndFunction
    \end{algorithmic}
\end{algorithm}

%!TeX root=./torneio.tex

\FloatBarrier
\subsection{Inserção e remoção em torneio} \label{trni:secao}

Assim como fizemos na seção \ref{abb}, além das operações
\textsc{advance}$(t)$, \textsc{change}$(j, v)$ e
\textsc{query\_max}$()$, poderíamos querer dar suporte a operações
como:

\begin{itemize}
    \item \textsc{insert}$(v, x_t)\rightarrow$ insere um elemento
    com velocidade $v$ e valor $x_t$ no instante \now;
    \item \textsc{delete}$(i) \rightarrow$ remove o elemento $i$ no
    instante \now.
\end{itemize}
Agora, diferentemente da seção \ref{abb}, não utilizaremos uma nova
estrutura para dar suporte a essas operações, pois um torneio já
suporta operações como inserção e remoção de elementos em tempo
logarítmico. Porém, da maneira como se encontra a interface,
poderíamos ter problemas como espaços de memória ociosos após várias
remoções ou um gasto elevado de tempo redimensionando vetores para
que suportem a inserção de novos elementos. Dessa forma,
descreveremos a seguir alterações a serem feitas na interface para
evitar problemas como os citados.

Inicialmente o vetor que guarda o torneio começa com os elementos
ocupando as suas últimas posições e construímos o torneio de acordo
com o valor de cada elemento no instante $t = 0$.

Uma vez montado o torneio, construímos um certificado para cada
elemento no torneio. Agora os certificados não serão mais mantidos
em um vetor; serão mantidos junto aos elementos para facilitar a
inserção e remoção de certificados, já que estas vêm junto com a
inserção e remoção de elementos. O certificado de um elemento $e$ se
refere à relação estabelecida entre o elemento $e$ e o elemento $k$,
que é o elemento que venceu $e$ na última partida que $e$ disputou,
e consiste no instante de tempo em que o elemento $e$ passará a ter
um valor maior que o valor do elemento $k$, se esse instante for
maior que o instante atual. Do contrário, o certificado consiste em
$+\infty$.

Note que o elemento que está na primeira posição do torneio não é
vencido por ninguém no instante \now. Portanto, daremos o valor
$+\infty$ para o seu certificado.

Esses $n$ certificados serão colocados em uma fila com prioridades,
com o prazo de validade como prioridade. O certificado com menor
prazo de validade estará ocupando a primeira posição da fila. Na
verdade, como os certificados estarão diretamente ligados aos
elementos, colocaremos os elementos nessa fila.

Para descrever as implementações das operações
\textsc{advance}$(t)$, \textsc{change}$(j, v)$,
\textsc{query\_max}$()$, \textsc{insert}$(v, x_t)$ e
\textsc{delete}$(i)$ vamos estabelecer os nomes dos objetos,
variáveis e rotinas auxiliares utilizados:
\begin{enumerate}
    \item $n$: número de elementos no instante \now;
    \item \elemento: elemento com os seguintes atributos:
    \begin{enumerate}
        \item \id: atributo para identificar o elemento. Daqui em
        diante, usaremos elemento $i$ para se referir ao elemento
        cujo \id~é $i$;

        \item \speed: a velocidade do elemento;

        \item \initv: é o valor que o elemento possuía no
        instante $t = 0$;

        \item \cert: o tempo de validade do certificado do
        elemento;

        \item \pqpos: atributo que aponta para a posição do elemento
        na fila com prioridades;

        \item \lastmatch: atributo que aponta para a posição do
        vetor \torneio~em que o elemento disputou sua última
        partida.
    \end{enumerate}
    \item \torneio: vetor, de $2n - 1$ posições, que guarda
    apontadores para os elementos formando um torneio de acordo com
    seus valores no instante \now;

    \item \Q: fila com prioridades que contém os elementos, com o
    elemento com certificado de menor valor à frente;

    \item \textsc{insertTourn}$(e) \rightarrow$ insere $e$, que é um
    elemento, no torneio e atualiza os certificados necessários no
    processo;

    \item \textsc{deleteTourn}$(e) \rightarrow$ remove $e$, que é um
    elemento, do torneio e atualiza os certificados necessários no
    processo.
\end{enumerate}
Para a implementação das operações \textsc{change}$(j, v)$ e
\textsc{delete}$(i)$, precisamos de alguma maneira recuperar um
elemento baseado no seu \id. Para tal, podemos utilizar estruturas
como uma árvore binária balanceada de busca ou uma tabela de
dispersão. A seguir~estão três operações que nos ajudarão a
recuperar os elementos:
\begin{enumerate}
    \item \textsc{getObject}$(i)\rightarrow$ retorna o elemento $i$;
    \item \textsc{insertObject}$(e) \rightarrow$ insere $e$, que é
    um elemento, na estrutura;
    \item \textsc{deleteObject}$(e) \rightarrow$ remove $e$, que é
    um elemento, da estrutura.
\end{enumerate}
Para permitir a inserção e remoção de certificados, a interface da
fila com prioridades será reformulada, contando com duas operações
extras:
\begin{enumerate}
    \item \textsc{insertPQ}$(Q, e) \rightarrow$ insere $e$ na fila
    com prioridades $Q$;
    \item \textsc{deletePQ}$(Q, e) \rightarrow$ remove $e$ da fila
    com prioridades $Q$;
    \item \textsc{updatePQ}$(Q,e,t) \rightarrow$ muda o valor do
    certificado de $e$ para $t$ e atualiza a fila com prioridades
    $Q$;
    \item \textsc{minPQ}$(Q) \rightarrow$ devolve o elemento com o
    certificado de menor valor da fila com prioridades $Q$.
\end{enumerate}
A operação \textsc{updatePQ}$(Q,e,t)$ pode ser implementada em tempo
logarítmico no número de elementos em $Q$ graças ao atributo
\pqpos~dos elementos.

Um evento está associado a um certificado $(e, t)$ que expira no
instante $t$. Na implementação da operação \textsc{event}, no
Algoritmo \ref{torneioi:evento}, utilizaremos a rotina
\textsc{update}$(e)$, do Algoritmo \ref{torneioi:update}, que
calcula a nova validade $t$ do certificado do elemento $e$, e chama
a rotina $\textsc{updatePQ}(Q, e, t)$ e a rotina
\Call{expire}{$e,e'$} calcula a validade do certificado entre os
elementos $e$ e $e'$; se $e'$ é \textsc{null} retorna $+\infty$.

\input{conteudo/capitulos/maximo/algoritmos/torneio/insertdelete/update}

\input{conteudo/capitulos/maximo/algoritmos/torneio/insertdelete/event}

No trecho das linhas 5 - 11 do Algoritmo \ref{torneioi:evento}, o
resultado da partida entre o elemento~$j$ e seu adversário que se
encontra na posição $k$ de \torneio~é recalculado, e o certificado
correspondente é atualizado. Caso o resultado da partida tenha sido
alterado, a verificação se propaga para o nível de cima.

\input{conteudo/capitulos/maximo/algoritmos/torneio/insertdelete/query_max}

\input{conteudo/capitulos/maximo/algoritmos/torneio/insertdelete/change}

A operação \textsc{query\_max}$()$, no Algoritmo
\ref{torneioi:query}, consiste em devolver \torneio$[1]$, enquanto
que a operação \textsc{change}$(j, v)$, no Algoritmo
\ref{torneioi:change}, consiste em recuperar o elemento~$j$, alterar
seu atributo \initv~para $x_0+(\mathit{speed}-v)\cdot now$,
\textit{speed} para \textit{v} e recalcular os eventuais
certificados de que $j$ participa. Para tanto, a partir da posição
$i$ mais alta em que $j$ se encontra no vetor \torneio, podemos
recalcular \textit{cert}$[j]$ e então continuamos visitando as
partidas em que $j$ participou para atualizar os certificados
daqueles que perderam de $j$, acionando a rotina \textsc{update}
para fazer os devidos acertos em $Q$ correspondentes a estas
modificações.

A operação \textsc{insert}$(v, x_t)$, no Algoritmo
\ref{torneioi:insert}, consiste em criar um novo elemento,
inicializando seus atributos com os devidos valores, inseri-lo no
torneio e na estrutura que usamos para recuperá-lo, calcular o seu
certificado e inseri-lo na fila de prioridade. Uma importante
observação é que se \now~$\neq 0$, então $x_t \neq$~\initv. Para
calcular \initv, podemos utilizar a relação $x_t = now\cdot speed +
x_0 \Rightarrow x_0 = x_t - speed\cdot now$. Veja o exemplo da
Figura \ref{fig:torneioi:insert}. Utilizamos a função auxiliar
\textsc{insertTourn}$(e)$, do Algoritmo \ref{torneioi:inserttourn},
que consiste em criar uma nova partida, usando o elemento que está
na posição $n$ para completar a partida, depois subir para o nível
de cima no torneio, corrigindo os vencedores das partidas e
atualizando os certificados correspondentes. O certificado do
elemento inserido não será calculado nessa função, mas será
calculado posteriormente. No Algoritmo \ref{torneioi:inserttourn},
$\textsc{resize}()$ checa se \torneio~é capaz de suportar a inserção
de novos elementos e, se não for, redimensiona \torneio.

\input{conteudo/capitulos/maximo/figuras/torneio/insertdelete/exemplo}

\input{conteudo/capitulos/maximo/algoritmos/torneio/insertdelete/insert}

\input{conteudo/capitulos/maximo/algoritmos/torneio/insertdelete/insert_tourn}

\input{conteudo/capitulos/maximo/figuras/torneio/insertdelete/insert}

A operação \textsc{delete}$(i)$, no Algoritmo \ref{torneioi:delete},
consiste em recuperar o elemento~$i$, removê-lo da fila de
prioridade, do torneio e da estrutura que usamos para recuperá-lo
depois. Utilizamos a função auxiliar \textsc{deleteTourn}$(e)$, do
Algoritmo \ref{torneioi:deletetourn} que consiste em usar o perdedor
da partida travada entre os elementos que estão nas duas últimas
posições de \torneio~para substituir o elemento $e$. Além disso,
desfazemos essa partida para que os $n$ elementos continuem a ocupar
as $2n - 1$ primeiras posições do torneio após a remoção de $e$. O
perdedor substituirá o elemento $e$ na posição da primeira partida
de que $e$ participou. Todas as partidas desde essa posição, se
propagando para o nível de cima no caminho até a primeira posição,
serão recalculadas com os devidos certificados atualizados. Essa
propagação até a primeira posição é importante para que não hajam
resquícios do elemento removido no torneio. Veja o exemplo da Figura
\ref{fig:torneioi:delete}. Na implementação, no Algoritmo
\ref{torneioi:deletetourn}, a rotina \textsc{substitute}$(e)$ faz a
substituição citada retornando a posição da primeira partida de que
$e$ participou e a o rotina \Call{compare}{$i, j$} retorna se o
valor de $i$ é maior que o valor de $j$.

\input{conteudo/capitulos/maximo/algoritmos/torneio/insertdelete/delete}

\input{conteudo/capitulos/maximo/algoritmos/torneio/insertdelete/delete_tourn}

\input{conteudo/capitulos/maximo/figuras/torneio/insertdelete/delete}


\FloatBarrier
\subsection{Análise de desempenho}

O torneio cinético é uma estrutura \textit{responsiva}, pois o custo de
processar um certificado é $O(\lg^2{n})$. O custo de processar um certificado
corresponde a uma iteração da linha $3$ da operação \textsc{event}, que troca o
resultado da partida entre os dois elementos envolvidos no certificado e
atualiza o certificado em tempo $O(\lg{n})$. Além disso, existem $O(\lg{n})$
certificados a serem alterados com a propagação da verificação para o nível de
cima, resultando no custo $O(\lg^2{n})$.

O torneio cinético é uma estrutura \textit{eficiente}, pois a quantidade de
eventos internos é $O(n\lg{n})$ para trajetórias lineares. Como a quantidade de
eventos externos é $\Theta(n)$ temos uma razão $O(\lg{n})$, que é logarítmica em
$n$, tornando o torneio cinético uma estrutura eficiente em trajetórias
lineares.

Vamos mostrar que a quantidade de eventos internos é $O(n\lg{n})$. Cada evento
do torneio é ocasionado pela troca do resultado de uma partida. Sendo assim,
para contar a quantidade de eventos internos podemos contar a quantidade máxima
de vezes que o valor de um nó se altera. É fácil ver que os elementos que podem
ocupar um determinado nó só podem ser aqueles que estão nas folhas da subárvore
com raiz naquele nó. Se uma subárvore enraizada num nó possui $k$ folhas, então
podem ocorrer no máximo $k-1$ trocas naquele nó. O número de folhas que a
subárvore enraizada no nó possui é $k = 2^h$, sendo $h$ a altura do nó. Como o
torneio é uma árvore binária balanceada, sua altura é $\left\lceil \lg{n}
\right\rceil$. Temos $\left\lfloor \frac{n}{2^h}\right\rfloor$ nós de altura
$h$. Logo, sendo $Q$ a quantidade de eventos:
\begin{align*}
    Q & \leq \displaystyle\sum_{h = 0}^{\left\lceil \lg{n}\right\rceil} \dfrac{n}{2^h}\cdot (2^h - 1) \\\nonumber
    & \leq \displaystyle\sum_{h = 0}^{\left\lceil \lg{n}\right\rceil} n - \displaystyle\sum_{h = 0}^{\left\lceil \lg{n}\right\rceil} \dfrac{n}{2^h} \\ \nonumber
    & \leq \displaystyle\sum_{h = 0}^{\left\lceil \lg{n}\right\rceil} n \\ \nonumber
    & \leq n\left\lceil \lg{n}\right\rceil + n \\\nonumber
    & = O(n \lg{n})\nonumber
\end{align*}

O torneio cinético é uma estrutura \textit{compacta}, pois tem no máximo $n$
certificados na fila com prioridades, um para cada objeto.

O torneio cinético é uma estrutura \textit{local}, pois um elemento está envolvido
em $O(\lg{n})$ certificados da fila ao mesmo tempo.

As operações \textsc{insert} e \textsc{delete} custam $O(\lg^2{n})$, por contas
respectivas operações \textsc{insertTourn} e \textsc{deleteTourn} que atualizam
$O(\lg{n})$ posições com custo $O(\lg{n})$. Além disso, também vale observar que
determinadas chamadas de \textsc{insertTourn} podem ter um custo $O(n)$, caso
seja necessário redimensionar o vetor na operação \textsc{resize}, que dobra o
tamanho de \textit{tourn}.



%!TeX root=../../tcc.tex
\chapter{Par mais próximo}

Considere o seguinte problema cinético. São dados $n$ pontos
movendo-se linearmente no plano. Cada ponto é representado por um
par $(s_0, \vec{v})$ onde $s_0 = (x_0, y_0)$ é a sua posição inicial
e $\vec{v} = (v_x, v_y)$ um vetor velocidade. A posição de um
determinado ponto $p$, num instante arbitrário $t \geq 0$, é $s_p =
(x_p, y_p) = (x_0, y_0)~+~t\cdot \vec{v}$. Queremos saber o par $(p,
q)$ cuja distância $d(p, q) = \sqrt{(x_p - x_q)^2 + (y_p - y_q)^2}$
é~mínima, num instante arbitrário $t \geq 0$.

Por exemplo, se tivermos $5$ pontos na coleção, representados na
figura \ref{fig:parestatico:exemplo}: $((1, 0), (2, 1))$, $((5, -1),
(-1, 2))$, $((0, 2), (1, -1))$, $((3, 2), (1, -2))$ e $((3, 1), (-1,
0))$.

\begin{figure}[H]
    \centering
    \begin{tikzpicture}[thick, scale=0.7]
        \node[label={1},circle,draw,minimum size=1cm]
        (1) at (0,0) {$3$};
        \node[label={2},circle,draw,minimum size=1cm]
        (2) at (-4,-2) {$3$};
        \node[label={3},circle,draw,minimum size=1cm]
        (3) at (4,-2) {$4$};
        \node[label={4},circle,draw,minimum size=1cm]
        (4) at (-6,-4) {$1$};
        \node[label={5},circle,draw,minimum size=1cm]
        (5) at (-2,-4) {$3$};
        \node[label={6},circle,draw,minimum size=1cm]
        (6) at (2,-4) {$4$};
        \node[label={7},circle,draw,minimum size=1cm]
        (7) at (6,-4) {$6$};
        \node[label={8},circle,draw,minimum size=1cm]
        (8) at (-7,-6) {$8$};
        \node[label={9},circle,draw,minimum size=1cm]
        (9) at (-5,-6) {$1$};
        \node[label={10},circle,draw,minimum size=1cm]
        (10) at (-3,-6) {$2$};
        \node[label={11},circle,draw,minimum size=1cm]
        (11) at (-1,-6) {$3$};
        \node[label={12},circle,draw,minimum size=1cm]
        (12) at (1,-6) {$4$};
        \node[label={13},circle,draw,minimum size=1cm]
        (13) at (3,-6) {$5$};
        \node[label={14},circle,draw,minimum size=1cm]
        (14) at (5,-6) {$6$};
        \node[label={15},circle,draw,minimum size=1cm]
        (15) at (7,-6) {$7$};
        \node[label={16},circle,draw,minimum size=1cm]
        (16) at (-8,-8) {$8$};
        \node[label={17},circle,draw,minimum size=1cm]
        (17) at (-6,-8) {$9$};

        \draw[thick] (1) -- (2);
        \draw[thick] (2) -- (4);
        \draw[thick] (4) -- (8);
        \draw[thick] (4) -- (9);
        \draw[thick] (8) -- (16);
        \draw[thick] (8) -- (17);
        \draw[thick] (2) -- (5);
        \draw[thick] (5) -- (10);
        \draw[thick] (5) -- (11);
        \draw[thick] (1) -- (3);
        \draw[thick] (3) -- (6);
        \draw[thick] (3) -- (7);
        \draw[thick] (6) -- (12);
        \draw[thick] (6) -- (13);
        \draw[thick] (7) -- (14);
        \draw[thick] (7) -- (15);
    \end{tikzpicture}
    \caption[Representação da estrutura torneio]{Torneio com $9$
        elementos em que $3$ é o elemento com valor máximo.}
    \label{fig:torneio:exemplo}
\end{figure}

Queremos dar suporte às seguintes operações:
\begin{itemize}
    \item \textsc{advance}$(t)$ $\rightarrow$ avança o tempo corrente
    para $t$;
    \item \textsc{change}$(j, \vec{v})$ $\rightarrow$ altera a
    velocidade do ponto $j$ para $\vec{v}$;
    \item \textsc{query\textunderscore closest}$()$ $\rightarrow$
    devolve os pontos que formam o par mais próximo no instante
    atual.
\end{itemize}

%!TeX root=./par.tex

\FloatBarrier


\section{Algoritmo estático}\label{sec:algoritmo-estatico}

O algoritmo que será aqui apresentado foi proposto por Basch, Guibas
e Hershberger [\cite{BASCH19991}] e admite uma boa cinetização, usando a ideia de linha
de varredura.

O algoritmo é baseado na ideia de dividir o plano, para cada ponto, em seis cones iguais.
Os cones são delimitados pela reta paralela ao eixo $y$ que passa pelo ponto e pelas retas $x \pm
30^\circ$, isto é, as retas que passam pelo ponto e formam $\pm 30^\circ$ com o eixo $x$ como mostra a
Figura~\ref{fig:parestatico:cones}.

\begin{figure}[H]
    \centering
    \begin{tikzpicture}[thick]
        \draw (0, -3) -- (0, 3) node[anchor=north west] {$y$};
        \draw (-3, 1.7302) -- (3, -1.7302)
        node[anchor=north west] {$x - 30^\circ$};
        \draw (-3, -1.7302) -- (3, 1.7302)
        node[anchor=south west] {$x + 30^\circ$};
        \node[label=250:$p$] (p) at (0, 0) {\textbullet};
    \end{tikzpicture}
    \caption[Exemplo de cones do algoritmo estático]{A reta paralela
    ao eixo $y$ que passa por $p$ e as retas $x \pm 30^\circ$.}
    \label{fig:parestatico:cones}
\end{figure}

Tendo dividido o plano em cones, a ideia é achar o ponto mais próximo de $p$ dentro de cada um
desses cones.
Se assim o fizermos para todos os pontos, um desses pares possui a menor distância entre si e será
um par mais próximo que buscamos.

Se $(p, q)$ formam um par mais próximo, então $(q, p)$ também forma um par mais próximo;
na verdade, são o mesmo par.
Dessa maneira, não precisamos dos seis cones para buscar os pares, somente de três deles.
Para uma varredura da direita para a esquerda, apenas buscaremos os pares mais próximos nos três
cones à direita de p.

Vamos começar analisando o cone cujo eixo central é paralelo ao eixo $x$.
Chamaremos esse cone de \textit{dominância de p} e o representaremos por $\Dom(p)$.
Consideraremos que um ponto em cima da linha $x + 30^\circ$ pertence a $\Dom(p)$ e um ponto em cima de
$x - 30^\circ$ não pertence a $\Dom(p)$ como mostra a Figura~\ref{fig:parestatico:dominancia}.
O mesmo algoritmo poderá ser aplicado aos outros dois cones se rotacionarmos o sistema de
coordenadas~$\pm 60^\circ$.

\begin{figure}[H]
    \centering
    \begin{tikzpicture}[thick, scale=0.8]
        % \draw (0, -3) -- (0, 3) node[anchor=north west] {$y$};
        \draw (0, 0) -- (4, 2.309);
        \draw[dashed] (0, 0) -- (4, -2.309);
        \draw (0, 0) circle (2pt) node[label=250:$p$] {};
        \node[label=0:$q$] (q) at (1, 0) {\textbullet};
        \node[label=250:$r$] (r) at (3, 1.73) {\textbullet};
        \draw (3, -1.73) circle (2pt) node[label=250:$s$] {};
        % \node[label=250:$s$] (s) at (3, -1.73);
    \end{tikzpicture}
    \caption[Exemplo de $\Dom(p)$]{Os pontos $q$ e $r$ pertencem a $\Dom(p)$,
    mas o ponto $s$ não.}
    \label{fig:parestatico:dominancia}
\end{figure}

A Figura~\ref{fig:parestatico:conjuntos}, inspirada em~\cite{BASCH19991}, ilustra cada uma das
definições a seguir.
Definiremos como $\Maxima(p)$ o conjunto dos pontos à direita de $p$ que não pertencem à
dominância de nenhum ponto à direita de $p$.
Isso nos permite definir o conjunto de \textit{candidatos} de $p$ representado por $\Cands(p)$:
$\Cands(p) = \Dom(p) \cap\Maxima(p)$, ou seja, os candidatos de $p$ são aqueles pontos à
direita de $p$ que não pertencem à dominância de nenhum ponto à direita de $p$ e
pertencem à dominância de $p$.
Chamaremos o ponto de $\Maxima$ de menor ordenada que está acima de $\Dom(p)$ de $\up(p)$ e
chamaremos o ponto de $\Maxima$ de maior ordenada que está abaixo de $\Dom(p)$ de $\low(p)$.
Caso não existam tais pontos, $\up(p)$ e $\low(p)$ são \nnull.
Os pontos de $\Maxima$ estritamente entre $\low(p)$ e $\up(p)$ são justamente os de $\Cands(p)$.
Dentre os candidatos de $p$, chamaremos o ponto com menor coordenada $x$ de $\lcand(p)$.

\begin{figure}[H]
    \centering
    \begin{tikzpicture}[thick, scale=0.8]
        \node[label={[label distance = -3mm]160:$p$}]
        at (0.00, 0.00) {\textbullet};
        \node[label={[label distance = -3mm]160:$a$}]
        (a) at (3.00, 3.00) {\textbullet};
        \node[label={[label distance = -3mm]160:$b$}]
        (b) at (2.50, 2.00) {\textbullet};
        \node[label={[label distance = -3mm]160:$c$}]
        (c) at (3.50, 1.00) {\textbullet};
        \node[label={[label distance = -3mm]160:$d$}]
        at (4.00, -0.60) {\textbullet};
        \node[label={[label distance = -3mm]160:$e$}]
        at (5.00, -2.00) {\textbullet};
        \node[label={[label distance = -3mm]220:$f$}]
        (f) at (4.50, -3.00) {\textbullet};

        % e cone
        \draw (5.00, -2.00) -- (6.00, -2.58);
        \draw (5.00, -2.00) -- (6.00, -1.42);
        % f cone
        \draw (4.50, -3.00) -- (6.00, -3.87);
        \draw (4.50, -3.00) -- (5.62, -2.36);
        % d cone
        \draw (4.00, -0.60) -- (5.71, -1.59);
        \draw (4.00, -0.60) -- (6.00, 0.55);
        % c cone
        \draw (3.50, 1.00) -- (5.14, 0.06);
        \draw (3.50, 1.00) -- (6.00, 2.44);
        % a cone
        \draw (3.00, 3.00) -- (4.98, 1.86);
        \draw (3.00, 3.00) -- (6.00, 4.73);
        % b cone
        \draw (2.50, 2.00) -- (3.87, 1.21);
        \draw (2.50, 2.00) -- (3.62, 2.64);
        % p cone
        \draw[line width = 0.5mm] (0.00, 0.00) -- (4.85, -2.80);
        \draw[line width = 0.5mm] (0.00, 0.00) -- (2.98, 1.72);

        \draw[dashed] (6,-2.5) -- (8, -2.5)
        node[anchor=west, label=90:$\Cands(p)$] {};
        \draw[dashed] (8, 1.5) -- (8, -2.5);
        \draw[dashed] (5,1.5) -- (8, 1.5);

        \draw[dashed] (4, -4) -- (10, -4);
        \draw[dashed] (10, -4) -- (10, 4);
        \draw[dashed] (5, 4) -- (10, 4)
        node[anchor=south, label=$\Maxima(p)$] {};

        \node[label={[label distance = -3mm]270:$\low(p)$}]
        (low) at (3, -4) {};
        \draw[->] (low) edge[out=90,in=160] (f);

        \node[label={[label distance = -3mm]270:$\lcand(p)$}]
        (lc) at (2, 0) {};
        \draw[->] (lc) edge[out=90,in=200] (c);

        \node[label={[label distance = -3mm]90:$\up(p)$}]
        (up) at (1.5, 3) {};
        \draw[->] (up) edge[out=270,in=180] (b);
    \end{tikzpicture}
    \caption[Exemplo dos conjuntos utilizados no algoritmo]{Os pontos
        $c$, $d$ e $e$ pertencem a $\Cands(p)$, e todos os pontos exceto
        $p$ pertencem a $\Maxima(p)$. O ponto $b$ é $\up(p)$ e o ponto
        $f$ é $\low(p)$. O ponto $c$ é $\lcand(p)$.}
    \label{fig:parestatico:conjuntos}
\end{figure}

Consideraremos apenas os pares $(p, \lcand(p))$ como possíveis candidatos a par mais próximo.
Caso, para algum $p$, mais de um ponto atenda à condição de ser $\lcand(p)$ poderemos escolher
qualquer um deles como $\lcand(p)$, pois, em um caso em que há mais de um possível $\lcand(p)$,
esses pontos formarão um par mais próximo entre si do que o par $(p, \lcand(p))$, como por exemplo
na Figura~\ref{fig:parestatico:lcands}.

\begin{figure}[H]
    \centering
    \begin{tikzpicture}[thick, scale=0.8]
        % \draw (0, -3) -- (0, 3) node[anchor=north west] {$y$};
        \draw (0, 0) -- (4, 2.309);
        \draw[dashed] (0, 0) -- (4, -2.309);
        \node[label=250:$p$] (p) at (0, 0) {\textbullet};
        \node[label=0:$q$] (q) at (4, -2) {\textbullet};
        \node[label=250:$r$] (r) at (4, 0) {\textbullet};
        \draw[dotted] (4, -2) -- (4, 0);
        \draw[dotted] (0, 0) -- (r);
        \draw[dotted] (0, 0) -- (q);
    \end{tikzpicture}
    \caption[Exemplo de $\lcand(p)$]{A distância de $r$ até $q$ é menor do que a
    distância de $p$ até $r$ e do que a distância de $p$ até $q$.}
    \label{fig:parestatico:lcands}
\end{figure}

O Algoritmo~\ref{parestatico:horizontal} descreve a sequência de operações a serem feitas para
achar o par mais próximo em alguma das ordens $(-60^\circ, 0^\circ, 60^\circ)$ representadas pelo
ângulo $\theta$, dado em radianos.
Antes da rotina ser chamada, os pontos devem ser ordenados de acordo com a sua coordenada $x$.
Os pontos são então processados da direita para a esquerda.
No algoritmo, $a$ e $b$ são os pontos que representam o par mais próximo.
Se $p$ ou $q$ são nulos, $d(p,q)$ retorna $+\infty$.

A cada iteração do Algoritmo~\ref{parestatico:horizontal}, $\Maxima$ é igual a $\Maxima(p)$.
Na nossa implementação, $\Maxima$ estará armazenado em uma árvore binária de busca, mais
especificamente em uma \textit{splay tree} cuja chave é a coordenada $y$ dos pontos, determinada
de acordo com o valor de $\theta$.
Com isso, podemos buscar por $\up(p)$ e $\low(p)$ em tempo logarítmico, bem como podemos retirar
$\Cands(p)$ de $\Maxima$ em tempo logarítmico, isto é, atualizar $\Maxima$ de maneira que $\Maxima
= \Maxima \setminus \Cands(p)$.

\begin{algorithm}
    \caption[Algoritmo \textsc{closest\_pair} do par mais próximo]{Função \textsc{closest\_pair}$(p, n, \theta)$.}
    \label{alg:par-estatico:horizontal}
    \begin{algorithmic}[1]
        \Function{closest\_pair}{$p, n, \theta$}
            \State \Call{heapsort}{$p, n, \theta$} \Comment{$p[1].x > \cdots > p[n].x$}
            \State $(a,b) \leftarrow (\nnull, \nnull)$
            \State $\Maxima \leftarrow \varnothing$
            \For{$i \leftarrow 1\To n$}
                \State $\Cands \leftarrow \Maxima\cap \Dom(p[i])$
                \State $\Maxima \leftarrow (\Maxima \setminus \Cands) \cup \lk p[i]\rk$
                \State $\lcand \leftarrow \Call{min\_x}{\Cands}$
                \If{$d(p[i], \lcand) < d(a, b)$}
                    \State $(a, b) \leftarrow (p[i], \lcand)$
                \EndIf
            \EndFor
            \State \Return{$(a, b)$}
        \EndFunction
    \end{algorithmic}
\end{algorithm}

Para descrever a implementação do algoritmo, já considerando as versões rotacionadas dele, iremos
antes precisar estabelecer os nomes das variáveis e rotinas auxiliares utilizadas.
São elas:
\begin{enumerate}
    \item $n$: o número de pontos dados;
    \item \textit{point}: um ponto com os seguintes atributos:
    \begin{enumerate}
        \item $x$: coordenada $x$ do ponto;
        \item $y$: coordenada $y$ do ponto.
    \end{enumerate}
    \item \raiz: raiz da splay tree;
    \item \no: objeto que compõe a árvore binária de busca,
    atributos:
    \begin{enumerate}
        \item \esq$:$ aponta para a raiz da subárvore esquerda do nó.
        A subárvore esquerda é composta apenas por pontos que possuem
        \textit{value}~com menor ordenada que a \textit{value}~do nó;
        \item \dir$:$ aponta para a raiz da subárvore direita do nó.
        A subárvore direita é composta apenas por pontos que possuem
        \textit{value}~com ordenada maior ou igual que a \textit{value}~do nó;
        \item \pai$:$ aponta para o nó que é pai deste nó;
        \item \textit{value}$:$ aponta para um ponto.
    \end{enumerate}
    \item \angulo: ângulo de rotação do sistema de coordenadas;
    \item \pontos: vetor de $n$ posições que guarda os pontos;
    \item \textsc{getX}$(p, \angulo) \rightarrow$ retorna a coordenada $x$
    de um ponto $p$ baseada no ângulo de rotação \angulo;
    \item \textsc{getY}$(p, \angulo) \rightarrow$ retorna a coordenada $y$
    de um ponto $p$ baseada no ângulo de rotação \angulo;
    \item \textsc{heapsort}$(\pontos, n, \angulo) \rightarrow$ ordena o vetor \pontos,
    utilizando o algoritmo \textit{heapsort}, de acordo com a coordenada $x$ de cada ponto cujo
    valor é retornado pela rotina \textsc{getX}$(p, \angulo)$.
\end{enumerate}

Para um ponto $(r, \phi)$ em coordenadas polares, $x = r\cdot \cos{(\phi)}$ e $y = r\cdot
\sin{(\phi)}$.

Rotacionar o sistema de coordenadas por $\theta$ é o mesmo que transformar $\phi$ em $\phi -
\theta$, veja a Figura~\ref{fig:parestatico:rotacao}.
Isso significa que agora as novas coordenadas são descritas como:
\begin{align*}
    x^* & = r\cdot \cos{(\phi - \theta)}= r\cdot \cos{(\phi)}\cdot \cos{(\theta)}
    + r\cdot \sin{(\phi)}\cdot \sin{(\theta)} = x\cdot \cos{(\theta)} + y\cdot \sin{(\theta)} \\
    y^* & = r\cdot \sin{(\phi - \theta)} = r\cdot \sin{(\phi)}\cdot \cos{(\theta)}
    - r\cdot \cos{(\phi)}\cdot \sin{(\theta)} = y\cdot \cos{(\theta)} - x\cdot \sin{(\theta)}.
\end{align*}
Os valores $x^*$ e $y^*$ são os valores, respectivamente, retornados por \textsc{getX}$(p,
\angulo)$ e \textsc{getY}$(p, \angulo)$ para $\theta = \angulo$.

\begin{figure}[H]
    \centering
    \begin{tikzpicture}[thick]
        \coordinate (a) at (0, 0);
        \coordinate (b) at (4, 0);
        \coordinate (c) at (2.828, 2.828);
        \coordinate (p) at (1, 2);
        \draw[thick,->] (0,0) -- (4,0)
        node[anchor=north west] {$x$};
        \draw[thick,->] (0,0) -- (0,4)
        node[anchor=south east] {$y$};
        \draw[thick,->, dashed] (0,0) -- (2.828,2.828)
        node[anchor=north west] {$x'$};
        \draw[thick,->, dashed] (0,0) -- (-2.828,2.828)
        node[anchor=south east] {$y'$};
        \pic [draw, angle radius = 0.5cm] {angle = b--a--c};
        \node[anchor=west, label={[label distance = 0mm]180:$\theta$}]
        (angl) at (1, 0.3) {};
        \node[label=90:$p$] at (p) {\textbullet};
        \draw[dashed,->] (a) -- node[above] {$r$} (p);
        \pic [draw, angle radius = 1cm] {angle = b--a--p};
        \node[anchor=west, label={[label distance = -3mm]180:$\phi$}]
        (angp) at (1.2, 0.7) {};
    \end{tikzpicture}
    \caption{O ponto $p$ está numa inclinação de $\phi - \theta$ radianos
    em relação a reta que passa pela origem e por $x'$.}
    \label{fig:parestatico:rotacao}
\end{figure}

A interface da \textit{splay tree} que vai armazenar $\Maxima$, e cuja chave é a coordenada $y$ do
ponto, contará com as seguintes operações, além das usuais \textsc{insert}$(p)$ e
\textsc{splay}$(x)$:
\begin{enumerate}
    \item \textsc{successor}$(p) \rightarrow$ busca pelo nó
    cuja chave é $\up(p)$ na \textit{splay tree}.
    Esse nó corresponde ao sucessor de $p$ na árvore;
    \item \textsc{predecessor}$(p) \rightarrow$ busca pelo nó cuja chave é $low(p)$ na \textit{splay tree}.
    Esse nó corresponde ao predecessor de $p$ na árvore;
    % \item \textsc{splay}$(x) \rightarrow$ dá um \textit{splay} no nó $x$;
    \item \textsc{lcand}$(p) \rightarrow$ calcula $\Cands(p)$, remove da
    \textit{splay tree} e determina $\lcand(p)$, que pode ser \nnull;
    % \item \textsc{clearAll}$() \rightarrow$ remove todos os nós
    % da \textit{splay tree}.
\end{enumerate}

% As operações \textsc{insert}$(p)$, \textsc{splay}$(x)$ e
% \textsc{clearAll}$()$ não possuem nenhuma diferença quanto
% à sua implementação. São operações comuns de uma
% \textit{splay tree}. Portanto, focaremos em explicar as
% operações successor$(p)$, predecessor$(p)$ e lcand$(p)$.

No Algoritmo~\ref{parestatico:successor} e no Algoritmo~\ref{parestatico:predecessor}, a rotina
\textsc{checkLine}$(p, q, \theta)$ retorna se o ponto$q$ está à esquerda, sobre ou à direita da
reta $r$.
A reta $r$ é a reta que passa por $p$ e faz um ângulo de $\theta$ radianos com o eixo $x$.
Para $q$ à esquerda de $r$ o retorno é $1$, para $q$ sobre $r$ o retorno é $0$ e para $q$ à
direita de $r$, o retorno é $-1$.

O Algoritmo~\ref{parestatico:lcand} implementa a função \textsc{lcand}$(p)$.
Sabemos que $\lcand(p)$ é o elemento de $\Cands(p)$ com menor coordenada $x$ e que $\Cands(p)$ é
limitado por $\low(p)$ e $\up(p)$.
A ideia para retirar $\Cands(p)$ da árvore é reorganizá-la de modo que $\low(p)$ seja a raiz e
$\up(p)$ o filho direito da raiz.
Dessa forma $\Cands(p)$ é a subárvore esquerda do filho direito da raiz, veja a
Figura~\ref{fig:parestatico:loweup}.
Nem sempre tal configuração é possível, mas o algoritmo tratará dos casos de borda que são quando
$\low(p)$, ou $\up(p)$, ou ambos, não existem.
A rotina \textsc{split}$(x)$ separa a subárvore de raiz $x$ da \textit{splay tree} e retorna a
raiz dessa nova árvore.
A rotina \textsc{min\_x}$(z)$ retorna o ponto com menor coordenada $x$ da árvore de raiz~$z$.

\begin{figure}
    \centering
    \begin{tikzpicture}[thick]
        \tikzstyle{every node}=[font=\small]
        \node[every node,circle,draw, minimum size=1cm] (low) at (0, 0) {$l$};
        \node[every node,circle,draw, minimum size=1cm] (up) at (2, -2) {$u$};
        \node (esq) at (-2, -2) {$\gamma$};
        \node[every node] (esqesq) at (1, -4) {$\alpha$};
        \node[every node] (esqdir) at (3, -4) {$\beta$};
        \draw (low) -- (up);
        \draw (esqesq) -- (up);
        \draw (esqdir) -- (up);
        \draw (low) -- (esq);
    \end{tikzpicture}
    \caption{Na figura, $l$ é $low(p)$ e $u$ é $up(p)$. A subárvore
    $\alpha$ contém todos os pontos que estão entre $low(p)$ e
    $up(p)$ e, portanto, corresponde ao conjunto $Cands(p)$.}
    \label{fig:parestatico:loweup}
\end{figure}

\begin{algorithm}
    \caption[Algoritmo \textsc{successor} do par mais próximo]{Função \textsc{successor}$(p)$.}
    \label{alg:par-estatico:successor}
    \begin{algorithmic}[1]
        \Function{successor}{$p$}
            \State $x \leftarrow \raiz$ \Comment{raiz da \textit{splay tree}}
            \State $\up \leftarrow \nnull$
            \While{$x \neq \nnull$}
                \State $y \leftarrow x$
                \If{$\Call{checkLine}{p, x.\key, \pi/6}= -1$}
                    \State $x \leftarrow x.\dir$
                \Else
                    \State $\up \leftarrow x$
                    \State $x \leftarrow x.\esq$
                \EndIf
            \EndWhile
            \If{$y \neq \nnull$}
                \Comment{aciona \textsc{splay} no último nó visitado}
                \State \Call{splay}{$y$}
            \EndIf
            \State \Return{$\up$}
        \EndFunction
    \end{algorithmic}
\end{algorithm}

\begin{algorithm}[H]
    \caption[Algoritmo \textsc{predecessor} do par mais próximo]{Função \textsc{predecessor}$(p)$.}
    \label{alg:par-estatico:predecessor}
    \begin{algorithmic}[1]
        \Function{predecessor}{$p$}
            \State $x \leftarrow \raiz$ \Comment{raiz da \textit{splay tree}}
            \State $\low \leftarrow \nnull$
            \While{$x \neq \nnull$}
                \State $y \leftarrow x$
                \If{$\Call{checkLine}{p, x.\key, -\pi/6}\leq 0$}
                    \State $x \leftarrow x.\esq$
                \Else
                    \State $\low \leftarrow x$
                    \State $x \leftarrow x.\dir$
                \EndIf
            \EndWhile
            \If{$y \neq \nnull$}
                \Comment{aciona \textsc{splay} no último nó visitado}
                \State \Call{splay}{$y$}
            \EndIf
            \State \Return{$\low$}
        \EndFunction
    \end{algorithmic}
\end{algorithm}

\begin{algorithm}
    \caption{Função lcand$(p)$.} \label{parestatico:lcand}
\begin{algorithmic}[1]
    \Function{lcand}{$p$}
        \State $r \leftarrow root$
        \State $low \leftarrow \Call{predecessor}{p}$
        \If{$low \neq NULL$}
            \State \Call{splay}{$low$}
            \State $r \leftarrow \Call{split}{low.right}$
        \EndIf
        \State $up \leftarrow \Call{successor}{p}$
        \If{$up \neq NULL$}
            \State $\Call{splay}{up}$
            \State $r \leftarrow \Call{split}{up.left}$
        \EndIf
        \If{$up \neq NULL$ \AND $low \neq NULL$}
            \State $low.right \leftarrow up$
            \State $up.parent \leftarrow low$
        \EndIf
        \State \Return{\Call{min\_x}{$r$}}
    \EndFunction
\end{algorithmic}
\end{algorithm}

O Algoritmo~\ref{alg:par-estatico:closest} implementa a função \textsc{query\_closest} que retorna um
par $(a,b)$ que possui distância mínima em \pontos.

\begin{algorithm}[H]
    \caption{Função \textsc{query\_closest}.} \label{alg:par-estatico:closest}
    \begin{algorithmic}[1]
        \Function{query\_closest}{\null}
            \State $(m,n) \leftarrow (\nnull, \nnull)$
            \State \angulo~$\leftarrow -\frac{\pi}{3}$
            \While{\angulo~$\leq \frac{\pi}{3}$}
                \State \Call{heapsort}{$\pontos, n, \angulo$} \Comment{$\pontos.x[1] > \cdots >
                \pontos.x[n]$}
                \For{$i \leftarrow 1$\To$n$}
                    \State $p \leftarrow $~\pontos$[i]$
                    \State $\lcand \leftarrow $ \Call{lcand}{$p$}
                    \State \Call{insert}{$p$}
                    \If{$d(p, \lcand) < d(a, b)$}
                        \State $(a, b) \leftarrow (p, \lcand)$
                    \EndIf
                \EndFor
                \State \angulo~$\leftarrow$~\angulo~$ + \frac{\pi}{3}$
                \State \Call{clearAll}{\null} \Comment{esvazia a \textit{splay tree}}
            \EndWhile
            \State \Return{$(m,n)$}
        \EndFunction
    \end{algorithmic}
\end{algorithm}


%!TeX root=./par.tex

\FloatBarrier


\section{Algoritmo cinético}\label{sec:algoritmo-cinetico}

Para ``cinetizar'' o algoritmo estático, utilizaremos certificados para assegurar que as nossas
estruturas permanecerão corretas.
Primeiramente, teremos os certificados das três \textit{listas ordenadas cinéticas}, conforme a
Seção~\ref{sec:heap-cinetico}, que guardarão a ordem dos pontos de acordo com os eixos $x$, $x +
60^\circ$ e $x - 60^\circ$.

Para garantir qual, dentre os pares $(p, \lcand(p))$, é o par mais próximo, usaremos um
\textit{torneio cinético com inserção e remoção}, conforme a Seção~\ref{subsec:torneioi:secao},
com respeito ao mínimo em vez de ao máximo.
Temos um total de $3n$ pares, pois consideraremos também os pares $(p, \lcand(p))$ em que
$\lcand(p)$ é nulo e os certificados destes serão $+\infty$.

Também precisaremos manter informação guardada para atualizar com eficiência mudanças provocadas
por trocas na ordem dos pontos em relação a um dos três eixos.
Por exemplo, uma troca na ordem dos pontos pode acarretar numa mudança nos conjuntos $\Cands(p)$ e
$\Cands(q)$.
Mudanças nesses conjuntos ocorrerão quando $q = \up(p)$, $q = \low(p)$ ou $q \in \Cands(p)$.
Portanto, para que consigamos manter $\lcand(p)$ de maneira eficiente, cada ponto terá três árvores
binárias de busca associadas a ele com os conjuntos $\Cands(p)$, $\Hits_{up}(p)$ e $\Hits_{low}(p)
$.
A árvore $\Cands(p)$ guarda os pontos que pertencem ao conjunto $\Cands(p)$ ordenados pela
coordenada $y$.
A árvore $\Hits_{up}(p)$ guarda os pontos $q$ tais que $\up(q) = p$, ordenados pela coordenada $x$.
Similarmente, a árvore $\Hits_{low}(p)$ guarda os pontos $q$ tais que $\low(q) = p$, ordenados
pela coordenada $x$.
Utilizaremos as árvores $\Cands(p)$, $\Hits_{up}(p)$ e $\Hits_{low}(p)$ para cada um dos eixos,
logo, para cada ponto $p$, haverão nove \textit{splay trees} no total.

Cada uma das nove árvores têm sua raiz apontando para o nó $p$, e cada nó das árvores aponta para
o seu nó pai.
Na árvore $\Cands(p)$, cada nó deve apontar para o descendente que contém o ponto mais à esquerda
na ordenação horizontal.
Na nossa implementação, as árvores serão \textit{splay trees}.
Essas estruturas contêm toda a informação necessária para que mantenhamos nossas estruturas
atualizadas e, consequentemente, o par mais próximo do conjunto.

Na implementação do algoritmo, inicialmente inserimos os pontos nas três listas ordenadas.
Uma vez que as listas estejam montadas, percorremos os pontos da direita para a esquerda
preenchendo as estruturas $\Cands(p)$, $\Hits_{up}(p)$ e $\Hits_{low}(p)$ para cada ponto $p$ e
para cada um dos eixos.
Esta etapa é feita da mesma forma que foi apresentada na seção sobre o algoritmo estático, a
Seção~\ref{sec:algoritmo-estatico}.

A medida que as estruturas $\Cands(p)$ são inicializadas, inserimos o par $(p, \lcand(p))$ no
torneio.
Quando todos os pares forem inseridos no torneio, realizamos as partidas e calculamos os
certificados.
O par $(p, q)$ da partida que possuir menor distância é considerado o vencedor.

Todos os certificados são colocados em uma fila de prioridade $Q$.
Os certificados inseridos na fila possuem quatro informações:
\begin{itemize}
    \item $t~\rightarrow$ instante de tempo em que o certificado
    expira.
    É utilizado como chave para a fila de prioridade.
    Desempates são tratados de maneira especial e serão explicados
    mais adiante;
    \item $p~\rightarrow$ um dos pontos envolvidos no evento
    representado pelo certificado.
    Caso seja um certificado de troca na ordenação, $p$ é o ponto mais à direita naquela ordenação;
    \item $q~\rightarrow$ o outro ponto envolvido no evento representado pelo certificado.
    Caso seja um certificado de troca na ordenação, $q$ é o ponto mais à esquerda naquela
    ordenação;
    \item tipo $ \rightarrow$ o tipo de evento que o certificado representa.
    Pode representar uma troca em uma das três ordenações, denominadas por \textit{H} (horizontal =
    $0^\circ$-ordem), \textit{U} (up = $+60^\circ$-ordem) e \textit{D} (down = $-60^\circ$-ordem) ou pode
    representar a vitória do par $(p, q)$ em uma partida do torneio.
\end{itemize}

Vamos agora falar de um evento em que ocorre uma mudança na ordem horizontal.
No primeiro caso, $p$ se encontra à esquerda e abaixo de $q$, veja a
Figura~\ref{fig:parcinetico:eventohorizontalabaixo}.
O caso em que $q$ está à esquerda de $p$ será tratado de maneira parecida.
O Algoritmo~\ref{alg:par-cinetico:eventohorizontal} implementa a sequência de operações referentes a
esse tipo de evento.

\begin{figure}[H]
    \centering
    \begin{tikzpicture}[thick, scale=0.35]
        \node[label={[label distance = -3mm]160:$p$}] at
        (2.00, 2.00) {\textbullet};
        \node[label={[label distance = -3mm]160:$q$}] at
        (3.00, 5.00) {\textbullet};
        \node[label={[label distance = -2mm]90:$a = t$}] at
        (6.00, 5.00) {\textbullet};
        \node[label={[label distance = -3mm]160:$b$}] at
        (8.00, 3.00) {\textbullet};
        \node[label={[label distance = -3mm]160:$c = w$}] at
        (7.00, 1.00) {\textbullet};
        \node[label={[label distance = -3mm]160:$d$}] at
        (9.00, -1.00) {\textbullet};
        \node[label={[label distance = -3mm]160:$e$}] at
        (5.00, 8.00) {\textbullet};
        \node[label={[label distance = -3mm]160:$f$}] at
        (6.00, 10.00) {\textbullet};
        \node[label={[label distance = -3mm]160:$g$}] at
        (6.00, -2.00) {\textbullet};
        % d cone
        \draw (9.00, -1.00) -- (15.00, -4.46);
        \draw (9.00, -1.00) -- (15.00, 2.46);
        % b cone
        \draw (8.00, 3.00) -- (11.96, 0.71);
        \draw (8.00, 3.00) -- (15.00, 7.04);
        % c cone
        \draw (7.00, 1.00) -- (9.73, -0.58);
        \draw (7.00, 1.00) -- (9.23, 2.29);
        % f cone
        \draw (6.00, 10.00) -- (13.06, 5.92);
        \draw (6.00, 10.00) -- (15.00, 15.20);
        % g cone
        \draw (6.00, -2.00) -- (15.00, -7.20);
        \draw (6.00, -2.00) -- (9.10, -0.21);
        % a cone
        \draw (6.00, 5.00) -- (8.73, 3.42);
        \draw (6.00, 5.00) -- (10.33, 7.50);
        % e cone
        \draw (5.00, 8.00) -- (8.10, 6.21);
        \draw (5.00, 8.00) -- (7.23, 9.29);
        % q cone
        \draw[line width = 0.5mm] (3.00, 5.00) -- (8.46, 1.85);
        \draw[line width = 0.5mm] (3.00, 5.00) -- (6.60, 7.08);
        % p cone
        \draw[line width = 0.5mm] (2.00, 2.00) -- (7.46, -1.15);
        \draw[line width = 0.5mm] (2.00, 2.00) -- (5.10, 3.79);

        \draw[dashed] (2, -3) -- (2, 8);
        \draw[dashed] (3, 3) -- (3, 8);

        \node at (18, 5) {$ \leftrightarrow$};

        \node[label={[label distance = -3mm]160:$p$}] at
        (23.00, 2.00) {\textbullet};
        \node[label={[label distance = -3mm]160:$q$}] at
        (22.00, 5.00) {\textbullet};
        \node[label={[label distance = -2mm]90:$a = w$}] at
        (26.00, 5.00) {\textbullet};
        \node[label={[label distance = -3mm]160:$b$}] at
        (28.00, 3.00) {\textbullet};
        \node[label={[label distance = -1mm]90:$c = t$}] at
        (27.00, 1.00) {\textbullet};
        \node[label={[label distance = -3mm]160:$d$}] at
        (29.00, -1.00) {\textbullet};
        \node[label={[label distance = -3mm]160:$e$}] at
        (25.00, 8.00) {\textbullet};
        \node[label={[label distance = -3mm]160:$f$}] at
        (26.00, 10.00) {\textbullet};
        \node[label={[label distance = -3mm]160:$g$}] at
        (26.00, -2.00) {\textbullet};
        % d cone
        \draw (29.00, -1.00) -- (35.00, -4.46);
        \draw (29.00, -1.00) -- (35.00, 2.46);
        % b cone
        \draw (28.00, 3.00) -- (31.96, 0.71);
        \draw (28.00, 3.00) -- (35.00, 7.04);
        % c cone
        \draw (27.00, 1.00) -- (29.73, -0.58);
        \draw (27.00, 1.00) -- (29.23, 2.29);
        % f cone
        \draw (26.00, 10.00) -- (33.06, 5.92);
        \draw (26.00, 10.00) -- (35.00, 15.20);
        % g cone
        \draw (26.00, -2.00) -- (35.00, -7.20);
        \draw (26.00, -2.00) -- (29.10, -0.21);
        % a cone
        \draw (26.00, 5.00) -- (28.73, 3.42);
        \draw (26.00, 5.00) -- (30.33, 7.50);
        % e cone
        \draw (25.00, 8.00) -- (28.10, 6.21);
        \draw (25.00, 8.00) -- (27.23, 9.29);
        % p cone
        \draw[line width = 0.5mm] (23.00, 2.00) -- (27.96, -0.87);
        \draw[line width = 0.5mm] (23.00, 2.00) -- (27.10, 4.37);
        % q cone
        \draw[line width = 0.5mm] (22.00, 5.00) -- (25.10, 3.21);
        \draw[line width = 0.5mm] (22.00, 5.00) -- (26.10, 7.37);

        \draw[dashed] (22, -3) -- (22, 8);
        \draw[dashed] (23, -1) -- (23, 4);
    \end{tikzpicture}
    \caption[Exemplo de evento \textsc{horizontal}]{Da esquerda para a direita, o caso em que
    $p$ está em $\Hits_{up}(q)$.
    Da direita para a esquerda, o caso em que $q$ está em $\Hits_{low}(p)$.}
    \label{fig:parcinetico:eventohorizontal}
\end{figure}

Se $p$ está em $\Hits_{up}(q)$, como demonstrado na
Figura~\ref{fig:parcinetico:eventohorizontalabaixo}, então parte de $\Cands(q)$ terá de passar
para $\Cands(p)$.
Para tal, buscamos pelo novo $t = \up(p)$ em $\Cands(q)$ e chamamos a rotina \textit{splay} no
nó que contém $t$.
Após o \textit{splay}, chamamos um \textit{split} na subárvore esquerda desse nó e a unimos a
$\Cands(p)$.
Se $t$ não for encontrado em $\Cands(q)$, então $t = \up(q)$ e todos os pontos de $\Cands(q)$
devem ser transferidos para $\Cands(p)$.
Não podemos esquecer de remover $p$ de $\Hits_{up}(q)$ e adicioná-lo a $\Hits_{up}(t)$, além de
remover $q$ de $\Hits_{low}(w)$ e adicioná-lo a $\Hits_{low}(p)$.
Se $p$ não está em $\Hits_{up}(q)$, então não haverão mudanças, veja a
Figura~\ref{fig:parcinetico:eventohorizontalabaixosemmudancas}.

\begin{figure}[H]
    \centering
    \begin{tikzpicture}[thick, scale=0.4]
        \node[label={[label distance = -3mm]160:$p$}] at
        (0.00, 2.00) {\textbullet};
        \node[label={[label distance = -3mm]160:$e$}] at
        (3.00, 5.00) {\textbullet};
        \node[label={[label distance = -3mm]160:$a$}] at
        (6.00, 5.00) {\textbullet};
        \node[label={[label distance = -3mm]160:$b$}] at
        (8.00, 3.00) {\textbullet};
        \node[label={[label distance = -3mm]160:$c$}] at
        (7.00, 1.00) {\textbullet};
        \node[label={[label distance = -3mm]160:$d$}] at
        (9.00, -1.00) {\textbullet};
        \node[label={[label distance = -3mm]160:$q$}] at
        (1.00, 8.00) {\textbullet};
        \node[label={[label distance = -3mm]220:$f$}] at
        (6.00, -2.00) {\textbullet};

        % d cone
        \draw (9.00, -1.00) -- (15.00, -4.46);
        \draw (9.00, -1.00) -- (15.00, 2.46);
        % b cone
        \draw (8.00, 3.00) -- (11.96, 0.71);
        \draw (8.00, 3.00) -- (15.00, 7.04);
        % c cone
        \draw (7.00, 1.00) -- (9.73, -0.58);
        \draw (7.00, 1.00) -- (9.23, 2.29);
        % f cone
        \draw (6.00, -2.00) -- (15.00, -7.20);
        \draw (6.00, -2.00) -- (9.10, -0.21);
        % a cone
        \draw (6.00, 5.00) -- (8.73, 3.42);
        \draw (6.00, 5.00) -- (15.00, 10.20);
        % e cone
        \draw (3.00, 5.00) -- (8.46, 1.85);
        \draw (3.00, 5.00) -- (15.00, 11.93);
        % q cone
        \draw[line width = 0.5mm] (1.00, 8.00) -- (4.60, 5.92);
        \draw[line width = 0.5mm] (1.00, 8.00) -- (15.00, 16.08);
        % p cone
        \draw[line width = 0.5mm] (0.00, 2.00) -- (6.46, -1.73);
        \draw[line width = 0.5mm] (0.00, 2.00) -- (4.10, 4.37);

        \draw[dashed] (0, -1) -- (0, 7);
        \draw[dashed] (1, 3) -- (1, 9);
        \draw[dashed] (20, -1) -- (20, 7);
        \draw[dashed] (19, 3) -- (19, 9);

        \node at (16, 5) {$ \leftrightarrow$};

        \node[label={[label distance = -3mm]160:$p$}] at
        (20.00, 2.00) {\textbullet};
        \node[label={[label distance = -3mm]160:$e$}] at
        (23.00, 5.00) {\textbullet};
        \node[label={[label distance = -3mm]160:$a$}] at
        (26.00, 5.00) {\textbullet};
        \node[label={[label distance = -3mm]160:$b$}] at
        (28.00, 3.00) {\textbullet};
        \node[label={[label distance = -3mm]160:$c$}] at
        (27.00, 1.00) {\textbullet};
        \node[label={[label distance = -3mm]160:$d$}] at
        (29.00, -1.00) {\textbullet};
        \node[label={[label distance = -3mm]160:$q$}] at
        (19.00, 8.00) {\textbullet};
        \node[label={[label distance = -3mm]220:$f$}] at
        (26.00, -2.00) {\textbullet};

        % d cone
        \draw (29.00, -1.00) -- (35.00, -4.46);
        \draw (29.00, -1.00) -- (35.00, 2.46);
        % b cone
        \draw (28.00, 3.00) -- (31.96, 0.71);
        \draw (28.00, 3.00) -- (35.00, 7.04);
        % c cone
        \draw (27.00, 1.00) -- (29.73, -0.58);
        \draw (27.00, 1.00) -- (29.23, 2.29);
        % f cone
        \draw (26.00, -2.00) -- (35.00, -7.20);
        \draw (26.00, -2.00) -- (29.10, -0.21);
        % a cone
        \draw (26.00, 5.00) -- (28.73, 3.42);
        \draw (26.00, 5.00) -- (35.00, 10.20);
        % e cone
        \draw (23.00, 5.00) -- (28.46, 1.85);
        \draw (23.00, 5.00) -- (35.00, 11.93);
        % p cone
        \draw[line width = 0.5mm] (20.00, 2.00) -- (26.46, -1.73);
        \draw[line width = 0.5mm] (20.00, 2.00) -- (24.10, 4.37);
        % q cone
        \draw[line width = 0.5mm] (19.00, 8.00) -- (23.60, 5.35);
        \draw[line width = 0.5mm] (19.00, 8.00) -- (35.00, 17.24);
    \end{tikzpicture}
    \caption{Se $p$ não está em $\Hits_{up}(q)$, ou se
        $q$ não está em $\Hits_{low}(p)$, nada acontece.}
    \label{fig:parcinetico:eventohorizontalabaixosemmudancas}
\end{figure}

Similarmente, se $q$ está em $\Hits_{low}(p)$, como demonstrado na
Figura~\ref{fig:parcinetico:eventohorizontalabaixo}, parte de $\Cands(p)$ passará a $\Cands(q)$
.
Para realizar tal operação, buscamos pelo novo $t = low(q)$ em $\Cands(p)$, damos um \textit{splay}
no nó que contém $t$, separamos a subárvore direita desse nó e a unimos a $\Cands(q)$.
Se $t$ não for encontrado em $\Cands(p)$, então $t = low(p)$ e todos os pontos de $\Cands(p)$ devem
ser passados para $\Cands(q)$.
Devemos também remover $q$ de $Hits_{low}(p)$ e inseri-lo em $Hits_{low}(t)$, além de remover $p$
de $Hits_{up}(w)$ e adicioná-lo a $Hits_{up}(q)$.
Se $q$ não está em $Hits_{low}(p)$, então não haverão mudanças, veja a
Figura~\ref{fig:parcinetico:eventohorizontalabaixosemmudancas}.

\begin{algorithm}[H]
    \caption[Algoritmo \textsc{horizontalEvent} do par mais próximo cinético]{Função \textsc{horizontalEvent}.}
    \label{alg:par-cinetico:eventohorizontal}
    \begin{algorithmic}[1]
        \Function{horizontalEvent}{$p,q, dir$}
            \If{$q = \Call{owner}{p.\hitsup(dir)}$}
                \State \Call{horizontalEventLeft}{$p,q, dir$}
            \Else
                \If{$p = \Call{owner}{q.\hitslow(p, dir)}$}
                    \State \Call{horizontalEventRight}{$p,q, dir$}
                \EndIf
            \EndIf
            \State $v \leftarrow \Call{owner}{p.cands(dir)}$
            \State $v' \leftarrow \Call{owner}{q.cands(dir)}$
            \If{$v = v'$}
                \State \Call{updateLcand}{$v, dir$}
            \EndIf
        \EndFunction
    \end{algorithmic}
\end{algorithm}

No caso de um evento em que ocorre uma mudança na $60^\circ$-ordem, que é a ordem dos pontos
projetados no eixo $x + 60^\circ$, vamos assumir que $p$ é o ponto que está à esquerda e acima de $q$.
O evento pode provocar a entrada ou saída do ponto $q$ de $\Cands(p)$, veja a
Figura~\ref{fig:parcinetico:eventoup}.
O Algoritmo~\ref{alg:par-cinetico:eventoup} implementa a sequência de operações referentes a este
evento.

Se $p$ está em $\Hits_{low}(q)$, ou seja, $q$ está entrando em $\Dom(p)$ como demonstrado na
Figura~\ref{fig:parcinetico:eventoup} da esquerda para direita, então a troca na $60^\circ$-ordem
afetará o ponto $v$ tal que $q$ está em $\Cands(v)$.
Achamos esse ponto subindo em $\Cands(v)$, a partir do nó que contém $q$, até a raiz que aponta
para $v$.
Devemos então remover $q$ de $\Cands(v)$ e inseri-lo em $\Cands(p)$.
A mudança também afetará todos os pontos que estão à esquerda de $p$ e estão em $\Hits_{up}(q)$.
Para achar esses pontos, buscamos o ponto $t$ em $\Hits_{up}(q)$ mais à esquerda que está à
direita de $p$.
Chamamos \textit{splay} para o nó que contém $t$ e chamamos \textit{split} para a subárvore
esquerda desse nó e juntamos essa árvore em $\Hits_{up}(p)$, pois são todos os pontos à esquerda
de $p$ que tinham $q$ como $\up$ e agora seu novo $\up$ é $p$.
Se esse ponto $t$ não existe, todos os pontos de $\Hits_{up}(q)$ devem ser transferidos para
$\Hits_{up}(p)$, veja a Figura~\ref{fig:parcinetico:eventouptnaoexiste}, e buscamos pelo ponto $t$
tal que $q$ está em $\Hits_{low}(t)$.
Por fim, removemos o ponto $p$ de $\Hits_{low}(q)$ e o inserimos em $\Hits_{low}(t)$.
Se $p$ não está em $\Hits_{low}(q)$ não haverão mudanças.

Se $q$ está em $\Cands(p)$, ou seja, $q$ está saindo de $\Dom(p)$ como demonstrado na
Figura~\ref{fig:parcinetico:eventoup} da direita para a esquerda, então a troca afetará o ponto $t$
tal que $p$ está em $\Hits_{low}(t)$.
Se o ponto $t$ existe, removemos $p$ de $\Hits_{low}(t)$.
Devemos agora inserir $p$ em $\Hits_{low}(q)$, já que $q$ é o novo $\low(p)$.
A mudança também afetará os pontos de $\Hits_{up}(p)$ que agora deverão estar em $\Hits_{up}(q)$.
Para achar esses pontos, buscamos pelo ponto $v$ em $\Hits_{up}(p)$ mais à direita que não deveria
estar em $\Hits_{up}(q)$, chamamos \textit{splay} para o nó que contém $v$ e um \textit{split}
para sua subárvore direita.
Essa nova árvore deve ser incorporada a $\Hits_{up}(q)$.
Se tal ponto $v$ não existe, todos os nós de $\Hits_{up}(p)$ devem ser passados para $\Hits_{up}
(q)$.
Por fim, devemos achar o novo ponto $u$ tal que $q$ deve estar em $\Cands(u)$.
Se o ponto $v$ descrito anteriormente existe, então $u = v$.
Se $v$ não existe, então $u$ é o ponto tal que $p$ está em $\Cands(u)$.
Dessa forma, retiramos $q$ de $\Cands(p)$ e o inserimos em $\Cands(u)$.
Se $q$ não está em $\Cands(p)$, não haverão mudanças.

\begin{algorithm}[H]
    \caption{Função \textsc{upEvent}.}
    \label{alg:par-cinetico:eventoup}
    \begin{algorithmic}[1]
        \Function{upEvent}{$p, q, dir$}
            \If{$q = \Call{owner}{p.hitsLow(dir)}$}
                \State \Call{upEventLeft}{$p, q, dir$}
            \Else
                \If{$p = \Call{owner}{q.cands(dir)}$}
                    \State \Call{upEventRight}{$p, q, dir$}
                \EndIf
            \EndIf
        \EndFunction
    \end{algorithmic}
\end{algorithm}

\begin{algorithm}[H]
    \caption{Função \textsc{upEvent}.}
    \label{alg:par-cinetico:eventoup}
    \begin{algorithmic}[1]
        \Function{upEvent}{$p, q, dir$}
            \If{$q = \Call{owner}{p.hitsLow(dir)}$}
                \State \Call{upEventLeft}{$p, q, dir$}
            \Else
                \If{$p = \Call{owner}{q.cands(dir)}$}
                    \State \Call{upEventRight}{$p, q, dir$}
                \EndIf
            \EndIf
        \EndFunction
    \end{algorithmic}
\end{algorithm}

\begin{figure}[h]
    \centering
    \begin{tikzpicture}[thick, scale=0.4]
        \node[label={[label distance = -3mm]160:$p$}] at
        (2.00, 4.00) {\textbullet};
        \node[label={[label distance = -3mm]160:$q$}] at
        (6.00, 1.00) {\textbullet};
        \node[label={[label distance = -2mm]270:$a = t$}] at
        (7.25, -2.00) {\textbullet};
        \node[label={[label distance = -3mm]160:$c$}] at
        (-1.00, -4.00) {\textbullet};
        \node[label={[label distance = -1mm]0:$b = v$}] at
        (-2.00, 0.00) {\textbullet};
        \node[label={[label distance = -1mm]0:$e$}] at
        (0.50, -4.00) {\textbullet};

        % a cone
        \draw (7.25, -2.00) -- (10.00, -3.59);
        \draw (7.25, -2.00) -- (10.00, -0.41);
        % q cone
        \draw[line width = 0.5mm] (6.00, 1.00) -- (9.22, -0.86);
        \draw[line width = 0.5mm] (6.00, 1.00) -- (10.00, 3.31);
        % p cone
        \draw[line width = 0.5mm] (2.00, 4.00) -- (6.60, 1.35);
        \draw[line width = 0.5mm] (2.00, 4.00) -- (10.00, 8.62);
        % e cone
        \draw (0.50, -4.00) -- (10.00, -9.48);
        \draw (0.50, -4.00) -- (7.58, 0.09);
        % c cone
        \draw (-1.00, -4.00) -- (10.00, -10.35);
        \draw (-1.00, -4.00) -- (6.83, 0.52);
        % b cone
        \draw (-2.00, 0.00) -- (1.96, -2.29);
        \draw (-2.00, 0.00) -- (3.46, 3.15);


        \draw[dashed] (2.00, 4.00) -- (-1.00, 5.73);
        \draw[dashed] (6.00, 1.00) -- (2.00, 3.30);
        \node at (12, 0) {$ \leftrightarrow$};

        \node[label={[label distance = -3mm]160:$p$}] at
        (22.00, 4.00) {\textbullet};
        \node[label={[label distance = -3mm]160:$q$}] at
        (27.00, 2.00) {\textbullet};
        \node[label={[label distance = -2mm]270:$a = t$}] at
        (27.25, -2.00) {\textbullet};
        \node[label={[label distance = -3mm]160:$c$}] at
        (19.00, -4.00) {\textbullet};
        \node[label={[label distance = -1mm]0:$b = v$}] at
        (18.00, 0.00) {\textbullet};
        \node[label={[label distance = -1mm]0:$e$}] at
        (20.50, -4.00) {\textbullet};

        % a cone
        \draw (27.25, -2.00) -- (31.00, -4.17);
        \draw (27.25, -2.00) -- (31.00, 0.17);
        % q cone
        \draw[line width = 0.5mm] (27.00, 2.00) -- (30.59, -0.07);
        \draw[line width = 0.5mm] (27.00, 2.00) -- (31.00, 4.31);
        % p cone
        \draw[line width = 0.5mm] (22.00, 4.00) -- (29.82, -0.52);
        \draw[line width = 0.5mm] (22.00, 4.00) -- (31.00, 9.20);
        % e cone
        \draw (20.50, -4.00) -- (31.00, -10.06);
        \draw (20.50, -4.00) -- (28.18, 0.43);
        % c cone
        \draw (19.00, -4.00) -- (31.00, -10.93);
        \draw (19.00, -4.00) -- (27.43, 0.87);
        % b cone
        \draw (18.00, 0.00) -- (21.96, -2.29);
        \draw (18.00, 0.00) -- (23.46, 3.15);

        \draw[dashed] (22.00, 4.00) -- (19.00, 5.73);
        \draw[dashed] (27.00, 2.00) -- (22.00, 4.88);
    \end{tikzpicture}
    \caption{Da esquerda para a direita,
        todos os pontos de $\Hits_{up}(q)$
        são transferidos para $\Hits_{up}(p)$
        e $p$, que está em $\Hits_{low}(q)$,
        é transferido para $\Hits_{low}(t)$.
        Da direita para a esquerda, os pontos
        em $\Hits_{up}(p)$ são transferidos
        para $\Hits_{up}(q)$ e $p$, que está
        em $\Hits_{low}(t)$, é transferido
        para $\Hits_{low}(q)$.}
    \label{fig:parcinetico:eventouptnaoexiste}
\end{figure}

Um evento em que ocorre uma mudança na $-60^\circ$-ordem, a ordem dos pontos projetados no eixo $x -
60^\circ$, é simétrico a um evento na $60^\circ$-ordem.
Os pontos envolvidos no evento serão $p$e $q$ e vamos assumir que $p$ é o ponto mais à esquerda e
abaixo de $q$.
O evento pode provocar a entrada ou saída do ponto $q$ de $\Cands(p)$, veja a
Figura~\ref{fig:parcinetico:eventodown}.
O Algoritmo~\ref{fig:parcinetico:eventodown} implementa a sequência de operações referentes a esse
evento.

Se $p$ está em $\Hits_{up}(q)$ ($q$ está entrando em $Dom(p)$), como demonstrado na
Figura~\ref{fig:parcinetico:eventodown}, então a troca na $-60^\circ$-ordem afetará o ponto $v$ tal
que $q$ está em $\Cands(v)$.
Achamos esse ponto subindo em $\Cands(v)$, a partir do nó que contém $q$, até a raiz que aponta
para~$v$.
Devemos então remover $q$ de $\Cands(v)$ e inseri-lo em $\Cands(p)$.
A mudança também afetará todos os pontos que estão à esquerda de $p$ e estão em $\Hits_{low}(q)$.
Para achar esses pontos buscamos o ponto $t$ em $\Hits_{low}(q)$ mais à esquerda que está a
direita de $p$.
Chamamos \textit{splay} para o nó que contém $t$ e \textit{split} para a nova subárvore esquerda
desse nó e juntamos essa árvore em $\Hits_{low}(p)$, pois são todos pontos à esquerda de $p$ que
tinham $q$ como $low$ e agora seu novo $low$ é $p$.
Se esse ponto $t$ não existe, veja a Figura~\ref{fig:parcinetico:eventodowntnaoexiste}, então
buscamos pelo ponto $t$ tal que $q$ está em $\Hits_{up}(t)$.
Por fim, removemos o ponto $p$ de $\Hits_{up}(q)$ e o inserimos em $\Hits_{up}(t)$.
Se $p$ não está em $\Hits_{up}(q)$ não haverão mudanças.

Se $q$ está em $\Cands(p)$ ($q$ está saindo de $Dom(p)$), como demonstrado na
Figura~\ref{fig:parcinetico:eventodown}, então a troca afetará o ponto $t$ tal que $p$ está em
$\Hits_{up}(t)$.
Se o ponto $t$ existe, removemos $p$ de $\Hits_{up}(t)$.
Devemos agora inserir $p$ em $\Hits_{up}(q)$, já que $q$ é o novo $up(p)$.
A mudança também afetará os pontos de $\Hits_{low}(p)$ que agora atingem $Dom(q)$.
Para achar esses pontos, buscamos pelo ponto $v$ em $\Hits_{low}(p)$ mais à direita que não atinge
$Dom(q)$, chamamos \textit{splay} para o nó que contém $v$ e um \textit{split} para sua subárvore
direita.
Essa nova árvore deve ser incorporada a $\Hits_{low}(q)$.
Se tal ponto $v$ não existe, todos os nós de $\Hits_{low}(p)$ devem ser passados para $\Hits_{low}
(q)$.
Por fim, devemos achar o novo ponto~$u$ tal que $q$ deve estar em $\Cands(u)$.
Se o ponto $v$ descrito anteriormente existe, então~$u = v$.
Se $v$ não existe, então $u$ é o ponto tal que $p$ está em $\Cands(u)$.
Dessa forma, retiramos $q$ de $\Cands(p)$ e o inserimos em $\Cands(u)$.
Se $q$ não está em $\Cands(p)$, não haverão mudanças.

\begin{figure}[h]
    \centering
    \begin{tikzpicture}[thick, scale=0.4]
        \node[label={[label distance = -3mm]160:$p$}] at
            (6.00, 0.00) {\textbullet};
        \node[label={[label distance = -3mm]160:$q$}] at
            (8.00, 2.00) {\textbullet};
        \node[label={[label distance = -1mm]180:$a$}] at
            (0.00, 5.00) {\textbullet};
        \node[label={[label distance = -2mm]90:$b = v$}] at
            (2.00, 5.00) {\textbullet};
        \node[label={[label distance = 0mm]0:$c$}] at
            (4.00, 5.00) {\textbullet};
        \node[label={[label distance = 0mm]0:$d$}] at
            (10.00, 4.00) {\textbullet};
        \node[label={[label distance = 0mm]90:$e = t$}] at
            (8.00, 4.00) {\textbullet};

        % d cone
        \draw (10.00, 4.00) -- (14.00, 1.69);
        \draw (10.00, 4.00) -- (14.00, 6.31);
        % q cone
        \draw[dashed] (8.00, 2.00) -- (4.00, -0.30);
        \draw[line width = 0.5mm] (8.00, 2.00) -- (14.00, -1.46);
        \draw[line width = 0.5mm] (8.00, 2.00) -- (10.73, 3.58);
        % e cone
        \draw (8.00, 4.00) -- (9.73, 3.00);
        \draw (8.00, 4.00) -- (14.00, 7.46);
        % p cone
        \draw[dashed] (6.00, 0.00) -- (2.00, -2.30);
        \draw[line width = 0.5mm] (6.00, 0.00) -- (14.00, -4.62);
        \draw[line width = 0.5mm] (6.00, 0.00) -- (8.73, 1.58);
        % c cone
        \draw (4.00, 5.00) -- (8.60, 2.35);
        \draw (4.00, 5.00) -- (14.00, 10.77);
        % b cone
        \draw (2.00, 5.00) -- (8.33, 1.35);
        \draw (2.00, 5.00) -- (14.00, 11.93);
        % a cone
        \draw (0.00, 5.00) -- (7.33, 0.77);
        \draw (0.00, 5.00) -- (14.00, 13.08);

        \node at (17, 5) {$ \leftrightarrow$};

        \node[label={[label distance = -3mm]160:$p$}] at
            (26.00, 0.00) {\textbullet};
        \node[label={[label distance = -3mm]160:$q$}] at
            (29.00, 1.00) {\textbullet};
        \node[label={[label distance = -1mm]180:$a$}] at
            (20.00, 5.00) {\textbullet};
        \node[label={[label distance = -2mm]90:$b = v$}] at
            (22.00, 5.00) {\textbullet};
        \node[label={[label distance = 0mm]0:$c$}] at
            (24.00, 5.00) {\textbullet};
        \node[label={[label distance = 0mm]0:$d$}] at
            (30.00, 4.00) {\textbullet};
        \node[label={[label distance = 0mm]90:$e = t$}] at
            (28.00, 4.00) {\textbullet};

        % d cone
        \draw (30.00, 4.00) -- (35.00, 1.11);
        \draw (30.00, 4.00) -- (35.00, 6.89);
        % q cone
        \draw[dashed] (29.00, 1.00) -- (26.00, -0.73);
        \draw[line width = 0.5mm] (29.00, 1.00) -- (35.00, -2.46);
        \draw[line width = 0.5mm] (29.00, 1.00) -- (32.10, 2.79);
        % e cone
        \draw (28.00, 4.00) -- (31.10, 2.21);
        \draw (28.00, 4.00) -- (34.00, 7.46);
        % p cone
        \draw[dashed] (26.00, 0.00) -- (24.00, -1.15);
        \draw[line width = 0.5mm] (26.00, 0.00) -- (35.00, -5.20);
        \draw[line width = 0.5mm] (26.00, 0.00) -- (30.46, 2.58);
        % c cone
        \draw (24.00, 5.00) -- (29.33, 1.92);
        \draw (24.00, 5.00) -- (35.00, 11.35);
        % b cone
        \draw (22.00, 5.00) -- (28.33, 1.35);
        \draw (22.00, 5.00) -- (35.00, 12.51);
        % a cone
        \draw (20.00, 5.00) -- (27.33, 0.77);
        \draw (20.00, 5.00) -- (35.00, 13.66);
    \end{tikzpicture}
    \caption{Da esquerda para direita, o caso em que
    $p$ está em $\Hits_{up}(q)$, ou seja, $q$ está
    entrando em $\Dom(p)$. Da direita para esquerda,
    o caso em que $q$ está em $\Cands(p)$, saindo
    de $\Dom(p)$.}
    \label{fig:parcinetico:eventodown}
\end{figure}

\begin{figure}[h]
    \centering
    \begin{tikzpicture}[thick, scale=0.4]
        \node[label={[label distance = -3mm]160:$p$}] at
            (6.00, 0.00) {\textbullet};
        \node[label={[label distance = -3mm]160:$q$}] at
            (8.00, 2.00) {\textbullet};
        \node[label={[label distance = -1mm]180:$a$}] at
            (0.00, 5.00) {\textbullet};
        \node[label={[label distance = -2mm]90:$b = v$}] at
            (2.00, 5.00) {\textbullet};
        \node[label={[label distance = 0mm]0:$c$}] at
            (4.00, 5.00) {\textbullet};
        \node[label={[label distance = 0mm]0:$d$}] at
            (10.00, 4.00) {\textbullet};
        \node[label={[label distance = 0mm]90:$e = t$}] at
            (8.00, 4.00) {\textbullet};

        % d cone
        \draw (10.00, 4.00) -- (14.00, 1.69);
        \draw (10.00, 4.00) -- (14.00, 6.31);
        % q cone
        \draw[dashed] (8.00, 2.00) -- (4.00, -0.30);
        \draw[line width = 0.5mm] (8.00, 2.00) -- (14.00, -1.46);
        \draw[line width = 0.5mm] (8.00, 2.00) -- (10.73, 3.58);
        % e cone
        \draw (8.00, 4.00) -- (9.73, 3.00);
        \draw (8.00, 4.00) -- (14.00, 7.46);
        % p cone
        \draw[dashed] (6.00, 0.00) -- (2.00, -2.30);
        \draw[line width = 0.5mm] (6.00, 0.00) -- (14.00, -4.62);
        \draw[line width = 0.5mm] (6.00, 0.00) -- (8.73, 1.58);
        % c cone
        \draw (4.00, 5.00) -- (8.60, 2.35);
        \draw (4.00, 5.00) -- (14.00, 10.77);
        % b cone
        \draw (2.00, 5.00) -- (8.33, 1.35);
        \draw (2.00, 5.00) -- (14.00, 11.93);
        % a cone
        \draw (0.00, 5.00) -- (7.33, 0.77);
        \draw (0.00, 5.00) -- (14.00, 13.08);

        \node at (17, 5) {$ \leftrightarrow$};

        \node[label={[label distance = -3mm]160:$p$}] at
            (26.00, 0.00) {\textbullet};
        \node[label={[label distance = -3mm]160:$q$}] at
            (29.00, 1.00) {\textbullet};
        \node[label={[label distance = -1mm]180:$a$}] at
            (20.00, 5.00) {\textbullet};
        \node[label={[label distance = -2mm]90:$b = v$}] at
            (22.00, 5.00) {\textbullet};
        \node[label={[label distance = 0mm]0:$c$}] at
            (24.00, 5.00) {\textbullet};
        \node[label={[label distance = 0mm]0:$d$}] at
            (30.00, 4.00) {\textbullet};
        \node[label={[label distance = 0mm]90:$e = t$}] at
            (28.00, 4.00) {\textbullet};

        % d cone
        \draw (30.00, 4.00) -- (35.00, 1.11);
        \draw (30.00, 4.00) -- (35.00, 6.89);
        % q cone
        \draw[dashed] (29.00, 1.00) -- (26.00, -0.73);
        \draw[line width = 0.5mm] (29.00, 1.00) -- (35.00, -2.46);
        \draw[line width = 0.5mm] (29.00, 1.00) -- (32.10, 2.79);
        % e cone
        \draw (28.00, 4.00) -- (31.10, 2.21);
        \draw (28.00, 4.00) -- (34.00, 7.46);
        % p cone
        \draw[dashed] (26.00, 0.00) -- (24.00, -1.15);
        \draw[line width = 0.5mm] (26.00, 0.00) -- (35.00, -5.20);
        \draw[line width = 0.5mm] (26.00, 0.00) -- (30.46, 2.58);
        % c cone
        \draw (24.00, 5.00) -- (29.33, 1.92);
        \draw (24.00, 5.00) -- (35.00, 11.35);
        % b cone
        \draw (22.00, 5.00) -- (28.33, 1.35);
        \draw (22.00, 5.00) -- (35.00, 12.51);
        % a cone
        \draw (20.00, 5.00) -- (27.33, 0.77);
        \draw (20.00, 5.00) -- (35.00, 13.66);
    \end{tikzpicture}
    \caption{Da esquerda para direita, o caso em que
    $p$ está em $\Hits_{up}(q)$, ou seja, $q$ está
    entrando em $\Dom(p)$. Da direita para esquerda,
    o caso em que $q$ está em $\Cands(p)$, saindo
    de $\Dom(p)$.}
    \label{fig:parcinetico:eventodown}
\end{figure}

\begin{figure}[h]
    \centering
    \begin{tikzpicture}[thick, scale=0.4]
        \node[label={[label distance = -3mm]160:$p$}] at
            (6.00, 0.00) {\textbullet};
        \node[label={[label distance = -3mm]160:$q$}] at
            (8.00, 2.00) {\textbullet};
        \node[label={[label distance = -1mm]180:$a$}] at
            (0.00, 5.00) {\textbullet};
        \node[label={[label distance = -2mm]90:$b = v$}] at
            (2.00, 5.00) {\textbullet};
        \node[label={[label distance = 0mm]0:$c$}] at
            (4.00, 5.00) {\textbullet};
        \node[label={[label distance = 0mm]0:$d$}] at
            (10.00, 4.00) {\textbullet};
        \node[label={[label distance = 0mm]90:$e = t$}] at
            (8.00, 4.00) {\textbullet};

        % d cone
        \draw (10.00, 4.00) -- (14.00, 1.69);
        \draw (10.00, 4.00) -- (14.00, 6.31);
        % q cone
        \draw[dashed] (8.00, 2.00) -- (4.00, -0.30);
        \draw[line width = 0.5mm] (8.00, 2.00) -- (14.00, -1.46);
        \draw[line width = 0.5mm] (8.00, 2.00) -- (10.73, 3.58);
        % e cone
        \draw (8.00, 4.00) -- (9.73, 3.00);
        \draw (8.00, 4.00) -- (14.00, 7.46);
        % p cone
        \draw[dashed] (6.00, 0.00) -- (2.00, -2.30);
        \draw[line width = 0.5mm] (6.00, 0.00) -- (14.00, -4.62);
        \draw[line width = 0.5mm] (6.00, 0.00) -- (8.73, 1.58);
        % c cone
        \draw (4.00, 5.00) -- (8.60, 2.35);
        \draw (4.00, 5.00) -- (14.00, 10.77);
        % b cone
        \draw (2.00, 5.00) -- (8.33, 1.35);
        \draw (2.00, 5.00) -- (14.00, 11.93);
        % a cone
        \draw (0.00, 5.00) -- (7.33, 0.77);
        \draw (0.00, 5.00) -- (14.00, 13.08);

        \node at (17, 5) {$ \leftrightarrow$};

        \node[label={[label distance = -3mm]160:$p$}] at
            (26.00, 0.00) {\textbullet};
        \node[label={[label distance = -3mm]160:$q$}] at
            (29.00, 1.00) {\textbullet};
        \node[label={[label distance = -1mm]180:$a$}] at
            (20.00, 5.00) {\textbullet};
        \node[label={[label distance = -2mm]90:$b = v$}] at
            (22.00, 5.00) {\textbullet};
        \node[label={[label distance = 0mm]0:$c$}] at
            (24.00, 5.00) {\textbullet};
        \node[label={[label distance = 0mm]0:$d$}] at
            (30.00, 4.00) {\textbullet};
        \node[label={[label distance = 0mm]90:$e = t$}] at
            (28.00, 4.00) {\textbullet};

        % d cone
        \draw (30.00, 4.00) -- (35.00, 1.11);
        \draw (30.00, 4.00) -- (35.00, 6.89);
        % q cone
        \draw[dashed] (29.00, 1.00) -- (26.00, -0.73);
        \draw[line width = 0.5mm] (29.00, 1.00) -- (35.00, -2.46);
        \draw[line width = 0.5mm] (29.00, 1.00) -- (32.10, 2.79);
        % e cone
        \draw (28.00, 4.00) -- (31.10, 2.21);
        \draw (28.00, 4.00) -- (34.00, 7.46);
        % p cone
        \draw[dashed] (26.00, 0.00) -- (24.00, -1.15);
        \draw[line width = 0.5mm] (26.00, 0.00) -- (35.00, -5.20);
        \draw[line width = 0.5mm] (26.00, 0.00) -- (30.46, 2.58);
        % c cone
        \draw (24.00, 5.00) -- (29.33, 1.92);
        \draw (24.00, 5.00) -- (35.00, 11.35);
        % b cone
        \draw (22.00, 5.00) -- (28.33, 1.35);
        \draw (22.00, 5.00) -- (35.00, 12.51);
        % a cone
        \draw (20.00, 5.00) -- (27.33, 0.77);
        \draw (20.00, 5.00) -- (35.00, 13.66);
    \end{tikzpicture}
    \caption{Da esquerda para direita, o caso em que
    $p$ está em $\Hits_{up}(q)$, ou seja, $q$ está
    entrando em $\Dom(p)$. Da direita para esquerda,
    o caso em que $q$ está em $\Cands(p)$, saindo
    de $\Dom(p)$.}
    \label{fig:parcinetico:eventodown}
\end{figure}

\begin{figure}[h]
    \centering
    \begin{tikzpicture}[thick, scale=0.4]
        \node[label={[label distance = -3mm]160:$p$}] at
            (6.00, 0.00) {\textbullet};
        \node[label={[label distance = -3mm]160:$q$}] at
            (8.00, 2.00) {\textbullet};
        \node[label={[label distance = -3mm]160:$a$}] at
            (0.00, 5.00) {\textbullet};
        \node[label={[label distance = -3mm]160:$b$}] at
            (2.00, 5.00) {\textbullet};
        \node[label={[label distance = -3mm]160:$c$}] at
            (4.00, 5.00) {\textbullet};
        \node[label={[label distance = -3mm]160:$d = t$}] at
            (10.00, 4.00) {\textbullet};

        % d cone
        \draw (10.00, 4.00) -- (15.00, 1.11);
        \draw (10.00, 4.00) -- (15.00, 6.89);
        % q cone
        \draw[dashed] (8.00, 2.00) -- (4.00, -0.30);
        \draw[line width = 0.5mm] (8.00, 2.00) -- (15.00, -2.04);
        \draw[line width = 0.5mm] (8.00, 2.00) -- (10.73, 3.58);
        % p cone
        \draw[dashed] (6.00, 0.00) -- (2.00, -2.30);
        \draw[line width = 0.5mm] (6.00, 0.00) -- (15.00, -5.20);
        \draw[line width = 0.5mm] (6.00, 0.00) -- (8.73, 1.58);
        % c cone
        \draw (4.00, 5.00) -- (8.60, 2.35);
        \draw (4.00, 5.00) -- (15.00, 11.35);
        % b cone
        \draw (2.00, 5.00) -- (8.33, 1.35);
        \draw (2.00, 5.00) -- (15.00, 12.51);
        % a cone
        \draw (0.00, 5.00) -- (7.33, 0.77);
        \draw (0.00, 5.00) -- (15.00, 13.66);

        \node at (17, 5) {$ \leftrightarrow$};

        \node[label={[label distance = -3mm]160:$p$}] at
            (26.00, 0.00) {\textbullet};
        \node[label={[label distance = -3mm]160:$q$}] at
            (29.00, 1.00) {\textbullet};
        \node[label={[label distance = -3mm]160:$a$}] at
            (20.00, 5.00) {\textbullet};
        \node[label={[label distance = -3mm]160:$b$}] at
            (22.00, 5.00) {\textbullet};
        \node[label={[label distance = -3mm]160:$c$}] at
            (24.00, 5.00) {\textbullet};
        \node[label={[label distance = -3mm]160:$d = t$}] at
            (30.00, 4.00) {\textbullet};

        % d cone
        \draw (30.00, 4.00) -- (35.00, 1.11);
        \draw (30.00, 4.00) -- (35.00, 6.89);
        % q cone
        \draw[dashed] (29.00, 1.00) -- (26.00, -0.73);
        \draw[line width = 0.5mm] (29.00, 1.00) -- (35.00, -2.46);
        \draw[line width = 0.5mm] (29.00, 1.00) -- (32.10, 2.79);
        % p cone
        \draw[dashed] (26.00, 0.00) -- (24.00, -1.15);
        \draw[line width = 0.5mm] (26.00, 0.00) -- (35.00, -5.20);
        \draw[line width = 0.5mm] (26.00, 0.00) -- (31.46, 3.15);
        % c cone
        \draw (24.00, 5.00) -- (29.33, 1.92);
        \draw (24.00, 5.00) -- (35.00, 11.35);
        % b cone
        \draw (22.00, 5.00) -- (28.33, 1.35);
        \draw (22.00, 5.00) -- (35.00, 12.51);
        % a cone
        \draw (20.00, 5.00) -- (27.33, 0.77);
        \draw (20.00, 5.00) -- (35.00, 13.66);
    \end{tikzpicture}
    \caption{Da direita para esquerda, $p$, que estava em $\Hits_{up}(q)$
    foi transferido para $\Hits_{up}(t)$ e todos os elementos em
    $\Hits_{low}(q)$ foram transferidos para $\Hits_{low}(p)$. Da esquerda
    para direita, $p$, que estava em $\Hits_{up}(t)$, foi transferido para
    $\Hits_{up}(q)$ e os pontos à direita de $v$ em $\Hits_{low}(p)$ foram
    transferidos para $\Hits_{low}(q)$.}
    \label{fig:parcinetico:eventodowntnaoexiste}
\end{figure}

\FloatBarrier

\subsection{Análise de desempenho}\label{subsec:par:analise-de-desempenho}

As análises de desempenho aqui foram extraídas de~\cite{eduardo}.

A estrutura de dados cinética para manter um par de pontos mais próximo é uma estrutura
\textit{responsiva}, pois o custo de processar um certificado é $O(\lg{n})$, onde $n$ é o número
de pontos.
O custo de processar um certificado é o custo de realizar as trocas necessárias nas listas
ordenadas, o que consome tempo $O(\lg{n})$.
Além disso também há o custo de corrigir as árvores $\Cands$, $\Hits_{low}$, $\Hits_{up}$, o
torneio e os certificados associados.
Mas, essas operações são realizadas em sequência, consumindo um custo também de $O(\lg{n})$.

A estrutura é \textit{eficiente}, pois a razão entre o total de eventos e os eventos
\textit{externos}, isto é, as trocas de par mais próximo, de acordo com~\cite{eduardo}, é
$O(\epsilon \lg{n})$, resultando em uma estrutura eficiente.

A estrutura é \textit{compacta}, pois teremos $O(n)$ certificados na fila de prioridades
associados a mudanças nas listas ordenadas e $O(n)$ certificados do torneio cinético, resultando
em $O(n)$ certificados na fila com prioridades num determinado instante.

A estrutura é \textit{local}, pois um ponto pode estar envolvido em até seis certificados das
listas ordenadas, sendo dois para cada uma das ordenações, e pode estar envolvido em até
$O(\lg{n})$ certificados no torneio.



%%%%%%%%%%%%%%%%%%%%%%%%%%%% APÊNDICES E ANEXOS %%%%%%%%%%%%%%%%%%%%%%%%%%%%%%%%

% Um apêndice é algum conteúdo adicional de sua autoria que faz parte e
% colabora com a ideia geral do texto mas que, por alguma razão, não precisa
% fazer parte da sequência do discurso; por exemplo, a demonstração de um
% teorema intermediário, as perguntas usadas em uma pesquisa qualitativa etc.
%
% Um anexo é um documento que não faz parte da tese (em geral, nem é de sua
% autoria) mas é relevante para o conteúdo; por exemplo, a especificação do
% padrão técnico ou a legislação que o trabalho discute, um artigo de jornal
% apresentando a percepção do público sobre o tema da tese etc.
%
% Os comandos appendix e annex reiniciam a numeração de capítulos e passam
% a numerá-los com letras. "annex" não faz parte de nenhuma classe padrão,
% foi criado para este modelo. Se o trabalho não tiver apêndices ou anexos,
% remova estas linhas.
%
% Diferentemente de \mainmatter, \backmatter etc., \appendix e \annex não
% forçam o início de uma nova página. Em geral isso não é importante, pois
% o comando seguinte costuma ser "\chapter", mas pode causar problemas com
% a formatação dos cabeçalhos. Assim, vamos forçar uma nova página antes
% de cada um deles.

%%%% Apêndices %%%%

% \makeatletter
% \if@openright\cleardoublepage\else\clearpage\fi
% \makeatother

% \pagestyle{appendix}

% \appendix

% % \addappheadtotoc acrescenta a palavra "Apêndice" ao sumário; se
% % só há apêndices, sem anexos, provavelmente não é necessário.
% \addappheadtotoc

% %!TeX root=../tese.tex
%("dica" para o editor de texto: este arquivo é parte de um documento maior)
% para saber mais: https://tex.stackexchange.com/q/78101

\chapter{Código-fonte e pseudocódigo}
\label{ap:pseudocode}

Com a \textit{package} \textsf{listings}, programas podem ser inseridos
diretamente no arquivo, como feito no caso do Programa~\ref{prog:java},
ou importados de um arquivo externo com o comando
\textsf{\textbackslash{}lstinputlisting}, como no caso
do Programa~\ref{prog:mdcinput}.

% O exemplo foi copiado da documentação de algorithmicx
\begin{program}
  \lstinputlisting[
    language=pseudocode,
    style=pseudocode,
    style=wider,
    functions={},
    specialidentifiers={},
  ]
  {conteudo/euclid.psc}

  \caption{Máximo divisor comum (arquivo importado).\label{prog:mdcinput}}
\end{program}

Trechos de código curtos (menores que uma página) podem ou não ser
incluídos como \textit{floats}; trechos longos necessariamente incluem
quebras de página e, portanto, não podem ser \textit{floats}. Com
\textit{floats}, a legenda e as linhas separadoras são colocadas pelo
comando \textsf{\textbackslash{}begin\{program\}}; sem eles, utilize o
ambiente \textsf{programruledcaption} (atenção para a colocação do
comando \textsf{\textbackslash{}label\{\}}, dentro da legenda), como
no Programa~\ref{prog:mdc}\footnote{\textsf{listings} oferece alguns
recursos próprios para a definição de \textit{floats} e legendas, mas
neste modelo não os utilizamos.}:

\begin{programruledcaption}{Máximo divisor comum (em português).\label{prog:mdc}}
  \begin{lstlisting}[
    language={[brazilian]pseudocode},
    style=pseudocode,
    style=wider,
    functions={},
    specialidentifiers={},
  ]
      funcao euclides(a, b) // O máximo divisor comum de \textbf{a} e \textbf{b}
          r := a $\bmod$ b
	  enquanto r != 0 // Atingimos a resposta se \textbf{r} é zero
              a := b
              b := r
              r := a $\bmod$ b
          fim
	  devolva b // O máximo divisor comum é \textbf{b}
      fim
  \end{lstlisting}
\end{programruledcaption}

Além do suporte às várias linguagens incluídas em \textsf{listings},
este modelo traz uma extensão para permitir o uso de pseudocódigo,
útil para a descrição de algoritmos em alto nível. Ela oferece
diversos recursos:

\begin{itemize}

    \item Comentários seguem o padrão de C++ (\lstinline{//} e
          \lstinline{/* ... */}), mas o delimitador é impresso
          como ``$\triangleright$''.

    \item ``:='', ``<>'', ``<='', ``>='' e ``!='' são substituídos
          pelo símbolo matemático adequado.

    \item É possível acrescentar palavras-chave além de ``if'', ``and''
          etc. com a opção ``\textsf{morekeywords=\{pchave1,\linebreak[0]{}pchave2\}}''
          (para um trecho de código específico) ou com o comando
          \textsf{\textbackslash{}lstset\{morekeywords=\linebreak[0]{}\{pchave1,pchave2\}\}}
          (como comando de configuração geral).

    \item É possível usar pequenos trechos de código, como nomes de variáveis,
          dentro de um parágrafo normal com \textsf{\textbackslash{}lstinline\{blah\}}.

    \item ``\$\dots\$'' ativa o modo matemático em qualquer lugar.

    \item Outros comandos \LaTeX{} funcionam apenas em comentários; fora, a
          linguagem simula alguns pré-definidos (\textsf{\textbackslash{}textit\{\}},
          \textsf{\textbackslash{}texttt\{\}} etc.).

    \item O comando \textsf{\textbackslash{}label} também funciona em
          comentários; a referência correspondente (\textsf{\textbackslash{}ref})
          indica o número da linha de código. Se quiser usá-lo numa linha sem
          comentários, use \lstinline{///}~\textsf{\textbackslash{}label\{blah\}};
          ``\lstinline{///}'' funciona como \lstinline{//}, permitindo
          a inserção de comandos \LaTeX{}, mas não imprime o delimitador
          (\ensuremath{\triangleright}).

    \item Para suspender a formatação automática, use \textsf{\textbackslash{}noparse\{blah\}}.

    \item Para forçar a formatação de um texto como função, identificador,
          palavra-chave ou comentário, use \textsf{\textbackslash{}func\{blah\}},
          \textsf{\textbackslash{}id\{blah\}}, \textsf{\textbackslash{}kw\{blah\}} ou
          \textsf{\textbackslash{}comment\{blah\}}.

    \item Palavras-chave dentro de comentários não são formatadas
          automaticamente; se necessário, use \textsf{\textbackslash{}func\{\}},
          \textsf{\textbackslash{}id\{\}} etc. ou comandos \LaTeX{} padrão.

    \item As palavras ``Program'', ``Procedure'' e ``Function'' têm formatação
          especial e fazem a palavra seguinte ser formatada como função.
          Funções em outros lugares \emph{não} são detectadas automaticamente;
          use \textsf{\textbackslash{}func\{\}}, a opção ``\textsf{functions=\{func1,func2\}}''
          ou o comando ``\textsf{\textbackslash{}lstset\{functions=\{func1,func2\}\}}''
          para que elas sejam detectadas.

    \item Além de funções, palavras-chave, strings, comentários e
          identificadores, há ``\textsf{specialidentifiers}''. Você pode
          usá-los com \textsf{\textbackslash{}specialid\{blah\}}, com a opção
          ``\textsf{specialidentifiers=\{id1,id2\}}'' ou com o comando
          ``\textsf{\textbackslash{}lstset\{specialidentifiers=\{id1,id2\}\}}''.

\end{itemize}



% \par

% %%%% Anexos %%%%

% \makeatletter
% \if@openright\cleardoublepage\else\clearpage\fi
% \makeatother

% \pagestyle{appendix} % repete o anterior, caso você não use apêndices

% \annex

% % \addappheadtotoc acrescenta a palavra "Anexo" ao sumário; se
% % só há anexos, sem apêndices, provavelmente não é necessário.
% \addappheadtotoc

% %!TeX root=../tese.tex
%("dica" para o editor de texto: este arquivo é parte de um documento maior)
% para saber mais: https://tex.stackexchange.com/q/78101

\chapter[Perguntas frequentes sobre o modelo]{Perguntas frequentes sobre o modelo\footnote{Esta
seção não é de fato um anexo, mas sim um apêndice; ela foi definida desta
forma apenas para servir como exemplo de anexo.}}

\begin{itemize}

\item \textbf{Não consigo decorar tantos comandos!}\\
Use a colinha que é distribuída juntamente com este modelo (\url{gitlab.com/ccsl-usp/modelo-latex/raw/master/pre-compilados/colinha.pdf?inline=false}).

\item \textbf{Por que tantos arquivos?}\\
O preâmbulo \LaTeX{} deste modelo é muito longo; as partes que normalmente não precisam ser modificadas foram colocadas no diretório \texttt{extras}, juntamente com alguns arquivos acessórios.

\item \textbf{Estou tendo problemas com caracteres acentuados.}\\
Versões modernas de \LaTeX{} usam UTF-8, mas arquivos antigos podem usar outras codificações (como ISO-8859-1, também conhecido como latin1 ou Windows-1252). Nesses casos, use \textsf{\textbackslash{}usepackage[latin1]\{inputenc\}} no preâmbulo do documento. Você também pode representar os caracteres acentuados usando comandos \LaTeX{}: \textsf{\textbackslash\textquotesingle{}a} para á, \textsf{\textbackslash{}c\{c\}} para cedilha etc., independentemente da codificação usada no texto\footnote{Você pode consultar os comandos desse tipo mais comuns em \url{en.wikibooks.org/wiki/LaTeX/Special_Characters}. Observe que a dica sobre o pingo do i \emph{não} é mais válida atualmente; basta usar \textsf{\textbackslash\textquotesingle{}i}.}.

\item \textbf{Aparece uma folha em branco entre os capítulos.}\\
Essa característica foi colocada propositalmente, dado que todo capítulo deve (ou deveria) começar em uma página de numeração ímpar (lado direito do documento). Se quiser mudar esse comportamento, acrescente ``openany'' como opção da classe, i.e., \textsf{\textbackslash{}documentclass[openany,\dots]\{book\}}.

\item \textbf{É possível resumir o nome das seções/capítulos que aparece no topo das páginas e no sumário?}\\
Sim, usando a sintaxe \textsf{\textbackslash{}section[mini-titulo]\{titulo enorme\}}. Isso é especialmente útil nas legendas (\textit{captions}\index{Legendas}) das figuras e tabelas, que muitas vezes são demasiadamente longas para a lista de figuras/tabelas.

\item \textbf{Existe algum programa para gerenciar referências em formato bibtex?}\\
Sim, há vários. Uma opção bem comum é o JabRef; outra é usar Zotero\index{Zotero} ou Mendeley\index{Mendeley} e exportar os dados deles no formato .bib.

\item \textbf{Posso usar pacotes \LaTeX{} adicionais aos sugeridos?}\\
Com certeza! Você pode modificar os arquivos o quanto desejar, o modelo serve só como uma ajuda inicial para o seu trabalho.

\item \textbf{Como faço para usar o Makefile (comando make) no Windows?}\\
Lembre-se que a ferramenta recomendada para compilação do documento é o \textsf{latexmk}, então você não precisa do \textsf{make}. Mas, se quiser usá-lo, você pode instalar o MSYS2 (\url{www.msys2.org}) ou o Windows Subsystem for Linux (procure as versões de Linux disponíveis na Microsoft Store). Se você pretende usar algum dos editores sugeridos, é possível deixar a compilação a cargo deles, também dispensando o \textsf{make}.\looseness=-1

\end{itemize}

% \par


%%%%%%%%%%%%%%% SEÇÕES FINAIS (BIBLIOGRAFIA E ÍNDICE REMISSIVO) %%%%%%%%%%%%%%%%

% O comando backmatter desabilita a numeração de capítulos.
\backmatter

\pagestyle{backmatter}

% Espaço adicional no sumário antes das referências / índice remissivo
\addtocontents{toc}{\vspace{2\baselineskip plus .5\baselineskip minus .5\baselineskip}}

% A bibliografia é obrigatória

\printbibliography[
  title=\refname\label{bibliografia}, % "Referências", recomendado pela ABNT
  %title=\bibname\label{bibliografia}, % "Bibliografia"
  heading=bibintoc, % Inclui a bibliografia no sumário
]

\printindex % imprime o índice remissivo no documento (opcional)

\end{document}
