%!TeX root=../tcc.tex
%("dica" para o editor de texto: este arquivo é parte de um documento maior)
% para saber mais: https://tex.stackexchange.com/q/78101

%% ------------------------------------------------------------------------- %%

% "\chapter" cria um capítulo com número e o coloca no sumário; "\chapter*"
% cria um capítulo sem número e não o coloca no sumário. A introdução não
% deve ser numerada, mas deve aparecer no sumário. Por conta disso, este
% modelo define o comando "\unnumberedchapter".
\unnumberedchapter{Introdução}
\label{cap:introducao}

\enlargethispage{.5\baselineskip}

Quando desejamos criar algoritmos para resolver problemas com o computador,
utilizamos maneiras de organizar os dados, as chamadas estruturas de dados,
para que operações de acesso e alteração desses dados possam ser realizadas
rapidamente.
A forma como serão organizados os dados depende altamente das características do
problema em questão.

Neste trabalho estudaremos \textit{estruturas de dados cinéticas} (em inglês,
\emph{KDS -- Kinetic Data Structures}), propostas por Basch, Guibas e Hershberger
[\cite{BASCH19991}] para a resolução dos chamados problemas \textit{cinéticos}.

Problemas \textit{cinéticos} são problemas em que deseja-se manter um
determinado atributo sobre objetos que estão em movimento contínuo.
Os objetos nos problemas podem representar entidades do mundo físico: pontos podem
representar pessoas, aviões, aparelhos móveis, entre outras coisas;
retas podem representar trajetórias.
Devido à natureza desses problemas é razoável que estudemos problemas clássicos de
geometria computacional, mas dentro de um contexto cinético.
Por exemplo, num conjunto dado de pontos em movimento, qual par de pontos possui
distância mínima.

Quando se tem dado um conjunto fixo de objetos geométricos, e deseja-se saber
informações de um determinado atributo desses objetos (como, por exemplo, em um
conjunto dado de pontos, qual par de pontos possui distância mínima), dizemos
que esse é um problema \textit{estático}.

O mesmo problema pode ser formulado sobre um conjunto mutável.
Por exemplo, pontos poderiam ser inseridos e removidos ao longo do tempo.
Queremos calcular o atributo sem ter que resolver do zero a nova instância do problema estático.
Chamamos esse tipo de problema de \textit{dinâmico} ou \textit{on-line}.

As \emph{estruturas de dados cinéticas} recebem esse nome para diferenciá-las
das estruturas de dados \textit{estáticas} e \textit{dinâmicas}, pois têm como
foco manter a descrição combinatória do problema que se altera frequentemente
com a passagem de tempo, já que os objetos estão em movimento contínuo.

Essas estruturas nos permitem realizar consultas de um determinado atributo dos
objetos, no instante atual.
A garantia de que a estrutura permanece correta se dá através do uso de instrumentos chamados \textit{certificados}.
Os certificados estabelecem que uma relação entre um objeto da estrutura e outro se
mantém verdadeira até o seu vencimento e devem ajudar na manutenção da estrutura
para permitir as consultas desejadas.
Chamaremos o instante de tempo em que o certificado vence de valor ou
\textit{prazo de validade} do certificado.

As estruturas contarão com uma operação \textsc{advance}, responsável por
avançar até o instante de tempo atual $t$ mantendo a estrutura correta.
Para tal, é necessário que nenhum certificado esteja vencido no instante $t$, ou seja, o
certificado de menor prazo de validade expira após o instante $t$.
Sendo assim, estamos interessados nos certificados de menor prazo de validade, para que
possamos realizar os ajustes necessários enquanto existir um certificado que
expira antes do instante de tempo que desejamos alcançar.

Para identificar o instante de vencimento de certificados, manteremos os certificados em uma fila
com prioridades, utilizando como prioridade o prazo de validade do certificado.

Para calcular o prazo de validade dos certificados, utilizaremos o chamado
\textit{plano de vôo} dos objetos.
O plano de vôo de um objeto é uma função que, dado o instante de tempo atual, determina a posição
corrente do objeto.
Assim como na vida real, o plano de vôo pode sofrer mudanças.
Essas mudanças no plano de vôo geram a necessidade de atualização de certificados
e de ajustes nas estruturas.
A operação \textsc{change} será utilizada para atualizar o plano de vôo dos
objetos e realizar as mudanças necessárias para manter a estrutura correta.

Por fim, a operação \textsc{query} ficará responsável por retornar o valor do atributo
geométrico em questão no instante atual.

Uma questão natural a ser feita a respeito destas estruturas é como medir o
desempenho delas, já que as formas clássicas de determinar a complexidade de
algoritmos não se enquadram muito bem, por conta da adição da dimensão tempo.
Basch, Guibas e Hershberger [\cite{BASCH19991}] propuseram algumas formas de avaliar essas
estruturas.
São elas:
\begin{itemize}
    \item \textbf{Responsividade}: uma estrutura é dita \textit{responsiva} se o custo de
    processar um certificado, isto é, o custo de atualizar os certificados e as
    outras estruturas necessárias, é pequeno;

    \item \textbf{Eficiência}: uma estrutura é dita \textit{eficiente} se a razão entre a
    quantidade total de eventos processados e a quantidade de eventos
    \textit{externos} é pequena.
    Um evento diz respeito ao vencimento de um certificado.
    Os eventos chamados \textit{externos} são os que geram mudanças na descrição combinatória
    do atributo, enquanto os eventos chamados \textit{internos} não geram mudanças na descrição
    combinatória do atributo, mas ainda são necessários para manter a estrutura.
    O total de eventos processados é a soma da quantidade de eventos externos e internos;

    \item \textbf{Compacidade}: uma estrutura é dita \textit{compacta} se a quantidade máxima de
    certificados que podem estar na fila com prioridades em um determinado instante é linear no
    número de objetos;

    \item \textbf{Localidade}: uma estrutura é dita \textit{local} se a quantidade máxima
    de certificados na fila que estão relacionados com um determinado objeto é
    pequena.
\end{itemize}

O custo de uma operação é dito pequeno se o custo é assintoticamente
polilogarítmico no número de objetos ou polinomial para um valor pequeno no expoente.

Neste trabalho estudaremos alguns problemas cinéticos e as estruturas utilizadas
para resolvê-los.
Apesar do modelo proposto por Basch, Guibas e Hershberger [\cite{BASCH19991}]
funcionar para trajetórias não-lineares, nos restringiremos apenas a trajetórias
lineares.
Neste caso, cada objeto corresponde a um par $(x_0, v)$, onde $x_0$ representa
o valor inicial do objeto e $v$ é a velocidade atual do objeto.

No Capítulo 1 estudaremos o problema da ordenação cinética.
Neste problema, os pontos movem-se apenas em uma dimensão $x(t) = x_0 + vt$ e
desejamos saber, para um dado~$i$, qual dos pontos possui o $i$-ésimo maior $x$ valor na coleção em
determinado instante $t$.
As estruturas que consideraremos para resolver o problema são a lista
ordenada cinética e uma árvore binária balanceada de busca.
A árvore binária balanceada de busca, além de suportar as operação já citadas,
também será capaz de realizar operações \textsc{insert} e \textsc{delete}.

No Capítulo 2 estudaremos o problema do máximo cinético.
Assim como no problema anterior, os pontos movem-se apenas em uma
dimensão $x(t) = x_0~+~vt$ e desejamos saber qual dos pontos possui o maior valor
no instante atual.
Apesar deste ser um caso particular do problema anterior, pois basta buscar pelo primeiro
maior valor $x$ na coleção, veremos que é possível obter essa resposta de forma mais
eficiente utilizando outras estruturas.

No Capítulo 3 estudaremos o problema do par mais próximo cinético, cuja resolução
utilizaremos estruturas vistas nos dois capítulos anteriores.
Diferentemente dos problemas anteriores, os pontos dados movem-se em duas dimensões
de acordo com uma função $\gamma(t) = (x(t), y(t))$, sendo que $x(t) = x_0 + v_x t$ e
$y(t) = y_0 + v_y t$.
Desejamos saber qual dos pares possíveis dos pontos dados possui distância mínima.

No Capítulo 4 estudaremos o problema da triangulação de Delaunay cinética no plano.
Assim como no problema do par mais próximo, os pontos movem-se em duas dimensões de acordo com uma
função $\gamma(t) = (x(t), y(t))$, sendo que $x(t) = x_0 + v_x t$ e $y(t) = y_0 + v_y t$.
Desejamos manter uma descrição da triangulação de Delaunay dos pontos dados.

O material apresentado nos Capítulos 1, 2, 3 e 4 baseou-se inicialmente na dissertação de Eduardo
Freitas~[\cite{eduardo}].
O material do Capítulo 3 baseou-se também no artigo de Basch, Guibas e
Hershberger~[\cite{BASCH19991}] e na tese de Basch [\cite{basch-thesis}].
Já o material do Capítulo 4 baseou-se em~\cite{computationalgeometry}.

As estruturas de dados descritas no Capítulo~\ref{ch:ordenacao-cinetica},
Capítulo~\ref{ch:maximo-cinetico} e Capítulo~\ref{ch:par-mais-proximo-cinetico} foram implementadas
em linguagem C e estão disponíveis em \href{https://github
.com/siolinm/mac0499/tree/main/implementacao}{github.com/siolinm/mac0499}.
Para o algoritmo do Capítulo~\ref{ch:par-mais-proximo-cinetico}, foi construída uma animação
utilizando a biblioteca \href{https://www.cairographics.org/}{cairo}.
