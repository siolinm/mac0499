%!TeX root=../tcc.tex
%("dica" para o editor de texto: este arquivo é parte de um documento maior)
% para saber mais: https://tex.stackexchange.com/q/78101

%% ------------------------------------------------------------------------- %%

% "\chapter" cria um capítulo com número e o coloca no sumário; "\chapter*"
% cria um capítulo sem número e não o coloca no sumário. A introdução não
% deve ser numerada, mas deve aparecer no sumário. Por conta disso, este
% modelo define o comando "\unnumberedchapter".
\unnumberedchapter{Introdução}
\label{cap:introducao}

\enlargethispage{.5\baselineskip}

Quando desejamos criar algoritmos para resolver problemas com o computador,
utilizamos maneiras de organizar os dados para que operações de acesso e
alteração desses dados possam ser realizadas rapidamente, as chamadas estruturas
de dados. A forma como serão organizados os dados depende altamente das
características do problema em questão.

Neste trabalho estudaremos \textit{estruturas de dados cinéticas} (em inglês,
\emph{KDS - Kinetic Data Structures}), propostas por Basch, Guibas e Hershberger
% referencia bibliográfica aqui
para a resolução dos chamados problemas \textit{cinéticos}.

Problemas \textit{cinéticos} são problemas em que deseja-se manter um
determinado atributo sobre objetos que estão em movimento contínuo. Por exemplo,
num conjunto dado de pontos em movimento, qual par de pontos possui distância
mínima. Os objetos nos problemas podem representar entidades do mundo físico:
pontos podem representar pessoas, aviões, estabelecimentos, entre outras coisas,
retas podem representar trajetórias. Devido à natureza desses problemas é comum
que olhemos para problemas de geometria computacional, mas dentro de um contexto
cinético.

Quando é dado um conjunto fixo de objetos geométricos, e deseja-se saber
informações de um determinado atributo desses objetos (como, por exemplo, em um
conjunto dado de pontos, qual par de pontos possui distância mínima), dizemos
que esse é um problema \textit{estático}.

O mesmo problema pode ser formulado sobre um conjunto mutável. Por exemplo,
pontos poderiam ser inseridos e removidos ao longo do tempo. Queremos calcular o
atributo sem ter que resolver do zero a nova instância do problema estático.
Chamamos esse tipo de problema de \textit{dinâmico} ou \textit{on-line}.

As \emph{estruturas de dados cinéticas} recebem esse nome para diferenciá-las
das estruturas de dados \textit{estáticas} e \textit{dinâmicas}, pois tem como
foco em manter a descrição combinatória do problema, que agora também se altera
% definir o que é descrição combinatória
com a passagem de tempo, já que os objetos estão em movimento contínuo.

% definir o que é o atributo geométrico dos objetos
Essas estruturas nos permitem realizar consultas de um determinado atributo dos
objetos, no instante atual. A garantia de que a estrutura permanece correta se
dá através do uso de instrumentos chamados \textit{certificados}. Os
certificados estabelecem que uma relação entre um objeto da estrutura e outro se
mantém verdadeira até o seu vencimento e devem ajudar na manutenção da estrutura
para permitir as consultas desejadas.

Nos problemas estudados neste trabalho os objetos serão pontos. Precisaremos
saber como os pontos se movimentam para determinar as relações entre os objetos
na estrutura e calcular o vencimento de certificados, o chamado \textit{plano de
vôo}. Uma operação de atualização pode ser realizada neste \emph{plano de vôo}, o que
implicará em mudanças a serem feitas nos certificados da estrutura. Neste
trabalho, a trajetória dos pontos será linear, descrita por uma função
$\gamma(t) = (x(t), y(t))$, sendo $x(t)$ e $y(t)$ funções afim. O modelo
proposto por Basch, Guibas e Hershberger também pode ser aplicado para
trajetórias não-lineares.

Falar sobre forma trivial de resolver, porque não funciona e usar de gatilho
para explicar as estruturas de dados e as medidas de eficiência.

% Poderíamos usar certificados por exemplo entre alguns pares de pontos que,
% considerando suas trajetórias atuais, garantissem que, até o instante $t$, a
% ordenada de um dos pontos é maior que a ordenada do outro ponto. Ao atingirmos o
% instante $t$, o certificado vence, implicando em uma mudança estrutural no
% conjunto de pontos que pode afetar o resultado de futuras consultas, requerendo
% assim possíveis ajustes na estrutura de dados, e eventual cálculo de novos
% certificados.
