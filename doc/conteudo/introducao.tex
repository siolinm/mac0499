%!TeX root=../tcc.tex
%("dica" para o editor de texto: este arquivo é parte de um documento maior)
% para saber mais: https://tex.stackexchange.com/q/78101

%% ------------------------------------------------------------------------- %%

% "\chapter" cria um capítulo com número e o coloca no sumário; "\chapter*"
% cria um capítulo sem número e não o coloca no sumário. A introdução não
% deve ser numerada, mas deve aparecer no sumário. Por conta disso, este
% modelo define o comando "\unnumberedchapter".
\unnumberedchapter{Introdução}
\label{cap:introducao}

\enlargethispage{.5\baselineskip}

Quando desejamos criar algoritmos para resolver problemas com o computador,
utilizamos maneiras de organizar os dados para que operações de acesso e
alteração desses dados possam ser realizadas rapidamente, as chamadas estruturas
de dados. A forma como serão organizados os dados depende altamente das
características do problema em questão.

Neste trabalho estudaremos \textit{estruturas de dados cinéticas} (em inglês,
\emph{KDS - Kinetic Data Structures}), propostas por Basch, Guibas e Hershberger
% referencia bibliográfica aqui
para a resolução dos chamados problemas \textit{cinéticos}.

Problemas \textit{cinéticos} são problemas em que deseja-se manter um
determinado atributo sobre objetos que estão em movimento contínuo. Os objetos
nos problemas podem representar entidades do mundo físico: pontos podem
representar pessoas, aviões, estabelecimentos, entre outras coisas, retas podem
representar trajetórias. Devido à natureza desses problemas é razóavel que
estudemos problemas clássicos de geometria computacional, mas dentro de um
contexto cinético. Por exemplo, num conjunto dado de pontos em movimento, qual
par de pontos possui distância mínima.

Quando é dado um conjunto fixo de objetos geométricos, e deseja-se saber
informações de um determinado atributo desses objetos (como, por exemplo, em um
conjunto dado de pontos, qual par de pontos possui distância mínima), dizemos
que esse é um problema \textit{estático}.

O mesmo problema pode ser formulado sobre um conjunto mutável. Por exemplo,
pontos poderiam ser inseridos e removidos ao longo do tempo. Queremos calcular o
atributo sem ter que resolver do zero a nova instância do problema estático.
Chamamos esse tipo de problema de \textit{dinâmico} ou \textit{on-line}.

As \emph{estruturas de dados cinéticas} recebem esse nome para diferenciá-las
das estruturas de dados \textit{estáticas} e \textit{dinâmicas}, pois têm como
foco manter a descrição combinatória do problema que se altera frequentemente
% definir o que é descrição combinatória
com a passagem de tempo, já que os objetos estão em movimento contínuo.

% definir o que é o atributo geométrico dos objetos
Essas estruturas nos permitem realizar consultas de um determinado atributo dos
objetos, no instante atual. A garantia de que a estrutura permanece correta se
dá através do uso de instrumentos chamados \textit{certificados}. Os
certificados estabelecem que uma relação entre um objeto da estrutura e outro se
mantém verdadeira até o seu vencimento e devem ajudar na manutenção da estrutura
para permitir as consultas desejadas. Chamaremos o instante de tempo em que o
certificado vence de valor ou \textit{prazo de validade} do certificado.

As estruturas contarão com uma operação \textsc{advance}, responsável por
avançar até o instante de tempo atual mantendo a estrutura correta. Para tal, é
necessário que nenhum certificado esteja vencido no instante $t$, ou seja, o
certificado de menor prazo de validade expira após o instante $t$. Sendo assim,
estamos interessados nos certificados de menor prazo de validade, para que
possamos realizar os ajustes necessários enquanto existir um certificado que
expira antes do instante de tempo que desejamos alcançar.

Para identificar o instante de vencimento de certificados manteremos uma fila
com prioridades utilizando como prioridade o prazo de validade do certificado.

Para calcular o prazo de validade dos certificados, utilizaremos o chamado
\textit{plano de vôo} dos objetos. O plano de vôo de um objeto é uma função que,
dado o instante de tempo atual, determina sua trajetória. Assim como na vida
real, o plano de vôo pode sofrer mudanças. Essas mudanças no plano de vôo geram
a necessidade de atualização de certificados e de ajustes nas estruturas. A
operação \textsc{change} será utilizada para atualizar o plano de vôo dos
objetos e realizar as mudanças necessárias para manter a estrutura correta.

% detalhe importante: a operação query só responde para um instante >= atual
Por fim, a operação \textsc{query} ficará responsável por responder o atributo
geométrico que desejamos saber num dado instante.

Uma questão natural a ser feita a respeito destas estruturas é como medir o
desempenho delas, já que as formas clássicas de determinar a complexidade de
algoritmos não se enquadram muito bem, por conta da adição da dimensão tempo. % Ainda meio confuso...
Basch, Guibas e Hershberger propuseram algumas formas de analisá-las e medi-las.
São elas:
\begin{itemize}
    \item Responsividade: uma estrutura é dita \textit{responsiva} se o custo de
    processar um certificado, isto é, o custo de atualizar os certificados e as
    outras estruturas necessárias é pequeno;
    \item Eficiência: uma estrutura é dita \textit{eficiente} se a razão entre a
    quantidade total de eventos processados e a quantidade de eventos
    \textit{externos} é pequena. Um evento diz respeito ao vencimento de um
    certificado, os eventos chamados \textit{externos} são eventos que geram
    mudanças na descrição combinatória, enquanto eventos chamados
    \textit{internos} não geram mudanças na descrição combinatória, mas ainda
    são necessários para manter a estrutura. O total de eventos processados é a
    soma da quantidade de eventos externos e internos;
    \item Compacidade: uma estrutura é dita \textit{compacta} se a quantidade máxima de
    certificados que podem estar na fila com prioridades em um determinado é
    instante é linear;
    \item Localidade: uma estrutura é dita \textit{local} se a quantidade máxima
    de certificados na fila que estão relacionados com um determinado objeto é
    pequena.
\end{itemize}

O custo de uma operação é dito pequeno se o custo é assintoticamente
polilogarítmico ou polinomial para um valor pequeno no expoente.

Neste trabalho estudaremos alguns problema cinéticos e as estruturas utilizadas
para resolvê-los. Apesar do modelo proposto por Basch, Guibas e Hershberger
funcionar para trajetórias não-lineares, nos restringiremos apenas a trajetórias
lineares.

No capítulo 1 estudaremos o problema da ordenação cinética. Neste problema os
pontos movem-se apenas em uma dimensão $y(t) = x_0 + vt$ e desejamos saber qual
dos pontos possui o $i$-ésimo maior valor na coleção em determinado instante
$t$. As estruturas que consideraremos para resolver o problema são a lista
ordenada cinética e uma árvore binária balanceada de busca. A árvore binária
balanceada de busca além de suportar as operação já citadas também será capaz de
realizar operações \textsc{insert} e \textsc{delete}.

No capítulo 2 estudaremos o problema do máximo cinético. Assim como no problema
anterior, os pontos movem-se apenas em uma dimensão $y(t) = x_0 + vt$ e
desejamos saber qual dos pontos possui o maior valor no instante atual. Apesar
do problema anterior resolver este problema, pois basta buscar pelo primeiro
maior valor na coleção, veremos que é possível obter essa resposta de forma mais
eficiente utilizando outras estruturas.

No capítulo 3 estudaremos o problema do par mais próximo cinético, em que
utilizaremos estruturas vistas nos dois capítulos anteriores. Diferentemente dos
problemas anteriores, os pontos movem-se em duas dimensões de acordo com uma
função $\gamma(t) = (x(t), y(t))$, sendo que $x(t) = x_0 + v_xt$ e $y(t) = y_0 +
v_yt$. Desejamos saber qual dos pares possíveis de pontos possui distância
mínima.

No capítulo 4 


% Poderíamos usar certificados por exemplo entre alguns pares de pontos que,
% considerando suas trajetórias atuais, garantissem que, até o instante $t$, a
% ordenada de um dos pontos é maior que a ordenada do outro ponto. Ao atingirmos o
% instante $t$, o certificado vence, implicando em uma mudança estrutural no
% conjunto de pontos que pode afetar o resultado de futuras consultas, requerendo
% assim possíveis ajustes na estrutura de dados, e eventual cálculo de novos
% certificados.
