%!TeX root=../tcc.tex
%("dica" para o editor de texto: este arquivo é parte de um documento maior)
% para saber mais: https://tex.stackexchange.com/q/78101

%% ------------------------------------------------------------------------- %%

% "\chapter" cria um capítulo com número e o coloca no sumário; "\chapter*"
% cria um capítulo sem número e não o coloca no sumário. A introdução não
% deve ser numerada, mas deve aparecer no sumário. Por conta disso, este
% modelo define o comando "\unnumberedchapter".
\unnumberedchapter{Introdução}
\label{cap:introducao}

\enlargethispage{.5\baselineskip}

Em Geometria Computacional visa-se desenvolver algoritmos que
resolvam problemas geométricos de maneira eficiente. Há problemas,
em que esses algoritmos podem ser usados, que advêm de outras áreas
como: tráfego aéreo, computação gráfica, telefonia celular,
movimento de partículas, entre outras. Para o desenvolvimento de
tais algoritmos, são utilizados resultados de áreas como: geometria
euclidiana, teoria dos grafos, combinatória, estruturas de dados e
análise de algoritmos.

Os objetos nos problemas podem representar entidades do mundo
físico. Por exemplo, pontos podem representar pessoas, aviões,
estabelecimentos, entre outras coisas, retas podem representar
trajetórias.

Quando é dado um conjunto de objetos geométricos fixo, e deseja-se
saber informações de um determinado atributo desses objetos (como,
por exemplo, em um conjunto dado de pontos, qual par de pontos
possui distância mínima) dizemos que esse é um problema
\textit{estático}.

O mesmo problema pode ser formulado, porém sobre um conjunto
mutável. Por exemplo, pontos poderiam ser inseridos e removidos ao
longo do tempo. Queremos calcular o atributo sem ter que resolver do
zero a nova instância do problema estático. Chamamos esse tipo de
problema de \textit{dinâmico} ou \textit{on-line}.

Em uma outra formulação, os pontos poderiam estar em movimento
contínuo. Essa é a chamada versão \textit{cinética} do problema. É
nesse contexto que entram as chamadas \textit{estruturas de dados
cinéticas} (\emph{KDS - Kinetic Data Structures}).

Essas estruturas nos permitem realizar consultas de um determinado
atributo dos objetos, no instante atual. A garantia de que a
estrutura permanece correta se dá através do uso de instrumentos
chamados \textit{certificados}. Os certificados estabelecem que uma
relação entre um objeto da estrutura e outro se mantém verdadeira
até o seu vencimento e devem ajudar na manutenção da estrutura para
permitir as consultas desejadas. Poderíamos usar certificados por
exemplo entre alguns pares de pontos que, considerando suas
trajetórias atuais, garantissem que, até o instante $t$, a ordenada
de um dos pontos é maior que a ordenada do outro ponto. Ao
atingirmos o instante $t$, o certificado vence, implicando em uma
mudança estrutural no conjunto de pontos que pode afetar o resultado
de futuras consultas, requerendo assim possíveis ajustes na
estrutura de dados, e eventual cálculo de novos certificados.
