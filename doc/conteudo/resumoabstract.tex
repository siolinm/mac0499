%!TeX root=../tcc.tex
%("dica" para o editor de texto: este arquivo é parte de um documento maior)
% para saber mais: https://tex.stackexchange.com/q/78101

% As palavras-chave são obrigatórias, em português e em inglês, e devem ser
% definidas antes do resumo/abstract. Acrescente quantas forem necessárias.
\palavrachave{Estruturas de dados cinéticas}
\palavrachave{Geometria Computacional}
\palavrachave{Algoritmos}

\keyword{Kinetic data structures}
\keyword{Computational Geometry}
\keyword{Algorithms}

% O resumo é obrigatório, em português e inglês. Estes comandos também
% geram automaticamente a referência para o próprio documento, conforme
% as normas sugeridas da USP.
\resumo{

Estruturas de dados cinéticas são um modelo proposto por Basch, Guibas e
Hershberger para a resolução de problemas cinéticos de maneira eficiente.
Problemas cinéticos são problemas em que os objetos envolvidos estão em
movimento contínuo e desejamos saber um determinado atributo destes objetos no
instante atual. Por exemplo, consultar qual o par de pontos mais próximo no
momento, num conjunto de pontos em movimento.

As interface proposta para as estruturas oferece suporte a três operações:
consulta, que retorna o atributo mantido, avançar no tempo, que determina o
instante atual, e alterar a trajetória de objetos, caracterizada pela alteração
de uma função cujo nome é plano de vôo. O plano de vôo é necessário para manter
o atributo desejado sobre os elementos, ele define a trajetória dos pontos ao
longo do tempo e será utilizado para calcular os certificados, os certificados
são objetos que garantem que a organização interna da estrutura está correta até
um determinado instante, denominado prazo de validade do certificado. As
estruturas serão orientadas a eventos, que são o vencimento de certificados. Os
certificados serão colocados numa fila de prioridades com o seu prazo de
validade como prioridade. No momento em que um certificado se torna inválido é
necessário realizar mudanças na estrutura para que esta volte a se tornar
válida, removendo os certificados inválidos e gerando novos certificados no
processo.

Com a inclusão da dimensão tempo, a forma tradicional de analisar algoritmos não
é adequada para a análise dessas estruturas. Por isso, Basch, Guibas e
Hershberger também propuseram outros quatro critérios com o intuito de
determinar a eficiência de cada estrutura: responsividade, eficiência,
localidade e compacidade.

Neste trabalho, estudaremos os problemas da ordenação, máximo, par mais próximo
e triangulação de delaunay, todos num contexto cinético. Também estudaremos as
respectivas estruturas utilizadas na solução de cada problema e discutiremos
sobre os critérios de eficiência em cada uma delas. Para algumas das estruturas
também discutiremos um cenário dinâmico-cinético, ou seja, operações de inserção
e remoção dentro de um contexto cinético.

}

\abstract{
Elemento obrigatório, elaborado com as mesmas características do resumo em
língua portuguesa. De acordo com o Regimento da Pós-Graduação da USP (Artigo
99), deve ser redigido em inglês para fins de divulgação. É uma boa ideia usar
o sítio \url{www.grammarly.com} na preparação de textos em inglês.
Text text text text text text text text text text text text text text text text
text text text text text text text text text text text text text text text text
text text text text text text text text text text text text text text text text
text text text text text text text text text text text text.
Text text text text text text text text text text text text text text text text
text text text text text text text text text text text text text text text text
text text text. Preencher depois.
}
