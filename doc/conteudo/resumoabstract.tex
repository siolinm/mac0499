%!TeX root=../tcc.tex
%("dica" para o editor de texto: este arquivo é parte de um documento maior)
% para saber mais: https://tex.stackexchange.com/q/78101

% As palavras-chave são obrigatórias, em português e em inglês, e devem ser
% definidas antes do resumo/abstract. Acrescente quantas forem necessárias.
\palavrachave{Estruturas de dados cinéticas}
\palavrachave{Geometria computacional}
\palavrachave{Algoritmos}

\keyword{Kinetic data structures}
\keyword{Computational geometry}
\keyword{Algorithms}

% O resumo é obrigatório, em português e inglês. Estes comandos também
% geram automaticamente a referência para o próprio documento, conforme
% as normas sugeridas da USP.
\resumo{

    Estruturas de dados cinéticas são um modelo proposto por Basch, Guibas e
    Hershberger para a resolução de problemas cinéticos de maneira eficiente.
    Problemas cinéticos são problemas em que os objetos envolvidos estão em
    movimento contínuo e desejamos saber um determinado atributo destes objetos no
    instante atual. Por exemplo, consultar qual o par de pontos mais próximo no
    momento, num conjunto de pontos em movimento.

    A interface proposta para as estruturas oferece suporte a três operações:
    consulta, que retorna o atributo mantido, avançar no tempo, que altera o
    instante atual para um dado instante no futuro, e alterar a trajetória de objetos,
    caracterizada pela alteração de uma função cujo nome é plano de vôo.
    O plano de vôo é necessário para manter o atributo desejado sobre os elementos: ele
    define a trajetória dos pontos ao longo do tempo e será utilizado para calcular os chamados
    certificados.
    Os certificados são objetos que garantem que a organização interna da estrutura está correta até
    um determinado instante, denominado prazo de validade do certificado.
    As estruturas serão orientadas a eventos, que são o vencimento de certificados.
    Os certificados serão mantidos numa fila de prioridades com o seu prazo de
    validade como prioridade.
    O vencimento de um certificado significa que a estrutura ficou inválida e é necessário realizar
    mudanças para que ela volte a se tornar válida, removendo os certificados vencidos e gerando novos
    certificados no processo.

    Com a inclusão da dimensão tempo, a forma tradicional de analisar algoritmos não
    é adequada para a análise dessas estruturas. Por isso, Basch, Guibas e
    Hershberger também propuseram outros quatro critérios com o intuito de
    determinar a eficiência de cada estrutura: responsividade, eficiência,
    localidade e compacidade.

    Neste trabalho, estudaremos quatro problemas: ordenação, máximo, par mais próximo
    e triangulação de Delaunay, todos num contexto cinético. Também estudaremos as
    respectivas estruturas utilizadas na solução de cada problema e discutiremos
    sobre os critérios de eficiência em cada uma delas. Para algumas das estruturas
    também discutiremos um cenário dinâmico-cinético, ou seja, consideraremos operações
    de inserção e remoção dentro de um contexto cinético.

}

\abstract{

    Kinetic data structures are a model proposed by Basch, Guibas and Hershberger to
    efficiently solve kinetic problems. Kinetic problems are problems in which the
    objects involved are in continuous motion and we want to know a certain
    attribute about these objects at the current moment. For example, we would like to
    query the closest pair of points at the moment, in a set of moving points.

    The proposed interface for the data structures supports three operations: query,
    that returns the maintained attribute, advance, that adjusts the current moment to a given instant in
    the future, and changing the object motion, given by updates on a function called flight
    plan.
    The flight plan is necessary to maintain the attribute of interest over
    the elements, it defines the object motion through time and will be used for
    computing the so called certificates.
    The certificates are objects that ensure the internal state of our data structure is correct until a
    certain instant of time: the certificate's expiration time.
    Our data structures will be event driven where the events are the expirations of certificates.
    The certificates will be kept in a priority queue with their expiration time as priority.
    The expiration of a certificate means the data structure has become invalid and we need to change it
    in order to be in a correct state again, deleting the expired certificates and generating
    new ones in the process.

    The traditional way of analyzing algorithms does not work well with these
    structures.
    Because of that, Basch, Guibas and Hershberger also proposed four
    criteria to determine the quality of the structures: responsiveness, efficiency,
    locality, and compactness.

    In this work, we address four problems: sorting, maximum, closest pair
    and Delaunay triangulation, all in a kinetic context.
    We study the respective kinetic data structures used to solve these problems and discuss
    the quality criteria for each one of them.
    For some of the structures we also consider a kinetic-dynamic scenario,
    where insert and delete operations will also be supported in a kinetic context.

}
