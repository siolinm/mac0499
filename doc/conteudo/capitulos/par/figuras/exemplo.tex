\begin{figure}[H]
    \centering
    \begin{tikzpicture}[thick]
        % \draw[step=1cm,lightgray,very thin] (-2,-2) grid (10,10);
        \draw[thick,->] (0,0) -- (7,0) node[anchor=north west]
            {$x$};
        \draw[thick,->] (0,0) -- (0,5) node[anchor=south east]
            {$y$};
        \foreach \Point in {(1, 0), (5, -1), (0, 2), (3, 2), (3, 1)}{
            \fill \Point circle[radius=2pt];
            % \node at \Point {\textbullet};
        }
        \draw[->] (1, 0) -- (1.5, 0.25);
        \draw[->] (5, -1) -- (4.75, -0.5);
        \draw[->] (0, 2) -- (0.25, 1.75);
        \draw[->] (3, 2) -- (3.25, 1.5);
        \draw[->] (3, 1) -- (2.5, 1);
        \foreach \Point in {(5, -2), (1, 1), (5, 2), (2,0), (3, 3)}{
            % \fill \Point circle[radius=2pt];
            \draw \Point circle[radius=2pt];
        }
        \draw[dashed] (3, 1) -- (3, 2);
        \draw[dotted] (1, 1) -- (2, 0);
    \end{tikzpicture}
    \caption[Exemplo do problema par mais próximo]{Os pontos
    preenchidos em preto representam a coleção no instante $t = 0$.
    Os pontos não preenchidos representam a coleção no instante $t =
    2$. As setas representam a direção e sentido do vetor velocidade
    de cada ponto. A linha tracejada representa a distância entre o
    par mais próximo no instante $t = 0$, enquanto a linha
    pontilhada representa a distância entre o par mais próximo no
    instante $t = 2$.}
    \label{fig:parestatico:exemplo}
\end{figure}