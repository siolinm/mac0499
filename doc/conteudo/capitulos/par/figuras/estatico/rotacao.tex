\begin{figure}[H]
    \centering
    \begin{tikzpicture}[thick,scale=0.8]
        \coordinate (a) at (0, 0);
        \coordinate (b) at (4, 0);
        \coordinate (c) at (2.828, 2.828);
        \coordinate (p) at (1, 2);
        \draw[thick,->] (0,0) -- (4,0)
        node[anchor=north west] {$x$};
        \draw[thick,->] (0,0) -- (0,4)
        node[anchor=south east] {$y$};
        \draw[thick,->, dashed] (0,0) -- (2.828,2.828)
        node[anchor=north west] {$x'$};
        \draw[thick,->, dashed] (0,0) -- (-2.828,2.828)
        node[anchor=south east] {$y'$};
        \pic [draw, angle radius = 0.5cm] {angle = b--a--c};
        \node[anchor=west, label={[label distance = 0mm]180:$\theta$}]
        (angl) at (1, 0.3) {};
        \node[label=90:$p$] at (p) {\textbullet};
        \draw[dashed,->] (a) -- node[above] {$r$} (p);
        \pic [draw, angle radius = 1cm] {angle = b--a--p};
        \node[anchor=west, label={[label distance = -3mm]180:$\phi$}]
        (angp) at (1.2, 0.7) {};
    \end{tikzpicture}
    \caption{O ponto $p$ está numa inclinação de $\phi - \theta$ radianos
    em relação a reta que passa pela origem e por $x'$.}
    \label{fig:parestatico:rotacao}
\end{figure}