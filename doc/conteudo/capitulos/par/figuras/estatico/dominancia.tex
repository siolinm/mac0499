\begin{figure}[H]
    \centering
    \begin{tikzpicture}[thick, scale=0.8]
        % \draw (0, -3) -- (0, 3) node[anchor=north west] {$y$};
        \draw (0, 0) -- (4, 2.309);
        \draw[dashed] (0, 0) -- (4, -2.309);
        \draw (0, 0) circle (2pt) node[label=250:$p$] {};
        \node[label=0:$q$] (q) at (1, 0) {\textbullet};
        \node[label=250:$r$] (r) at (3, 1.715) {\textbullet};
        \draw (3, -1.73) circle (2pt) node[label=250:$s$] {};
        % \node[label=250:$s$] (s) at (3, -1.73);
    \end{tikzpicture}
    \caption[Exemplo de $\Dom(p)$]{Os pontos $q$ e $r$ pertencem a $\Dom(p)$,
    mas os pontos $p$ e $s$ não.}
    \label{fig:parestatico:dominancia}
\end{figure}