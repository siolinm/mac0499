\begin{figure}[H]
    \centering
    \begin{tikzpicture}
        \coordinate (a) at (-1, 0.5);
        \coordinate (b) at (4, 0);
        \coordinate (c) at (4.5, 3);
        \coordinate (d) at (0, 4);
        \draw (a) -- (b);
        \draw (b) -- (c);
        \draw (c) -- (d);
        \draw (d) -- (a);
        \draw (a) -- (c);
        \pic [draw, angle radius = 1cm] {angle = c--b--a};
        \pic [draw, angle radius = 1cm] {angle = a--c--b};
        \pic [draw, angle radius = 1cm] {angle = b--a--c};

        \pic [draw, angle radius = 0.5cm] {angle = c--a--d};
        \pic [draw, angle radius = 0.5cm] {angle = d--c--a};
        \pic [draw, angle radius = 0.5cm] {angle = a--d--c};

        \begin{scope}
            [shift={(7, 0)}]
            \coordinate (a) at (-1, 0.5);
            \coordinate (b) at (4, 0);
            \coordinate (c) at (4.5, 3);
            \coordinate (d) at (0, 4);
            \draw (a) -- (b);
            \draw (b) -- (c);
            \draw (c) -- (d);
            \draw (d) -- (a);
            \draw (b) -- (d);
            \pic [draw, angle radius = 0.5cm] {angle = c--b--d};
            \pic [draw, angle radius = 0.5cm] {angle = d--c--b};
            \pic [draw, angle radius = 0.5cm] {angle = b--d--c};

            \pic [draw, angle radius = 1cm] {angle = b--a--d};
            \pic [draw, angle radius = 1cm] {angle = d--b--a};
            \pic [draw, angle radius = 1cm] {angle = a--d--b};
        \end{scope}

    \end{tikzpicture}
    \caption[Exemplo troca de aresta]{Após trocar os vértices da aresta do quadrilátero
    convexo, seis novos ângulos se formaram.}\label{fig:delaunay:flip}
\end{figure}
