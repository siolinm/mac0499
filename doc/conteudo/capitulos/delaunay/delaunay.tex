%!TeX root=../../tcc.tex

\chapter{Triangulação de Delaunay cinética}

% Descrever o problema cinético

Considere o seguinte problema cinético. São dados $n$ pontos movendo-se
linearmente no plano. Cada ponto é representado por um par $(s_0, \vec{v})$ onde
$s_0 = (x_0, y_0)$ é a sua posição inicial e $\vec{v} = (v_x, v_y)$ um vetor
velocidade. A posição de um determinado ponto $p$, num instante arbitrário $t
\geq 0$, é $s_p = (x_p, y_p) = (x_0, y_0)~+~t\cdot \vec{v}$. Queremos saber o
conjunto de arestas que define a triangulação de Delaunay desses pontos, num
instante arbitrário $t \geq 0$.

Por exemplo, se tivermos 10 pontos na coleção: 

% Descrever triangulação de delaunay como dual do diagrama de Voronoi (colocar
% uma imagem)

% Algoritmo
% Descrever como representar os objetos (livro de geometria computacional) DCEL
% Descrever um algoritmo simples para triangulação de delaunay (flipar arestas
% até que elas se tornem válidas)
% Fazer pseudocódigo das operações Event, change, etc. (manipulando a DCEL)


