%!TeX root=../tcc.tex

\unnumberedchapter{Conclusão}\label{cap:conclusao}

Este trabalho se propõe a apresentar de forma simplificada as estruturas de dados cinéticas, analisar o seu desempenho e
também implementar as estruturas.

O plano inicial era implementar e explicar com mais detalhes todas as estruturas, porém a estrutura para manter a
triangulação de Delaunay, apresentada no Capítulo~\ref{ch:triangulacao-de-delaunay-cinetica},
não foi implementada por restrições de tempo e por este motivo também não foi detalhada neste trabalho.

As estruturas de dados descritas no Capítulo~\ref{ch:ordenacao-cinetica},
Capítulo~\ref{ch:maximo-cinetico} e Capítulo~\ref{ch:par-mais-proximo-cinetico} foram implementadas
em linguagem C e estão disponíveis em \href{https://github
.com/siolinm/mac0499/tree/main/implementacao}{github.com/siolinm/mac0499}.
Para o algoritmo do Capítulo~\ref{ch:par-mais-proximo-cinetico}, foi construída uma animação
utilizando a biblioteca \href{https://www.cairographics.org/}{cairo}.

Um grande obstáculo durante o estudo do Capítulo~\ref{ch:par-mais-proximo-cinetico} foram casos degenerados em que $2$
ou mais pontos


