%!TeX root=../tcc.tex

\unnumberedchapter{Conclusão}\label{cap:conclusao}

Neste trabalho, apresentamos de forma simplificada as estruturas de dados cinéticas, analisamos o seu desempenho e
implementamos as estruturas.

O plano inicial era implementar e explicar com mais detalhes a implementação de cada uma das estruturas.
Porém, a estrutura para manter a triangulação de Delaunay, que é apresentada no
Capítulo~\ref{ch:triangulacao-de-delaunay-cinetica}, não foi implementada e por este motivo não foram apresentados
tantos detalhes de implementação como nas outras.

As estruturas de dados descritas no Capítulo~\ref{ch:ordenacao-cinetica},
Capítulo~\ref{ch:maximo-cinetico} e Capítulo~\ref{ch:par-mais-proximo-cinetico} foram implementadas
em linguagem C e estão disponíveis em \href{https://github
.com/siolinm/mac0499/tree/main/implementacao}{github.com/siolinm/mac0499}.
Para o algoritmo do Capítulo~\ref{ch:par-mais-proximo-cinetico}, foi construída uma animação
utilizando a biblioteca \href{https://www.cairographics.org/}{cairo}.

Um grande obstáculo que tentamos superar durante o estudo do Capítulo~\ref{ch:par-mais-proximo-cinetico}
foi o de casos degenerados em que $2$ ou mais pontos assumem a mesma coordenada em um determinado instante de tempo.
Dependendo da ordem em que os eventos eram processados, as estruturas entravam em um estado inconsistente para
o instante $t + \epsilon$.
Após algum tempo, não conseguimos adaptar o algoritmo para conseguir lidar com esses casos, e por isso seguimos para
outros assuntos sem resolver esse problema.

